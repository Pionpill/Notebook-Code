\section{Lombok 注解}

Lombok 依赖坐标如下:

\begin{xml}
<dependency>
    <groupId>org.projectlombok</groupId>
    <artifactId>lombok</artifactId>
</dependency>
\end{xml}

和 Validation,Lombok 没有被集成到 spring 中,为什么呢?因为没必要,Lombok 是在编译阶段改动语法树达到效果(所以几乎所有注解 Retention 都是 SOURCE,CLASS),相当于改变了 Java 源码,和 spring 没什么关系。

\subsection{验证性注解}

\subsubsection{@NonNull}

用于限定被注解的属性不能为 Null。

\begin{Java}
@Target({ElementType.FIELD, ElementType.METHOD, ElementType.PARAMETER, ElementType.LOCAL_VARIABLE, ElementType.TYPE_USE})
@Retention(RetentionPolicy.CLASS)
@Documented
public @interface NonNull { }
\end{Java}

等效代码:

\begin{Java}
// Lombok
public NonNullExample(@NonNull Person person) {
    this.name = person.getName();
}
// Java
public NonNullExample(@NonNull Person person) {
    if (person == null) {
      throw new NullPointerException("person is marked non-null but is null");
    }
    this.name = person.getName();
}
\end{Java}

\subsection{辅助性注解}
\subsubsection{@Cleanup}

该注解用于自动调用 close() 方法释放资源,和 try-with-resource 语句有异曲同工之处。

\begin{Java}
@Target(ElementType.LOCAL_VARIABLE)
@Retention(RetentionPolicy.SOURCE)
public @interface Cleanup {
	String value() default "close";
}
\end{Java}

等效代码:

\begin{Java}
// Lombok
@Cleanup InputStream in = new FileInputStream(args[0]);
@Cleanup OutputStream out = new FileOutputStream(args[1]);
byte[] b = new byte[10000];
while (true) {
    int r = in.read(b);
    if (r == -1) break;
    out.write(b, 0, r);
}
// Java
InputStream in = new FileInputStream(args[0]);
    try {
        OutputStream out = new FileOutputStream(args[1]);
        try {
            byte[] b = new byte[10000];
            while (true) {
                int r = in.read(b);
                if (r == -1) break;
                out.write(b, 0, r);
            }
        } finally {
            if (out != null) {
                out.close();
            }
        }
    } finally {
        if (in != null) {
            in.close();
        }
    }
\end{Java}

\subsubsection{@SneakyThrows}

这个注解是用来解决这样的代码的:

\begin{Java}
try{
    ...
}catch(Exception e){
    throw new RuntimeException(e);
}
\end{Java}

使用了该注解后可以自动向上抛错误:

\begin{Java}
@Target({ElementType.METHOD, ElementType.CONSTRUCTOR})
@Retention(RetentionPolicy.SOURCE)
public @interface SneakyThrows {
    Class<? extends Throwable>[] value() default java.lang.Throwable.class;
}
\end{Java}

等价效果如下:

\begin{Java}
// Lombok
@SneakyThrows
public void run() {
    throw new Throwable();
}
// Java
public void run() {
    try {
        throw new Throwable();
    } catch (Throwable t) {
        throw Lombok.sneakyThrow(t);
    }
}
\end{Java}

\subsection{增强性注解}

\subsubsection{@Getter @Setter}

这些注解用于自动生成对应的 getXXX setXXX 方法。

\begin{Java}
@Target({ElementType.FIELD, ElementType.TYPE})
@Retention(RetentionPolicy.SOURCE)
public @interface Getter {
    boolean lazy() default false;
    ...
}
\end{Java}

等效代码:

\begin{Java}
// Lombok
@Getter @Setter private int age = 10;
// Java
private int age = 10;
public int getAge() {
    return age;
}
public void setAge(int age) {
    this.age = age;
}
\end{Java}

默认的 setXXX 方法返回 void,链式编程参考接下来的注解。

getXXX 有一个重要的参数 lazy,默认关闭,如果启用,会缓存 getXXX 返回的对象,创建一个private final变量,以方便接下来使用。

\begin{Java}
// Lombok
@Getter(lazy=true) private final double[] cached = expensive();
// Java
private final java.util.concurrent.AtomicReference<java.lang.Object> cached = new java.util.concurrent.AtomicReference<java.lang.Object>();
public double[] getCached() {
    java.lang.Object value = this.cached.get();
    if (value == null) {
        synchronized(this.cached) {
            value = this.cached.get();
            if (value == null) {
                final double[] actualValue = expensive();
                value = actualValue == null ? this.cached : actualValue;
                this.cached.set(value);
            }
        }
    }
    return (double[])(value == this.cached ? null : value);
}
\end{Java}

\subsubsection{@Accessors}
@Accessors 不是 stable 的注解,是一个 experimental 注解,比较常用所以也写一下。

该注解主要作用是:当属性字段在生成 getter 和 setter 方法时,做一些相关的设置。

\begin{Java}
@Target({ElementType.TYPE, ElementType.FIELD})
@Retention(RetentionPolicy.SOURCE)
public @interface Accessors {
	boolean fluent() default false;
	boolean chain() default false;
	boolean makeFinal() default false;
	String[] prefix() default {};
}
\end{Java}

它的几个属性作用如下:
\begin{itemize}
    \item fluent: 如果改为 true: getter 方法不会有 get 前缀,setter 方法不会有 set 前缀。
    \item chain: 链式编程,如果改为 true,setter 方法将会返回当前对象。
    \item prefix: 当该数组有值时,表示忽略字段中对应的前缀,生成对应的 getter 和 setter 方法。
    \item makeFinal: 如果改为 true,生成的方法会被标记为 final。
\end{itemize}

\subsubsection{@ToString}

这个注解用于自动生成 toString 方法。

\begin{Java}
@Target(ElementType.TYPE)
@Retention(RetentionPolicy.SOURCE)
public @interface ToString {
    ...
}
\end{Java}

默认情形下,它生成的 toString 格式如下:

\begin{Java}
private String name;
private int age;
@Override
public String toString() {
    return "Student{" +
            "name='" + name + '\'' +
            ", age=" + age +
            '}';
}
\end{Java}

他还有很多参数,提供自定义输出格式,另外有两个相关注解 @ToString.Exclude @ToString.Include。 其他几个增强注解基本上都有类似的相关注解。

\subsubsection{@EqualsAndHashCode}

该注解用于生成 equals 和 hashCode 方法。该注解参数很多,实现也比较复杂。

\begin{Java}
@Target(ElementType.TYPE)
@Retention(RetentionPolicy.SOURCE)
public @interface EqualsAndHashCode {
    boolean callSuper() default false;
    ...
}
\end{Java}

有个 callSuper 和生成 hashCode 有关:
\begin{itemize}
    \item callSuper = true: 根据子类和从父类继承的字段生成 hashCode。
    \item callSuper = false: 根据子类本身字段生成 hashCode。
\end{itemize}

该注解生成的 equals 方法逻辑和 String 类型的 equals 逻辑类似。下面仅给出官方实例生成的两个方法。

\begin{Java}
@Override public boolean equals(Object o) {
    if (o == this) return true;
    if (!(o instanceof EqualsAndHashCodeExample)) return false;
    EqualsAndHashCodeExample other = (EqualsAndHashCodeExample) o;
    if (!other.canEqual((Object)this)) return false;
    if (this.getName() == null ? other.getName() != null : !this.getName().equals(other.getName())) return false;
    if (Double.compare(this.score, other.score) != 0) return false;
    if (!Arrays.deepEquals(this.tags, other.tags)) return false;
    return true;
}  
@Override public int hashCode() {
    final int PRIME = 59;
    int result = 1;
    final long temp1 = Double.doubleToLongBits(this.score);
    result = (result*PRIME) + (this.name == null ? 43 : this.name.hashCode());
    result = (result*PRIME) + (int)(temp1 ^ (temp1 >>> 32));
    result = (result*PRIME) + Arrays.deepHashCode(this.tags);
    return result;
}
\end{Java}

\subsubsection{Constructor}

Lombok 有三个用于生成构造函数的注解: @NoArgsConstructor, @RequiredArgsConstructor, @AllArgsConstructor。

首先看一下源代码:

\begin{Java}
@Target(ElementType.TYPE)
@Retention(RetentionPolicy.SOURCE)
public @interface NoArgsConstructor {
    String staticName() default "";
    AnyAnnotation[] onConstructor() default {};
	AccessLevel access() default lombok.AccessLevel.PUBLIC;
	boolean force() default false;
    ...
}
\end{Java}

它的几个常用属性如下:
\begin{itemize}
    \item access: 设置构造器的访问修饰符,如果是单例模式,可以设置为 AccessLevel.PRIVATE。
    \item staticName: 用于设置一个静态构造函数。
    \item force: 如果有未赋值的 final 字段,强制初始化默认值。
\end{itemize}

\begin{Java}
// Lombok
@NoArgsConstructor(staticName = "UserHa")
public class User {
    private String username;
    private String password;
}
// Java
public class User {
    private String username;
    private String password;
    private User() { }
    public static User UserHa() {
        return new User();
    }
}
\end{Java}

@RequiredArgsConstructor也是在类上使用,但是这个注解可以生成带参或者不带参的构造方法。若带参数,只能是类中所有带有 @NonNull注解的和以final修饰的未经初始化的字段。

\begin{Java}
// Lombok
@RequiredArgsConstructor
public class User {
    private final String gender;
    @NonNull
    private String username;
    private String password;
}
// Java
public class User {
    private final String gender;
    @NonNull
    private String username;
    private String password;

    public User(String gender, @NonNull String username) {
        if (username == null) {
            throw new NullPointerException("username is marked @NonNull but is null");
        } else {
            this.gender = gender;
            this.username = username;
        }
    }
}
\end{Java}

最后一个是 @AllArgsConstructor 需要注意的是,这里的全参不包括已初始化的final字段(主要是final字段,一旦被赋值不允许再被修改),没啥好说的。

\subsubsection{@Builder}

@Builder 注解是一个相当复杂的注解,被他注解的类可以实现 Builder 模式\footnote{Builder 模式不解释,可以看这篇文章: \url{https://blog.csdn.net/qq_17678217/article/details/86507693}}。总而言之,它可以完成 Builder 模式所需要的所有代码。

\begin{Java}
@Target({TYPE, METHOD, CONSTRUCTOR})
@Retention(SOURCE)
public @interface Builder {
    ...
}
\end{Java}

@Builder 的属性很多,建议查源代码了解。这个注解本身没什么好说的,对应的属性也都是和 Builder 模式相关。

\subsubsection{@Synchronized}

用于替换 synchronize 关键字或 lock 锁。

\begin{Java}
@Target({ElementType.METHOD, ElementType.CONSTRUCTOR})
@Retention(RetentionPolicy.SOURCE)
public @interface SneakyThrows {
	Class<? extends Throwable>[] value() default java.lang.Throwable.class;
}
\end{Java}

可以传入一个属性值用于设置锁对象。

\begin{Java}
// Lombok
public final String NAME = "唐嫣";

@Synchronized(value = "NAME")
public void name() {
	System.out.println(NAME);
}
// Java
public void name() {
	super.getClass();
	synchronized ("唐嫣") {
		System.out.println("唐嫣");
	}
}
\end{Java}

\subsubsection{@With}

该注解用于“改变”final属性,具体的方法是,通过 withXXX 函数,返回一个新的对象,这样 final 对应的属性就变相改变了。

\begin{Java}
@Target({ElementType.FIELD, ElementType.TYPE})
@Retention(RetentionPolicy.SOURCE)
public @interface With {
    AccessLevel value() default AccessLevel.PUBLIC;
    AnyAnnotation[] onMethod() default {};
    AnyAnnotation[] onParam() default {};
    ...
}
\end{Java}

示例代码:

\begin{Java}
// Lombok
public class WithExample {
    @With(AccessLevel.PROTECTED) @NonNull private final String name;
    @With private final int age;
    public WithExample(@NonNull String name, int age) {
        this.name = name;
        this.age = age;
    }
}
// Java
public class WithExample {
    private @NonNull final String name;
    private final int age;
    public WithExample(String name, int age) {
        if (name == null) throw new NullPointerException();
        this.name = name;
        this.age = age;
    }
    protected WithExample withName(@NonNull String name) {
        if (name == null) throw new java.lang.NullPointerException("name");
        return this.name == name ? this : new WithExample(name, age);
    }
    public WithExample withAge(int age) {
        return this.age == age ? this : new WithExample(name, age);
    }
}
\end{Java}

\subsubsection{@Log}

该注解用于创建对应的 Log 对象:

\begin{Java}
@Retention(RetentionPolicy.SOURCE)
@Target(ElementType.TYPE)
public @interface Log {
    String topic() default "";
}
\end{Java}

示例代码如下:

\begin{Java}
// Lombok
@Log
public class LogExample {
    public static void main(String... args) {
        log.severe("Something's wrong here");
    }
}
// Java
public class LogExample {
    private static final java.util.logging.Logger log = java.util.logging.Logger.getLogger    (LogExample.class.getName());
    
    public static void main(String... args) {
        log.severe("Something's wrong here");
    }
}
\end{Java}

\subsection{集成注解}

\subsubsection{@Data}

@Data 注解是 @ToString, @EqualsAndHashCode, @Getter, @Setter, @RequiredArgsConstructor 注解的集合。

\subsubsection{@Value}

@Value 是针对不可变对象(例如 String)的 @Data。它会将被修饰的类转换为如下形式:

\begin{itemize}
    \item 成员变量由 private final 修饰; 
    \item 提供带参数构造函数;
    \item 仅为成员变量提供带参数的构造器;
    \item 不允许子类覆盖方法;
\end{itemize}

详细文献: \url{https://blog.csdn.net/cauchy6317/article/details/102646009}

\newpage