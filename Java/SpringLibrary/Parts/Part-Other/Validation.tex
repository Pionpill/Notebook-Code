\section{Validation 注解}

导入依赖如下:
\begin{xml}
<dependency>
    <groupId>org.springframework.boot</groupId>
    <artifactId>spring-boot-starter-validation</artifactId>
</dependency>
\end{xml}

可以发现 validation 以及被集成到 spring 中了,为什么要这样做呢?简言之,validation 的一部分注解会在 mvc 的参数解析器中进行解析。

\subsection{约束性注解}

\subsubsection{@NotNull,@NotEmpty,@NotBlank}

这三个注解均用来哦按段是否为空,分别为:
\begin{itemize}
    \item @NotNull: 属性不可以为 null,但可以为空串。
    \item @NotEmpty: 属性不可以为 null, 且不可以为空串(长度大于0)。
    \item @NotBlank: 只能作用于 String 类型的属性上,属性不可以为 null,且 trim() 后不可以为空串。
\end{itemize}

其中 @NotNull 源代码如下:
\begin{Java}
@Target({ METHOD, FIELD, ANNOTATION_TYPE, CONSTRUCTOR, PARAMETER, TYPE_USE })
@Retention(RUNTIME)
@Repeatable(List.class)
@Documented
@Constraint(validatedBy = { })
public @interface NotNull {
    String message() default "{javax.validation.constraints.NotNull.message}";
    Class<?>[] groups() default { };
    Class<? extends Payload>[] payload() default { };
    @Target({ METHOD, FIELD, ANNOTATION_TYPE, CONSTRUCTOR, PARAMETER, TYPE_USE })
    @Retention(RUNTIME)
    @Documented
    @interface List {
        NotNull[] value();
    }
}
\end{Java}

@NotNull 对应的还有一个 @Null 注解。

\subsubsection{@Size,@Length,@Max,@Min}

这三个注解均用于判断数值大小/字符串长度:
\begin{itemize}
    \item @Min: 验证 Number 和 String 对象是否大于等于指定的值。
    \item @Max: 验证 Number 和 String 对象是否小于等于指定的值。
    \item @Size(min=,max=): 验证对象(Array,Collection,Map,String)长度是否在给定范围内。
    \item @Length(min=,max=): 验证字符串长度是否在给定范围内。
\end{itemize}

其中 @Size 源代码如下:
\begin{Java}
@Target({ METHOD, FIELD, ANNOTATION_TYPE, CONSTRUCTOR, PARAMETER, TYPE_USE })
@Retention(RUNTIME)
@Repeatable(List.class)
@Documented
@Constraint(validatedBy = { })
public @interface Size {
    String message() default "{javax.validation.constraints.Size.message}";
    Class<?>[] groups() default { };
    Class<? extends Payload>[] payload() default { };
    int min() default 0;
    int max() default Integer.MAX_VALUE;
    @Target({ METHOD, FIELD, ANNOTATION_TYPE, CONSTRUCTOR, PARAMETER, TYPE_USE })
    @Retention(RUNTIME)
    @Documented
    @interface List {
        Size[] value();
    }
}
\end{Java}

\subsubsection{@Digits,@DecimalMax,@DecimalMin}

被注解的元素必须为一个数字,其值必须在可接受的范围内。主要参数有两个:
\begin{itemize}
    \item integer: 整数部分位数。
    \item fraction: 小数部分位数。
\end{itemize}

\begin{Java}
@Digits(integer = 3,fraction = 0)
private String ccCVV;
\end{Java}

\begin{Java}
@Target({ METHOD, FIELD, ANNOTATION_TYPE, CONSTRUCTOR, PARAMETER, TYPE_USE })
@Retention(RUNTIME)
@Repeatable(List.class)
@Documented
@Constraint(validatedBy = { })
public @interface Digits {
    ...
}
\end{Java}

@DecimalMax 与 @DecimalMin 分别表示数字的最大值与最小值。

\subsubsection{@AssertFalse,@AssertTrue}

被注解的对象必须是 True 或 False,否则抛错。

\begin{Java}
@Target({ METHOD, FIELD, ANNOTATION_TYPE, CONSTRUCTOR, PARAMETER, TYPE_USE })
@Retention(RUNTIME)
@Repeatable(List.class)
@Documented
@Constraint(validatedBy = { })
public @interface AssertTrue {
    ...
}
\end{Java}

\subsubsection{@Future,@Past}

用于限定日期必须在当前日期的未来或者过期,注意源码中并没有 value,也就是说这里的日期是指程序运行时的日期。

\begin{Java}
@Target({ METHOD, FIELD, ANNOTATION_TYPE, CONSTRUCTOR, PARAMETER, TYPE_USE })
@Retention(RUNTIME)
@Repeatable(List.class)
@Documented
@Constraint(validatedBy = { })
public @interface Future {
    ...
}
\end{Java}

\subsubsection{@Pattern}

限定字符串必须满足正则表达式:

\begin{Java}
@Target({ METHOD, FIELD, ANNOTATION_TYPE, CONSTRUCTOR, PARAMETER, TYPE_USE })
@Retention(RUNTIME)
@Repeatable(List.class)
@Documented
@Constraint(validatedBy = { })
public @interface Pattern {
    ...
}
\end{Java}

\subsubsection{@Email,@URL}

用于限定被修饰的属性必须是邮箱或者URL。

\subsection{验证性注解}

\subsubsection{@Valid,@Validated}

@Valid注解用于校验。\footnote{参考: \url{https://blog.csdn.net/gaojp008/article/details/80583301}}

\begin{Java}
@Target({ METHOD, FIELD, CONSTRUCTOR, PARAMETER, TYPE_USE })
@Retention(RUNTIME)
@Documented
public @interface Valid {
}
\end{Java}

使用校验,首先要在实体类相应字段上添加用于校验的注解。如: @Min。其次在 Controller 层的方法的要校验的参数上添加 @Valid 注解,并且需要传入 BindingResult 对象,用于获取校验失败情况下的反馈信息,代码如下:

\begin{Java}
@PostMapping("/girls")  
public Girl addGirl(@Valid Girl girl, BindingResult bindingResult) {  
    if(bindingResult.hasErrors()){  
        System.out.println(bindingResult.getFieldError().getDefaultMessage());  
        return null;  
    }  
    return girlResposity.save(girl);  
}
\end{Java}

@Validated 是 @Valid 的一次封装,是Spring提供的校验机制使用。@Valid不提供分组功能。当一个实体类需要多种验证方式时,例:对于一个实体类的id来说,新增的时候是不需要的,对于更新时是必须的。可以通过groups对验证进行分组。

不分配 groups 时,默认每次都进行验证,对参数需要多种验证方式时,也可通过分配不同的组达到目的:

\begin{Java}
@NotEmpty(groups={First.class})  
@Size(min=3,max=8,groups={Second.class})  
private String name;  
\end{Java}

注意 Validated 是在 springframework 包下,不是 javax 包下:

\begin{Java}
@Target({ElementType.TYPE, ElementType.METHOD, ElementType.PARAMETER})
@Retention(RetentionPolicy.RUNTIME)
@Documented
public @interface Validated {
    Class<?>[] value() default {};
}
\end{Java}

\newpage