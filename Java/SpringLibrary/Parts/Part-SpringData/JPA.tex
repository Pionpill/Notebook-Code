\section{Spring Data JPA 注解}

导入依赖如下:

\begin{Java}
<dependency>
    <groupId>org.springframework.boot</groupId>
    <artifactId>spring-boot-starter-data-jpa</artifactId>
</dependency>
\end{Java}

\subsection{表相关注解}

\subsubsection{@Entity}

被该注解标注的实体类将会被 JPA 管理控制,在程序运行时,JPA会识别并映射到指定的数据库表。

\begin{Java}
@Documented
@Target(TYPE)
@Retention(RUNTIME)
public @interface Entity {
	String name() default "";
}
\end{Java}

属性 name 用于指定实体类名称,默认为实体类的非限定名。

\subsubsection{@Table}

若表名与实体类名称不同时,使用 @Table(name="表名"),与@Entity标注并列使用,置于实体类声明语句之前。如果表名和实体类名相同,那么@Table可以省略。

注意是 javax.persistence.Table 下的 @Table:

\begin{Java}
@Target(TYPE) 
@Retention(RUNTIME)
public @interface Table {
    String name() default "";
    String catalog() default "";
    String schema() default "";
    UniqueConstraint[] uniqueConstraints() default {};
    Index[] indexes() default {};
}
\end{Java}

@Table 的属性如下:
\begin{itemize}
    \item name: 对应表名。
    \item catalog: 表所属的数据库目录。通常为数据库名。
    \item schema: 表所属的数据库模式。通常为数据库名。
    \item uniqueConstraints: 设置约束条件。
\end{itemize}

\subsubsection{@Id}

@Id 用于实体类的一个属性或者属性对应的getter方法上,被标注的的属性将映射为数据库主键。

\begin{Java}
@Target({METHOD, FIELD})
@Retention(RUNTIME)
public @interface Id {}
\end{Java}

\subsubsection{@GeneratedValue}

与@Id一同使用,用于标注主键的生成策略,通过 strategy 属性指定。

\begin{Java}
@Target({METHOD, FIELD})
@Retention(RUNTIME)
public @interface GeneratedValue {
    GenerationType strategy() default AUTO;
    String generator() default "";
}
\end{Java}

在 javax.persistence.GenerationType 中定义了以下几种可供选择的策略:
\begin{itemize}
    \item AUTO: 默认方式,JPA 自动选择合适的策略。
    \item IDENTITY: 由数据库生成,采用数据库自增长,Oracle 不支持。
    \item SEQUENCE: 通过数据库序列生成,MySQL 不支持。
    \item TABLE:通过表产生主键,框架借由表模拟序列产生主键。
\end{itemize}

generator属性的值是一个字符串,默认为"",其声明了主键生成器的名称。使用非 AUTO 的策略需要结合其他注解使用,不是很常用,不讲了。

\subsubsection{@Basic}

@Basic 是实体类与数据库字段映射时最简单的类型。它可以用于持久类属性或实例变量,类型包含基本类型,包装类,枚举类,实现了 Serializable 接口的类型。

\begin{Java}
@Target({METHOD, FIELD}) 
@Retention(RUNTIME)
public @interface Basic {
    FetchType fetch() default EAGER;
    boolean optional() default true;
}
\end{Java}

两个属性意义如下:
\begin{itemize}
    \item fetch: 加载机制:默认 EAGER,即时加载,可以改为 LAZY 懒加载。
    \item optional: 判断属性是否能为空,默认可以。
\end{itemize}


简言之,与数据库对应的属性都要加上 @Basic,如果你在实体类属性上不加@Basic注解,它也会自动加上@Basic,并使用默认值。一般不需要显示书写,除非要改属性。

\subsubsection{@Transient}

表示该属性并非一个数据库表的字段的映射,ORM框架将忽略该属性,不会对其持久化。

\subsubsection{@Cloumn}

用于定义实体属性:

\begin{Java}
@Target({METHOD, FIELD}) 
@Retention(RUNTIME)
public @interface Column {
    String name() default "";
    boolean unique() default false;
    boolean nullable() default true;
    boolean insertable() default true;
    boolean updatable() default true;
    String columnDefinition() default "";
    String table() default "";
    int length() default 255;
    int precision() default 0;
    int scale() default 0;
}
\end{Java}

\subsubsection{@Embedded @Embeddable}

通过此注解可以在Entity模型中使用一般的java对象,不过此对象还需要用 @Embeddable 注解标注。

\begin{Java}
@Target({METHOD, FIELD})
@Retention(RUNTIME)
public @interface Embedded { }
\end{Java}

例如:User包括id,name,city,street,zip属性,我们希望city,street,zip属性映射为Address对象,这样,User对象将具有id,name和address这三个属性,Address对象要定义为@Embededable。

参考例子: \url{https://blog.csdn.net/je_ge/article/details/53678238}

\subsection{记录相关}

\subsubsection{@CreatedBy}

自动插入创建人。

\subsubsection{@CreatedDate}

自动插入创建时间

\subsubsection{@LastModifiedBy}

自动修改更新人

\subsubsection{@LastModifiedDate}

自动修改更新时间

\subsubsection{@Version}

自动更新版本号

\newpage