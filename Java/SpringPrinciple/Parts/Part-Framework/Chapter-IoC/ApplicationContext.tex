\section{ApplicationContext}

ApplicationContext除了拥有BeanFactory支持的所有功能之外,还进一步扩展了基本容器的功能,包括BeanFactoryPostProcessor、BeanPostProcessor以及其他特殊类型bean的自动识别、容器启动后bean实例的自动初始化、国际化的信息支持、容器内事件发布等。

\subsection{统一资源加载策略}

\subsubsection{Spring 中的 Resource}

Spring框架内部使用 Resource 接口作为所有资源的抽象和访问接口。

Resource接口可以根据资源的不同类型,或者资源所处的不同场合,给出相应的具体实现。Spring框架在这个理念的基础上,提供了一些实现类:

\begin{itemize}
    \item \textbf{ByteArrayResource}: 将字节(byte)数组提供的数据作为一种资源进行封装,如果通过InputStream形式访问该类型的资源,该实现会根据字节数组的数据,构造相应的ByteArrayInputStream并返回。
    \item \textbf{ClassPathResource}: 该实现从Java应用程序的ClassPath中加载具体资源并进行封装,可以使用指定的类加载器(ClassLoader)或者给定的类进行资源加载。
    \item \textbf{FileSystemResource}: 对java.io.File类型的封装,所以,我们可以以文件或者URL的形式对该类型资源进行访问,只要能跟File打的交道,基本上跟FileSystemResource也可以。
    \item \textbf{UrlResource}: 通过java.net.URL进行的具体资源查找定位的实现类,内部委派URL进行具体的资源操作。
    \item \textbf{InputStreamResource}: 将给定的InputStream视为一种资源的Resource实现类,较为少用。
\end{itemize}

如果以上资源实现还不能满足要求,那么我们可以根据相应场景给出自己的实现,只需要实现 \texttt{Resource} 接口,通常继承 \texttt{AbstractResource} 即可。

\subsubsection{\texttt{ResourceLoader}}

资源是有了,但如何去查找和定位这些资源,则应该是 \texttt{ResourceLoader} 的职责所在。\texttt{ResourceLoader} 接口是资源查找定位策略的统一抽象。

\begin{Java}
public interface ResourceLoader {   
    String CLASSPATH_URL_PREFIX = ResourceUtils.CLASSPATH_URL_PREFIX;   
    Resource getResource(String location);   
    ClassLoader  getClassLoader();  
} 
\end{Java}

其中最主要的就是 \texttt{getResource} 方法,通过它,我们就可以根据指定的资源位置,定位到具体的资源实例。

Spring 提供了几个 \texttt{getResource} 的实现类,关系如下:

\begin{figure}[H]
    \scriptsize
    \centering
    \begin{tikzpicture}[scale = 1]
        \begin{interface}[text width=2cm]{Resource}{0,0}    
        \end{interface}
        \begin{interface}[text width=2cm]{ResourceLoader}{5,0}    
        \end{interface}
        \begin{class}[text width=3cm]{DefaultResoourceLoader}{2,-2}    
            \implement{ResourceLoader}
        \end{class}
        \begin{interface}[text width=3cm]{ResourcePatternResolver}{8,-2}    
            \implement{ResourceLoader}
        \end{interface}
        \begin{class}[text width=3cm]{FileSystemResourceLoader}{2,-4}    
            \inherit{DefaultResoourceLoader}
        \end{class}
        \begin{interface}[text width=5cm]{PathMatchingResourcePatternResolver}{8,-4}    
            \inherit{ResourcePatternResolver}
        \end{interface}
        \unidirectionalAssociation{ResourceLoader}{}{}{Resource}
    \end{tikzpicture}
    \caption{ResourceLoader 类层次图}
    \label{fig:ResourceLoader 类层次图}
\end{figure}

\noindent\textbf{DefaultResourceLoader}

\texttt{DefaultResourceLoader} 顾名思义,是 \texttt{ResourceLoader} 是默认实现类,该类的默认查找处理逻辑如下:

\begin{itemize}
    \item 检查资源路径是否以 classpath: 前缀打头,如果是,则尝试构造\texttt{ClassPathResource} 类型资源并返回。
    \item 否则,(a) 尝试通过URL,根据资源路径来定位资源,如果没有抛出MalformedURLException,有则会构造UrlResource类型的资源并返回;(b)如果还是无法根据资源路径定位指定的资源,则委派getResourceByPath(String)方法来定位,DefaultResourceLoader的getResourceByPath(String)方法默认实现逻辑是,构造ClassPathResource类型的资源并返回。
\end{itemize}

\texttt{FileSystemResourceLoader} 覆写了 \texttt{getResourceByPath} 方法,使之从文件系统加载资源并以 \texttt{FileSystemResource} 类型返回。这样,我们就可以取得预想的资源类型。

\noindent\textbf{ResourcePatternResolver}

\texttt{ResourceLoader} 每次只能根据资源路径返回确定的单个 \texttt{Resource} 实例,而 \texttt{ResourcePa tternResolver} 则可以根据指定的资源路径匹配模式,每次返回多个Resource实例。

\texttt{ResourceLoader} 最常用的实现是 \texttt{PathMatchingResourcePatternResolver}。

\subsubsection{\texttt{ApplicationContext} 与 \texttt{ResourceLoader}}

\texttt{ApplicationContext} 继承了 \texttt{ResourcePatternResolver} 下面是其继承逻辑,有了前面的铺垫,结合下图可以很清楚地对 \texttt{ApplicationContext} 有一个更完善的认知。

\begin{figure}[H]
    \scriptsize
    \centering
    \begin{tikzpicture}[scale = 1]
        \begin{interface}[text width=2cm]{ResourceLoader}{0,0}
        \end{interface}
        \begin{interface}[text width=3cm]{ResourcePatternResolver}{5,-2}
            \inherit{ResourceLoader}
        \end{interface}
        \begin{interface}[text width=3cm]{ApplicationContext}{5,-4}
            \inherit{ResourcePatternResolver}
        \end{interface}
        \begin{interface}[text width=3cm]{DefaultResourceLoader}{0,-4}
            \implement{ResourceLoader}
        \end{interface}
        \begin{interface}[text width=4cm]{ConfigurableApplicationContext}{5,-6}
            \inherit{ApplicationContext}
        \end{interface}
        \begin{class}[text width=6cm]{AbstractApplicationContext}{0,-8}
            \inherit{DefaultResourceLoader}
            \implement{ConfigurableApplicationContext}
            \operation{getResourcePatternResolver():ResourcePatternResolver}
        \end{class}
        \begin{class}[text width=4cm]{ConfigurableApplicationContext}{8,-8}
            \implement{ResourcePatternResolver}
        \end{class}
        \aggregation{AbstractApplicationContext}{}{}{ConfigurableApplicationContext}
    \end{tikzpicture}
    \caption{ApplicationContext 继承关系}
    \label{fig:ApplicationContext 继承关系}
\end{figure}

\subsection{容器内部事件发布}

Spring的 \texttt{ApplicationContext} 容器提供的容器内事件发布功能,是通过提供一套基于Java  SE标准自定义事件类而实现的。为了更好地了解这组自定义事件类,我们可以先从Java SE的标准自定义事件类实现的推荐流程说起。

\subsubsection{自定义事件发布}

Java SE提供了实现自定义事件发布(Custom Event publication)功能的基础类,即 java.util. EventObject 类和 java.util.EventListener 接口。所有的自定义事件类型可以通过扩展 \texttt{EventObje ct} 来实现,而事件的监听器则扩展自 \texttt{EventListener}。

我们可以通过扩展 \texttt{EventObject} 类来实现自定义的事件类型:

\begin{Java}
public class MethodExecutionEvent extends EventObject {   
    private static final long serialVersionUID = -71960369269303337L;   
    private String methodName;   
    public MethodExecutionEvent(Object source) {     
        super(source);    
    }    
    public MethodExecutionEvent(Object source,String methodName)   {      
        super(source);      
        this.methodName = methodName;   
    }    
    public String getMethodName() {     
        return  methodName;    
    }    
    public void setMethodName(String methodName) {     
        this.methodName = methodName;   
    }  
}
\end{Java}

当该类型的事件发布之后,相应的监听器即可对该类型的事件进行处理。如果需要,自定义事件类可以根据情况提供更多信息,不用担心自定义事件类的“承受力”。

\textbf{实现针对自定义事件类的事件监听器接口}。自定义的事件监听器需要在合适的时机监听自定义的事件,我们可以在方法开始执行的时候发布该事件,也可以在方法执行即将结束之际发布该事件。

\begin{Java}
public interface MethodExecutionEventListener extends EventListener {   
    // 处理方法开始执行的时候发布的MethodExecutionEvent事件   
    void onMethodBegin(MethodExecutionEvent evt);   
    // 处理方法执行将结束时候发布的MethodExecutionEvent事件  
    void onMethodEnd(MethodExecutionEvent evt);
}
\end{Java}

\textbf{组合事件类和监听器,发布事件}。有了自定义事件和自定义事件监听器,剩下的就是发布事件,然后让相应的监听器监听并处理事件了。通常情况下,我们会有一个事件发布者(EventPublisher),它本身作为事件源,会在合适的时间点,将相应事件发布给对应的事件监听器。

\begin{Java}
public class MethodExeuctionEventPublisher { 
    private List<MethodExecutionEventListener> listeners = new ArrayList<MethodExecutionEventListener>();  
    
    public void methodToMonitor()   {      
        MethodExecutionEvent event2Publish = new  MethodExecutionEvent(this,"methodToMonitor");      
        publishEvent(MethodExecutionStatus.BEGIN,event2Publish);  
        //  执行实际的方法逻辑    
        //  ...      
        publishEvent(MethodExecutionStatus.END,event2Publish);    
    }  

    protected void publishEvent(MethodExecutionStatus status, MethodExecutionEvent methodExecutionEvent) {     
        List<MethodExecutionEventListener> copyListeners = new ArrayList<MethodExecutionEventListener>(listeners);      
        for(MethodExecutionEventListener  listener:copyListeners) {        
            if(MethodExecutionStatus.BEGIN.equals(status))          
                listener.onMethodBegin(methodExecutionEvent);        
            else          
                listener.onMethodEnd(methodExecutionEvent);      
        }
    }

    ......

    public static void main(String[] args) {     
        MethodExeuctionEventPublisher eventPublisher = new  MethodExeuctionEventPublisher();     
        eventPublisher.addMethodExecutionEventListener(new SimpleMethodExecutionEventListener()); 
        eventPublisher.methodToMonitor();    
    }
}
\end{Java}

\subsubsection{Spring 的容器内部事件发布类结构分析}

Spring 的 \texttt{ApplicationContext} 容器内部允许以 \texttt{ApplicationEvent} 的形式发布事件,容器内注册的 \texttt{ApplicationListener} 类型的 bean 定义会被 \texttt{ApplicationContext} 容器自动识别,它们负责监听容器内发布的所有 \texttt{ApplicationEvent} 类型的事件。也就是说,一旦容器内发布 \texttt{ApplicationEvent} 及其子类型的事件,注册到容器的 \texttt{ApplicationListener} 就会对这些事件进行处理。

\begin{itemize}
    \item \texttt{ApplicationEvent}: Spring容器内自定义事件类型,继承自 \texttt{EventObject} ,它是一个抽象类,需要根据情况提供相应子类以区分不同情况。默认情况下,Spring提供了三个实现。
    \begin{itemize}
        \item \texttt{ContextClosedEvent}: 容器在即将关闭的时候发布的事件类型。
        \item \texttt{ContextRefreshedEvent}: 容器在初始化或者刷新的时候发布的事件类型。
        \item \texttt{RequestHandledEvent}: Web请求处理后发布的事件。
    \end{itemize}
    \item \texttt{ApplicationListener}: 继承自 \texttt{Eve ntListener}。容器启动时,会自动识别并加载 \texttt{EventListener} 类型 bean 定义,一旦容器内有事件发布,将通知这些注册到容器的 \texttt{EventListener}。
    \item \texttt{ApplicationContext}: 它继承了 \texttt{ApplicationEventPublisher} 接口,担当的就是事件发布者的角色。
\end{itemize}

\texttt{ApplicationContext} 内部实现比较复杂,这里单独讲一下。\texttt{ApplicationContext} 容器的具体实现类在实现事件的发布和事件监听器的注册方面,并没有亲自处理,而是把这些活儿转包给了一个称作 \texttt{ApplicationEventMulticaster} 的接口。该接口定义了具体事件监听器的注册管理以及事件发布的方法。他有一个抽象子类: \texttt{AbstractApplicationEventMulyicaster}, Spring 提供了一个子类实现: \texttt{SimpleApplicationEventMulticaster}。

因为 \texttt{ApplicationContext} 容器的事件发布功能全部委托给了 \texttt{ApplicationEventMulti caster} 来做,所以,容器启动伊始,就会检查容器内是否存在名称为 \texttt{applicationEventMultic aster} 的 \texttt{ApplicationEventMulticaster} 对象实例。有的话就使用提供的实现,没有则默认初始化一个 \texttt{SimpleApplicationEventMulticaster} 作为将会使用的 \texttt{ApplicationEventMultic aster} 。

\begin{figure}[H]
    \scriptsize
    \centering
    \begin{tikzpicture}[scale = 1]
        \begin{interface}[text width=6cm]{ApplicationListener}{0,0}
            \operation{+ void onApplicationEvent (event:ApplicationEvent)}
        \end{interface}
        \begin{interface}[text width=7.5cm]{ApplicationEventMulticaster}{0,-2.5}
            \operation{+ void multicastEvent (event:ApplicationEvent)}
            \operation{+ void addApplicationListener (listener: ApplicationListener)}
            \operation{+ void removeApplicationListener (listener: ApplicationListener)}
            \operation{+ void removeAllListener ()}
        \end{interface}
        \begin{class}[text width=5cm]{AbstracrApplicationEventMulticaster}{0,-5.5}
            \implement{ApplicationEventMulticaster}
        \end{class}
        \begin{class}[text width=5cm]{SimpleApplicationEventMulticaster}{0,-7}
            \inherit{AbstracrApplicationEventMulticaster}
        \end{class}
        \begin{class}[text width=3cm]{ApplicationEvent}{7,-0.5}
        \end{class}
        \begin{class}[text width=7cm]{AbstractApplicationContext}{9,-3}
            \attribute{- applicationEventMulticaster:ApplicationEventMulticaster}
        \end{class}
        \composition{AbstractApplicationContext}{}{}{ApplicationEventMulticaster}
        \unidirectionalAssociation{ApplicationEventMulticaster}{}{notify}{ApplicationListener}
        \unidirectionalAssociation{ApplicationEventMulticaster}{}{publish}{ApplicationEvent}
        \unidirectionalAssociation{ApplicationListener}{}{recive}{ApplicationEvent}
    \end{tikzpicture}
    \caption{Spring 容器内事件发布实现类图}
    \label{fig:Spring 容器内事件发布实现类图}
\end{figure}

\subsubsection{Spring 容器内事件发布的应用}

Spring的 \texttt{ApplicationContext} 容器内的事件发布机制,主要用于单一容器内的简单消息通知和处理,并不适合分布式、多进程、多容器之间的事件通知。我们应该在合适的地点、合适的需求分析的前提下,合理地使用Spring提供的 \texttt{ApplicationContext} 容器内的事件发布机制。

要让我们的业务类支持容器内的事件发布,需要它拥有 \texttt{ApplicationEventPublisher} 的事件发布支持。所以,需要为其注入 \texttt{ApplicationEventPublisher} 实例。可以通过如下两种方式为我们的业务对象注入 \texttt{ApplicationEventPublisher} 的依赖。

\begin{itemize}
    \item \texttt{ApplicationEventPublisherAware}: 在 \texttt{ApplicationContext} 类型的容器启动时,会自动识别该类型的bean定义并将 \texttt{ApplicationContext} 容器本身作为 \texttt{ApplicationEve ntPublisher} 注入当前对象,而 \texttt{ApplicationContext} 容器本身就是一个 \texttt{ApplicationE ventPublisher}。
    \item \texttt{ApplicationContextAware}: 既然 \texttt{ApplicationContext} 本身就是一个 \texttt{Application EventPublisher},那么通过 \texttt{ApplicationContextAware} 几乎达到第一种方式相同的效果。
\end{itemize}