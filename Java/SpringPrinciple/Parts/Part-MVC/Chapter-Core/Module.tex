\section{Module}

\subsection{Module 简介}

这里主要介绍在 DispatchServlet 中初始化的九大组件,这些组件内部还可能使用到一些子组件。

\subsubsection*{HandlerMapping}

HandlerMapping 的作用是根据 request 找到对应的处理器 Handler 和 Interceptors,HandlerMapping 接口里只有一个公有方法:

\begin{Java}
@Nullable
HandlerExecutionChain getHandler(HttpServletRequest request) throws Exception;
\end{Java}

在 HandlerExecutionChain 里只包含两个主要元素:

\begin{Java}
private final Object handler;
private final List<HandlerInterceptor> interceptorList = new ArrayList<>();
\end{Java}

在纯 Spring MVC 中,多个 HandlerMapping 的搜索顺序需要手动配置。

\subsubsection*{HandlerAdapter}

HandlerAdapter 是用来处理 Handler 的,他有三个主要方法:
\begin{itemize}
    \item supports: 判断是否可以使用某个 Handler;
    \item handle: 使用 handler 干活;
    \item getLastModified: 获取资源的 Last-Modified。(被废弃!)
\end{itemize}

\begin{Java}
public interface HandlerAdapter {
    boolean supports(Object handler);
    @Nullable
    ModelAndView handle(HttpServletRequest request, HttpServletResponse response, Object handler) throws Exception;
    @Deprecated
    long getLastModified(HttpServletRequest request, Object handler);
}
\end{Java}

之所以要使用HandlerAdapter是因为Spring MVC中并没有对处理器做任何限制,处理器可以以任意合理的方式来表现,可以是一个类,也可以是一个方法,还可以是别的合理的方式,从handle方法可以看出它是Object的类型。这种模式就给开发者提供了极大的自由。

使用哪个 HandlerAdapter 的过程在 DispatchServlet 的 getHandlerAdapter 方法中,通过遍历素有 Adapter 检查哪个可以处理当前的 Handler:

\begin{Java}
protected HandlerAdapter getHandlerAdapter(Object handler) throws ServletException {
    if (this.handlerAdapters != null) {
        for (HandlerAdapter adapter : this.handlerAdapters) {
            if (adapter.supports(handler)) {
                return adapter;
            }
        }
    }
    throw new ServletException(XXX);
}
\end{Java}

\subsubsection*{HandlerExceptionResolver}

HandlerExceptionResolver 根据异常设置ModelAndView,之后再交给render方法进行渲染。HandlerExceptionResolver 是在 Render 之前工作的,只是用于解析对请求做处理的过程中产生的异常。

\begin{Java}
public interface HandlerExceptionResolver {
    @Nullable
    ModelAndView resolveException(HttpServletRequest request, HttpServletResponse response, @Nullable Object handler, Exception ex);
}
\end{Java}

HandlerExceptionResolver结构非常简单,只有一个方法,只需要从异常解析出Model-AndView就可以了。具体实现可以维护一个异常为key、View为value的Map,解析时直接从Map里获取View,如果在Map里没有相应的异常可以返回默认的View。

\subsubsection*{ViewResolver}

ViewResolver用来将String类型的视图名和Locale解析为View类型的视图,ViewResolve接口也非常简单,只有一个方法,定义如下:

\begin{Java}
public interface ViewResolver {
    @Nullable
    View resolveViewName(String viewName, Locale locale) throws Exception;
}
\end{Java}

View是用来渲染页面的,通俗点说就是要将程序返回的参数填入模板里,生成html(也可能是其他类型)文件。这里有两个关键的问题:使用哪个模板?用什么技术填入参数?

ViewResolver需要找到渲染所用的模板和所用的技术(也就是视图的类型)进行渲染,具体的渲染过程则交给不同的视图自己完成。我们最常使用的UrlBasedViewResolver系列的解析器都是针对单一视图类型进行解析的,只需要找到使用的模板就可以了,比如,InternalResourceViewResolver只针对jsp类型的视图。

\subsubsection*{RequestToViewNameTranslator}

ViewResolver是根据ViewName查找View,但有的Handler处理完后并没有设置View也没有设置viewName,这时就需要从request获取viewName了,而如何从request获取view-Name就是RequestToViewNameTranslator要做的事情。RequestToViewNameTranslator接口定义如下:

\begin{Java}
public interface RequestToViewNameTranslator {
	@Nullable
	String getViewName(HttpServletRequest request) throws Exception;
}
\end{Java}

RequestToViewNameTranslator在Spring MVC容器里只可以配置一个,所以所有request到ViewName的转换规则都要在一个Translator里面全部实现。

\subsubsection*{LocalResolver}

解析视图需要两个参数:一个是视图名,另一个是Locale。视图名是处理器返回的(或者使用RequestToViewNameTranslator解析的默认视图名),Locale是从哪里来的呢?这就是LocaleResolver要做的事情。

LocaleResolver 用于从 request 解析出 Locale。Locale就是zh-cn之类,表示一个区域。有了这个就可以对不同区域的用户显示不同的结果,这就是i18n(国际化)的基本原理,LocaleResolver是i18n的基础。LocaleResolver接口定义如下:

\begin{Java}
public interface LocaleResolver {
    Locale resolveLocale(HttpServletRequest request);
    void setLocale(HttpServletRequest request, @Nullable HttpServletResponse response, @Nullable Locale locale);
}
\end{Java}

接口定义非常简单,只有2个方法,分别表示:从request解析出Locale和将特定的Locale设置给某个request。容器会将localeResolver设置到request的attribute中,代码如下:

\begin{Java}
request.setAttribute(LOCALE_RESOLVER_ATTRIBUTE, this.localeResolver);
\end{Java}

Spring MVC中主要在两个地方用到了Locale:
\begin{itemize}
    \item ViewResolver 解析视图的时候。
    \item 使用到国际化资源或者主题的时候。
\end{itemize}

\subsubsection*{ThemeResolver}

ThemeResolver从名字就可以看出是解析主题用的。ThemeResolver接口定义如下:

\begin{Java}
public interface ThemeResolver {
    String resolveThemeName(HttpServletRequest request);
    void setThemeName(HttpServletRequest request, @Nullable HttpServletResponse response, @Nullable String themeName);
}
\end{Java}

不同的主题其实就是换了一套图片、显示效果以及样式等。Spring MVC中一套主题对应一个properties文件,里面存放着跟当前主题相关的所有资源,如图片、css样式表等。

Spring MVC中跟主题有关的类主要有ThemeResolver、ThemeSource和Theme。ThemeResolver的作用是从request解析出主题名; ThemeSource则是根据主题名找到具体的主题; Theme是ThemeSource找出的一个具体的主题,可以通过它获取主题里具体的资源。

\subsubsection*{MultipartResolver}

MultipartResolver用于处理上传请求,处理方法是将普通的request包装成MultipartHttpServletRequest,后者可以直接调用getFile方法获取到File,如果上传多个文件,还可以调用getFileMap得到FileName→File结构的Map,这样就使得上传请求的处理变得非常简单。

当然,这里做的其实是锦上添花的事情,如果上传的请求不用MultipartResolver封装成MultipartHttpServletRequest,直接用原来的request也是可以的,所以在Spring MVC中此组件没有提供默认值。MultipartResolver定义如下:

\begin{Java}
public interface MultipartResolver {
    boolean isMultipart(HttpServletRequest request);
    MultipartHttpServletRequest resolveMultipart(HttpServletRequest request) throws MultipartException;
    void cleanupMultipart(MultipartHttpServletRequest request);
}
\end{Java}

这里一共有三个方法,作用分别是判断是不是上传请求、将request包装成MultipartHttpServletRequest、处理完后清理上传过程中产生的临时资源。对上传请求可以简单地判断是不是multipart / form-data类型。

\subsubsection*{FlashMapManager}

FlashMap 主要用在 redirect 中传递参数。而FlashMapManager是用来管理FlashMap的。

\begin{Java}
public interface FlashMapManager {
    @Nullable
    FlashMap retrieveAndUpdate(HttpServletRequest request, HttpServletResponse response);
    void saveOutputFlashMap(FlashMap flashMap, HttpServletRequest request, HttpServletResponse response);
}
\end{Java}

retrieveAndUpdate方法用于恢复参数,并将恢复过的和超时的参数从保存介质中删除;saveOutputFlashMap用于将参数保存起来。

默认实现是 SessionFlashMapManager,它是将参数保存到session中。

\subsection{HandlerMapping}

