\section{认证流程}

\subsection{登录流程分析}

与登录流程相关的三个基本组件: \texttt{AuthenticationManager}, \texttt{ProviderManager} 和 \texttt{Authen ticationProvider}. 相关过滤器: \texttt{AbstractAuthenticationProcessingFilter}.

\subsubsection{\texttt{AuthenticationManager}}

\texttt{AuthenticationManager} 是一个认证管理器, 定义了 Spring Security 要如何执行认证操作. 认证成功后会返回一个 \texttt{Authentication} 对象,这个对象会被设置到 \texttt{SecurityContextHolder} 中.

\begin{Java}
public interface AuthenticationManager {
    Authentication authenticate(Authentication authentication) throws AuthenticationException;
}
\end{Java}

\texttt{AuthenticationManager} 对传入的 \texttt{Authentication} 进行身份认证, 此时传入的 \texttt{Authentication} 只有用户名, 密码等简单的数学,如果认证成功,会得到补充.

\texttt{AuthenticationManager} 常用的的实现类是 \texttt{ProviderManager} 也是 Spring Security 默认的实现类.

\subsubsection{\texttt{AuthenticationProvider}}

\texttt{AuthenticationProvider} 针对不同的身份类型执行具体的身份认证. 常见的认证方案及对应实现类如下:
\begin{itemize}
  \item \texttt{DaoAuthenticationProvider}: 支持用户名/密码登录认证.
  \item \texttt{RememberMeAuthenticationProvider}: 支持``记住我''认证.
\end{itemize}

\begin{Java}
public interface AuthenticationProvider {
    Authentication authenticate(Authentication authentication) throws AuthenticationException;
    // 判断是否支持对应的身份类型
    boolean supports(Class<?> authentication);
}
\end{Java}

大部分实现类都继承自 \texttt{AbstractUserDetailsAuthenticationProvider}, 它的几个主要属性和方法如下:

\begin{itemize}
  \item \textbf{\texttt{userCache}}: 声明一个用户缓存对象,默认情况下不启用.
  \item \textbf{\texttt{hideUserNotFoundExceptions}}: 隐藏失败异常,抛出一个模糊的 \texttt{BadCredentialsException} 异常代替查不到用户, 验证错误等异常, 默认开启.
  \item \textbf{\texttt{forcePrincipalAsString}}: 默认关闭,返回 \texttt{UserDetails} 对象, 开启后仅返回用户名.
  \item \textbf{\texttt{preAuthenticationChecks}}: 状态检查, 校验前
  \item \textbf{\texttt{postAuthenticationChecks}}: 状态检测, 校验后
  \item \textbf{\texttt{additionalAuthentication()}}: 校验密码
  \item \textbf{\texttt{authenticate()}}: 核心校验方法
\end{itemize}

\texttt{authenticate()} 方法检验用户名密码登录的逻辑如下:
\begin{itemize}
  \item 从登陆数据中获取用户名;
  \item 查用户获取用户对象;
  \begin{itemize}
    \item 根据用户名去缓存中查询用户对象;
    \item 如果查不到,仅数据库加载用户;
    \item 如果还是查不到,抛出异常(用户不存在);
  \end{itemize}
  \item 获取到用户对象后,调用 \texttt{preAuthenticationChecks.check()} 进行用户状态检查.
  \item 调用 \texttt{additionalAuthenticationChecks()} 进行密码校验;
  \item 调用 \texttt{postAuthenticationChecks.check()} 检查密码是否过期,
  \item 调用 \texttt{createSuccessAuthentication} 方法创建一个认证后的 \texttt{UsernamePasswordA uthenticationToken} 对象并返回.
\end{itemize}

\subsubsection{\texttt{ProviderManager}}

\texttt{ProviderManager} 是 \texttt{AuthenticationManager} 的一个重要实现类:

\begin{figure}[H]
  \small
  \centering
  \begin{tikzpicture}[font=\small]
    \node[block, fill=blue!20, drop shadow] (pm) at (0,0) {ProviderManager};
    \node[dashed, draw, minimum width=4cm, minimum height=4cm] (rect) at (5,0) {};
    \begin{scope}[font=\footnotesize]
      \node[block, fill=green!20, drop shadow] (p1) at (5,2) {AuthenticationProvider};
      \node[block, fill = red!10, minimum width=2cm, minimum height=0.25cm] (s1) at (5,0.5) {};
      \node[block, fill=green!20, drop shadow] (p2) at (5,-2) {AuthenticationProvider};
      \node[block, fill = red!10, minimum width=2cm, minimum height=0.25cm] (s2) at (5,-0.5) {};
      \draw[Stealth-Stealth] (p1) -- (s1);
      \draw[Stealth-Stealth] (p2) -- (s2);
    \end{scope}
    \draw[-Stealth] (pm) -- (rect);
  \end{tikzpicture}
  \caption{Provider Manager}
  \label{fig:Provider Manager}
\end{figure}

多个 \texttt{AuthenticationProvider} 将组成一个列表, 这个列表将由 \texttt{ProviderManager} 代理, 在 \texttt{ProviderManager} 中遍历每一个 \texttt{AuthenticationProvider} 去执行身份认证, 最终得到认证结果.

理论上, 多个 \texttt{ProviderManager} 本身也可以再配置一个 \texttt{AuthenticationManager} 作为 parent. 一直套娃下去.