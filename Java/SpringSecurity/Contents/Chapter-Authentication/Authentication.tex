\section{Spring Security 认证}

\subsection{默认认证}
\subsubsection{流程分析}

首先我们需要引入 SpringSecurity 依赖:

\begin{xml}
<dependency>
  <groupId>org.springframework.boot</groupId>
  <artifactId>spring-boot-starter-security</artifactId>
</dependency>
\end{xml}

此后我们请求任意接口,都会被强制跳转到登录界面(/login),Spring Security 提供了一个简单的账号密码登录验证。默认用户名为 user,密码会在控制台中显示。在这个过程中发生了下图所示请求(假设访问 /hello)。

\begin{figure}[H]
  \footnotesize
  \centering
  \begin{sequencediagram}
    \newthread{browser}{:Browser}
    \newinst[1]{app}{:Application}
    \newinst{login}{:DefaultLoginPageGeneratingFilter}
    \newinst{security}{:FilterSecurityInterceptor}
    \newinst{exception}{:ExceptionTranslationFilter}
    \begin{call}{browser}{请求 /hello 接口}{app}{重定向到 /login}
      \mess{app}{经过}{login}
      \mess{login}{经过}{security}
      \mess{security}{抛出异常: AuthenticationException}{exception}
      \mess{exception}{调用  LoginUrlAuthenticationEntryPoint, 返回 302}{app};
    \end{call}
    \begin{call}{browser}{请求 /login 接口}{app}{获取 login page}
      \mess{app}{经过}{login}
      \mess{login}{返回登陆界面}{app}
    \end{call}
  \end{sequencediagram}
  \caption{认证请求流程}
  \label{fig:认证请求流程}
\end{figure}

整个流程如下:
\begin{enumerate}
  \item \textbf{客户端访问 /hello 接口}
  \item \textbf{请求走过滤器链}, 在 \texttt{FilterSecurityInterceptor} 过滤器中被拦截,发现用户未认证,抛出 \texttt{AuthenticationException} 异常。
  \item \textbf{异常被捕获}: \texttt{ExceptionTranslationFilter} 过滤器捕获异常,调用 \texttt{LoginUrlAuthent icationEntryPoint} 方法返回 302 状态码,重定向到 /login 界面。
  \item \textbf{客户端发送 /login 请求}
  \item \textbf{/login 请求被拦截}: /login 请求被 \texttt{DefaultLoginPageGeneratingFilter} 拦截,并在过滤器中返回该界面。
\end{enumerate}

我们只引入了一个依赖,Spring Security 就做了这么多事情,主要包括:

\begin{itemize}
  \item 开启 Spring Security 自动化配置,创建 \texttt{springSecurityFilterChain} 并注入到 Spring 容器中。
  \item 创建一个 \texttt{UserDetailService} 实例,负责提供用户数据,默认用户数据是基于内容的。
  \item 生成一个默认的登陆界面。
  \item 开启各项攻击防御。
\end{itemize}

\subsubsection{原理分析}

\subsubsection*{默认用户生成}

Spring Security中定义了UserDetails接口来规范开发者自定义的用户对象,UserDetails接口定义如下:

\begin{Java}
public interface UserDetails extends Serializable {
    // 获取账户具备的权限
    Collection<? extends GrantedAuthority> getAuthorities();
    String getPassword();
    String getUsername();
    // 账户是否未过期
    boolean isAccountNonExpired();
    // 账户是否未被锁
    boolean isAccountNonLocked();
    // 账户凭证(如密码)是否未过期
    boolean isCredentialsNonExpired();
    boolean isEnabled();
}
\end{Java}

负责提供用户数据源的接口是: \texttt{UserDetailService}:

\begin{Java}
public interface UserDetailsService {
    UserDetails loadUserByUsername(String username) throws UsernameNotFoundException;
}
\end{Java}

\texttt{loadUserByUsername} 有一个参数是 \texttt{username}, 这是瀛湖认证时传入的用户名,最常见的就是用户在登陆时输入的用户名,实际上在单点登录等其他情况下,可能是别的用户名参数。开发者在这里拿到 \texttt{username} 之后再去数据库中查询用户,最终返回一个 \texttt{UserDetails} 实例。

在实际项目中,需要开发者自己实现 \texttt{UserDetailsService} 接口,如果没有实现,Spring Security 也提供了默认实现:

\begin{itemize}
  \item \textbf{UserDetailManager}: 在 \texttt{UserDetailsService} 基础上,增加了添加用户,更新用户,删除用户,修改密码,判断用户是否存在5个方法。
  \item \textbf{JdbcDaoImpl}: \texttt{UserDetailsService} 基础上,通过 spring-jdbc 实现了从数据库中查询用户的方法。
  \item \textbf{InMemoryUserDetailsManager}: 实现了 \texttt{UserDetailsService} 中关于用户增删改查的方法,不过都是基于内存的操作,没有持久化。
  \item ......
\end{itemize}

如果我们只引入一个 Spring Security 依赖,则默认使用 \texttt{InMemoryUserDetailsManager}。

默认情况下 Spring Security 是怎么处理的呢?可以看一下 \texttt{UserDetailsServiceAutoCon figuration} 源代码(比较长,不贴了),这里只讲关键的地方:
\begin{itemize}
  \item 默认明文密码需要前缀: {noop}
  \item 加载一个 \texttt{InMemoryUserDetailsManager} 注入到容器中。
\end{itemize}

同时启用改配置还有一些前置要求:
\begin{itemize}
  \item 当前 classPath 下存在 \texttt{AuthenticationManager} 类。
  \item 项目中,系统没有提供 \texttt{AuthenticationManager, AuthenticationProvider...} 实例。
\end{itemize}

\subsubsection*{默认页面生成}

在过滤器链中有两个和页面相关的过滤器: \texttt{DefaultLoginPageGeneratingFilter} 和 \texttt{Def aultLogoutPageGeneratingFilter}。分别负责登录,登出。

这两个类实现的主要功能(以Login为例)是: 判断当前请求是否为 登陆出错,注销成功,登录请求。如果是这三者中的任一个就会在 \texttt{DefaultLoginPageGeneratingFilter} 过滤器中生成登录页并返回。否则请求继续往下走,执行下一个过滤器。

\subsection{自定义登陆表}

Spring Security5.4 之后,通过创建 \texttt{SecurityFilterChain} bean 来配置 \texttt{HttpSecurity}。\footnote{后续均以 Spring Security 6 为准,不再讨论之前的写法}:

\begin{Java}
@Configuration
public class SecurityConfiguration {
    @Bean
    public SecurityFilterChain filterChain(HttpSecurity http) throws Exception {
        http.authorizeHttpRequests((authz) -> authz
                .anyRequest().authenticated()
            )
            .httpBasic(withDefaults());
        return http.build();
    }
}
\end{Java}