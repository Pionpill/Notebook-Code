\section{Spring Security 认证}

\subsection{默认认证}
\subsubsection{流程分析}

首先我们需要引入 SpringSecurity 依赖:

\begin{xml}
<dependency>
  <groupId>org.springframework.boot</groupId>
  <artifactId>spring-boot-starter-security</artifactId>
</dependency>
\end{xml}

此后我们请求任意接口,都会被强制跳转到登录界面(/login),Spring Security 提供了一个简单的账号密码登录验证。默认用户名为 user,密码会在控制台中显示。在这个过程中发生了下图所示请求(假设访问 /hello)。

\begin{figure}[H]
  \footnotesize
  \centering
  \begin{sequencediagram}
    \newthread{browser}{:Browser}
    \newinst[1]{app}{:Application}
    \newinst{login}{:DefaultLoginPageGeneratingFilter}
    \newinst{security}{:FilterSecurityInterceptor}
    \newinst{exception}{:ExceptionTranslationFilter}
    \begin{call}{browser}{请求 /hello 接口}{app}{重定向到 /login}
      \mess{app}{经过}{login}
      \mess{login}{经过}{security}
      \mess{security}{抛出异常: AuthenticationException}{exception}
      \mess{exception}{调用  LoginUrlAuthenticationEntryPoint, 返回 302}{app};
    \end{call}
    \begin{call}{browser}{请求 /login 接口}{app}{获取 login page}
      \mess{app}{经过}{login}
      \mess{login}{返回登陆界面}{app}
    \end{call}
  \end{sequencediagram}
  \caption{认证请求流程}
  \label{fig:认证请求流程}
\end{figure}

整个流程如下:
\begin{enumerate}
  \item \textbf{客户端访问 /hello 接口}
  \item \textbf{请求走过滤器链}, 在 \texttt{FilterSecurityInterceptor} 过滤器中被拦截,发现用户未认证,抛出 \texttt{AuthenticationException} 异常。
  \item \textbf{异常被捕获}: \texttt{ExceptionTranslationFilter} 过滤器捕获异常,调用 \texttt{LoginUrlAuthent icationEntryPoint} 方法返回 302 状态码,重定向到 /login 界面。
  \item \textbf{客户端发送 /login 请求}
  \item \textbf{/login 请求被拦截}: /login 请求被 \texttt{DefaultLoginPageGeneratingFilter} 拦截,并在过滤器中返回该界面。
\end{enumerate}

我们只引入了一个依赖,Spring Security 就做了这么多事情,主要包括:

\begin{itemize}
  \item 开启 Spring Security 自动化配置,创建 \texttt{springSecurityFilterChain} 并注入到 Spring 容器中。
  \item 创建一个 \texttt{UserDetailService} 实例,负责提供用户数据,默认用户数据是基于内容的。
  \item 生成一个默认的登陆界面。
  \item 开启各项攻击防御。
\end{itemize}

\subsubsection{原理分析}

\subsubsection*{默认用户生成}

Spring Security中定义了UserDetails接口来规范开发者自定义的用户对象,UserDetails接口定义如下:

\begin{Java}
public interface UserDetails extends Serializable {
    // 获取账户具备的权限
    Collection<? extends GrantedAuthority> getAuthorities();
    String getPassword();
    String getUsername();
    // 账户是否未过期
    boolean isAccountNonExpired();
    // 账户是否未被锁
    boolean isAccountNonLocked();
    // 账户凭证(如密码)是否未过期
    boolean isCredentialsNonExpired();
    boolean isEnabled();
}
\end{Java}

负责提供用户数据源的接口是: \texttt{UserDetailService}:

\begin{Java}
public interface UserDetailsService {
    UserDetails loadUserByUsername(String username) throws UsernameNotFoundException;
}
\end{Java}

\texttt{loadUserByUsername} 有一个参数是 \texttt{username}, 这是瀛湖认证时传入的用户名,最常见的就是用户在登陆时输入的用户名,实际上在单点登录等其他情况下,可能是别的用户名参数。开发者在这里拿到 \texttt{username} 之后再去数据库中查询用户,最终返回一个 \texttt{UserDetails} 实例。

在实际项目中,需要开发者自己实现 \texttt{UserDetailsService} 接口,如果没有实现,Spring Security 也提供了默认实现:

\begin{itemize}
  \item \textbf{UserDetailManager}: 在 \texttt{UserDetailsService} 基础上,增加了添加用户,更新用户,删除用户,修改密码,判断用户是否存在5个方法。
  \item \textbf{JdbcDaoImpl}: \texttt{UserDetailsService} 基础上,通过 spring-jdbc 实现了从数据库中查询用户的方法。
  \item \textbf{InMemoryUserDetailsManager}: 实现了 \texttt{UserDetailsService} 中关于用户增删改查的方法,不过都是基于内存的操作,没有持久化。
  \item ......
\end{itemize}

如果我们只引入一个 Spring Security 依赖,则默认使用 \texttt{InMemoryUserDetailsManager}。

默认情况下 Spring Security 是怎么处理的呢?可以看一下 \texttt{UserDetailsServiceAutoCon figuration} 源代码(比较长,不贴了),这里只讲关键的地方:
\begin{itemize}
  \item 默认明文密码需要前缀: {noop}
  \item 加载一个 \texttt{InMemoryUserDetailsManager} 注入到容器中。
\end{itemize}

同时启用改配置还有一些前置要求:
\begin{itemize}
  \item 当前 classPath 下存在 \texttt{AuthenticationManager} 类。
  \item 项目中,系统没有提供 \texttt{AuthenticationManager, AuthenticationProvider...} 实例。
\end{itemize}

\subsubsection*{默认页面生成}

在过滤器链中有两个和页面相关的过滤器: \texttt{DefaultLoginPageGeneratingFilter} 和 \texttt{Def aultLogoutPageGeneratingFilter}。分别负责登录,登出。

这两个类实现的主要功能(以Login为例)是: 判断当前请求是否为 登陆出错,注销成功,登录请求。如果是这三者中的任一个就会在 \texttt{DefaultLoginPageGeneratingFilter} 过滤器中生成登录页并返回。否则请求继续往下走,执行下一个过滤器。

\subsection{自定义登陆表}

Spring Security5.4 之后,通过创建 \texttt{SecurityFilterChain} bean 来配置 \texttt{HttpSecurity}。\footnote{后续均以 Spring Security 6 为准,均使用注入 bean 的方法实现},两篇推荐的文献:
\begin{itemize}
  \item 注入 \texttt{SecurityFilterChain} Bean 完成配置: \url{https://blog.csdn.net/lazy_LYF/article/details/127284459}
  \item 基于数据库的配置: \url{https://blog.csdn.net/lazy_LYF/article/details/127284982}
\end{itemize}

简单地进行自定义配置如下:

\begin{Java}
@Configuration
public class SecurityConfig {
    @Bean
    public SecurityFilterChain filterChain(HttpSecurity http) throws Exception {
        http.authorizeHttpRequests()
                .anyRequest()     // 针对任意请求
                .authenticated()  // 进行认证
                .and()            // 回到 HttpSecurity 实例
                .formLogin()      // 开启表单登录
                .loginPage("/login")  // 配置登录页面
                .loginProcessingUrl("/doLogin") // 配置登录接口地址
                .defaultSuccessUrl("/index")    // 默认成功登陆后跳转的地址
                .failureUrl("/login") // 失败后跳转的地址
                .usernameParameter("username")  // 用户名参数名
                .passwordParameter("pwd")       // 密码参数名
                .permitAll();     // 允许所有请求进入该界面
        return http.build();
    }
}
\end{Java}

这里我们通过注入自定义过滤器链替代了 Spring Security 默认的过滤连,并做了一些基础配置。大致流程如下:
\begin{itemize}
  \item 确定请求类型。
  \item 配置登录界面及成功与失败处理。
  \item 返回 \texttt{http.build()} 注入到容器中替代默认配置。
\end{itemize}

上述代码有几个注意的地方:
\begin{itemize}
  \item \texttt{and()}: 表示回到 HttpSecurity 实例,官方提供的链式编程方法,与重新使用 \texttt{http.formLogin()} 等效。
  \item \texttt{loginPage()} 表示登录界面,\texttt{loginProcessingUrl()} 表示后端处理的接口地址。
\end{itemize}

\subsubsection{登陆成功}

在配置方法中,与成功登录相关的方法是 \texttt{defaultUrl()}: 表示成功登录之后,会自动重定向到登陆前的地址上。例如访问的是 /hello 接口,302重定向到 /login 接口认证之后会回到之前的 /hello 界面。

这存在一个问题,如果之前访问的是 /login 接口,那么认证成功后会停留在当前界面,这对用户不是很友好。有两种解决方案:

\begin{itemize}
  \item 使用 \texttt{successForwardUrl},只要成功登录,强制跳转到指定的界面。
  \item \texttt{defaultSuccess} 有一个重载方法,第二个参数传入 \texttt{true},则不考虑用户之前的访问地址,进入 \texttt{defaultSuccess} 页面。
\end{itemize}

这两者有一个不同之处,\texttt{successForwardUrl} 通过服务器跳转实现,\texttt{defaultSuccess} 则通过让客户端重定向实现。

登陆成功后会由 \texttt{AuthenticationSuccessHandler} 接口处理:
\begin{Java}
public interface AuthenticationSuccessHandler {
    default void onAuthenticationSuccess(HttpServletRequest request, HttpServletResponse response, FilterChain chain, Authentication authentication) throws IOException, ServletException {
        this.onAuthenticationSuccess(request, response, authentication);
        chain.doFilter(request, response);
    }

    void onAuthenticationSuccess(HttpServletRequest request, HttpServletResponse response, Authentication authentication) throws IOException, ServletException;
}
\end{Java}

\texttt{AuthenticationSuccessHandler} 有两个直接实现类:
\begin{itemize}
  \item \texttt{SimpleUrlAuthenticationSuccessHandler}: 继承自 \texttt{AbstractAuthenticationTarget UrlRequestHandler},通过它的 \texttt{handle} 方法实现请求重定向。
  \item \texttt{ForwardAuthenticationSuccessHandler}: 服务端跳转。
\end{itemize}

比较重要的是 \texttt{SaveRequestAwareAuthenticationSuccessHandler},继承自 \texttt{SimpleUrl AuthenticationSuccessHandler},在其基础上添加了请求缓存功能,可以记录之前的地址,进而在登陆成功之后重定向到一开始访问的地址。\texttt{defaultSuccessUrl} 对应的实现类就是它。它的处理逻辑如下:
\begin{itemize}
  \item 获取缓存请求,如果没有,说明在访问登录页之前没有访问其他页面,交给父类处理,重定向到 \texttt{defaultSuccessUrl} 地址。
  \item 如果存在缓存,且存在一个 \texttt{targetUrlParameter} 或者 \texttt{defaultSuccessUrl} 第二个参数为 \texttt{true} 则直接重定向到默认地址。
  \item 否则,获取重定向地址进行重定向。
\end{itemize}


我们可以自定义对应的逻辑:
\begin{Java}
@Bean
SavedRequestAwareAuthenticationSuccessHandler savedRequestAwareAuthenticationSuccessHandler() {
    SavedRequestAwareAuthenticationSuccessHandler handler = new SavedRequestAwareAuthenticationSuccessHandler();
    handler.setDefaultTargetUrl("/index");
    handler.setTargetUrlParameter("target");
    return handler;
}
\end{Java}

\texttt{successForwardUrl} 对应 \texttt{ForwardAuthenticationSuccessHandler},其功能特别简单,就是一个服务器转发。

现在流行的前后端分离架构中,一般不需要后端处理这些逻辑,应该返回 json 数据由前端进行判断:

\begin{Java}
public class MyAuthenticationSuccessHandler implements AuthenticationSuccessHandler {
    @Override
    public void onAuthenticationSuccess(HttpServletRequest request, HttpServletResponse response, Authentication authentication) throws IOException, ServletException {
        response.setContentType("application/json; charset=utf-8");
        Map<String, Object> resp = new HashMap<>();
        resp.put("status",200);
        resp.put("msg","成功登录!");
        ObjectMapper om = new ObjectMapper();
        String s = om.writeValueAsString(resp);
        response.getWriter().write(s);
    }
}
\end{Java}

然后将其写入带 Config 文件中即可:

\begin{Java}
@Bean
public SecurityFilterChain filterChain(HttpSecurity http) throws Exception {
    http.authorizeHttpRequests()
            ...
            .successHandler(new MyAuthenticationSuccessHandler())
            ...
    return http.build();
}
\end{Java}

\subsubsection{登录失败}

登录失败和成功逻辑类似,相关方法是 \texttt{failureUrl},表示登录失败后重定向的路由。如果希望展示请求失败的异常信息,可以使用这种方式:

\begin{Java}
@Bean
public SecurityFilterChain filterChain(HttpSecurity http) throws Exception {
    http.authorizeHttpRequests()
        ...
        .failureForwardUrl("mylogin")
        ...
    return http.build();
}
\end{Java}

\texttt{failureForwardUrl} 是一种服务器跳转,好处是可以携带登陆异常信息。

这两种方式所配置的都是 \texttt{AuthenticationFailureHandler} 接口:

\begin{Java}
public interface AuthenticationFailureHandler {
    void onAuthenticationFailure(HttpServletRequest request, HttpServletResponse response, AuthenticationException exception) throws IOException, ServletException;
}
\end{Java}

它对应的实现类有很多,这里列出重要的几个:
\begin{itemize}
  \item \textbf{\texttt{SimpleUrlAuthenticationFailureHandler}}: 默认处理逻辑,重定向到登陆界面,是 \texttt{failureUrl()} 方法的底层实现。
  \item \textbf{\texttt{ExceptionMappingAuthenticationFailureHandler}}: 根据异常类型映射到不同路径。
  \item \textbf{\texttt{ForwardAuthenticationFailureHandler}}: 通过服务端跳转到登陆界面,是 \texttt{failure ForwardUrl} 方法的底层实现。
\end{itemize}

自定义登陆失败的方法和登陆成功类似,不再赘述。

\subsubsection{注销登录}

Spring Security 提供了默认的注销界面:

\begin{Java}
@Bean
public SecurityFilterChain filterChain(HttpSecurity http) throws Exception {
    http.authorizeHttpRequests()
        ...
        .and()
        .logout()   // 开启注销登录配置
        .logoutUrl("logout")  // 指定请求地址
        .invalidateHttpSession(true)  // 是否使 session 失效
        .clearAuthentication(true)    // 是否清除认证信息
        .logoutSuccessUrl("/myLogout")  // 注销后跳转的地址
        ...
    return http.build();
}
\end{Java}

自定义配置方法类似,不再赘述。

\subsection{登录用户数据获取}

登录成功之后,用户信息将默认保存在 HttpSession 中,Spring Security 对齐进行了封装,开发者获取用户数据有两种主流的思路:

\begin{itemize}
  \item 从 \texttt{SecurityContextHolder} 中获取。
  \item 从当前请求对象中获取。
\end{itemize}

无论使用哪种方案都离不开 \texttt{Authentication} 对象:

\begin{Java}
public interface Authentication extends Principal, Serializable {
    // 获取用户权限
    Collection<? extends GrantedAuthority> getAuthorities();
    // 获取用户凭证,一般是密码
    Object getCredentials();
    // 获取用户携带的详细信息,可能是当前请求
    Object getDetails();
    // 获取当前用户,用户名或用户对象
    Object getPrincipal();
    // 当前用户是否认证成功
    boolean isAuthenticated();
    void setAuthenticated(boolean isAuthenticated) throws IllegalArgumentException;
}
\end{Java}

\texttt{Authentication} 对象主要有两方面的功能:
\begin{itemize}
  \item 作为 \texttt{AuthenticationManager} 输入参数,提供用户身份认证的凭证。
  \item 代表已经经过身份认证的用户,此时的 \texttt{Authentication} 可以从 \texttt{SecurityContext} 中获取。
\end{itemize}

一个 \texttt{Authentication} 对象主要包括三方面信息:
\begin{itemize}
  \item \textbf{\texttt{principal}}: 定义认证的用户,通常是 \texttt{UserDetail} 对象。
  \item \textbf{\texttt{credentials}}: 登录凭证,比如密码。登陆成功后,登录凭证会被自动擦除。
  \item \textbf{\texttt{authorities}}: 用户被赋予的权限信息。
\end{itemize}

Java 本身提供了 \texttt{Principal} 接口来描述认证主体,\texttt{Authentication} 则继承自 \texttt{Principal}。\texttt{Authentication} 有多个实现类,常用的有两个:

\begin{itemize}
  \item \textbf{\texttt{UsernamePasswordAuthenticationToken}}: 表单登录时封装的用户对象。
  \item \textbf{\texttt{RememberMeAuthenticationToken}}: 使用 RememberMe 方式时封装的用户对象。
\end{itemize}

\subsubsection{从 \texttt{SecurityContextHolder} 中获取}

一般的,获取用户信息的代码如下:

\begin{Java}
@GetMapping("/user")
public void userInfo() {
    Authentication authentication = SecurityContextHolder.getContext().getAuthentication();
    String username = authentication.getName();
    ......
}
\end{Java}

其中 \texttt{SecurityContextHolder.getContext()} 是一个静态方法,用于返回一个  \texttt{Security Context} 对象。

\begin{figure}[H]
  \small
  \centering
  \begin{tikzpicture}[font=\small]
    \draw[fill=yellow!20] (0,0) rectangle (7,-3);
    \node at (3.5,-0.3) {SecurityContextHolder};
    \draw[fill=orange!20] (0.25,-0.6) rectangle (6.75,-2.75);
    \node at (3.5,-0.9) {SecurityContext};
    \draw[fill=red!20] (0.5,-1.2) rectangle (6.5,-2.5);
    \node at (3.5,-1.5) {Authentication};
    \begin{scope}[font=\footnotesize]
      \node[block, fill=red!50] at (1.6,-2) {Principal};
      \node[block, fill=red!50] at (3.5,-2) {Credentials};
      \node[block, fill=red!50] at (5.4,-2) {Authorities};
    \end{scope}
  \end{tikzpicture}
  \caption{SecurityContextHolder 关系}
  \label{fig:SecurityContextHolder 关系}
\end{figure}

\texttt{SecurityContextHolder} 中存放的是 \texttt{SecurityContext},共定义了三种不同的数据存储策略:
\begin{itemize}
  \item \textbf{MODE\_THREADLOCAL}: 将 \texttt{SecurityContext} 存放到 \texttt{ThreadLocal} 中,是默认的存储策略。如果开启了子线程,则无法获取用户信息。
  \item \textbf{MODE\_INHERITABLETHREADLOCAL}: 适用于多线程模式。
  \item \textbf{MODE\_GLOBAL}: 实际上将数据保存在静态变量中,Web 开发中很少使用.
\end{itemize}

\texttt{SecurityContextHolderStrategy} 用于规范存储策略中的方法:

\begin{Java}
public interface SecurityContextHolderStrategy {
    void clearContext();
    SecurityContext getContext();
    default Supplier<SecurityContext> getDeferredContext() {
        return () -> {
            return this.getContext();
        };
    }
    void setContext(SecurityContext context);
    default void setDeferredContext(Supplier<SecurityContext> deferredContext) {
        this.setContext((SecurityContext)deferredContext.get());
    }
    SecurityContext createEmptyContext();
}
\end{Java}

对应三种存储策略,有三个 \texttt{SecurityContextHolderStrategy}  实现类. 都比较简单,读者有兴趣可以自行查源代码.

\subsubsection*{\texttt{SecurityContextHolderFilter}}

\texttt{SecurityContextHolderFilter} 是专为存储 \texttt{Security Context} 而设计的. 主要做了两件事:
\begin{itemize}
  \item 请求到来,从 \texttt{HttpSession} 中获取 \texttt{SecurityContext} 并存入 \texttt{SecurityContextHolder} 中.
  \item 请求完成, 从 \texttt{SecurityContextHolder} 中获取 \texttt{SecurityContext} 并存入 \texttt{HttpSession} 中.
\end{itemize}

在此之前,先要了解 \texttt{SecurityContextRepository} 接口:

\begin{Java}
public interface SecurityContextRepository {
    /** @deprecated */
    @Deprecated
    SecurityContext loadContext(HttpRequestResponseHolder requestResponseHolder);
    default DeferredSecurityContext loadDeferredContext(HttpServletRequest request) {
        Supplier<SecurityContext> supplier = () -> {
            return this.loadContext(new HttpRequestResponseHolder(request, (HttpServletResponse)null));
        };
        return new SupplierDeferredSecurityContext(SingletonSupplier.of(supplier), SecurityContextHolder.getContextHolderStrategy());
    }
    void saveContext(SecurityContext context, HttpServletRequest request, HttpServletResponse response);
    boolean containsContext(HttpServletRequest request);
}
\end{Java}

它的几个方法很好理解, 主要实现类是 \texttt{HttpSessionSecurityContextRepository}. 实现了将 \texttt{SecurityContext} 存储到 \texttt{HttpSession} 以及从 \texttt{HttpSession} 中加载 \texttt{SecurityContext} 出来. 具体的实现方案比较复杂, 有兴趣请自行查源代码.

\texttt{SecurityContextHolderFilter} 主要通过 \texttt{SecurityContextRepository} 获取用户信息, 主要方法如下:

\begin{Java}
public class SecurityContextHolderFilter extends GenericFilterBean {
    private final SecurityContextRepository securityContextRepository;
    private SecurityContextHolderStrategy securityContextHolderStrategy = SecurityContextHolder.getContextHolderStrategy();

    // 获取 SecurityContextRepository 实例
    public SecurityContextHolderFilter(SecurityContextRepository securityContextRepository) {
        Assert.notNull(securityContextRepository, "securityContextRepository cannot be null");
        this.securityContextRepository = securityContextRepository;
    }

    private void doFilter(HttpServletRequest request, HttpServletResponse response, FilterChain chain) throws ServletException, IOException {
        if (request.getAttribute(FILTER_APPLIED) != null) {
            chain.doFilter(request, response);
        } else {
          request.setAttribute(FILTER_APPLIED, Boolean.TRUE);
          // 获取信息
          Supplier<SecurityContext> deferredContext = this.securityContextRepository.loadDeferredContext(request);
          try {
              // 写入信息
              this.securityContextHolderStrategy.setDeferredContext(deferredContext);
              chain.doFilter(request, response);
          } finally {
              this.securityContextHolderStrategy.clearContext();
              request.removeAttribute(FILTER_APPLIED);
          }

        }
    }
}
\end{Java}

一言以蔽之,请求在到达过滤器之后,先从HttpSession中读取SecurityContext出来,并存入SecurityContextHolder之中以备后续使用;当请求离开过滤器的时候,获取最新的SecurityContext并存入HttpSession中,同时清空SecurityContextHolder中的登录用户信息。

\subsubsection{从当前请求对象中获取}

开发者可以直接在 Controller 请求参数中放入 \texttt{Authentication} 对象来获取登录用户信息. 

\begin{Java}
@RequestMapping("/authentication")
public void authentication(Authentication authentication) {
    System.out.println(authentication);
}
\end{Java}

Controller 中的方法是当前请求 \texttt{HttpServletRequest} 带来的. 那么这些数据(\texttt{Authentication}参数)是如何放入的呢?

如果使用了Spring Security框架,那么我们在Controller参数中拿到的\texttt{HttpServletRequest}实例将是\texttt{Servlet3SecurityContextHolderAwareRequestWrapper},很明显,这是被Spring Security封装过的请求。

我们直接将\texttt{Authentication}或者\texttt{Principal}写到Controller参数中,实际上就是Spring MVC框架从\texttt{Servlet3SecurityContextHolderAwareRequestWrapper}中提取的用户信息。

此外,还有一个 \texttt{SecurityContextHolderAwareRequestFilter} 过滤器,对 \texttt{HttpServlet Request} 请求进行再包装. 

对请求的\texttt{HttpServletRequest}包装之后,接下来在过滤器链中传递的\texttt{HttpServlet Request}对象,它的多个方法就可以直接使用了。\texttt{HttpServletRequest}中\texttt{getUserPrincipal()}方法有了返回值之后,最终在Spring MVC的\texttt{ServletRequestMethodArgumentResolver\#resolve Argument(Class<?>,HttpServletRequest)}进行默认参数解析,自动解析出Principal对象。开发者在Controller中既可以通过\texttt{Principal}来接收参数,也可以通过\texttt{Authentication}对象来接收.

\subsection{用户定义}

自定义用户其实是使用 \texttt{UserDetailService} 的不同实现类来提供用户数据,同时将配置好的 \texttt{UserDetailService} 配置给 \texttt{AuthenticationManagerBuilder}, 系统再将 \texttt{UserDetailService} 提供给 \texttt{AuthenticationProvider} 使用.

一种配置方案如下(部分代码省略):

\begin{Java}
@Bean
AuthenticationManager authenticationManager(HttpSecurity httpSecurity) throws Exception {
    UserDetailsService userServiceImpl;
    AuthenticationManager authenticationManager = httpSecurity.getSharedObject(AuthenticationManagerBuilder.class)
            .userDetailsService(userServiceImpl)
            .passwordEncoder(passwordEncoder())
            .and()
            .build();
    return authenticationManager;
}
\end{Java}

\newpage