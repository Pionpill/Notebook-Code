\section{概览}

安全框架(包括 Spring Security)的两个主要功能: 认证,授权。认证否则身份验证,授权负责访问控制。

Spring Security 的优点有很多,比如:
\begin{itemize}
  \item Spring 家族成员。
  \item 良好的微服务配置。
  \item 自动防止网络攻击。
\end{itemize} 

同时也有一个主要缺点: 臃肿。和其他 Spring 项目类似,很快入门,极难精通。

\subsection{认证与授权}
\subsubsection*{认证}

在 Spring Security 架构设计中,认证和授权是分开的,两者互不影响。

用户认证信息主要由 \texttt{Authentication} 实现类来保存:

\begin{Java}
public interface Authentication extends Principal, Serializable {
    // 获取用户权限
    Collection<? extends GrantedAuthority> getAuthorities();
    // 获取用户凭证,一般是密码
    Object getCredentials();
    // 获取用户携带的详细信息,可能是当前请求
    Object getDetails();
    // 获取当前用户,用户名或用户对象
    Object getPrincipal();
    // 当前用户是否认证成功
    boolean isAuthenticated();
    void setAuthenticated(boolean isAuthenticated) throws IllegalArgumentException;
}
\end{Java}

认证工作主要由 \texttt{AuthenticationManager} 接口来负责:

\begin{Java}
public interface AuthenticationManager {
    Authentication authenticate(Authentication authentication) throws AuthenticationException;
}
\end{Java}

\texttt{authenticate} 方法用于做认证,有三个不同的返回值:
\begin{itemize}
  \item 返回 \texttt{Authentication},表认证成功。
  \item 抛出 \texttt{AuthenticationException},表示输入了无效的凭证。
  \item 返回 \texttt{null},表示不能断定。
\end{itemize}

\texttt{AuthenticationManager} 最主要的实现类是 \texttt{ProviderManager},它管理了众多 \texttt{Authenti cationProvider} \texttt{实例,AuthenticationProvider} 类似于 \texttt{AuthenticationManager,但多了一个} \texttt{supports} 方法用来判断是否支持给定的 \texttt{Authentication} 类型。

\begin{Java}
public interface AuthenticationProvider {
    Authentication authenticate(Authentication authentication) throws AuthenticationException;
    boolean supports(Class<?> authentication);
}
\end{Java}

由于 \texttt{Authentication} 拥有众多不同的实现类,这些不同的实现类又由不同的 \texttt{Authentic ationProvider} 来处理,所以 \texttt{AuthenticationProvider} 会有一个 \texttt{supports} 方法,用来判断
当前的 \texttt{AuthenticationProvider} 是否支持对应的\texttt{Authentication}。

在一次完整的认证流程中,可能会同时存在多个 \texttt{AuthenticationProvider}(例如账号密码登录,短信验证登录),多个 \texttt{ProviderManager} 统一由 \texttt{ProviderManager} 管理。同时 \texttt{ProviderManager} 具有一个可选的 \texttt{parent},如果所有认证都失败,就调用 \texttt{parent} 进行认证(备胎)。

\subsubsection*{授权}

授权体系中,有两个关键接口:
\begin{itemize}
  \item \texttt{AccessDecisionManager}: 决策器,判断此次访问是否被允许。
  \item \texttt{AccessDecisionVoter}: 投票器,检查用户是否具有应有的角色。
\end{itemize}

\texttt{AccessDecisionManager} 会挨个遍历 \texttt{AccessDecisionVoter} 进而决定是否允许用户访问。

在 Spring Security 中,用户请求一个资源所需要的角色会被封装成一个 \texttt{ConfigAttribute} 对象,\texttt{ConfigAttribute} 只有一个 \texttt{getAttribute} 方法,但会角色的名称。

\subsection{过滤器}

SpringSecurity 中认证授权等功能都是基于过滤器来完成的。过滤器按照既定的优先级排列,最终形成一个过滤器链,开发者可以自定义过滤器链,并通过 \texttt{@Order} 注解去调整过滤器位置。

SpringSecurity 中的过滤器链通过 \texttt{FilterChainProxy} 嵌入到 Web 项目的原生过滤器中,过滤器链可以不止一个,多个过滤器链之间要指定优先级:

\texttt{FilterChainProxy} 作为一个顶层管理者,将统一管理 Security Filter。\texttt{FilterChainProxy} 本身通过 Spring 框架提供的 \texttt{DelegatingFilterProxy} 整合到原生过滤器链中:

\begin{figure}[H]
  \small
  \centering
  \begin{tikzpicture}[font=\small]
    \node(browser) [block, fill=blue!30] at (0,0) {Client(Browser)};
    \node(filter1) [block, fill=orange!20] at (0,-1.5) {Filter};
    \node(proxy) [block, fill=orange!20] at (0,-3) {DelegatingFilterProxy \\ \footnotesize{(FilterChainProxy)}};
    \node(filter2) [block, fill=orange!20] at (0,-4.5) {Filter};
    \node(servlet) [block, fill=blue!30] at (0,-6) {Servlet};
    \begin{scope}[xshift=5cm, yshift=-2cm]
      \draw[dashed] (-0.3,-0.5) rectangle (2.3,1.5);
      \draw[fill=orange!20, draw=white] (0,0) rectangle (2,0.25);
      \draw[fill=orange!20, draw=white] (0,0.5) rectangle (2,0.75);
      \draw[fill=orange!20, draw=white] (0,1) rectangle (2,1.25);
      \node[font=\footnotesize] at (1,-0.25) {SecurityFilterChain};
    \end{scope}
    \begin{scope}[xshift=5cm, yshift=-5cm]
      \draw[dashed] (-0.3,-0.5) rectangle (2.3,1.5);
      \draw[fill=orange!20, draw=white] (0,0) rectangle (2,0.25);
      \draw[fill=orange!20, draw=white] (0,0.5) rectangle (2,0.75);
      \draw[fill=orange!20, draw=white] (0,1) rectangle (2,1.25);
      \node[font=\footnotesize] at (1,-0.25) {SecurityFilterChain};
    \end{scope}
    \draw[-Stealth] (proxy) -- +(3,0) |- node [pos=0.2,fill=white] {/admin/*} (4.75,-1);
    \draw[-Stealth] (proxy) -- +(3,0) |- node [pos=0.2, fill=white] {/user/*} (4.75,-5);
    \draw[Stealth-Stealth] (browser) -- (filter1);
    \draw[Stealth-Stealth] (filter1) -- (proxy);
    \draw[Stealth-Stealth] (proxy) -- (filter2);
    \draw[Stealth-Stealth] (filter2) -- (servlet);
  \end{tikzpicture}
  \caption{过滤器链}
  \label{fig:过滤器链}
\end{figure}

\subsection{登录数据}

Spring Security 将数据保存在 Session 中,但做了一些调整。

用户成功登陆后,数据会被保存在  \texttt{SecurityContextHolder} 中,\texttt{SecurityContextHolder} 中的数据保存默认通过 \texttt{ThreadLocal} 实现。只允许当前线程获取。当登录请求处理完毕后,Spring Security会将\texttt{SecurityContextHolder}中的数据拿出来保存到Session中,同时将\texttt{Sec urityContextHolder}中的数据清空。以后每当有请求到来时,Spring Security就会先从Session中取出用户登录数据,保存到\texttt{SecurityContextHolder}中,方便在该请求的后续处理过程中使用,同时在请求结束时将\texttt{SecurityContextHolder}中的数据拿出来保存到Session中,然后将\texttt{SecurityContextHolder}中的数据清空。

\newpage