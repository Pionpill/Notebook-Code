\documentclass{PionpillNote-book}
\usetikzlibrary {intersections,through,arrows.meta,graphs,shapes.misc,positioning,shapes.misc,positioning,calc}
\usetikzlibrary{animations}
\usetikzlibrary {shapes.geometric}
\usetikzlibrary {animations}
\usetikzlibrary {shapes.multipart}
\usetikzlibrary {positioning}
\usetikzlibrary {fit,shapes.geometric}
\usetikzlibrary {automata}
\usetikzlibrary {quotes}
\usetikzlibrary {matrix}
\usetikzlibrary {backgrounds}
\usetikzlibrary {scopes}
\usetikzlibrary {calc}
\usetikzlibrary {intersections}
\usetikzlibrary {svg.path}
\usetikzlibrary {decorations}
\usetikzlibrary {patterns}
\usetikzlibrary {decorations.pathmorphing}
\usetikzlibrary {shadows}
\usetikzlibrary {bending}

\title{Spring Security Principle}
\author{
    Pionpill \footnote{笔名:北岸,电子邮件:673486387@qq.com,Github:\url{https://github.com/Pionpill}} \\
    本文档为作者学习 SpringSecurity 理论时的笔记。\\
}

\date{\today}

\begin{document}

\pagestyle{plain}
\maketitle

\noindent\textbf{前言:}

笔者为软件工程系在校本科生,有计算机学科理论基础,本文前置技术栈:
\begin{itemize}
    \item Java: Java 基础
    \item Spring: SpringBoot 基础
\end{itemize}

本文重点是 Spring Secruity 原理,入门及实战可以参考网上的一些视频教程。主要参考资料:
\begin{itemize}
    \item SpringSecurity框架教程(视频): \url{https://www.bilibili.com/video/BV1mm4y1X7Hc/}
\end{itemize}

本人的编写及开发环境如下:
\begin{itemize}
    \item Java: Java17
    \item SpringBoot: 3.0.2
    \item SpringSecurity: 6.0.1
    \item OS: Windows11
\end{itemize}

\date{\today}
\newpage

\tableofcontents

\newpage

\setcounter{page}{1} 
\pagestyle{fancy}

\chapter{Spring Security 基础}
\import{Contents}{Tikz.tex}
\import{Contents/Chapter-Authentication}{Abstract.tex}
\import{Contents/Chapter-Authentication}{Authentication.tex}

\end{document}