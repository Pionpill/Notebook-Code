\section{容器基础与工具}

\subsection{Arrays 数组工具类}

\subsubsection{Array 动态数组类}

与这几章大部分类不同的是,Array 类位于 java.lang.reflect 包下,其他都位于 java.util 包下。可能是因为 Array 都是 native 方法。

Array 为数组类,为我们提供了动态创建和访问Java数组的方法。它是最高效的,但是其容量是固定的,并且无法动态改变,且存放同一数据类型的数据。与 Arrays 相同,他的所有方法都是静态的,并且全部方法根本上都是 native 的。

Array 只包含三类方法: newInstance 用于创建数组;get, set 方法用于获取与设置数组元素,由于是 native 的不讲了。

\subsubsection*{newInstance 创建数组}

newInstance 方法是唯一不直接被 native 修饰的方法,用于创建一个定长数组:

\begin{Java}
public static Object newInstance(Class<?> componentType, int length)
    throws NegativeArraySizeException {
    return newArray(componentType, length);
}

@IntrinsicCandidate
private static native Object newArray(Class<?> componentType, int length)
    throws NegativeArraySizeException;
\end{Java}

可以看到,本质上还是用 native 方法实现,这也是 Array 高效的原因。

newInstance 还有一个重载的同名函数,用于创建多维数组。

\subsubsection{Arrays 工具类}

Arrays包含各种操作数组的方法(例如排序和搜索)。该类还包含一个静态工厂,允许将数组视为列表。Arrays类里的方法都是静态方法, 下面介绍几个常用方法。

\subsubsection*{sort 排序}

排序,默认使用双轴快速排序算法,在排序前会对传入的下标(如果有)进行检查。

\begin{Java}
public static void sort(int[] a, int fromIndex, int toIndex) {
    rangeCheck(a.length, fromIndex, toIndex);
    DualPivotQuicksort.sort(a, 0, fromIndex, toIndex);
}
\end{Java}

JDK8 提供了另一种排序方案: parallelSort 用于处理大数据。它的实现逻辑和 sort 类似。

\begin{Java}
public static void parallelSort(int[] a, int fromIndex, int toIndex) {
    rangeCheck(a.length, fromIndex, toIndex);
    DualPivotQuicksort.sort(a, ForkJoinPool.getCommonPoolParallelism(), fromIndex, toIndex);
}
\end{Java}

可以看出,两种排序方式仅有一个值不同,这两个啊脾虚方法特点如下:
\begin{itemize}
    \item \textbf{sort}: 单线程,适用于处理小数据。
    \item \textbf{parallelSort}: 多线程并行排序,将数据分段调用多核心处理再合并,是用于处理大量数据。
\end{itemize}

一般的,数组长度大于 10000,采用并行排序,其他采用普通排序即可。

此外,还有一个私有的 legacyMergeSort 方法:

\begin{Java}
private static void legacyMergeSort(Object[] a) {
    Object[] aux = a.clone();
    mergeSort(aux, a, 0, a.length, 0);
}
\end{Java}

JDK7 之前用的是归并排序,这个方法是用来兼容旧版的,此外所有归并排序都是私有方法,只有特性情况下会在 sort 中调用该方法。

\subsubsection*{parallelPrefix 并行相关计算}

给出某种运算方法,对数组中下标为 1 的元素开始进行计算:

\begin{Java}
public static void parallelPrefix(int[] array, IntBinaryOperator op) {
    Objects.requireNonNull(op);
    if (array.length > 0)
        new ArrayPrefixHelpers.IntCumulateTask
                (null, op, array, 0, array.length).invoke();
}
\end{Java}

其中 IntBinaryOperator(不同类型有对应不同的接口) 是函数式编程接口:

\begin{Java}
@FunctionalInterface
public interface IntBinaryOperator {
    int applyAsInt(int left, int right);
}
\end{Java}

一般的,配合 lambda 表达式使用,例如:
\begin{Java}
int arr[] = {1,2,3,4};
Arrays.parallelPrefix(arr, (left, right) -> left + right);
// arr: {1,3,6,10}
\end{Java}

有两点需要注意:
\begin{itemize}
    \item 运算从第二个元素开始,即下标为 1 的元素。
    \item 每次运算都会修改元素值。
\end{itemize}

可以将其看作循环运算的一种简便写法,不过用的人不多。

\subsubsection*{setAll 迭代计算}

和 parallelPrefix 类似,不过 parallelPrefix 会计算前后相关的元素,但 setAll 不会。还有一个并行方法: parallelSetAll

\begin{Java}
public static <T> void setAll(T[] array, IntFunction<? extends T> generator) {
    Objects.requireNonNull(generator);
    for (int i = 0; i < array.length; i++)
        array[i] = generator.apply(i);
}
\end{Java}

\subsubsection*{copyOf 复制数组}

有两个复制数组的函数如下:

\begin{Java}
public static int[] copyOf(int[] original, int newLength)
public static int[] copyOfRange(int[] original, int from, int to)
\end{Java}

用于复制一个新的数组,可以指定长度或范围。

copyOf 比较常用,源码如下:
\begin{Java}
@IntrinsicCandidate
public static <T,U> T[] copyOf(U[] original, int newLength, Class<? extends T[]> newType) {
    @SuppressWarnings("unchecked")
    T[] copy = ((Object)newType == (Object)Object[].class)
        ? (T[]) new Object[newLength]
        : (T[]) Array.newInstance(newType.getComponentType(), newLength);
    // 这是个 native 方法
    System.arraycopy(original, 0, copy, 0, Math.min(original.length, newLength));
    return copy;
}
\end{Java}


\subsubsection*{其他常用方法}

\begin{Java}
// 元素交换
private static void swap(Object[] x, int a, int b)
// 二分查找
public static int binarySearch(int[] a, int fromIndex, int toIndex,int key)
// 等于判断
public static boolean equals(int[] a, int[] a2)
// 比较判断
public static int compare(int[] a, int[] b)
// 找第一个不相等的元素
public static int mismatch(int[] a, int[] b)
// 填充:改变范围内元素值
public static void fill(int[] a, int fromIndex, int toIndex, int val)
// 计算 hashCode
public static int hashCode(int a[])
// toString
public static String toString(int[] a)
// 流处理,返回 Stream
public static <T> Stream<T> stream(T[] array, int startInclusive, int endExclusive)
// 转换为集合
public static <T> List<T> asList(T... a);
\end{Java}

并行迭代:在多线程中会应用,参考文章: \url{https://blog.csdn.net/sunboylife/article/details/103307072}

\begin{Java}
public static <T> Spliterator<T> spliterator(T[] array, int startInclusive, int endExclusive)
\end{Java}

\subsection{Collections 集合工具类}

Collections 不仅对 Collection 接口提供了方法,对 Map 也提供了一些处理方法。

\subsubsection*{功能性函数}

Collection 通用:

\begin{Java}
// 找最小值
public static <T extends Object & Comparable<? super T>> T min(Collection<? extends T> coll)
// 找最大值
public static <T extends Object & Comparable<? super T>> T max(Collection<? extends T> coll)
// 返回视图
public static <E> Collection<E> checkedCollection(Collection<E> c, Class<E> type)
// 获取对象出现次数
public static int frequency(Collection<?> c, Object o)
// 判断结合是否相交
public static boolean disjoint(Collection<?> c1, Collection<?> c2)
// 添加元素
public static <T> boolean addAll(Collection<? super T> c, T... elements)
\end{Java}

针对 List 类型,Arrays 有的方法几乎都有,此外还添加的函数有:

\begin{Java}
// 洗牌,打乱
public static void shuffle(List<?> list)
// 左移/右移
public static void rotate(List<?> list, int distance)
// 找子列表
public static int indexOfSubList(List<?> source, List<?> target)
// 找最后一个子列表
public static int lastIndexOfSubList(List<?> source, List<?> target)
\end{Java}

\subsubsection*{获取集合对象}

Collections 写了很多内置类,方便开发者使用。

\begin{Java}
// 获取不可变的集合对象
public static <T> Collection<T> unmodifiableCollection(Collection<? extends T> c)
// 获取线程安全的集合对象
public static <T> Collection<T> synchronizedCollection(Collection<T> c)
// 返回空且不可变对象
public static final <T> List<T> emptyList()
// 返回特定且不可变对象
public static <T> List<T> singletonList(T o)
// 转换为枚举
public static <T> Enumeration<T> enumeration(final Collection<T> c)
public static <T> ArrayList<T> list(Enumeration<T> e)
\end{Java}

\newpage