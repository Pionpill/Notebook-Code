\section{高精度运算}
\subsection{BigInteger}

\begin{Java}
public class BigInteger extends Number implements Comparable<BigInteger>
\end{Java}

BigInteger 有两个基础属性:

\begin{Java}
final int signum;     // 存储符号位 1:整  0:0 -1:负
final int[] mag;      // 存储数据
\end{Java}

注意这两个属性都是 final 修饰的,这决定了 \texttt{BigInteger} 是不可变对象。

总的来说,BigInteger 的构造函数有两类:
\begin{Java}
// 通过数组构造
public BigInteger(byte[] val, int off, int len)
public BigInteger(int signum, byte[] magnitude, int off, int len)   // 指定符号位
// 通过字符串构造
public BigInteger(String val, int radix)
\end{Java}

同时还提供了一个静态方法:
\begin{Java}
public static BigInteger valueOf(long val)
\end{Java}

首先看一下加法运算,首先通过 signum 进行了优化,如果符号位不同,则转换为减法运算再处理结果:
\begin{Java}
public BigInteger add(BigInteger val) {
    if (val.signum == 0)
        return this;
    if (signum == 0)
        return val;
    if (val.signum == signum)
        return new BigInteger(add(mag, val.mag), signum);

    int cmp = compareMagnitude(val);    // 忽略符号位进行比较
    if (cmp == 0)
        return ZERO;
    int[] resultMag = (cmp > 0 ? subtract(mag, val.mag)
                       : subtract(val.mag, mag));
    resultMag = trustedStripLeadingZeroInts(resultMag);
    return new BigInteger(resultMag, cmp == signum ? 1 : -1);
}
\end{Java}

其次是减法运算,运算逻辑类似:

\begin{Java}
public BigInteger subtract(BigInteger val) {
    if (val.signum == 0)
        return this;
    if (signum == 0)
        return val.negate();
    if (val.signum != signum)
        return new BigInteger(add(mag, val.mag), signum);

    int cmp = compareMagnitude(val);
    if (cmp == 0)
        return ZERO;
    int[] resultMag = (cmp > 0 ? subtract(mag, val.mag)
                       : subtract(val.mag, mag));
    resultMag = trustedStripLeadingZeroInts(resultMag);
    return new BigInteger(resultMag, cmp == signum ? 1 : -1);
}
\end{Java}

乘除法运算,在这部分的算法就比较复杂了,只给出扩展文献连接:
\begin{Java}
public BigInteger multiply(BigInteger val)
public BigInteger divide(BigInteger val)
\end{Java}

\begin{itemize}
    \item 乘法运算: \url{https://zhuanlan.zhihu.com/p/391716853}
\end{itemize}

类似的,还提供了一些数字处理的常用方法,没什么好讲的。

\subsection{BigDecimal }

\begin{Java}
public class BigDecimal extends Number implements Comparable<BigDecimal>
\end{Java}

BigDecimal 有几个关键属性:

\begin{Java}
private final BigInteger intVal;    // 实际存储的数据
private final int scale;            // 标度,表示小数位数
private transient int precision;    // 精度
\end{Java}

注意这里有个 transient 关键字,表示该属性不会被序列化,大概意思就是,该属性不会进入存储和传输过程(这两个过程需要序列化)。

在构造过程中,会将浮点数膨胀成为整数,并记录标度和精度。标度和精度有什么区别呢?标度是不会改变的,表示实际的小数点位数,辅助 intVal。而精度会在计算过程中改变,例如 1.01 + 1.001 实际上是 101(2) + 1001(3),这时候就需要改为 1010(3) + 1001(3),第一个数的精度由此改变了,精度只是一个辅助运算值,对于读取对象记录的大整数具体值没有影响,所以他被 transient 修饰,标度才是真爱。

BigDecimal 精度计算原理比较简单暴力,转换为整数运算再返回。例如 2.00001 会被转换为 BigInteger 的 200001 再计算。由于 BigDecimal 将实际存储值 intVal 与 scale 小数位数分开存储,所以这个过程非常高效,实际上并没有涉及到 BigDecimal 与 BigInteger 的转换。

BigDecimal 提供了两个比较相关函数: equals 和 compareTo:
\begin{itemize}
    \item \textbf{equals}: 比较属性,本质上比较 intVal, scale。
    \item \textbf{compareTo}: 比较值,会比较实际的 BigDecimal 含义值。
\end{itemize}

什么叫比较实际含义值,例如 1.0 与 1.00 进行比较:
\begin{itemize}
    \item equals: 两个值的 intVal 和 scale 分别是 (10,1) 和 (100,2) 不同,直接返回 false;
    \item compareTo: 比较标度 1,2 发现不同,进行标度对其操作: (10,1) $\rightarrow$ (100,2) 再进行比较,返回 true。
\end{itemize}

有一个常见的坑,不要使用 double, float 这样精度不确定的值构造 Decimal,应该优先使用 String,具体原因不解释了。

\newpage