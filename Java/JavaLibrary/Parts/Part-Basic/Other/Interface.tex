\section{Interface 接口}

\subsection{Appendable}

如果某个类的实例打算接收取自 java.util.Formatter 的格式化输出,那么该类必须实现 Appendable 接口。这个接口在 IO 相关的类中被经常使用。

\begin{Java}
public interface Appendable {
    Appendable append(CharSequence csq) throws IOException;
    Appendable append(CharSequence csq, int start, int end) throws IOException;
    Appendable append(char c) throws IOException;
}
\end{Java}

\subsection{AutoCloseable}

AutoCloseable 和 try-with-resource 语句相关,如果要在该语句中声明某个对象,该对象必须实现 AutoCloseable 接口,try-with-resource 语句块会在结束后自动调用 close 方法。

\begin{Java}
public interface AutoCloseable {
    void close() throws Exception;
}
\end{Java}

\subsection{CharSequence}

CharSequence 是用于统一字符串操作的,String, StringBuilder, StringBuffer 都实现了 CharSequence。该接口提供了很多方法,部分给予了默认实现:

\begin{Java}
public interface CharSequence {
    int length();
    char charAt(int index);
    default boolean isEmpty() {
        return this.length == 0;
    }
    @NotNull
    CharSequence subSequence(int start, int end);
    @NotNull
    public String toString();
    @Contract(pure=true) @NotNull
    public default IntStream chars() {...}
    @Contract(pure=true) @NotNull
    public default IntStream codePoints() {...}
    public static int compare(CharSequence cs1, CharSequence cs2) {...}
}
\end{Java}

\subsection{Cloneable}

Cloneable 接口是一个标记接口,只有实现这个接口后,然后在类中重写Object中的clone方法,然后通过类调用clone方法才能克隆成功,如果不实现这个接口,则会抛出CloneNotSupportedException(克隆不被支持)异常。

默认的 Object 的 clone 方法是浅拷贝,如果我们实现了 Cloneable 并重写 clone 方法,需要尽量实现深拷贝。

\begin{Java}
@IntrinsicCandidate
protected native Object clone() throws CloneNotSupportedException;
\end{Java}

\subsection{Comparable}

Comparable 是一个很常见的接口,用于实现比较:

\begin{Java}
public interface Comparable<T> {
    @Contract(pure = true)
    public int compareTo(T o);
}
\end{Java}

\subsection{Iterator}

Iterator 为迭代器对象,还有一个 Iterable 为可迭代对象,Iterable 有一个 iterator 方法返回了 Iterator 对象。

首先看一下 Iterator 接口:

\begin{Java}
public interface Iterator<E> {
    boolean hasNext();
    E next();
    default void remove() {
        throw new UnsupportedOperationException("remove");
    }
    default void forEachRemaining(Consumer<? super E> action) {
        Objects.requireNonNull(action);
        while (hasNext())
            action.accept(next());
    }
}
\end{Java}

前两个方法是 Iterator 的主要方法,用于迭代下一个对象,具体实现由子类负责,最后一个方法和 for-each 语句相关。

其次是 Iterable 接口:

\begin{Java}
public interface Iterable<T> {
    Iterable<T> iterator();
    default void forEach(Consumer<? super T> action) {
        Objects.requireNonNull(action);
        for (T t : this) {
            action.accept(t);
        }
    }
    default Spliterator<T> spliterator() {
        return Spliterators.spliteratorUnknownSize(iterator(), 0);
    }
}
\end{Java}

这里只有一个主要方法 iterator,用于返回一个 Iterable 对象。

在看完这两个接口后,我们看一下 for-each 语法糖的实际处理过程:

\begin{Java}
// 语法糖
for (Integer i:list) {
    System.out.println(i);
}
// 反编译
Integer i;
for (Iterator iterator = list.iterator(); i = (Integer)iterator.next();) {
    System.out.println(i);
}
\end{Java}

所以,为什么 Iterable 返回 Iterator 却不直接在 Iterable 内部实现 Iterator 方法? 原因是有些集合类可能不止一种遍历方式,实现了Iterable的类可以再实现多个Iterator内部类。

\subsection{Readable}

如果一个类继承了readable接口并实现了read方法,我们就可以使用scanner类来进行操作。

\begin{Java}
public interface Readable {
    public int read(java.nio.CharBuffer cb) throws IOException;
}
\end{Java}

\subsection{Runnable}

Runnable 是实现函数式接口,他有一个 run 方法,多用在线程中:

\begin{Java}
@FunctionalInterface
public interface Runnable {
    public abstract void run();
}
\end{Java}

\newpage