\section{Java 基本程序设计结构}
\subsection{Java 程序与注释}
\subsubsection{Java 程序}

一个 Java 源文件的文件名必须和此文件中的公有类\footnote{这也代表着一个 Java 程序中只能由一个类提供对外接口。}名相同,我们可以使用
\begin{Java}
java sample arg
\end{Java}
运行对应的java程序,注意不需要添加后缀名。其中 args 代表传入的参数。

\subsubsection{Java 注释}

Java 提供了三种注释方式:
\begin{itemize}
    \item 单行注释: //
    \item 界定注释: /* \quad */
    \item 多行注释: /** \quad */
\end{itemize}
在多行注释中可以通过 @ + 关键字 书写文档,比如 @author Pionpill 表示这段代码由 Pionpill 书写。要养成良好的注释习惯捏。

\subsection{数值型}
\subsubsection{Java 的一些性质}

不同于 C++,Java 数字的值不会因平台不同而有变动,也即与机器硬件无关。在 A 机器上存储的整型 X 在 B 机器上也一定会是 X,C++ 在这个过程中可能会将 X 变为某个不知道的值。

Java 的所有函数传参都是传值,而不是传引用。我在 Python 和 JavaScript 笔记中已经多次说明过这两者的区别,相信有经验的 coder 知道这点有多重要。

Java 中除了基本类型(数值,字符串,布尔值)不是对象,其他数据类型均为对象。

\subsubsection{数值的基本数据类型}

下面简单过一下 Java 的基础数据类型。

\noindent 整型:Java 提供了以下四种整型数:

\begin{table}[H]
    \centering
    \caption{Java 整型}
    \label{table:Java 整型}
    \setlength{\tabcolsep}{10mm}
    \begin{tabular}{c|c|cc}
        \toprule
        \textbf{类型} & \textbf{存储空间} & \textbf{取值范围} \\
        \midrule
        int & 4 byte & -20亿 —— 20 亿 \\
        short & 2 byte & -32768 —— 32767 \\
        long & 8 byte & $-9^{19} \text{——} 9^{19}$ \\
        byte & 1 byte & -128 —— 127 \\
        \bottomrule
    \end{tabular}
\end{table}

有以下注意点:
\begin{itemize}
    \item Java 没有 unsigned 形式的整型数。
    \item 可以给数字字面量加下划线,编译器会自动去除,这只是为了添加可读性。
    \item 可以通过在字面量后加 l/L 指定长整型。
\end{itemize}

\noindent 浮点数:Java 提供了两种浮点类型:

\begin{table}[H]
    \centering
    \caption{浮点类型}
    \label{table:浮点类型}
    \setlength{\tabcolsep}{12mm}
    \begin{tabular}{c|c|cc}
        \toprule
        \textbf{类型} & \textbf{存储空间} & \textbf{有效位} \\
        \midrule
        float & 4 byte & 6-7 位 \\
        double & 8 byte & 15 位 \\
        \bottomrule
    \end{tabular}
\end{table}

有以下注意点:
\begin{itemize}
    \item 三个特殊浮点数值: 正无穷大,负无穷大,NAN。
    \item 可以通过在字面量后加 f/F d/D 指定浮点数类型。
\end{itemize}

\noindent 布尔型(boolean)

注意Java 中的 boolean 型不是整型就行了\footnote{Python 中布尔值就是 0/1。}。

\subsubsection{数值转换}

\begin{figure}[H]
    \centering
    \begin{tikzpicture}[font=\small]
        \node [draw] (byte) at (0,0) {byte};
        \node [draw] (short) at (3,0) {short};
        \node [draw] (int) at (6,0) {int};
        \node [draw] (long) at (9,0) {long};
        \node [draw] (char) at (6,2) {char};
        \node [draw] (float) at (6,-2) {float};
        \node [draw] (double) at (9,-2) {double};
        \draw [-Stealth] (byte) -- (short); 
        \draw [-Stealth] (short) -- (int); 
        \draw [-Stealth] (char) -- (int); 
        \draw [-Stealth] (int) -- (long); 
        \draw [-Stealth] (int) -- (double);
        \draw [-Stealth] (float) -- (double);
        \draw [-Stealth,dashed] (int) -- (float);
        \draw [-Stealth,dashed] (long) -- (double);
        \draw [-Stealth,dashed] (long) -- (float);
    \end{tikzpicture}
    \caption{数值类型之间的合法转换}
    \label{fig:数值类型之间的合法转换}
\end{figure}

上图所示,实线代表无信息丢失,虚线代表可能有信息丢失。此外,在进行数值计算时会自动向精度更高的数值类型转换。

\subsubsection{大数}

在 \texttt{java.math} 包中包含两个大数类: BigInteger 和 BigDecimal。这两个类可以处理包含任意长度数字序列的数值。

\subsection{字符串}

字符串基本操作和其他语言一样,下面讲一下易混淆的 equals 和 == 比较运算。

\subsubsection{== 比较}

对于基本数据类型,== 比较值是否相同(因为基本数据类型没有引用捏)。

对于引用数据类型,== 比较引用的地址是否相同。

\subsubsection{equals()}

Object 中的 \texttt{equals} 方法是这样写的:
\begin{Java}
public boolean equals(Object obj) {
    return (this == obj);
}
\end{Java}

也即和 == 相同,对引用数据类型比较地址是否相同。但绝大多数类都会重写 \texttt{equals} 方法,在 string 中就这样重写了 \texttt{equals} 方法:

\begin{Java}
public boolean equals(Object anObject) {
    if (this == anObject) {
        return true;
    }
    if (anObject instanceof String) {
        String anotherString = (String)anObject;
        int n = value.length;
        if (n == anotherString.value.length) {
            char v1[] = value;
            char v2[] = anotherString.value;
            int i = 0;
            while (n-- != 0) {
                if (v1[i] != v2[i])
                    return false;
                i++;
            }
            return true;
        }
    }
    return false;
}
\end{Java}

可以理解为: 比较内容是否相同。

\subsection{输入与输出}

\subsubsection{标准输入}

利用 Java 进行标准输出非常简单,调用 \texttt{System.out.print()} 即可,而构建标准输入却显得不那么简单。想要通过控制台进行输入,往往需要执行以下几个步骤:

\begin{itemize}
    \item 构建 \texttt{Scanner} 对象:
\begin{Java}
Scanner in = new Scanner(System.in);
\end{Java}

    \item 调用对应的方法:
    
    Scanner 对象针对不同的数据类型设计了多种方法,例如最简单的 \texttt{nextLine()} 读取一行输入, \texttt{next()} 读取一个单词,\texttt{nextInt()} 读取一个整数。
\end{itemize}

同时针对读取密码,Java 还设计了 \texttt{Console} 类,但它不如 \texttt{Scanner} 那么好用。

\subsubsection{文件的输入输出}

文件的读取同样需要构建 Scanner 对象:

\begin{Java}
Scanner in = new Scanner(Path.of("myfile.txt"),StandardCharsets.UTF_8);
\end{Java}

文件写入则需要构造 PrintWriter 对象。

\begin{Java}
PrintWriter out = new PrintWriter("myfile.txt",StandardCharsets.UTF_8);
\end{Java}

在这个过程中可能会因为文件不存在或无法在指定路径创建文件等 IO 错误。

\subsection{数组}
\subsubsection{数组基础}

数组与字符串一样是不可变数据类型。其基础的创建语句如下(其中数字代表数组长度):

\begin{Java}
int[] a = new int[100];
\end{Java}

Java 提供了一种创建数组的简写形式:

\begin{Java}
int[] a = {1,2,3,4,5};
\end{Java}

也可以通过这种方式创建匿名数组:

\begin{Java}
new int[] {1,2,3,4,5};
\end{Java}

多维数组的申明也十分简单:
\begin{Java}
double[][] a = new double[m][n];
\end{Java}

\subsubsection{for each 循环}

Java 提供了一种遍历可迭代对象的简单语法:
\begin{Java}
for(int element:a)
    System.out.println(element);
\end{Java}

这和传统的 \texttt{for} 循环结合下标达到的效果是相同的。

\subsubsection{数组拷贝}

由于 Java 的所有参数传递都是值传递,直接使用下面这种形式拷贝数组仅是一种浅拷贝。

\begin{Java}
int[] a = {1,2,3};
int[] b = a;
\end{Java}

如果想要深拷贝数组,则需要使用到 \texttt{Arrays} 类的 \texttt{copyOf} 方法。

\begin{Java}
int[] b = Arrays.copyOf(luckyNumbers, 2*luckyNumbers.length);
\end{Java}

其中,第二个参数是数组长度。

\newpage