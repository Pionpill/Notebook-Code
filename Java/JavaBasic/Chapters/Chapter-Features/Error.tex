\section{异常,断言和日志}

\subsection{错误处理}

\subsubsection{异常分类}

在 Java 程序设计语言中,异常对象都是派生于 \texttt{Throwable} 类的一个实例对象。

\begin{figure}[H]
    \scriptsize
    \centering
    \begin{tikzpicture}[scale = 1]
        \begin{class}[text width=2cm]{Throwable}{0,0}
        \end{class}
        \begin{class}[text width=2cm]{Error}{-2,-2}
            \inherit{Throwable}
        \end{class}
        \begin{class}[text width=2cm]{Exception}{2,-2}
            \inherit{Throwable}
        \end{class}
        \begin{class}[text width=2cm]{IOException}{0,-4}
            \inherit{Exception}
        \end{class}
        \begin{class}[text width=2.2cm]{RuntimeException}{4,-4}
            \inherit{Exception}
        \end{class}
    \end{tikzpicture}
    \caption{Java 中的异常层次结构}
    \label{fig:Java 中的异常层次结构}
\end{figure}

\texttt{Exception} 派生的两个分支需要重点关注,一般的规则是: 由编程错误导致的异常属于 \texttt{RuntimeException};如果程序本省没问题,但由于 I/O 错误这类问题导致的异常属于其他异常。

Java 语言规范将派生于 \texttt{Error} 类或 \texttt{RuntimeException} 类的所有异常称为\textit{非检查型}(unchecked)异常,所有其他异常称为\textit{检查型}(checked)异常。编译器将检查你是否为所有的检查型异常提供了异常处理器。

\subsubsection{声明检查型异常}

如果遇到了无法处理的情况,Java 方法可以抛出一个异常。方法不仅需要告诉编译器将要返回什么值,还要告诉编译器可能发生什么错误。比如在读取文件时,可能文件不存在,或者文件内容为空。

要在方法的首部指出这个方法可能抛出一个异常:
\begin{Java}
public FileInputStream(String name) throws FileNotFoundException
\end{Java}

如果发生了糟糕的情况,构造器将不会重新初始化一个对象,而是抛出对应的错误,并搜索如何处理错误。

不需要声明 Java 的内部错误,即从 Error 继承的异常。任何程序代码都有可能抛出那些异常,比如数组下标问题。

\subsubsection{创建异常类}

如果有必要,我们可以定义一个派生自 \texttt{Exception} 或其子类的异常类。自定义的这个类应该包含两个构造器,一个默认构造器,一个包含详细描述信息的构造器(超类 \texttt{Throwable} 的 \texttt{toString} 方法会返回一个字符串,其中包含这个详细信息。)。

\begin{Java}
class FileFormatException extends IOException {
    public FileFormatException() {}
    public FileFormatException(String gripe) {
        super(gripe);
    }
}
\end{Java}

\subsection{捕获异常}
\subsubsection{捕获异常}

如果发生了某个异常,但没有在任何地方捕获这个异常,程序就会终止。想要捕获一个异常,就需要 \texttt{try/catch} 语句。异常语句很好理解,不再赘述,下面给出基本语法:

\begin{Java}
try {
    code
}
catch (FileNotFoundException | UnknownHostException e) {
    XXX
}
catch (IOException e) {
    XXX
}
finally {
    XXX
}
\end{Java}

介绍一个特殊的方法 \texttt{initCause} 这个方法是对异常来进行包装的。目的就是为了出了问题的时候追根究底。举个例子,在底层中有一个异常 A 被捕获到后进行了处理,这导致上层抛出了异常 B,虽然此时捕获到了 B 异常,但是只能说明 B 出现了异常,而不知道是什么导致了 B 异常。

如果此时我们使用 \texttt{initCause} 方法对异常进行包装,我们就可以通过 \texttt{getCause} 方法获得原始的 A 异常。

\begin{Java}
class A {
    try {
        ...
    } catch(AException a) {
        BException b = new BEexception();
        b.initCause(a);
        throw b;
    }
}
...
class B {
    try {
        ...
    } catch(BException b) {
        b.getCause(); //得到导致B异常的原始异常
    }
}
\end{Java}

\subsubsection{\texttt{try-with-Resources} 语句}

\texttt{try-with-Resources} 语句最简单的形式为:

\begin{Java}
try (Resource res = ...) {
    work with res
}
\end{Java}

\texttt{try} 块退出时,会自动调用 \texttt{res.close()}。下面给出一个典型的例子:

\begin{Java}
try (var in = new Scanner (
    new FileInputStream("/usr/share/dict/words"), StandardCharsets.UTF_8))
{
    while (in.hasNext())
        System.out.println(in.next());
}
\end{Java}

这个块正常退出时,或者存在异常时,都会自动调用 \texttt{in.close()} 方法,就好像使用了 \texttt{finally} 块一样。

\subsubsection{处理异常的技巧}

\begin{enumerate}
    \item 异常不能代替测试: 捕获并抛出异常消耗的资源远高于简单的测试。
    \item 不要将语句分装在多个 try 语句块中,这会导致代码块激增。
    \item 充分利用异常层次结构,派生合适的异常子类。
    \item 不要压制异常,即使异常出现的概率很低。
    \item 检查错误时,``苛刻''要比放任更好。比起返回 null,直接报错往往更好。
    \item 不要羞于传递异常,如果不是必要,传递比自己捕获往往更合适。
\end{enumerate}

\subsection{使用断言}

\subsubsection{断言的概念}

断言机制允许在\textit{代码测试}期间向代码中插入一些检查,而在生产代码中会\textit{自动删除}这些检查。Java 语言引入了关键字 assert。

\begin{Java}
assert condition;
assert condition:expression;
\end{Java}

其中,表达式将传给 \texttt{AssertionError} 对象,以便以后显示。

需要重申的是,断言仅在测试阶段起作用。而且断言往往是针对致命性错误。因此在程序设计中不可以过于依赖断言。

\subsubsection{启用很禁用断言}

默认情况下,断言是禁用的。可以在运行程序时,使用 \texttt{-enableassertions -ea} 选择启用断言。

也可以指定某个类或某个包启用断言:
\begin{Java}
java -ea:MyClass -ea:com.company.lib MyApp
\end{Java}

也可以使用 \texttt{-disableassertions -da} 选择禁用断言。

\subsection{日志}

最简单的 debug 方式莫过于使用 \texttt{System.out.print} 观察程序行为,但是在我们解决完问题后需要删除这些语句,如果出问题了,还需要新增语句。日志 API 可以很好的解决这个繁琐的过程,该 API 优点如下:

\begin{itemize}
    \item 可以很容易取消全部日志记录,或者仅仅取消某个级别以下的日志,而且很容易打开日志。
    \item 日志的代码开销很小,灵活度很高。
    \item 日志记录可以被定向到控制台,文件等等。
    \item 日志系统有单独的控制器,可以被格式化。
\end{itemize}

\subsubsection{使用日志}

简单的日志记录,可以使用全局日志记录器并调用其 \texttt{info} 方法:

\begin{Java}
Logger.getGlobal().info("File->Open menu item selected");
\end{Java}

在合适的地方可以通过调用如下函数取消所有日志:
\begin{Java}
Logger.getGlobal().setLevel(Level.OFF);
\end{Java}

可以调用 \texttt{getLogger} 方法创建或获取日志记录器:

\begin{Java}
private static final Logger myLogger = Logger.getLogger("com.mycompany.myapp");
\end{Java}

\fbox{
    \parbox{0.87\textwidth}{
        \begin{notice}
            未被任何变量引用的日志记录器可能会被垃圾回收。为了防止这种情况发生,要像上面的例子中一样,用静态变量存储日志记录器的引用。
        \end{notice}
    }
}

通常有 7 个日志级别: \texttt{SERVER, WARNING, INFO, CONFIG, FINE, FINER, FINEST}

默认情况下,实际只记录前三个级别,也可以设置一个不同的级别:

\begin{Java}
logger.setLevel(level.FINE);
\end{Java}

这样,FINE 以及所有更高级别的日志都会被记录。此外还可以通过 \texttt{LEVEL.ALL LEVEL.OFF} 开关所有级别的日志记录。或者指定级别:

\begin{Java}
logger.log(level.FINE, message);
\end{Java}

记录日志的最常见用途是记录那些预料之外的异常。可以使用下面两个便利方法在日志记录中包含异常的描述:

\begin{Java}
void throwing(String className, String methodName, Throwable t);
void log(Level l, String message, Throwable t);
\end{Java}

Java 的日志还包含过滤器,处理器等等功能。本人觉得脱了项目谈这些过于抽象,这里省略,请读者自行查阅资料。

\newpage