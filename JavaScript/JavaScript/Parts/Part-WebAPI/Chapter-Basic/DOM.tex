\section{DOM}

文档对象模型(DOM,Document  Object  Model)是HTML和XML文档的编程接口。DOM表示由多层节点构成的文档,通过它开发者可以添加、删除和修改页面的各个部分。

本章与下一章将说明 DOM 最基础也是核心的用法,随着 DOM 的发展,新增了更多 DOM 相关的 API,不过掌握了核心用法,熟悉其他 API 也只是时间问题。

\subsection{节点层级}

任何HTML或XML文档都可以用DOM表示为一个由节点构成的层级结构。节点分很多类型,每种类型对应着文档中不同的信息和(或)标记,也都有自己不同的特性、数据和方法,而且与其他类型有某种关系。这些关系构成了层级,让标记可以表示为一个以特定节点为根的树形结构。

\texttt{document} 节点表示每个文档的根节点。一般的 HTML 界面中 \texttt{document} 节点的唯一子节点是 \texttt{<html>} 元素,也叫文档元素。。文档元素是文档最外层的元素,所有其他元素都存在于这个元素之内。每个文档只能有一个文档元素。

HTML中的每段标记都可以表示为这个树形结构中的一个节点。元素节点表示HTML元素,属性节点表示属性,文档类型节点表示文档类型,注释节点表示注释。DOM中总共有12种节点类型,这些类型都继承一种基本类型。

\subsubsection{\texttt{Node} 类型}

\texttt{Node} 接口时所有 DOM 节点类型都必须实现的。在JavaScript中,所有节点类型都继承\texttt{Node}类型,因此所有类型都共享相同的基本属性和方法。

每个节点都有\texttt{nodeType}属性,表示该节点的类型。节点类型由定义在\texttt{Node}类型上的12个数值常量表示:

\begin{itemize}
    \item \texttt{Node.ELEMENT\_NODE}: 1
    \item \texttt{Node.ATTRIBUTE\_NODE}: 2
    \item \texttt{Node.TEXT\_NODE}: 3
    \item \texttt{Node.CDATA\_SECTION\_NODE}: 4
    \item \texttt{Node.ENTITY\_REFERENCE\_NODE}: 5
    \item \texttt{Node.ENTITY\_NODE}: 6
    \item \texttt{Node.PROCESSING\_INSTRUCTION\_NODE}: 7
    \item \texttt{Node.COMMENT\_NODE}: 8
    \item \texttt{Node.DOCUMENT\_NODE}: 9
    \item \texttt{Node.DOCUMENT\_TYPE\_NODE}: 10
    \item \texttt{Node.DOCUMENT\_FRAGMENT\_NODE}: 11
    \item \texttt{Node.NOTATION\_NODE}: 12
\end{itemize}

并不是所有的浏览器都支持全部的节点,常用的节点类型时元素节点和文本节点。

此外,节点还有两个属性: \texttt{nodeName} 和 \texttt{nodeValue}。这两个属性的值完全取决于节点本身。

每一个节点都有一个 \texttt{childNodes} 属性,包含一个 \texttt{NodeList} 实例。\texttt{NodeList}  是一个类数组对象,用于存储可以按位置存取的有序节点。\texttt{NodeList} 的很多操作都类似于 \texttt{Array},比如可以通过中括号取值,但 \texttt{NodeList} 并不是 \texttt{Array}。

\texttt{NodeList}对象独特的地方在于,它其实是一个对DOM结构的查询,因此DOM结构的变化会自动地在\texttt{NodeList}中反映出来。我们通常说\texttt{NodeList}是实时的活动对象,而不是第一次访问时所获得内容的快照。

每个节点都有一个\texttt{parentNode}属性,指向其DOM树中的父元素。此外,\texttt{childNodes}列表中的每个节点都是同一列表中其他节点的同胞节点。而使用\texttt{previousSibling}和\texttt{nextSib ling}可以在这个列表的节点间导航。这个列表中第一个节点的\texttt{previousSibling}属性是\texttt{null},最后一个节点的\texttt{nextSibling}属性也是\texttt{null}。

父节点和它的第一个及最后一个子节点也有专门属性:\texttt{firstChild}和\texttt{lastChild}分别指向\texttt{childNodes}中的第一个和最后一个子节点。\texttt{firstChild}的值始终等于\texttt{childNodes[0]}。

最后还有一个所有节点都共享的关系。\texttt{ownerDocument}属性是一个指向代表整个文档的文档节点的指针。

\subsubsection*{操纵节点}

因为所有关系指针都是只读的,所以DOM又提供了一些操纵节点的方法。

最常见的是 \texttt{appendChild()} 用于在 \texttt{childNodes} 列表末尾添加节点。添加新节点会更新相关的关系指针,\texttt{appendChild()}方法返回新添加的节点。

如果把文档中已经存在的节点传给\texttt{appendChild()},则这个节点会从之前的位置被转移到新位置。即使DOM树通过各种关系指针维系,一个节点也不会在文档中同时出现在两个或更多个地方。因此,如果调用\texttt{appendChild()}传入父元素的第一个子节点,则这个节点会成为父元素的最后一个子节点,可以这样改变 DOM 树的结构。

类似的,还有 \texttt{insertBefore()} 方法,接收两个参数:要插入的节点和参照节点。调用这个方法后,要插入的节点会变成参照节点的前一个同胞节点,参照节点可以为 \texttt{null} 表示插入为最后一个节点。

前面两个方法都不会删除任何节点,\texttt{replaceChild()} 方法接受两个参数: 要插入的节点和要替换的节点。要替换的节点会被返回并从文档树中完全移除,要插入的节点会取而代之。

要移除节点而不是替换节点,可以使用\texttt{removeChild()}方法。这个方法接收一个参数,即要移除的节点。被移除的节点会被返回.

此外还有两个所有节点共享的方法:
\begin{itemize}
    \item \texttt{cloneNode()}: 会返回与调用它的节点一模一样的节点。有一个布尔值参数,如果为 \texttt{true} 则返回深复制,否则返回节点本身。复制返回的节点属于文档所有,但尚未指定父节点,所以可称为孤儿节点(orphan)。
    \item \texttt{normalize()}: 处理文档子树中的文本节点。由于解析器实现的差异或DOM操作等原因,可能会出现并不包含文本的文本节点,或者文本节点之间互为同胞关系。在节点上调用\texttt{normalize()}方法会检测这个节点的所有后代,从中搜索上述两种情形。如果发现空文本节点,则将其删除;如果两个同胞节点是相邻的,则将其合并为一个文本节点。
\end{itemize}

\subsubsection{\texttt{Document} 类型}

\texttt{Document} 类型是JavaScript中表示文档节点的类型。在浏览器中,文档对象\texttt{document}是\texttt{HTMLDocument}的实例(\texttt{HTMLDocument}继承\texttt{Document}),表示整个HTML页面。\texttt{document}是\texttt{window}对象的属性,因此是一个全局对象。\texttt{Document}类型的节点有以下特征:

\begin{itemize}
    \item \texttt{nodeType} 为 \texttt{Node.DOCUMENT\_NODE};
    \item \texttt{nodeName} 值为 \texttt{``\#document''};
    \item \texttt{nodeValue} 值为 \texttt{null};
    \item \texttt{parentNode} 值为 \texttt{null};
    \item \texttt{ownerDocument} 值为 \texttt{null};
    \item 子节点可以是\texttt{DocumentType}(最多一个)、\texttt{Elemen}t(最多一个)、\texttt{ProcessingInstruction}或\texttt{Comment}类型。
\end{itemize}

DOM 为 \texttt{Document} 提供了两个访问子节点的快捷方式。第一个是\texttt{documentElement}属性,始终指向HTML页面中的\texttt{<html>}元素。第二个是 \texttt{body}属性,直接指向\texttt{<body>}元素。因为这个元素是开发者使用最多的元素。\texttt{Document}类型另一种可能的子节点是\texttt{DocumentType}。\texttt{<!doctype>}标签是文档中独立的部分。

\subsubsection*{文档信息}

\texttt{document}作为\texttt{HTMLDocument}的实例,还有一些标准\texttt{Document}对象上所没有的属性。这些属性提供浏览器所加载网页的信息。

\begin{itemize}
    \item \texttt{title}: 获取 \texttt{<title>} 元素中的文本,可以读写,但是修改 \texttt{title} 属性并不会改变 \texttt{<title>} 元素。
    \item \texttt{URL}: 当前页面的完整URL。
    \item \texttt{domain}: 包含页面的域名。
    \item \texttt{reference}: 包含链接到当前页面的那个页面的URL。
\end{itemize}

其中 \texttt{domain} 是可以设置的,当页面中包含来自某个不同子域的窗格(\texttt{<frame>})或内嵌窗格(\texttt{<iframe>})时,设置\texttt{document.domain}是有用的。因为跨源通信存在安全隐患,所以不同子域的页面间无法通过JavaScript通信。此时,在每个页面上把\texttt{document.domain}设置为相同的值,这些页面就可以访问对方的JavaScript对象了。

\texttt{domain} 属性的设置是有限制的,只能基于当前域设置,且放松后不能再收紧:

\begin{JavaScript}
// 页面来自p2p.wrox.com 
document.domain = "wrox.com";     // 放松,成功
document.domain = "p2p.wrox.com"; // 收紧,错误!
\end{JavaScript}

\subsubsection*{定位元素}

使用DOM最常见的情形可能就是获取某个或某组元素的引用,然后对它们执行某些操作。为此 \texttt{Document} 提供了三个方法:
\begin{itemize}
    \item \texttt{getElementById()}:  接收一个参数,要获取元素的ID, ID 区分大小写,多个则只返回第一个; 如果找到了则返回这个元素,如果没找到则返回\texttt{null}。
    \item \texttt{getElementsByTagName()}: 接收一个参数,即要获取元素的标签名,返回包含零个或多个元素的\texttt{NodeList},标签名可以是 ``*'' 表示所有。在HTML文档中,这个方法返回一个\texttt{HTMLCollection}对象。
    \item \texttt{getElementsByName()}: 这个方法会返回具有给定\texttt{name}属性的所有元素,这个方法返回一个\texttt{HTMLCollection}对象。\texttt{getElementsByName()}方法最常用于单选按钮,因为同一字段的单选按钮必须具有相同的\texttt{name}属性才能确保把正确的值发送给服务器。
\end{itemize}

\texttt{HTMLCollection}对象还有一个额外的方法\texttt{namedItem()},可通过标签的\texttt{name}属性取得某一项的引用,当然也可以通过中括号获取。

\begin{JavaScript}
<img src="myimage.gif" name="myImage"> 
let myImage = images.namedItem("myImage"); 
let myImage = images["myImage"]; 
\end{JavaScript}

\subsubsection*{特殊集合}

\texttt{document}对象上还暴露了几个特殊集合,这些集合也都是\texttt{HTMLCollection}的实例。这些集合是访问文档中公共部分的快捷方式,列举如下:

\begin{itemize}
    \item \texttt{document.anchors}: 包含文档中所有带\texttt{name}属性的\texttt{<a>}元素。
    \item \texttt{document.applets}: 包含文档中所有\texttt{<applet>}元素(因为\texttt{<applet>}元素已经不建议使用,所以这个集合已经废弃)。
    \item \texttt{document.forms}: 包含文档中所有\texttt{<form>}元素。
    \item \texttt{document.images}: 包含文档中所有\texttt{<img>}元素(。
    \item \texttt{document.links}: 包含文档中所有带\texttt{href}属性的\texttt{<a>}元素。
\end{itemize}

\subsubsection{\texttt{Element} 类型}

除了\texttt{Document}类型,\texttt{Element}类型就是Web开发中最常用的类型了。\texttt{Element}表示XML或HTML元素,对外暴露出访问元素标签名、子节点和属性的能力。\texttt{Element}类型的节点具有以下特征:

\begin{itemize}
    \item \texttt{nodeType} 为 \texttt{Node.ELEMENT\_NODE};
    \item \texttt{nodeName} 值为元素的标签名;
    \item \texttt{nodeValue} 值为 \texttt{null};
    \item \texttt{parentNode} 值为 \texttt{Document} 和 \texttt{Element} 对象;
    \item 子节点可以是\texttt{Element、Text、Comment、ProcessingInstruction、CDATASection、EntityReference}类型。
\end{itemize}

可以通过\texttt{nodeName}或\texttt{tagName}属性来获取元素的标签名,这两个属性返回同样的值。在 HTML 中,返回的值是标签名的大写,如 \texttt{DIV},而 XML 中则返回相同的原标签名。

HTML 的节点包含的诸多属性如 \texttt{class, title, id} 等都可以用同名的 \texttt{Element} 变量获得并修改,同时也提供了以下方法来操作属性:
\begin{itemize}
    \item \texttt{getAttribute()}: 通过属性名获取属性,属性名不区分大小写。
    \item \texttt{setAttribute()}: 设置属性值。
    \item \texttt{removeAttribute()}: 删除属性,会直接移除属性,用的很少。
\end{itemize}

\texttt{Element}类型是唯一使用\texttt{attributes}属性的DOM节点类型。\texttt{attributes}属性包含一个\texttt{NamedNodeMap}实例,是一个类似\texttt{NodeList}的“实时”集合。元素的每个属性都表示为一个\texttt{Attr}节点,并保存在这个\texttt{NamedNodeMap}对象中。\texttt{NamedNodeMap}对象包含下列方法:
\begin{itemize}
    \item \texttt{getNamedItem(name)}: 返回\texttt{nodeName}属性等于\texttt{name}的节点;
    \item \texttt{removeNamedItem(name)},删除\texttt{nodeName}属性等于\texttt{name}的节点;
    \item \texttt{setNamedItem(node)},向列表中添加\texttt{node}节点,以其\texttt{nodeName}为索引;
    \item \texttt{item(pos)},返回索引位置\texttt{pos}处的节点。
\end{itemize}

\subsubsection*{创建元素}

可以使用 \texttt{document.createElement()}  方法创建新元素。这个方法接收一个擦拭农户,即要创建元素的标签名。使用\texttt{createElement()}方法创建新元素的同时也会将其\texttt{ownerDocument}属性设置为\texttt{document}。此时,可以再为其添加属性、添加更多子元素:

\begin{JavaScript}
let div = document.createElement("div"); 
div.id = "myNewDiv"; 
div.className = "box"; 
\end{JavaScript}

在新元素上设置这些属性只会附加信息。因为这个元素还没有添加到文档树,所以不会影响浏览器显示。

元素可以拥有任意多个子元素和后代元素,因为元素本身也可以是其他元素的子元素。childNodes属性包含元素所有的子节点,这些子节点可能是其他元素、文本节点、注释或处理指令。不同浏览器在识别这些节点时的表现有明显不同。比如下面的代码:

\begin{HTML}
<ul id="myList">    
    <li>Item 1</li>   
    <li>Item 2</li>   
    <li>Item 3</li> 
</ul> 
\end{HTML}

在解析以上代码时,\texttt{<ul>}元素会包含7个子元素,其中3个是\texttt{<li>}元素,还有4个\texttt{Text}节点(表示\texttt{<li>}元素周围的空格)。

\subsubsection{\texttt{Text} 类型}

\texttt{Text}节点由\texttt{Text}类型表示,包含按字面解释的纯文本,也可能包含转义后的HTML字符,但不含HTML代码。\texttt{Text}类型的节点具有以下特征:

\begin{itemize}
    \item \texttt{nodeType} 为 \texttt{Node.Text\_NODE};
    \item \texttt{nodeName} 值为 \texttt{``\#text''};
    \item \texttt{nodeValue} 值为节点中包含的文本;
    \item \texttt{parentNode} 值为 \texttt{Element} 对象;
    \item 不支持子节点。
\end{itemize}

\texttt{Text}节点中包含的文本可以通过\texttt{nodeValue}属性访问,也可以通过\texttt{data}属性访问,这两个属性包含相同的值。修改\texttt{nodeValue}或\texttt{data}的值,也会在另一个属性反映出来。文本节点暴露了以下操作文本的方法:
\begin{itemize}
    \item \texttt{appendData(text)}: 向节点末尾添加文本 \texttt{text}。
    \item \texttt{deleteData(offset,count)}: 从位置 \texttt{offset} 开始删除 \texttt{count} 个字符。
    \item \texttt{insertData(offset,text)}: 从位置 \texttt{offset} 插入 \texttt{text}。
    \item \texttt{replaceData(offset,count,text)}: 用\texttt{text}替换从位置\texttt{offset}到\texttt{offset+count}的文本;
    \item \texttt{splitText(offset)}:  在位置\texttt{offset}将当前文本节点拆分为两个文本节点。
    \item \texttt{substringData(offset,count)}: 提取从位置\texttt{offset}到\texttt{offset + count}的文本。
\end{itemize}

默认情况下,包含文本内容的每个元素最多只能有一个文本节点。只要开始标签和结束标签之间有内容,就会创建一个文本节点。

\begin{HTML}
<div>Hello World!</div>
\end{HTML}

下列代码可以用来访问并修改这个文本节点:

\begin{JavaScript}
let textNode = div.firstChild;
div.firstChild.nodeValue = "Some other message"; 
\end{JavaScript}

修改文本节点有一点要注意,就是HTML或XML代码(取决于文档类型)会被转换成实体编码,即小于号、大于号或引号会被转义,如下所示:

\begin{JavaScript}
// 输出为"Some &lt;strong&gt;other&lt;/strong&gt; message" 
div.firstChild.nodeValue = "Some <strong>other</strong> message"; 
\end{JavaScript}

\subsubsection*{操作文本}

\texttt{document.createTextNode()}可以用来创建新文本节点,它接收一个参数,即要插入节点的文本。创建新文本节点后,其\texttt{ownerDocument}属性会被设置为\texttt{document}。在把这个节点添加到文档树之前,我们不会在浏览器中看到它。

一般来说一个元素只包含一个文本子节点,也可以添加多个文本子节点,不过这样做没有实际意义。在将一个文本节点作为另一个文本节点的同胞插入后,两个文本节点的文本之间不会包含空格。

DOM文档中的同胞文本节点可能导致困惑,因为一个文本节点足以表示一个文本字符串。同样,DOM文档中也经常会出现两个相邻文本节点。为此,有一个方法可以合并相邻的文本节点: \texttt{normalize()}。是在\texttt{Node}类型中定义的(因此所有类型的节点上都有这个方法)。在包含两个或多个相邻文本节点的父节点上调用\texttt{normalize()}时,所有同胞文本节点会被合并为一个文本节点,这个文本节点的\texttt{nodeValue}就等于之前所有同胞节点\texttt{nodeValue}拼接在一起得到的字符串。

\texttt{Text}类型定义了一个与\texttt{normalize()}相反的方法:\texttt{splitText()}。这个方法可以在指定的偏移位置拆分\texttt{nodeValue},将一个文本节点拆分成两个文本节点。拆分之后,原来的文本节点包含开头到偏移位置前的文本,新文本节点包含剩下的文本。这个方法返回新的文本节点,具有与原来的文本节点相同的parentNode。

\subsubsection{其他节点类型}

\subsubsection*{\texttt{Comment} 类型}

DOM中的注释通过\texttt{Comment}类型表示。\texttt{Comment}类型的节点具有以下特征:

\begin{itemize}
    \item \texttt{nodeType} 为 \texttt{Node.Comment\_NODE};
    \item \texttt{nodeName} 值为 \texttt{``\#comment''};
    \item \texttt{nodeValue} 值为节注释的内容;
    \item \texttt{parentNode} 值为 \texttt{Document, Element} 对象;
    \item 不支持子节点。
\end{itemize}

\texttt{Comment}类型与\texttt{Text}类型继承同一个基类(\texttt{CharacterData}),因此拥有除\texttt{splitText()}之外\texttt{Text}节点所有的字符串操作方法。

注释节点可以作为父节点的子节点来访问。比如下面的HTML代码:

\begin{HTML}
<div id="myDiv"><!-- A comment --></div>
\end{HTML}

这里的注释是\texttt{<div>}元素的子节点,这意味着可以像下面这样访问它:

\begin{JavaScript}
let div = document.getElementById("myDiv"); 
let comment = div.firstChild; 
alert(comment.data); // "A comment" 
\end{JavaScript}

可以使用\texttt{document.createComment()}方法创建注释节点,参数为注释文本。

\begin{JavaScript}
let comment = document.createComment("A comment"); 
\end{JavaScript}

注释节点很少使用,浏览器不承认 \texttt{</html>} 标签之后的注释。

\subsubsection*{\texttt{CDATASection} 类型}

\texttt{CDATASection}类型表示XML中特有的\texttt{CDATA}区块。\texttt{CDATASection}类型继承\texttt{Text}类型,因此拥有包括\texttt{splitText()}在内的所有字符串操作方法。

\begin{itemize}
    \item \texttt{nodeType} 为 \texttt{Node.CDATA\_SECTION\_NODE};
    \item \texttt{nodeName} 值为 \texttt{``\#cdata-section''};
    \item \texttt{nodeValue} 值为CDATA区块的内容;
    \item \texttt{parentNode} 值为 \texttt{Document, Element} 对象;
    \item 不支持子节点。
\end{itemize}

\texttt{CDATA}区块只在XML文档中有效,因此某些浏览器比较陈旧的版本会错误地将\texttt{CDATA}区块解析为\texttt{Comment}或\texttt{Element}。比如下面这行代码:

\begin{HTML}
<div id="myDiv"><![CDATA[This is some content.]]></div> 
\end{HTML}

在真正的XML文档中,可以使用\texttt{document.createCDataSection()}并传入节点内容来创建\texttt{CDATA}区块。

\subsubsection*{\texttt{DocumentType} 类型}

\texttt{DocumentType}类型的节点包含文档的文档类型(\texttt{doctype})信息,具有以下特征:
\begin{itemize}
    \item \texttt{nodeType} 为 \texttt{Node.DOCUMENT\_TYPE\_NODE};
    \item \texttt{nodeName} 值为文档类型的名称;
    \item \texttt{nodeValue} 值为 \texttt{null};
    \item \texttt{parentNode} 值为 \texttt{Document} 对象;
    \item 不支持子节点。
\end{itemize}

\texttt{DocumentType}对象在DOM Level 1中不支持动态创建,只能在解析文档代码时创建。对于支持这个类型的浏览器,\texttt{DocumentType}对象保存在\texttt{document.doctype}属性中。DOM Level 1 规定了\texttt{DocumentType}对象的3个属性:\texttt{name, entities, notations}:
\begin{itemize}
    \item \texttt{name}: 文档类型的名称。
    \item \texttt{entities}: 文档类型描述的实体的\texttt{NamedNodeMap}。
    \item \texttt{notations}: 文档类型描述的表示法的\texttt{NamedNodeMap}。
\end{itemize}

因为浏览器中的文档通常是HTML或XHTML文档类型,所以\texttt{entities}和\texttt{notations}列表为空。(这个对象只包含行内声明的文档类型。)无论如何,只有\texttt{name}属性是有用的。这个属性包含文档类型的名称,即紧跟在<!DOCTYPE后面的那串文本。比如下面的HTML 4.01严格文档类型:

\begin{HTML}
<!DOCTYPE HTML PUBLIC "-// W3C// DTD HTML 4.01// EN"   
    "http:// www.w3.org/TR/html4/strict.dtd"> 
\end{HTML}

\begin{JavaScript}
alert(document.doctype.name); // "html" 
\end{JavaScript}

\subsubsection*{\texttt{DocumentFragment} 类型}

在所有节点类型中,\texttt{DocumentFragment}类型是唯一一个在标记中没有对应表示的类型。DOM将文档片段定义为“轻量级”文档,能够包含和操作节点,却没有完整文档那样额外的消耗。\texttt{DocumentFragment}节点具有以下特征:

\begin{itemize}
    \item \texttt{nodeType} 为 \texttt{Node.DOCUMENT\_FRAGMENT\_NODE};
    \item \texttt{nodeName} 值为 \texttt{``\#document-fragment''};
    \item \texttt{nodeValue} 值为 \texttt{null};
    \item \texttt{parentNode} 值为 \texttt{null};
    \item 子节点可以是\texttt{Element, ProcessingInstruction, Comment, Text, CDATASection, EntityReference}。
\end{itemize}

不能直接把文档片段添加到文档。相反,文档片段的作用是充当其他要被添加到文档的节点的仓库。可以使用\texttt{document.createDocumentFragment()}方法像下面这样创建文档片段:

\begin{JavaScript}
let fragment = document.createDocumentFragment(); 
\end{JavaScript}

文档片段从\texttt{Node}类型继承了所有文档类型具备的可以执行DOM操作的方法。如果文档中的一个节点被添加到一个文档片段,则该节点会从文档树中移除,不会再被浏览器渲染。添加到文档片段的新节点同样不属于文档树,不会被浏览器渲染。可以通过\texttt{appendChild()}或\texttt{insertBefore()}方法将文档片段的内容添加到文档。在把文档片段作为参数传给这些方法时,这个文档片段的所有子节点会被添加到文档中相应的位置。文档片段本身永远不会被添加到文档树。

\subsubsection*{\texttt{Attr} 类型}

元素数据在DOM中通过\texttt{Attr}类型表示。\texttt{Attr}类型构造函数和原型在所有浏览器中都可以直接访问。技术上讲,属性是存在于元素\texttt{attributes}属性中的节点。\texttt{Attr}节点具有以下特征:

\begin{itemize}
    \item \texttt{nodeType} 为 \texttt{Node.ATTRIBUTE\_NODE};
    \item \texttt{nodeName} 值为属性名;
    \item \texttt{nodeValue} 值为属性值;
    \item \texttt{parentNode} 值为 \texttt{null};
    \item 在 HTML 中不支持子节点,在 XML 中子节点可以是: \texttt{Text, EntityReference}。
\end{itemize}

属性节点尽管是节点,却不被认为是DOM文档树的一部分。\texttt{Attr}节点很少直接被引用,通常开发者更喜欢使用\texttt{getAttribute()、removeAttribute()、setAttribute()}方法操作属性。

Attr对象上有3个属性:\texttt{name、value、specified}。\texttt{specified}是一个布尔值,表示属性使用的是默认值还是被指定的值。

可以使用\texttt{document.createAttribute()}方法创建新的\texttt{Attr}节点,参数为属性名。

\subsection{DOM 编程}

\subsubsection*{动态加载}

动态脚本就是在页面初始加载时不存在,之后又通过DOM包含的脚本。与之对应的,在 HTML 中插入 JavaScript 脚本有两种方式: 外部文件和直接引用。

动态加载外部文件很容易实现,比如:
\begin{HTML}
<script src="foo.js"></script> 
\end{HTML}

\begin{JavaScript}
let script = document.createElement("script"); 
script.src = "foo.js"; 
document.body.appendChild(script); 
\end{JavaScript}

另一个动态插入JavaScript的方式是嵌入源代码,如下面的例子所示:
\begin{HTML}
<script>    
    function sayHi() {     
        alert("hi");  
    } 
</script> 
\end{HTML}

\begin{JavaScript}
let script = document.createElement("script"); 
script.appendChild(document.createTextNode("function sayHi(){alert('hi');}")); 
document.body.appendChild(script); 
\end{JavaScript}

\subsubsection*{动态样式}

CSS样式在HTML页面中可以通过两个元素加载。\texttt{<link>}元素用于包含CSS外部文件,而<style>元素用于添加嵌入样式。与动态脚本类似,动态样式也是页面初始加载时并不存在,而是在之后才添加到页面中的。

动态加载外部文件很简单, 也是主流的方案:

\begin{HTML}
<link rel="stylesheet" type="text/css" href="styles.css">
\end{HTML}

\begin{JavaScript}
let link = document.createElement("link"); 
link.rel = "stylesheet"; 
link.type = "text/css"; 
link.href = "styles.css"; 
let head = document.getElementsByTagName("head")[0]; 
head.appendChild(link); 
\end{JavaScript}

当然也可以嵌入源代码,但是比较繁琐:

\begin{HTML}
<style type="text/css"> 
body {   
    background-color: red; 
} 
</style> 
\end{HTML}

\begin{JavaScript}
let style = document.createElement("style"); 
style.type = "text/css"; 
style.appendChild(document.createTextNode("body{background-color:red}")); 
let head = document.getElementsByTagName("head")[0]; 
head.appendChild(style); 
\end{JavaScript}

不同的浏览器可能会不支持部分操作,我们可以使用 \texttt{try...catch} 对有问题的语句捕捉。

\subsubsection*{操作表格}

表格是 HTML 最复杂的结构之一,用 DOM 操作表格往往十分繁琐,为此 DOM 给 \texttt{<table>, <tbody>, <tr>} 元素增加了一些属性和方法,下面展示 \texttt{<table>} 元素部分属性方法,其余几个元素方法类似:
\begin{itemize}
    \item \texttt{tBodies}: 指向 \texttt{<tbody>} 元素的 \texttt{HTMLCollection}。
    \item \texttt{tHead}: 指向 \texttt{thead} 元素(如果存在)。
    \item \texttt{createTHead()},创建\texttt{<thead>}元素,放到表格中,返回引用。
    \item \texttt{insertRow(pos)},在行集合中给定位置插入一行。
\end{itemize}

\subsubsection*{\texttt{NodeList}}

DOM 中有三个关键的集合对象: \texttt{NodeList, NamedNodeMap, HTMLCollection},后两个继承自 \texttt{NodeList}。这3个集合类型都是“实时的”,意味着文档结构的变化会实时地在它们身上反映出来。

任何时候要迭代NodeList,最好再初始化一个变量保存当时查询时的长度,然后用循环变量与这个变量进行比较,因为往 \texttt{NodeList} 中添加元素会实时改变它的长度,最好的方式如下:

\begin{JavaScript}
let divs = document.getElementsByTagName("div");  
for (let i = 0, len = divs.length; i < len; ++i) {   
    let div = document.createElement("div");   
    document.body.appendChild(div); 
}
\end{JavaScript}

一般来说,最好限制操作\texttt{NodeList}的次数。因为每次查询都会搜索整个文档,所以最好把查询到的\texttt{NodeList}缓存起来。

\subsection{\texttt{MutationObserver} 接口}

\texttt{MutationObserver}接口可以在DOM被修改时异步执行回调。使用\texttt{MutationObserver}可以观察整个文档、DOM树的一部分,或某个元素。此外还可以观察元素属性、子节点、文本,或者前三者任意组合的变化。

\texttt{MutationObserver}的实例要通过调用构造函数并传入一个回调函数来创建:

\begin{JavaScript}
let observer = new MutationObserver(() => console.log('DOM was mutated!')); 
\end{JavaScript}

\subsubsection*{\texttt{observe()} 方法}

新创建的\texttt{MutationObserver}实例不会关联DOM的任何部分。要把这个\texttt{observer}与DOM关联起来,需要使用\texttt{observe()}方法。这个方法接收两个必需的参数:要观察其变化的DOM节点,以及一个\texttt{MutationObserverInit}对象。

\texttt{MutationObserverInit}对象用于控制观察哪些方面的变化,是一个键/值对形式配置选项的字典。例如,下面的代码会创建一个观察者(\texttt{observer})并配置它观察\texttt{<body>}元素上的属性变化:

\begin{JavaScript}
let observer = new MutationObserver(() => console.log('<body> attributes changed')); 
observer.observe(document.body, { attributes: true }); 
\end{JavaScript}

执行以上代码后,\texttt{<body>}元素上任何属性发生变化都会被这个\texttt{MutationObserver}实例发现,然后就会异步执行注册的回调函数。

\subsubsection*{回调与\texttt{MutationRecord}}

每个回调都会收到一个\texttt{MutationRecord}实例的数组。\texttt{MutationRecord}实例包含的信息包括发生了什么变化,以及DOM的哪一部分受到了影响。因为回调执行之前可能同时发生多个满足观察条件的事件,所以每次执行回调都会传入一个包含按顺序入队的\texttt{MutationRecord}实例的数组。

\begin{JavaScript}
let observer = new MutationObserver((mutationRecords) => console.log(mutationRecords));
\end{JavaScript}

\texttt{mutationRecords} 有很多属性,例如 \texttt{target, type, oldValue} 等,开发者可以调用这些属性自定义回调函数。

默认情况下,只要被观察的元素不被垃圾回收,\texttt{MutationObserver}的回调就会响应DOM变化事件,从而被执行。要提前终止执行回调,可以调用\texttt{disconnect()}方法。

调用\texttt{disconnect()}并不会结束\texttt{MutationObserver}的生命。还可以重新使用这个观察者,再用 \texttt{observer()} 将它关联到新的目标节点。

\subsubsection*{异步回调与记录队列}

\texttt{MutationObserver}接口是出于性能考虑而设计的,其核心是异步回调与记录队列模型。为了在大量变化事件发生时不影响性能,每次变化的信息(由观察者实例决定)会保存在\texttt{MutationRecord}实例中,然后添加到记录队列。这个队列对每个\texttt{MutationObserver}实例都是唯一的,是所有DOM变化事件的有序列表。

每次\texttt{MutationRecord}被添加到\texttt{MutationObserver}的记录队列时,仅当之前没有已排期的微任务回调时(队列中微任务长度为0),才会将观察者注册的回调(在初始化\texttt{MutationO bserver}时传入)作为微任务调度到任务队列上。这样可以保证记录队列的内容不会被回调处理两次。

不过在回调的微任务异步执行期间,有可能又会发生更多变化事件。因此被调用的回调会接收到一个\texttt{MutationRecord}实例的数组,顺序为它们进入记录队列的顺序。回调要负责处理这个数组的每一个实例,因为函数退出之后这些实现就不存在了。回调执行后,这些\texttt{MutationRecord}就用不着了,因此记录队列会被清空,其内容会被丢弃。

调用\texttt{MutationObserver}实例的\texttt{takeRecords()}方法可以清空记录队列,取出并返回其中的所有\texttt{MutationRecord}实例。

\subsubsection*{性能,内存与垃圾回收}

DOM  Level  2规范中描述的\texttt{MutationEvent}定义了一组会在各种DOM变化时触发的事件。由于浏览器事件的实现机制,这个接口出现了严重的性能问题。因此,DOM Level 3规定废弃了这些事件。\texttt{MutationObserver}接口就是为替代这些事件而设计的更实用、性能更好的方案。

无论如何,使用\texttt{MutationObserver}仍然不是没有代价的。因此理解什么时候避免出现这种情况就很重要了。

\texttt{MutationObserver}实例与目标节点之间的引用关系是非对称的。\texttt{MutationObserver}拥有对要观察的目标节点的弱引用。因为是弱引用,所以不会妨碍垃圾回收程序回收目标节点。然而,目标节点却拥有对\texttt{MutationObserver}的强引用。如果目标节点从DOM中被移除,随后被垃圾回收,则关联的\texttt{MutationObserver}也会被垃圾回收。

记录队列中的每个\texttt{MutationRecord}实例至少包含对已有DOM节点的一个引用。如果变化是\texttt{childList}类型,则会包含多个节点的引用。记录队列和回调处理的默认行为是耗尽这个队列,处理每个\texttt{MutationRecord},然后让它们超出作用域并被垃圾回收。

有时候可能需要保存某个观察者的完整变化记录。保存这些\texttt{MutationRecord}实例,也就会保存它们引用的节点,因而会妨碍这些节点被回收。如果需要尽快地释放内存,建议从每个\texttt{MutationRecord}中抽取出最有用的信息,然后保存到一个新对象中,最后抛弃\texttt{MutationReco rd}。

\newpage