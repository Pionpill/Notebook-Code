\section{BOM}

BOM(Browser Object Model)是使用JavaScript开发Web应用程序的核心。BOM提供了与网页无关的浏览器功能对象。很不幸的是,多年以来 BOM是在缺乏规范的背景下发展起来的,许多浏览器开发商有着自己的规范,因此 BOM 像 ECMAScript 一样灵活(乱)。

\subsection{\texttt{window} 对象}

BOM的核心是\texttt{window}对象,表示浏览器的实例。\texttt{window}对象在浏览器中有两重身份,一个是ECMAScript中的\texttt{Global}对象(\texttt{window} 代理 \texttt{Global} 对象),另一个就是浏览器窗口的JavaScript接口。

\subsubsection{操作 \texttt{window} 对象}

\subsubsection*{\texttt{Global} 作用域}

因为 \texttt{window} 对象被复用为 ECMAScript 的 \texttt{Global} 对象,所以通过 \texttt{var} 声明的所有全局变量和函数都会变成 \texttt{window} 对象的属性和方法。

\begin{JavaScript}
var age = 29;
alert(window.age); // 29
\end{JavaScript}

如果将 \texttt{var} 换成 \texttt{let} 则不会将变量添加给全局对象。前面已经讲过。

\subsubsection*{窗口关系}

\texttt{top} 对象始终指向最上层(外层)窗口,即浏览器窗口本身。而 \texttt{parent} 对象则始终指向当前窗口的父窗口。

还有一个 \texttt{self} 对象,他是终极 \texttt{window} 属性,始终指向 \texttt{window}。实际上 \texttt{self} 和 \texttt{window} 就是同一个对象。

这些都是 \texttt{window} 对象的属性,因此访问 \texttt{window.parent}, \texttt{window.top} 和 \texttt{window.self} 都可以。这意味着可以把访问多个窗口的 \texttt{window} 对象串联起来。比如 \texttt{window.parent.parent}。

\subsubsection*{窗口位置与像素化}

\texttt{window} 对象的位置可以通过不同的属性个方法来确定。现代浏览器提供了 \texttt{parentLeft} 和 \texttt{screenTop} 属性,用于表示窗口相对于屏幕左侧和顶部的位置,返回单位是 CSS 像素\footnote{CSS 像素参考博客: \url{https://blog.csdn.net/u014465934/article/details/97040694}}。

可以使用 \texttt{moveTo()} 和 \texttt{moveBy()} 方法移动窗口。这两个方法都接收两个参数,其中 \texttt{moveTo()} 接收要移动到的新位置的绝对坐标\texttt{x}和\texttt{y}。而 \texttt{moveBy()} 则接受相对当前位置在两个方向上移动的像素数(相对位置)。

\subsubsection*{窗口大小}

所有现代浏览器都支持4个属性:
\begin{itemize}
    \item 返回浏览器窗口自身大小: \texttt{outerWidth, outerHeight}。
    \item 返回浏览器窗口中页面大小: \texttt{innerWidth, innerHeight}。
\end{itemize}

\texttt{document.documentElement.clientWidth}和\texttt{document.documentElement.clientHeight}返回页面视图的宽度和高度。

浏览器窗口自身的精确尺寸不好确定,但可以确定页面视口的大小,如下所示:

\begin{JavaScript}
let pageWidth = window.innerWidth,
    pageHeight = window.innerHeight;

// 如果单位不是数值(物理像素)
if (typeof pageWidth != "number") {
    // 如果是 CSS 像素,返回页面高宽
    if (document.compatMode == "CSS1Compat") {
        pageWidth = document.documentElement.clientWidth;
        pageHeight = document.documentElement.clientHeight;
    // 如果不是 CSS 像素,返回 body 元素高宽
    } else {
        pageWidth = document.body.clientWidth;
        pageHeight = document.body.clientHeight;
    }
}
\end{JavaScript}

此外,还有两个方法用于缩放窗口大小:
\begin{itemize}
    \item \texttt{window.resizeTo(x,y)}: 缩放到绝对大小。
    \item \texttt{window.resizeBy(x,y)}: 相对缩放。
\end{itemize}

大部分浏览器都支持上诉方法,但部分浏览器则会禁用。

\subsubsection*{窗口位置}

浏览器窗口尺寸通常无法满足完整显示整个页面,为此用户可以通过滚动在有限的视口中查看文档。相关的方法是 \texttt{window.scroll} 系方法。用法和 \texttt{window.resize} 类似。

\subsubsection*{定时器}

JavaScript在浏览器中是单线程执行的,但允许使用定时器指定在某个时间之后或每隔一段时间就执行相应的代码。\texttt{setTimeout()}用于指定在一定时间后执行某些代码,而\texttt{setInter val()}用于指定每隔一段时间执行某些代码。

\begin{itemize}
    \item \texttt{setTimeout()}: 接收两个参数:要执行的代码和在执行回调函数前等待的时间(毫秒), 返回一个表示该超时排期的数值ID。
\begin{JavaScript}
let timeoutId = setTimeout(() => alert("Hello world!"), 1000); 
clearTimeout(timeoutId);    // 取消超时任务
\end{JavaScript}
    \item \texttt{setInterval()}: 接收两个参数:要执行的代码(字符串或函数),以及把下一次执行定时代码的任务添加到队列要等待的时间(毫秒)。同样返回一个 ID。
\end{itemize}

\subsubsection{新窗口}

\texttt{window.open()} 可以用于导航到指定URL,也可以用于打开新浏览器窗口。这个方法接收4个参数:要加载的URL、目标窗口、特性字符串和表示新窗口在浏览器历史记录中是否替代当前加载页面的布尔值。通常,调用这个方法时只传前3个参数,最后一个参数只有在不打开新窗口时才会使用:

\begin{JavaScript}
// 与<a href="http://www.wrox.com" target="topFrame"/>相同
window.open("http://www.wrox.com/", "topFrame"); 
\end{JavaScript}

特征字符串是一个逗号分隔的设置字符串,用于指定新窗口包含的特性。下表列出了一些选项:

\begin{table}[H]
    \centering
    \small
    \caption{特征字符串}
    \label{table:特征字符串}
    \setlength{\tabcolsep}{4mm}
    \begin{tabular}{l|l|l}
        \toprule
        \textbf{设置} & \textbf{值} & \textbf{说明} \\
        \midrule
        fullscreen & ``yes''|``no'' & 新窗口是否最大化,仅“IE”支持 \\
        height & 数值 & 新窗口高度,值不能小于 100 \\
        width & 数值 & 新窗口宽度,值不能小于 100 \\
        left & 数值 & 新窗口 x 坐标,不能为负 \\
        top & 数值 & 新窗口 y 坐标,不能为负 \\
        \midrule
        location & ``yes''|``no'' & 是否显示地址栏,不同显示器默认值不一样 \\
        Menubar & ``yes''|``no'' & 表示是否显示菜单栏。默认为``no'' \\
        resizable & ``yes''|``no'' & 表示是否可以拖动改变新窗口大小。默认为``no'' \\
        scrollbars & ``yes''|``no'' & 表示是否可以在内容过长时滚动。默认为``no'' \\
        status & ``yes''|``no'' & 表示是否显示状态栏。不同浏览器的默认值也不一样 \\
        toolbar & ``yes''|``no'' & 表示是否显示工具栏。默认为"no" \\
        \bottomrule
    \end{tabular}
\end{table}

这些设置需要以逗号分隔的名值对形式出现,其中名值对以等号连接。

\begin{JavaScript}
let wroxWin = window.open("http://www.wrox.com/",
    "wroxWindow",
    "height=400,width=400,top=10,left=10,resizable=yes"); 
\end{JavaScript}

\texttt{window.open()}方法返回一个对新建窗口的引用。这个对象与普通window对象没有区别,只是为控制新窗口提供了方便。

新创建窗口的\texttt{window}对象有一个属性\texttt{opener},指向打开它的窗口。这个属性只在弹出窗口的最上层\texttt{window}对象(\texttt{top})有定义,是指向调用\texttt{window.open()}打开它的窗口或窗格的指针。

\begin{JavaScript}
alert(wroxWin.opener === window); // true 
\end{JavaScript}

虽然新建窗口中有指向打开它的窗口的指针,但反之则不然。窗口不会跟踪记录自己打开的新窗口,因此开发者需要自己记录。

在某些浏览器中,每个标签页会运行在独立的进程中。如果一个标签页打开了另一个,而window对象需要跟另一个标签页通信,那么标签便不能运行在独立的进程中。在这些浏览器中,可以将新打开的标签页的opener属性设置为null,表示新打开的标签页可以运行在独立的进程中。比如:

\begin{JavaScript}
wroxWin.opener = null; 
\end{JavaScript}

把opener设置为null表示新打开的标签页不需要与打开它的标签页通信,因此可以在独立进程中运行。这个连接一旦切断,就无法恢复了。

\subsubsection*{安全限制}

由于弹出窗口被在线广告用烂了,因此很多浏览器会屏蔽弹出窗口,为此我们需要检测新窗口是否被正确弹出,最简单的方法是判断返回值:

\begin{JavaScript}
let wroxWin = window.open("http://www.wrox.com", "_blank"); 
if (wroxWin == null){    
    alert("The popup was blocked!"); 
} 
\end{JavaScript}

但有的浏览器会让 \texttt{window.open()} 抛出错误,更好的方法是使用 \texttt{try catch} 语句:

\begin{JavaScript}
let blocked = false; try {
    let wroxWin = window.open("http://www.wrox.com", "_blank");
    if (wroxWin == null) { 
        blocked = true;
    }
} catch (ex) {
    blocked = true;
} if (blocked) {
    alert("The popup was blocked!");
} 
\end{JavaScript}

\subsubsection*{系统对话框}

使用\texttt{alert()、confirm()}和\texttt{prompt()}方法,可以让浏览器调用系统对话框向用户显示消息。这些对话框与浏览器中显示的网页无关,而且也不包含HTML。它们的外观由操作系统或者浏览器决定,无法使用CSS设置。此外,这些对话框都是同步的模态对话框,即在它们显示的时候,代码会停止执行,在它们消失以后,代码才会恢复执行。

\begin{itemize}
    \item \texttt{alert()}: 接收一个要显示给用户的字符串。传入的字符串会显示在一个系统对话框中。对话框只有一个“OK”(确定)按钮。如果传给\texttt{alert()}的参数不是一个原始字符串,则会调用这个值的\texttt{toString()}方法将其转换为字符串。
    \item \texttt{confirm()}: 和 \texttt{alert()} 类似,但会有 \texttt{Cancel} 和 \texttt{OK} 两个按钮。通过返回的 \texttt{boolean} 值判断用户按下了哪个按钮。
    \item \texttt{prompt()}: 在 \texttt{confirm()} 的基础上,增加了文本框,用于提示用户输入信息。
    \begin{itemize}
        \item \texttt{prompt()} 有两个参数: 显示给用户的信息,文本框默认值。
        \item 如果用户按下 \texttt{OK} 按钮,则返回文本框的内容;否则返回 \texttt{null}。
    \end{itemize}
\end{itemize}

JavaScript还可以显示另外两种对话框:\texttt{find()}和\texttt{print()}。这两种对话框都是异步显示的,即控制权会立即返回给脚本。用户在浏览器菜单上选择“查找”(find)和“打印”(print)时显示的就是这两种对话框。通过在window对象上调用\texttt{find()}和\texttt{print()}可以显示它们:

\begin{JavaScript}
window.print();
window.find(); 
\end{JavaScript}

\subsection{其他 BOM 对象}

\subsubsection*{\texttt{location} 对象}

\texttt{location}是最有用的BOM对象之一,提供了当前窗口中加载文档的信息,以及通常的导航功能。这个对象独特的地方在于,它既是\texttt{window}的属性,也是\texttt{document}的属性。也就是说,\texttt{window.location}和\texttt{document.location}指向同一个对象。

\texttt{location}对象不仅保存着当前加载文档的信息,也保存着把URL解析为离散片段后能够通过属性访问的信息。比如:
\begin{itemize}
    \item \texttt{location.host}: 服务器名以及端口号。
    \item \texttt{location.hostname}: 服务器名。
    \item \texttt{location.search}: URL的查询字符串。这个字符串以问号开头。
\end{itemize}

\subsubsection*{查询字符串}

虽然 \texttt{location.search} 能获取 URL 的查询字符串,但是\texttt{location}并没有日工获取单个查询内容的接口(也提供不了)。如果我们要解析查询内容,则需要手动编写函数处理 \texttt{location.search} 返回的字符串。

\texttt{URLSearchParams} 提供了一组标准API方法,通过它们可以检查和修改查询字符串。

\begin{JavaScript}
let qs = "?q=javascript&num=10";
let searchParams = new URLSearchParams(qs);
alert(searchParams.toString());  // " q=javascript&num=10" 
searchParams.has("num");         // true 
searchParams.get("num");         // 10 
searchParams.set("page", "3");
alert(searchParams.toString());  // " q=javascript&num=10&page=3" 
searchParams.delete("q");
alert(searchParams.toString());  // " num=10&page=3"
\end{JavaScript}

大多数支持\texttt{URLSearchParams}的浏览器也支持将\texttt{URLSearchParams}的实例用作可迭代对象:

\begin{JavaScript}
let qs = "?q=javascript&num=10";
let searchParams = new URLSearchParams(qs);
for (let param of searchParams) {
    console.log(param);
}
// ["q", "javascript"] 
// ["num", "10"] 
\end{JavaScript}

\subsubsection*{修改地址}

可以通过修改\texttt{location}对象修改浏览器的地址。首先,最常见的是使用\texttt{assign()}方法并传入一个URL,如下所示:

\begin{JavaScript}
location.assign("http://www.wrox.com"); 
\end{JavaScript}

这行代码会立即启动导航到新URL的操作,同时在浏览器历史记录中增加一条记录。如果给\texttt{location.href}或\texttt{window.location}设置一个URL,也会以同一个URL值调用\texttt{assign()}方法。比如,下面两行代码都会执行与显式调用\texttt{assign()}一样的操作:

\begin{JavaScript}
window.location = "http://www.wrox.com"; 
location.href = "http://www.wrox.com";  // 最常见
\end{JavaScript}

修改location对象的其他属性也会修改当前加载的页面。

在以前面提到的方式修改URL之后,浏览器历史记录中就会增加相应的记录。如果不希望增加历史记录,可以使用\texttt{replace()}方法。这个方法接收一个URL参数,但重新加载后不会增加历史记录。调用\texttt{replace()}之后,用户不能回到前一页。

最后一个修改地址的方法是\texttt{reload()},它能重新加载当前显示的页面。调用\texttt{reload()}而不传参数,页面会以最有效的方式重新加载。也就是说,如果页面自上次请求以来没有修改过,浏览器可能会从缓存中加载页面。如果想强制从服务器重新加载,可以给\texttt{reload()}传个\texttt{true}。

脚本中位于\texttt{reload()}调用之后的代码可能执行也可能不执行,这取决于网络延迟和系统资源等因素。为此,最好把\texttt{reload()}作为最后一行代码。

\subsubsection*{\texttt{navigator} 对象}

navigator对象的属性通常用于确定浏览器的类型,表示用户代理的状态和标识。举几个常用的属性/方法:

\begin{itemize}
    \item \texttt{appName}: 浏览器全名。
    \item \texttt{cookieEnabled}: 返回布尔值,表示是否启用了cookie。
    \item \texttt{deviceMemory}: 返回单位为GB的设备内存容量。
    \item \texttt{language}: 返回浏览器的主语言。
    \item \texttt{mediaDevices}: 返回可用的媒体设备。
    \item \texttt{plugins}: 返回浏览器安装的插件数组。
\end{itemize}

最常见的用法是检查浏览器插件,\texttt{plugins} 数组包含以下内容:
\begin{itemize}
    \item \texttt{name}: 插件名称。
    \item \texttt{description}: 插件介绍。
    \item \texttt{filename}: 插件的文件名。
    \item \texttt{length}: 由当前插件处理的MIME类型数量。
\end{itemize}

\subsubsection*{\texttt{screen} 对象}

\texttt{screen} 对象很少用,这个对象中保存的纯粹是客户端能力信息,也就是浏览器窗口外面的客户端显示器的信息,比如像素宽度和像素高度。举几个例子:

\begin{itemize}
    \item availHeight: 屏幕像素高度减去系统组件高度(只读)。
    \item colorDepth: 表示屏幕颜色的位数;多数系统是32(只读)。
    \item height: 屏幕像素高度。
\end{itemize}

\subsubsection*{\texttt{history} 对象}

\texttt{history}对象表示当前窗口首次使用以来用户的导航历史记录。因为\texttt{history}是\texttt{window}的属性,所以每个\texttt{window}都有自己的\texttt{history}对象。

\texttt{go()} 方法可以在用户历史记录中沿任何方向导航,可以前进也可以后退。这个方法只接收一个参数,这个参数可以是一个整数,表示前进或后退多少步。

\begin{JavaScript}
// 后退一页
history.go(-1); 
// 前进一页
history.go(1); 
// 前进两页
history.go(2); 
\end{JavaScript}

\texttt{go()}有两个简写方法:\texttt{back()}和\texttt{forward()}。顾名思义,这两个方法模拟了浏览器的后退按钮和前进按钮。

\texttt{history}对象还有一个\texttt{length}属性,表示历史记录中有多个条目。

\newpage