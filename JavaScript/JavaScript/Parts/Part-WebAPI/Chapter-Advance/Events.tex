\section{DOM 事件}

JavaScript与HTML的交互是通过事件实现的,事件代表文档或浏览器窗口中某个有意义的时刻。可以使用仅在事件发生时执行的监听器(也叫处理程序)订阅事件。

\subsection{事件流}

在第四代Web浏览器(IE4和Netscape Communicator 4)开始开发时,开发团队碰到了一个有意思的问题:页面哪个部分拥有特定的事件呢?要理解这个问题,可以在一张纸上画几个同心圆。把手指放到圆心上,则手指不仅是在一个圆圈里,而且是在所有的圆圈里。两家浏览器的开发团队都是以同样的方式看待浏览器事件的。当你点击一个按钮时,实际上不光点击了这个按钮,还点击了它的容器以及整个页面。

事件流描述了页面接收事件的顺序。结果非常有意思,IE和Netscape开发团队提出了几乎完全相反的事件流方案。IE将支持事件冒泡流,而Netscape Communicator将支持事件捕获流。

\subsubsection*{事件冒泡}

IE事件流被称为事件冒泡,这是因为事件被定义为从最具体的元素(文档树中最深的节点)开始触发,然后向上传播至没有那么具体的元素(文档)。

比如说,我们点击 \texttt{<body>} 标签中的一个 \texttt{<div>}, \texttt{click} 事件会由如下顺序发生: \texttt{<div> $\rightarrow$ <body> $\rightarrow$ <html> $\rightarrow$ document}。

也就是说,\texttt{<div>}元素,即被点击的元素,最先触发\texttt{click}事件。然后,\texttt{click}事件沿DOM树一路向上,在经过的每个节点上依次触发,直至到达\texttt{document}对象。

所有现代浏览器都支持事件冒泡,只是在实现方式上会有一些变化。现代浏览器中的事件会一直冒泡到\texttt{window}对象。

\subsubsection*{事件捕获}

事件捕获事件流由网景公司提出,沿着 DOM 树向下传播和事件冒泡流顺序完全相反。\texttt{click} 事件会由如下顺序发生: \texttt{document $\rightarrow$ <html> $\rightarrow$ <body> $\rightarrow$ <div>}。

事件捕获得到了所有现代浏览器的支持。实际上,所有浏览器都是从\texttt{window}对象开始捕获事件,而DOM2 Events规范规定的是从\texttt{document}开始。由于旧版本浏览器不支持,因此实际当中几乎不会使用事件捕获。通常建议使用事件冒泡,特殊情况下可以使用事件捕获。

\subsubsection*{DOM 事件流}

DOM2 Events规范规定事件流分为3个阶段:事件捕获、到达目标和事件冒泡。事件捕获最先发生,为提前拦截事件提供了可能。然后,实际的目标元素接收到事件。最后一个阶段是冒泡,最迟要在这个阶段响应事件。

在DOM事件流中,实际的目标(\texttt{<div>}元素)在捕获阶段不会接收到事件。这是因为捕获阶段从\texttt{document}到\texttt{<html>}再到\texttt{<body>}就结束了。下一阶段,即会在\texttt{<div>}元素上触发事件的“到达目标”阶段,通常在事件处理时被认为是冒泡阶段的一部分。然后,冒泡阶段开始,事件反向传播至文档。

虽然DOM2 Events规范明确捕获阶段不命中事件目标,但现代浏览器都会在捕获阶段在事件目标上触发事件。最终结果是在事件目标上有两个机会来处理事件。

\subsection{事件处理程序}

事件意味着用户或浏览器执行的某种动作。比如,单击(click)、加载(load)、鼠标悬停(mouseover)。为响应事件而调用的函数被称为事件处理程序(或事件监听器)。事件处理程序的名字以"on"开头,因此click事件的处理程序叫作\texttt{onclick},而load事件的处理程序叫作\texttt{onload}。有很多方式可以指定事件处理程序。

\subsubsection*{HTML 事件处理程序}

特定元素支持的每个事件都可以使用事件处理程序的名字以HTML属性的形式来指定。此时属性的值必须是能够执行的JavaScript代码。例如,要在按钮被点击时执行某些JavaScript代码,可以使用以下HTML属性:

\begin{JavaScript}
<input type="button" value="Click Me" onclick="console.log('Clicked')"/> 
\end{JavaScript}

在HTML中定义的事件处理程序可以包含精确的动作指令,也可以调用在页面其他地方定义的脚本,比如:

\begin{JavaScript}
<script>   
    function showMessage() {      
        console.log("Hello world!");   
    }  
</script> 
<input type="button" value="Click Me" onclick="showMessage()"/> 
\end{JavaScript}

以这种方式指定的事件处理程序有一些特殊的地方。首先,会创建一个函数来封装属性的值。这个函数有一个特殊的局部变量\texttt{event},其中保存的就是\texttt{event}对象(\texttt{event 具体的属性不讨论})。有了这个对象,就不用开发者另外定义其他变量,也不用从包装函数的参数列表中去取了。

这个动态创建的包装函数还有一个特别有意思的地方,就是其作用域链被扩展了。在这个函数中,document和元素自身的成员都可以被当成局部变量来访问。这意味着事件处理程序可以更方便地访问自己的属性。

在HTML中指定事件处理程序有一些问题。第一个问题是时机问题。有可能HTML元素已经显示在页面上,用户都与其交互了,而事件处理程序的代码还无法执行。为此,大多数HTML事件处理程序会封装在try/catch块中,以便在这种情况下静默失败。另一个问题是对事件处理程序作用域链的扩展在不同浏览器中可能导致不同的结果。不同JavaScript引擎中标识符解析的规则存在差异,因此访问无限定的对象成员可能导致错误。

使用HTML指定事件处理程序的最后一个问题是HTML与JavaScript强耦合。如果需要修改事件处理程序,则必须在两个地方,即HTML和JavaScript中,修改代码。这也是很多开发者不使用HTML事件处理程序,而使用JavaScript指定事件处理程序的主要原因。

\subsubsection*{DOM0 事件处理程序}

在JavaScript中指定事件处理程序的传统方式是把一个函数赋值给(DOM元素的)一个事件处理程序属性。使用JavaScript指定事件处理程序,必须先取得要操作对象的引用。

每个元素(包括\texttt{window}和\texttt{document})都有通常小写的事件处理程序属性,比如\texttt{onclick}。只要把这个属性赋值为一个函数即可:

\begin{JavaScript}
let btn = document.getElementById("myBtn"); 
btn.onclick = function() {    
    console.log("Clicked");  
}; 
\end{JavaScript}

像这样使用DOM0方式为事件处理程序赋值时,所赋函数被视为元素的方法。因此,事件处理程序会在元素的作用域中运行,即this等于元素。下面的例子演示了使用this引用元素本身:

\begin{JavaScript}
let btn = document.getElementById("myBtn"); 
btn.onclick = function() {  
    console.log(this.id);  // "myBtn" 
}; 
\end{JavaScript}

通过将事件处理程序属性的值设置为null,可以移除通过DOM0方式添加的事件处理程序。

\subsubsection*{DOM2 事件处理程序}

DOM2  Events为事件处理程序的赋值和移除定义了两个方法:\texttt{addEventListener()}和\texttt{removeEventListener()}。这两个方法暴露在所有DOM节点上,它们接收3个参数:事件名、事件处理函数和一个布尔值,\texttt{true}表示在捕获阶段调用事件处理程序,\texttt{false}(默认值)表示在冒泡阶段调用事件处理程序。

仍以给按钮添加\texttt{click}事件处理程序为例,可以这样写:

\begin{JavaScript}
let btn = document.getElementById("myBtn"); 
btn.addEventListener("click", () => {   
    console.log(this.id); 
}, false); 
\end{JavaScript}

与DOM0方式类似,这个事件处理程序同样在被附加到的元素的作用域中运行。使用DOM2方式的主要优势是可以为同一个事件添加多个事件处理程序。

通过\texttt{addEventListener()}添加的事件处理程序只能使用\texttt{removeEventListener()}并传入与添加时同样的参数来移除。这意味着使用\texttt{addEventListener()}添加的匿名函数无法移除。

\begin{JavaScript}
let btn = document.getElementById("myBtn");
let handler = function () {
    console.log(this.id);
};
btn.addEventListener("click", handler, false);
btn.removeEventListener("click", handler, false);
\end{JavaScript}

大多数情况下,事件处理程序会被添加到事件流的冒泡阶段,主要原因是跨浏览器兼容性好。把事件处理程序注册到捕获阶段通常用于在事件到达其指定目标之前拦截事件。如果不需要拦截,则不要使用事件捕获。