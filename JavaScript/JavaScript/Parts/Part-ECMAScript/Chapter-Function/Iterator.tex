\section{迭代器与生成器}

\subsection{迭代器}

迭代器基础不讲,ECMAScript 的这些内置类型实现了 \texttt{Iterable} 接口: 字符串,数组,映射,集合,\texttt{arguments 对象}, \texttt{NodeList} 等 DOM 集合类型。

如果要自定义可迭代对象,写一个 \texttt{next()} 方法即可。\texttt{next()} 方法返回的是一个对象,包含 \texttt{done} 和 \texttt{value} 两个元素,其中 \texttt{done} 为 \texttt{Boolean} 表示迭代器返回成功与否,\texttt{value} 为返回的数据。

\subsection{生成器}

生成器是 ES6 新增的一个极为灵活的结构,拥有在一个函数块内暂停和恢复代码执行的能力。这种新能力具有深远的影响,比如,使用生成器可以自定义迭代器和实现协程。

生成器的形式是一个函数,函数名称前面加一个星号(*)表示它是一个生成器。只要是可以定义函数的地方,就可以定义生成器。

\begin{JavaScript}
// 生成器函数声明
function * generatorFn() {...}
// 生成器哈桑农户表达式
let generationFn = function*() {...}
// 作为对象字面量方法的生成器函数
let foo = {    
    * generatorFn() {} 
} 
// 作为类实例方法的生成器函数
class Foo {   
    * generatorFn() {} 
} 
// 作为类静态方法的生成器函数
class Bar {   
    static * generatorFn() {} 
} 
\end{JavaScript}

\fbox{
    \parbox{0.87\textwidth}{
        \begin{notice}
            箭头函数不能用来定义生成器函数。
        \end{notice}
    }
}

调用生成器函数会产生一个生成器对象。生成器对象一开始处于暂停执行(suspended)的状态。与迭代器相似,生成器对象也实现了\texttt{Iterator}接口,因此具有\texttt{next()}方法。调用这个方法会让生成器开始或恢复执行。

\begin{JavaScript}
function* generatorFn() {} 
const g = generatorFn(); 
console.log(g);       // generatorFn {<suspended>} 
console.log(g.next);  // f next() { [native code] } 
\end{JavaScript}

\texttt{yield}关键字可以让生成器停止和开始执行,也是生成器最有用的地方。生成器函数在遇到\texttt{yield}关键字之前会正常执行。遇到这个关键字后,执行会停止,函数作用域的状态会被保留。停止执行的生成器函数只能通过在生成器对象上调用\texttt{next()}方法来恢复执行:

\texttt{yield}关键字有点像函数的中间返回语句,它生成的值会出现在\texttt{next()}方法返回的对象里。通过\texttt{yield}关键字退出的生成器函数会处在\texttt{done: false}状态;通过\texttt{return}关键字退出的生成器函数会处于\texttt{done: true}状态。

\begin{JavaScript}
function* generatorFn() {  
    yield 'foo'; yield 'bar';   
    return 'baz'; 
} 
let generatorObject = generatorFn(); 
console.log(generatorObject.next());  // { done: false, value: 'foo' } 
console.log(generatorObject.next());  // { done: false, value: 'bar' } 
console.log(generatorObject.next());  // { done: true, value: 'baz' } 
\end{JavaScript}

生成器函数内部的执行流程会针对每个生成器对象区分作用域。在一个生成器对象上调用\texttt{next()}不会影响其他生成器:

\begin{JavaScript}
function* generatorFn() {   
    yield 'foo';    
    yield 'bar';   
    return 'baz'; 
} 
let generatorObject1 = generatorFn(); 
let generatorObject2 = generatorFn(); 
console.log(generatorObject1.next()); // { done: false, value: 'foo' } 
console.log(generatorObject2.next()); // { done: false, value: 'foo' } 
\end{JavaScript}

\fbox{
    \parbox{0.87\textwidth}{
        \begin{advise}
            由于生成器在实际开发过程中用的并不多,这里仅作基本介绍。JS 的生成器一定程度上参考了 python 的生成器,有兴趣的读者请自行了解。
        \end{advise}
    }
}










\newpage