\section{语言基础}
\subsection{语法}

ECMAScript 语言很大程度上借鉴了 C 语言,作为 C 语言系中的一门,它和 Java 与许多相似之处,比如区分大小写。

\subsubsection*{标识符}

JavaScript 对标识符组成规定如下:
\begin{itemize}
    \item 第一个字符必须是一个字母,下划线(\_)或美元符号(\$)。
    \item 剩下的字符可以是字母,数字,下划线(\_)或美元符号(\$)。
\end{itemize}

标识符中的字符是可扩展的 ASCII 中的字母,也可以是 Unicode 的字母字符,但不建议使用 Unicode 字符。

\fbox{
    \parbox{0.87\textwidth}{
        \begin{advise}
            下划线(\_)和美元符号(\$)往往有特殊作用,不建议在自己命名的变量中使用。
        \end{advise}
    }
}

ECMAScript 标识符使用驼峰大小写形式。

\subsubsection*{注释}
ECMAScript 采用 C 语言风格的注释。
\begin{itemize}
    \item 单行注释:
    
    \begin{JavaScript}
    // 当行注释
    \end{JavaScript}
    \item 多行注释(块注释):
    \begin{JavaScript}
    /*多行
    注释*/
    \end{JavaScript}
\end{itemize}

\subsubsection*{严格模式}

ES5 增加了严格模式,用于处理之前版本(ES3)的一些不规范书写问题。启用严格模式需要在脚本头部加上这一行:

\begin{JavaScript}
"use strict"
\end{JavaScript}

也可以单独在一个函数中执行严格模式:

\begin{JavaScript}
function doSomething() {
    "use strict";
    // 函数体
}
\end{JavaScript}

严格模式会影响到 JavaScript 执行的方方面面,后文会经常出现这个概念。所有现代浏览器都支持严格模式。

\subsubsection*{语句}

ECMAScript 中的语句和 Java 几乎一样。他理论上有几种宽松的写法,但并不推荐这样用。

\begin{itemize}
    \item ECMAScript 允许省略分号\footnote{可能影响性能。同时降低可读性。}。
    \item 单条控制语句可以省略括号\footnote{影响不大,但建议保留。}。
\end{itemize}

\subsection{关键字与保留字}

JavaScript 有很多关键字,关键字不能用作标识符或属性名,具体的关键字视 ES 版本而定,这里不一一列举\footnote{参考文章:\url{https://zhuanlan.zhihu.com/p/257105802}}。

此外,规范中还有一些未来的保留字,同样不能作为标识符或属性名使用。这些保留字多是其他语言中的关键字,有一定其他语言基础的人一般都不会用到。

\subsection{变量}

ECMAScript 变量视松散类型,意思是变量可以用于保存任何类型的数据。共有3个关键字可以声明变量: \texttt{var, let, const}。其中 \texttt{let, const} 只有在 ES6 级之后的版本才可以使用。

\subsubsection{\texttt{var} 关键字}

\texttt{var} 关键字是最原始的声明变量关键字。用法如下:

\begin{JavaScript}
var message = "hi";
\end{JavaScript}

如果没有赋值,仅声明变量,变量会保存一个特殊值 \texttt{undefined}\footnote{下一节讨论。}。这里 \texttt{message} 被定义为一个保存字符串值的变量,但这样初始化变量不会将它标识为字符串类型,只是一个简单的赋值而已。

此外,虽然 ECMAScript 允许变量重新赋值,但是并不推荐这样使用。

\noindent\textbf{\texttt{var} 声明作用域}

使用 \texttt{var} 操作符定义的变量会成为包含它的函数的局部变量。在函数外是不起作用的。

\begin{JavaScript}
function test() {
    var message = "hi";
}
console.log(message)    // 出错!
\end{JavaScript}

如果我们需要在函数创建一个全局变量,只需要省略 \texttt{var} 关键字。

\begin{JavaScript}
function test() {
    message = "hi";
}
console.log(message)    // 正确
\end{JavaScript}

\fbox{
    \parbox{0.87\textwidth}{
        \begin{warning}
            虽然省略 \texttt{var} 操作符可以定义全局变量,但是不建议这么做。绝大多数面向对象的高级语言也不建议这么做。
        \end{warning}
    }
}

如果要定义多个变量,可以在一条语句中用逗号分隔各个变量\footnote{这种做法也不建议使用。}。

\noindent\textbf{\texttt{var} 声明提升}

下面 JavaScript 代码不会报错。这是因为使用 \texttt{var} 关键字声明的变量会自动提升到函数作用域顶部:

\begin{JavaScript}
function foo() {
    console.log(age);
    var age = 26;
}
\end{JavaScript}

上述代码等价于:

\begin{JavaScript}
function foo() {
    var age;
    console.log(age);
    age = 26;
}   
\end{JavaScript}

这就是所谓的 "提升"(hoist),也就是把所有变量声明都拉到函数作用域的顶部。此外,反复多次使用 \texttt{var} 声明同一个变量也没有问题。

\begin{JavaScript}
function foo() {
    var age = 16;
    var age = 26;
    var age = 36;
    console.log(age);
}
\end{JavaScript}

\subsubsection{\texttt{let} 声明}

\texttt{let} 和 \texttt{var} 作用差不多, 设计 \texttt{let} 就是为了在大部分情况下取代 \texttt{var}。它们最明显的区别是, \texttt{let} 声明的范围是块作用域,而 \texttt{var} 是函数作用域。\texttt{let} 更符合一般高级语言的语法逻辑。

\begin{JavaScript}
if (true) {
    var name = 'Matt';
    console.log(name);
}
console.log(name);  // 正确
\end{JavaScript}

\begin{JavaScript}
if (true) {
    let name = 'Matt';
    console.log(name);
}
console.log(name);  // 错误
\end{JavaScript}

在这里, \texttt{age} 变量 之所以不能在 \texttt{if} 块之外被引用,是因为它的作用域仅限于该块内部。块作用域是函数作用域的子集,因此适用于 \texttt{var} 的作用域同样适用于 \texttt{let}。

\texttt{let} 也不允许同一个块作用域中出现冗余声明。这样会报错:

\begin{JavaScript}
var name;
var name;   // 出现两次,无误

let age;
let age;    // 出现两次,SyntaxError
\end{JavaScript}

此外,对声明冗余的报错,不会因混用 \texttt{let} 和 \texttt{var} 而受影响。这两个关键字声明的并不是不同类型的变量,它们只是指出变量在相关作用域如何存在。

\noindent\textbf{暂时性死区}

\texttt{let} 和 \texttt{var} 的另一个重要区别是: 使用\texttt{let}声明的变量不会再作用域中被提升。

在解析代码时, JavaScript 引擎会注意到块后面的 \texttt{let} 声明,只不过在此之前不能以任何方式来引用未声明的变量。在 \texttt{let} 声明之前的执行瞬间被称为 ``暂时性死区'',在此阶段引用任何后面才声明的变量都会抛出 \texttt{ReferenceError}。

\noindent\textbf{全局声明}

与 \texttt{var} 关键字不同,使用 \texttt{let} 在全局作用域中声明的变量不会称为 \texttt{windows} 对象的属性( \texttt{var} 声明会 )。

\begin{JavaScript}
var name = 'Matt';
console.log(window.name);   // 'Matt'

let age = 26;
console.log(window.age);    // undefined
\end{JavaScript}

不过 \texttt{let} 声明仍然是在全局作用域中发生的。相应变量会在页面的生命周期内存续。因此,为了避免 \texttt{SyntaxError},必须确保页面不会重复声明同一个变量。

\noindent\textbf{条件声明}

在使用 \texttt{var} 声明变量时,由于声明被提升, JavaScript 引擎会自动将多余的声明在作用于顶部合并。因为 \texttt{let} 的作用域是块,所以不可能检查前面是否已经使用 \texttt{let} 声明郭同名变量,同时也就不可能在没有声明的情况下声明它。

\begin{JavaScript}
<script>
    var name = 'Pionpill';
    let age = 22;
</script>

<script>
    var name = 'Matt';
    // 没有问题,同时被作为一个提升声明来处理
    // 不需要检查之前是否声明过同名变量
    let age = 23;
    // age 之前声明过,报错
</script>
\end{JavaScript}

使用 \texttt{try/catch} 语句或 \texttt{typeof} 操作符也不能解决,因此条件快中 \texttt{let} 声明的作用域仅限于该块。

\noindent\textbf{\texttt{for} 循环中的 \texttt{let} 声明}

在 \texttt{let} 出现之前, \texttt{for} 循环定义的迭代变量会渗透到循环体外部:

\begin{JavaScript}
for(var i=0; i<5; i++) {
    // 循环逻辑
}
console.log(i);     // 5
\end{JavaScript}

改用 \texttt{let} 后就不会出现这个问题:

\begin{JavaScript}
for(let i=0; i<5; i++) {
    // 循环逻辑
}
console.log(i);     // ReferenceError
\end{JavaScript}

此外,由于 \texttt{var} 对应的是函数作用域,在 \texttt{for} 循环,\texttt{for-in},\texttt{for-of} 循环中会出现种种意想不到的问题。

\subsubsection{\texttt{const} 声明}

\texttt{const} 的行为与 \texttt{let} 基本相同,唯一一个重要的区别是它声明的变量必须同时初始化变量,且尝试修改 \texttt{const} 声明的变量会报错。

此外, \texttt{const} 变量不可变本质上是指其引用不可变,并不是引用的对象不可变。

\begin{JavaScript}
const name = "Pionpill";
name = "Matt";      // 报错

const person = {};
person.name = "Matt";   // 正确
\end{JavaScript}

此外,不能用 \texttt{const} 来声明迭代变量(应为迭代变量会自增):
\begin{JavaScript}
for (const i=0; i<10; i++) {}   // TypeError
\end{JavaScript}

不过,如果只想用 const 声明一个不会被修改的 \texttt{for} 循环变量,那是可以的。也就是说,每次迭代只是创建一个新变量。这对 \texttt{for-in},\texttt{for-of} 循环特别有意义:

\begin{JavaScript}
let i = 0;
for(const j=7; i<5; ++i) {
    console.log(j);
}
// 7,7,7,7,7

for(const key in {a:1,b:2}) {
    console.log(key);
}
// a,b

for (const value of [1,2,3,4,5]) {
    console.log(value);
}
// 1,2,3,4,5
\end{JavaScript}

\subsubsection*{如何声明变量}

ES6 加入 \texttt{let} 和 \texttt{const} 关键字,其主要目的就是为了取代 \texttt{var} 关键字。由于 \texttt{var} 声明的变量会出现作用域,升格等问题,它已经在实践中被弃用了。至于为什么还要花大段时间讲 \texttt{var},自然是为了看得懂一些远古代码。

至于 \texttt{const} 关键字,由于浏览器会保持它不变,无论是静态分析还是处理效率上,都要优于 \texttt{let}。当然,使用最多的还是 \texttt{let} 关键字。

总而言之,\textbf{不要使用 \texttt{var},优先使用 \texttt{const}, \texttt{let} 次之。}

\subsection{操作符}

JavaScript 中的基本操作符和 C 语言系中的操作符运算是相同的,相同内容省略。此外,和 Python 一样,JavaScript 是一门十分灵活的语言,比如 \texttt{false == 0} 会返回 \texttt{true},这种数据类型转换在操作符运算中时常会出现。限于操作符类型太多,涉及数据类型转换的操作更多,所以不一一说明,如果遇到问题,读者可通过 node.js 自行检查。

\noindent\textbf{递增递减操作符}

和 C/Java 语言中一样的内容省略。这个操作符往往会降低可读性,不建议使用,Python 就完全抛弃了。

ECMA 中的递增递减操作符除了对数值类型起作用,其他类型也起作用:
\begin{itemize}
    \item 字符串,如果是有效的数值形式,则转换为数值进行计算,变量类型由此变成数值类型。
    \item 字符串,如果不是有效的数值形式,则转换为 \texttt{NaN},变量类型由此变成数值类型。
    \item 布尔型,变成对应的 0/1 再进行计算,变量类型变成数值。
    \item 浮点数,加1或减1。
    \item 对象,调用 \texttt{valueOf()} 方法取得可以操作的值。对得到的值采用上述规则。如果是 \texttt{NaN},调用 \texttt{toString()} 并再次应用其他规则。变量类型变成数值类型。
\end{itemize}

\noindent\textbf{指数操作符}

ES7 新增了指数操作符,其语法和 Python 中的指数操作符相同。

\begin{JavaScript}
console.log(Math.pow(3,2));     // 9
cnosole.log(3**2);              // 9
\end{JavaScript}

不止如此,ECMA还提供了指数赋值操作符 \texttt{**+}:

\begin{JavaScript}
let a = 3;
a **= 2;
console.log(a);     // 9
\end{JavaScript}

\noindent\textbf{全等于运算}

ECMA 中的相等(\texttt{==})符号默认会转换操作数,而全等(\texttt{===})符号则不会。

\begin{JavaScript}
console.log("55" == 55);    // true
console.log("55" === 55);   // false
\end{JavaScript}

与全等对应的还有不全等 (\texttt{!==}) 运算。

\fbox{
    \parbox{0.87\textwidth}{
        \begin{advise}
            多数情况下,由于等于运算涉及类型转换问题,更推荐使用全等运算,这有助于保持数据类型的完整性。
        \end{advise}
    }
}

\subsection{语句}

省略 \texttt{if,while,for} 语句的介绍。

\subsubsection{\texttt{for-in} 语句}

\texttt{for-in} 语句是一种严格的迭代语句,用于枚举对象中的非符号键属性,语法如下:

\begin{JavaScript}
for (property in expression) statement
\end{JavaScript}

下面是一个例子:

\begin{JavaScript}
for (const propName in window) {
    document.write(propName);
}
\end{JavaScript}

这个例子使用 \texttt{for-in} 循环显示了 BOM 对象 \texttt{window} 的所有属性。这里控制语句中的 \texttt{const} 不是必须的,不过为了确保局部变量不被修改,推荐使用 \texttt{const}。

值得说明的是,ECMAScript 中对象的属性是无序的。

\subsubsection{\texttt{for-of} 语句}

\texttt{for-of} 语句是一种严格的迭代语句,用遍历可迭代对象的元素,语法如下:

\begin{JavaScript}
for (property of expression) statement
\end{JavaScript}

下面是示例:

\begin{JavaScript}
for (const el of [2,4,6,8]) {
    document.write(el);
}
\end{JavaScript}

\subsubsection{标签语句}

标签语句用于给语句加标签,语法如下:
\begin{JavaScript}
label: statement
\end{JavaScript}
下面是一个例子:
\begin{JavaScript}
start: for(lei i=0; i<count; i++) {
    console.log(i);
}
\end{JavaScript}

这个语句中的 \texttt{start} 是一个标签,可以在后面通过 \texttt{break} 或 \texttt{continue} 语句引用。标签语句的典型应用场景是嵌套循环。

\subsubsection{\texttt{with} 语句}

\texttt{with} 语句的用途是将代码作用域设置为特定的对象,其语法是:

\begin{JavaScript}
with (expression) statement;
\end{JavaScript}

使用 \texttt{with} 语句的主要场景是针对一个对象反复操作,这时候将代码作用域设置为该对象能提供便利。

\begin{JavaScript}
// 不使用 with 语句
let qs = location.search.substring(1);
let hostname = location.hostname;
let url = location.href;

// 使用 with 语句
with(location) {
    let qs = search.substring(1);
    let hostName = hostname;
    let url = href;
}
\end{JavaScript}

在 \texttt{with} 语句中,每个变量会首先被认为是一个局部变量。如果没有找到局部变量,则会搜索 \texttt{location} 对象,看它是否有同名属性。

\fbox{
    \parbox{0.87\textwidth}{
        \begin{advise}
            严格模式下不允许使用 \texttt{with} 语句,由于 \texttt{with} 语句影响性能且难于调试,所以不建议使用。
        \end{advise}
    }
}

\newpage