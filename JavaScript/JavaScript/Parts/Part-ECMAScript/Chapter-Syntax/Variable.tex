\section{变量,作用域与内存}
\subsection{原始值与引用值}

原始值就是最简单的数据,引用值是由多个值构成的对象。JavaScript 中,6个基本数据类型是原始值,变量按保存的是数据本身,其他数据类型保存的是引用。

与 Java/Python 不同的是,JavaScript 中的 String 类型是基本数据类型,是原始值,对应变量保存数据本身。

\subsubsection*{动态属性}

对于引用值而言,可以随时添加,修改和删除其属性和方法。

\begin{JavaScript}
let person = new Object();
person.name = "Pionpill";
\end{JavaScript}

原始值不能拥有属性,尽管尝试添加属性不会报错(这样做没有意义),但是调用时会报错。

注意,原始类型的初始化可以只使用原始字面量形式。如果使用的是 \texttt{new} 关键字,则 JavaScript 会创建一个 \texttt{Object} 类型的实例,但其行为类似原始值。

\subsubsection*{复制值}

JavaScript 对值复制的方法同 Java 相同,原始值的复制将在内存中创建新的空间,而引用值的复制只是多加了一个引用(指向同一块内存区域)。

\subsubsection*{传递参数}

ECMAScript 中所有函数的参数都是按值传递的。也就是说在函数中会创建一个新的局部变量,是原变量的复制。

下面通过一个例子说明引用值是如何按值传递的。

\begin{JavaScript}
function setName(obj) {
    obj.name = "Pionpill";
    obj = new Object();
    obj.name = "Brandon";
}

let person = new Object();
setName(person)
console.log(person.name);   // "Pionpill"
\end{JavaScript}

上述例子中,最终的输出是 \texttt{"Pionpill"} 而不是 \texttt{"Brandon"},注意我们全局变量中的 \texttt{person} 和 传入函数的(局部变量) \texttt{person} 是两个不同的变量,他们都指向一个 \texttt{Object} 实例。在第二行,局部变量 \texttt{person} 修改了所指向的 \texttt{Object} 实例,创建 \texttt{name} 属性,值为 \texttt{"Pionpill"}。但在第3行,局部变量 \texttt{person} 创建了一个新的 \texttt{Object} 实例,不再指向原先的区域,因此此时再对局部变量 \texttt{person} 进行修改,不会影响到全局的 \texttt{person} 所指向的实例。

\fbox{
    \parbox{0.87\textwidth}{
        \begin{notice}
            上面示例代码仅因为要解释原理给出,实际作用中请不要随意在函数中改变属性或引用对象。
        \end{notice}
    }
}

\subsubsection*{确定类型}

我们知道 \texttt{typeof} 操作符可以用来确定数据类型,但是在面向对象编程中,我们往往需要知道某一实例是什么类型的对象。这时候需要用到 \texttt{instanceof} 操作符。 

\begin{JavaScript}
person instanceof Object;   // person 是 Object 吗?
colors instanceof Array;    // colors 是 Array 吗?
\end{JavaScript}

\fbox{
    \parbox{0.87\textwidth}{
        \begin{notice}
            ECMA-262 规定,任何实现内部 \texttt{[call]} 方法的对象都应该在 \texttt{typeof} 检测时返回 \texttt{"function"}。但并不是所有浏览器都这样做。
        \end{notice}
    }
}

\subsection{上下文与作用域}

\textit{基础的省略}。

需要指出的是,所有通过 \texttt{var} 定义的全局变量和函数都会成为 \texttt{window} 对象的属性和方法。使用 \texttt{let} 和 \texttt{const} 的顶级声明不会定义在全局上下文中,但在作用域链解析上效果是一样的。

\subsubsection*{作用域链增强}

执行上下文主要有全局上下文和函数上下文两种(\texttt{eval() 调用内部存在第三种上下文}),但有其他方式来增强作用域链。某些语句会导致在作用域链前端临时添加一个上下文,这个上下文在实行代码后会被删除:
\begin{itemize}
    \item \texttt{try/catch} 语句的 \texttt{catch} 块
    \item \texttt{with} 语句
\end{itemize}

\subsubsection*{变量声明}

\texttt{var,let,const} 关键字以在前文进行过说明,这里不再累述。

在子块中如果声明和父块同名的变量会掩盖父级变量,也即在子块中对这一变量的调用指的都是子块中的变量。如果父块是全局作用域,则可以使用 \texttt{windows.name} 调用对应变量。

\fbox{
    \parbox{0.87\textwidth}{
        \begin{advise}
            虽然 JavaScript 允许父块和子块中出现同名变量,但不建议这样使用。
        \end{advise}
    }
}

\subsection{垃圾回收}
\subsubsection*{标记清理}

所有现代浏览器采用的 JavaScript 垃圾回收机制都是标记清理。当变量进入上下文时,变量会被加上存在于上下文中的标记。当变量离开上下文时,也会被加上离开上下文的标记\footnote{加标记也是局部变量与全局变量不同的原因。}。

垃圾回收程序运行时,会标记内存中存储的所有变量。然后,它会将所有在上下文中的变量,以及被在上下文中的变量引用的变量的标记去掉。在此之后再被加上标记的变量就是待删除的了。随后垃圾回收程序做一次内存清理,销毁带标记的所有指并回收它们的内存。

\fbox{
    \parbox{0.87\textwidth}{
        \begin{notice}
            还有一种垃圾回收机制:引用计数。不过已经被绝大多数浏览器弃用了。但 Python 的垃圾回收机制采用的是这种方式,有兴趣请读者自行查阅资料。
        \end{notice}
    }
}

垃圾回收机制是涉及浏览器底层的操作,了解这些底层知识对开发者并没有太多的益处,而且,开发者也无法改善这种底层操作(需要浏览器改善)。因此,这里我省略了很多书上的理论内容。

\subsubsection*{内存管理}

出于优化考虑,系统只会给浏览器分配较少的内存。因此将内存占用量保存在一个较小的值可以让页面性能更好。优化内存最好的方法就是只保存必要的数据。如果数据不需要,那么把它设置为 \texttt{null},从而释放其引用。这也可以叫做解除引用。这个做法尤其适用于全局变量,因为局部变量在超出作用域后会自动被解除引用。

\begin{JavaScript}
let person = new Object();
person.name = "Pionpill";

let person = null;      // 解除引用
\end{JavaScript}

不过要注意,解除一个值的引用并不会自动导致相关内存被回收(可能多个变量指向同一对象)。解除引用的关键在于确保相关的值已经不在上下文。因此在下次垃圾回收时可能会将其回收。

\noindent\textbf{使用 \texttt{let,const}}

由于 \texttt{let,const} 关键字在声明变量时是块作用域,因此,在垃圾回收过程中变量会更加快速地被处理,而 \texttt{var} 则相对迟钝。

\noindent\textbf{隐藏类}

隐藏类是一种优化处理,Chrome 浏览器的 V8 引擎在解释后会采用这种优化处理,从而产生隐藏类。

简而言之,如果两个示例相同(构造方法相同,类相同),那么它们可能会指向同一个对象,这样就节省了开辟一个相同对象地空间。如果其中某个对象被改动了,则会智能地再创建一个新对象。这种处理方式和 Python 类似\footnote{我的另一本 《Fluent Python》 笔记有详细解释 Python 的这种做法。}。详细的处理过程请读者自行查阅文献,下面仅给出处理隐藏类的一些建议。

隐藏类固然有好处,但是如果使用不当,这会成为累赘,比如如果我们要创建两个不同的类实例:应该在创建时就区分它们,而不是使用相同的构造函数(这样两个实例名会先指向同一实例,更改后再重新创建实例,步骤多余)。

\begin{JavaScript}
function Article(opt) {
    this.title = "ababababa";
    this.author = opt;
}

let a1 = new Article();
let a2 = new Article("Pionpill");
\end{JavaScript}

此外,如果两个变量共享一个隐藏类,请不要使用 \texttt{delete} 语句修改类属性,这会导致生成相同的隐藏类片段,如果要删除,应该将值设为 \texttt{null}。

\begin{JavaScript}
let a1 = new Article();
let a2 = new Article();

a1.author = null;
\end{JavaScript}

\fbox{
    \parbox{0.87\textwidth}{
        \begin{notice}
            CSDN 上鲜有隐藏类的相关文章,只有极致追求性能的开发者才会关注隐藏类的优化处理。在开发阶段无需过多关于这些内容,等优化阶段再处理。
        \end{notice}
    }
}

\newpage