\section{JavaScript 介绍}
\subsection{历史回顾}

JavaScript 是网景公司率先设计出的,原名 LiveScript,改名是为了蹭 Java 热度。后来微软也曾开发过一款名为 JScript 的脚本语言。这两门语言是现在使用的 JavaScript 的前身。

网景和微软两个公司的 JavaScript 有许多不同之处,这段时间内造成了很多历史遗留问题。后来几个公司打造了 ECMAScript(ES) 这个新的脚本语言标准。JavaScript 才渐渐统一。

JavaScript 和 ECMAScript 基本上是同义词,但实际运用中,JavaScript 包含更宽泛的技术:
\begin{itemize}
    \item 核心:ECMAScript
    \item DOM:文档对象模型
    \item BOM:浏览器对象模型
\end{itemize}

ECMAScript 规定了语言的语法,包括语句,关键字,全局对象等,与一般的高级语言类似。不同的浏览器支持不同版本的 ECMAScript,感谢微软在 2022 年停止了 IE 浏览器的更新,这解决了绝大部分版本兼容问题。

DOM,即文档对象模型(Document Object Model)是一个应用编程接口,用于在 HTML 中使用扩展的 XML。可以将其理解为网页的层次结构,所有的浏览器 DOM 都是相同的。

BOM,即浏览器对象模型(Browser Object Model),用于支持访问和操作浏览器的窗口。

JavaScript 不同于一般的高级程序语言,其版本并是由浏览器决定的,所以在使用之前,最好查看你的浏览器所支持的版本\footnote{现代浏览器(Chrome,FireFox,Edge)都对 JS 有良好的支持,无需关心版本问题。但请不要用 IE浏览器,国产浏览器做开发。}。

\subsection{HTML 中的 JavaScript}
 
将 JavaScript 插入 HTML 的主要方法是使用 \texttt{<script>} 元素,这个元素有8个属性,其用法如下:
\begin{itemize}
    \item \texttt{src}: 可选。表示包含要执行的代码的外部文件。
    \item \texttt{async}: 可选。表示应该立即开始下载脚本,但不能阻止其他页面动作。只对外部脚本文件有效。
    \item \texttt{crossorigin}: 可选。配置相关请求的 CORS(跨源资源共享)设置。默认不使用 CORS。
    \begin{itemize}
        \item \texttt{crossorigin="anonymous"} 配置文件请求不必设置凭据标志。
        \item \texttt{crossorigin="use-credentials"} 设置凭据表指,意味着出站请求会包含凭据。
    \end{itemize}
    \item \texttt{defer}: 可选。表示脚本可以延迟到文档全部被解析和显示之后再执行。只对外部脚本文件有效。
    \item \texttt{integrity}: 可选。允许对比接收到的资源和指定的加密签名以验证自资源完整性。
    \item \texttt{charset}: 可选。使用 \texttt{src} 属性指定的代码字符集。这个属性很少用。
    \item \texttt{type}: 可选,代替 \texttt{language},表示代码块中脚本语言的内容类型,不同浏览器可能有不同值,一般为 "\texttt{text/javascript"}。
    \item \texttt{language}: 废弃。历史遗留问题,不再介绍。
\end{itemize}

\subsubsection*{行内 JavaScript 代码}

嵌入式 JavaScript 直接将代码放在 \texttt{<script>} 元素中就行:

\begin{HTML}
<script>
    function sayHi() {
        console.log("Hi!");
    }
</script>
\end{HTML}

包含在 \texttt{<script>} 内的代码会被从上到下解释。在  \texttt{<script>} 元素中的代码被计算完成之前,浏览器的其余内容不会被加载,也不会被显示。

在使用行内 JavaScript 代码时,注意代码中不能出现字符串  \texttt{<\textbackslash script>},因为这将被识别为脚本结束标签。如果一定要使用,需要转义字符 \texttt{"\textbackslash"}。

\begin{HTML}
<script>
    function sayHi() {
        console.log("<\/script>");
    }
</script>
\end{HTML}

\subsubsection*{外部 JavaScript 代码}

要包含外部文件中的 JavaScript,就必须使用 \texttt{src} 属性。这个属性值是一个 URL,指向包含 JavaScript 代码的文件。

\begin{HTML}
<script src = "example.js"></script>
\end{HTML}

与解释行内 JavaScript 一样,在解释外部 JavaScript 文件时,页面也会阻塞(包含下载时间)。

\fbox{
    \parbox{0.87\textwidth}{ 
        \begin{notice}
            按照惯例,外部 JavaScript 文件的扩展名是 .js。但这不是必需的,因为浏览器不会检查扩展名。这就为使用服务器端脚本语言动态生成 JavaScript 代码,或在浏览器中将 JavaScript 扩展语言(如 TypeScript)转译为 JavaScript 提供了可能性。
        \end{notice}
    }
}

使用了 \texttt{src} 属性的 \texttt{<script>} 元素不应在标签中再包含其他 JavaScript 代码。如果有,浏览器会直接忽略行内代码。

此外, \texttt{src} 属性可以包含来自外部域的 JavaScript 文件。

\begin{HTML}
<script src="http://somewhere//example.js"></script>
\end{HTML}

使用外部域的脚本文件时,会产生许多安全问题,除非对齐安全性有绝对的信息,否则不要随意使用。

\fbox{
    \parbox{0.87\textwidth}{
        \begin{advise}
            在实际运用中,一般都会选择外部文件来书写代码,这在可维护性,缓存,兼容等方面都要优于行内代码。
        \end{advise}
    }
}

\subsubsection*{标签位置}

过去,所有的 \texttt{<script>} 元素都被放在页面的 \texttt{<head>} 标签内,这样做会导致许多问题。脚本文件必须从上到下一个个被执行后,页面才能开始加载。这会导致页面渲染明显延迟,现代的许多脚本文件都被放在了 \texttt{<body>} 标签中。

\subsubsection*{执行脚本}

HTML 提供了两个属性用于推迟或异步执行脚本:

\noindent\textbf{\texttt{defer} 属性:推迟执行}

这个属性表示脚本在执行的时候不会改变页面的结构。也就是说,脚本会被延迟到整个页面解析完毕后再执行。在 \texttt{<script>} 元素上设置 \texttt{defer} 属性,相当于告诉浏览器立即下载,但延迟执行。

\begin{HTML}
<!DOCTYPE html>
<html lang="en">
  <head>
    <title> Example </title>
    <script defer src="example1.js"> </script>
    <script defer src="example2.js"> </script>
  </head>
  <body></body>
</html>
\end{HTML}

虽然上述例子中的 \texttt{<script>} 元素包含在页面的 \texttt{<head>} 中,但它们会在浏览器解析到结束时才执行。

\fbox{
    \parbox{0.87\textwidth}{
        \begin{advise}
            由于可能会产生顺序错误,一般建议只包含一个含 \texttt{defer} 属性的外部脚本。
        \end{advise}
    }
}

\noindent\textbf{\texttt{async} 属性:异步执行}

从改变脚本处理的方式上来看, \texttt{async} 属性与 \texttt{defer} 类似。不过 \texttt{async} 属性标记的脚本并不保证能按照它们出现的顺序来执行。

给脚本添加 \texttt{async} 属性的目的时告诉浏览器,不必等脚本下载和执行完后再加载页面,同样也不必等到该一部脚本下载和执行后再加载其他脚本。正因为如此,异步脚本不应该在加载期间修改 DOM。

\fbox{
    \parbox{0.87\textwidth}{
        \begin{advise}
            虽然存在这个属性,但一般不推荐使用。
        \end{advise}
    }
}

\subsubsection*{动态加载脚本}

除了 \texttt{<script>} 标签,还有其他方式加载脚本。因为 JavaScript 可以使用 DOM API,所以通过向 DOM 中动态添加 \texttt{script} 元素同样可以加载指定的脚本。只要创建一个 \texttt{script} 元素并将其添加到 DOM 即可。

\begin{JavaScript}
let script = document.createElement('script');
script.src = 'gibberish.js';
document.head.appendChild(script);
\end{JavaScript}

默认情况下,以这种方式创建的 \texttt{<script>} 元素是以异步方式加载的,相当于添加了 \texttt{async} 属性。不过这样做可能会有问题。开发者可以选择关闭 \texttt{async} 属性。

\begin{JavaScript}
script.async = false;
\end{JavaScript}

以这种方式获取的资源对六拉你去预加载器是不可见的。这会严重影响它们在资源获取队列中的优先级。这可能会严重影响性能。要想让预加载器直到这些动态请求文件的存在,可以在文档头部显示声明它们。

\begin{HTML}
<link rel="preload" href="gibberish.js">
\end{HTML}

\subsubsection*{\texttt{<noscript>}元素}

\texttt{<noscript>} 元素主要用于解决早期浏览器不支持 JavaScript 的问题,在以下两种情况下,将显示 <noscript> 中的内容:

\begin{itemize}
    \item 浏览器不支持脚本。
    \item 浏览器对脚本的支持被关闭。
\end{itemize}

若是不满足,则不会渲染对应内容。下面是一个使用例子:

\begin{JavaScript}
<!DOCTYPE html>
<html lang="en">
  <head>
    <meta charset="UTF-8" />
    <script defer="defer" src="exmaple1.js"></script>
    <script defer="defer" src="exmaple2.js"></script>
    <title>Document</title>
  </head>
  <body>
    <noscript>
      <p>This page needs a JavaScript-enabled browser.</p>
    </noscript>
  </body>
</html>
\end{JavaScript}

\newpage