\documentclass{PionpillNote-book}

\title{Three.js}
\author{
    Pionpill \footnote{笔名:北岸,电子邮件:673486387@qq.com,Github:\url{https://github.com/Pionpill}} \\
    Three.js 及相关 WebGL 库学习笔记\\
}

\date{\today}

\begin{document}

\pagestyle{plain}
\maketitle

\noindent\textbf{前言:}

本文阅读前提: 具备良好的 HTML,CSS,JS 基础,看得懂 React(hooks),TS 写法, 有一定的3D建模基础。此外,本人使用 Blender(一款开源3D建模与渲染软件) 获取或修改一些3D模型。

由于本人有三维建模基础,因此一些基础概念一带而过,如果读者不了解,可以先尝试学习一下 Blender。此外,3D 建模与渲染是一门很深的学问,本文可能涉及到材质漫射,高光,点光源,面光源,粒子,动画等专业知识。这是 3D 领域的理论知识,但不属于 three.js 技术层面,因此本文不会对这些概念做解释,读者如有问题,请自行查阅资料。本人强烈建议在系统学习 WebGL 之前,入门一下 Blender 或其他三维软件。

本文主要参考文献:
\begin{itemize}
    \item Three.js 官方文档: \url{https://threejs.org/docs}
    \item Three.js 源码仓库: \url{https://github.com/mrdoob/three.js}
\end{itemize}

本文撰写环境:

\begin{itemize}
    \item React: 18.2.0
    \item IDE: VSCode 1.72 
    \item Edge: 111.0
    \item Chrome: 91.0
    \item Node.js: 18.12
    \item OS: Window11
\end{itemize}

\date{\today}
\newpage

\tableofcontents

\newpage

\setcounter{page}{1} 
\pagestyle{fancy}

\import{Contents}{style.tex}
\part{Three.js}
\chapter{Three.js 基础概念}
\import{Contents/Part-Three/Chapter-Basic}{introduction.tex}
\import{Contents/Part-Three/Chapter-Basic}{basic.tex}
\chapter{材质,灯光与几何体}
\import{Contents/Part-Three/Chapter-Scene}{light.tex}
\import{Contents/Part-Three/Chapter-Scene}{material.tex}
\import{Contents/Part-Three/Chapter-Scene}{geometry.tex}
\part{拓展包}
\chapter{@react-three}
\import{Contents/Part-Package/Chapter-React-Three}{fiber.tex}

\end{document}