\section{@react-three/fiber}

@react-three/fiber(本章简称 fiber) 是目前最主流的在 React 框架上使用 Three.js 的方案。同时也是 @react-three 生态最重要的项目。
\begin{itemize}
  \item \link{https://docs.pmnd.rs/react-three-fiber}{官方文档}
\end{itemize}

本节开始需要具备 React 基础知识。

\begin{bash}
npm install three @types/three @react-three/fiber
\end{bash}

\subsection{入门项目}

\subsubsection*{\texttt{<Canvas>} 标签}

fiber 创建的组件需要用 \texttt{<Canvas>} 标签包裹,该标签有以下两个作用:
\begin{itemize}
  \item 设置一个默认的 Scene 和 Camera。
  \item 自动每帧渲染,无需手动书写渲染循环。
\end{itemize}

\begin{JavaScript}
<div>
  <Canvas/>
</div>
\end{JavaScript}

此外,\texttt{<Canvas>} 标签有如下效果:
\begin{itemize}
  \item 创建的 Three 组件长宽由父元素决定。
  \item 动画由 canvas 技术支持。
\end{itemize}

\subsubsection*{\texttt{<mesh>} 标签}

\texttt{<mesh>} 标签是一个原生标签(所有的 three.js 对象会被自动转换为原生 JSX 标签),它就相当于使用 \texttt{Three.Mesh()} 创建对象。

\begin{JavaScript}
<Canvas>
  <mesh/>
</Canvas>
\end{JavaScript}

我们可以用 React 风格的代码为 \texttt{<mesh/>} 标签设置一些属性:

\begin{JavaScript}
<Canvas>
  <mesh position={[0,0,0]}/>
</Canvas>
\end{JavaScript}

此外,单独的 \texttt{<mesh/>} 标签并不会显示什么,我们需要传入几何体和材质。以子组件的方式传入:

\begin{JavaScript}
<Canvas>
  <mesh>
    <boxGeometry />
    <meshStandardMaterial />
  </mesh>
</Canvas>
\end{JavaScript}

这段代码就相当于:

\begin{JavaScript}
const scene = new THREE.Scene()
const camera = new THREE.PerspectiveCamera(75, width / height, 0.1, 1000)

const renderer = new THREE.WebGLRenderer()
renderer.setSize(width, height)
document.querySelector('#canvas-container').appendChild(renderer.domElement)

const mesh = new THREE.Mesh()
mesh.geometry = new THREE.BoxGeometry()
mesh.material = new THREE.MeshStandardMaterial()

scene.add(mesh)

function animate() {
  requestAnimationFrame(animate)
  renderer.render(scene, camera)
}

animate()
\end{JavaScript}

此外,我们知道有些对象的构造函数必须传入一些参数,使用标签的 \texttt{args} 属性即可:

\begin{JavaScript}
// 原生 JS 写法
new THREE.BoxGeometry(2, 2, 2)
// React 组件写法
<boxGeometry args={[2, 2, 2]} />
\end{JavaScript}

此外,我们还需要加入光照:

\begin{JavaScript}
// 原生 JS 写法
const light = new THREE.AmbientLight()
light.intensity = 0.1
// React 组件写法
<ambientLight intensity={0.1} />
\end{JavaScript}

或者加入聚光灯:

\begin{JavaScript}
// 原生 JS 写法
const light = new THREE.DirectionalLight()
light.position.set(0, 0, 5)
light.color.set('red')
// React 组件写法
<directionalLight position={[0, 0, 5]} color="red" />
\end{JavaScript}

\subsection{基础 API}

\subsubsection*{\texttt{<Canvas>}}

\texttt{<Canvas>} 是 fiber 最基础最核心的标签,几乎所有的 fiber 组件都需要在 \texttt{<Canvas>} 标签(包括 hooks)内书写。

\texttt{<Canvas>} 有几个重要属性这里说明一下:
\begin{itemize}
  \item \texttt{camera}: 传入一个 \texttt{Three.Camera} 替代原本的相机或者传入相机构造函数参数对象。默认为 \texttt{{fov:75, near:0.1, far:1000, position:[0,0,5]}}。
  \item \texttt{orthographic}: 布尔值,如果为 \texttt{true} 则使用平视摄像机,默认使用透视。
  \item \texttt{scene}: 可以传入一个自定义 \texttt{Scene} 对象。
  \item \texttt{shadows}: 布尔值,决定是否渲染阴影。
\end{itemize}

\texttt{<Canvas>} 采用 \texttt{createRoot} 创建一个 \texttt{THREE.WebGLRenderer} 对象。同时创建摄像机,场景。如果 \texttt{shadows} 为 \texttt{true} 同时创建阴影。

\subsubsection*{创建对象}

原生 JSX 创建对象的过程为: 构造函数 \texttt{new} 对象,为对象设置属性,将对象加入到场景。

这三个步骤在 fiber 中被简化为:
\begin{itemize}
  \item 构造函数: 使用 \texttt{args} 传入一个数组即可。
  \item 设置属性: 利用现有的标签属性设置,例如 \texttt{position}。
  \item 加入场景: 作为父标签 \texttt{children} 属性即可。
\end{itemize}

对于构造函数,并不是所有对象都一定要传入 \texttt{args},如果构造函数没有参数或者参数都有默认值,可以不传入 \texttt{args}。

对于属性,所有存在 \texttt{.set()} 方法的属性均可以直接用过标签传入参数。

如果子标签是父标签的参数,可以通过 \texttt{attach} 属性传入,属性值即为副标签对象的属性名:

\begin{JavaScript}
<mesh>
  <meshBasicMaterial attach="material">
  <boxGeometry attach="geometry">
</mesh>
\end{JavaScript}

所有继承自 \texttt{THREE.Material} 的标签,\texttt{attach} 都默认为 \texttt{material}。同样,继承自 \texttt{THREE.BufferGeometry} 的标签 \texttt{attach} 默认为 \texttt{geometry}。因此,多数情况下我们没必要显式指定 \texttt{attach} 值。

\subsection{Hooks}

Hooks 仅能用在 \texttt{<Canvas>} 标签内部,下面写法是错误的:

\begin{JavaScript}
import { useThree } from '@react-three/fiber'

function App() {
  const { size } = useThree() // This will just crash
  return (
    <Canvas>
      <mesh>
\end{JavaScript}

正确的写法是:

\begin{JavaScript}
function Foo() {
  const { size } = useThree()
  ...
}

function App() {
  return (
    <Canvas>
      <Foo />
\end{JavaScript}

\subsubsection*{\texttt{useThree}}

\texttt{useThree} 允许我们获取当前 three.js 组件的状态,包括场景,相机等。

\begin{JavaScript}
// 获取相机
const camera = useThree((state) => state.camera)
// 获取视口
const viewport = useThree((state) => state.viewport)
\end{JavaScript}

我们只能获取 \texttt{state} 的属性,例如 camera,get,set。无法获取 \texttt{state} 属性的属性,例如 camera.zoom。

\subsubsection*{\texttt{useFrame}}

\texttt{useFrame} 用于在每次渲染帧内调用,该钩子声明如下:

\begin{JavaScript}
useFrame(callback: RenderCallback, renderPriority?: number): null
\end{JavaScript}

\texttt{renderPriority} 表示渲染优先级,如果为负值,表示不会再渲染循环中执行,只执行一次,此时数值越小,执行优先级越高。

回调函数 \texttt{callback} 声明如下:

\begin{JavaScript}
(state: RootState, delta: number, frame?: THREE.XRFrame) => void
\end{JavaScript}

其中,\texttt{delta} 表示子渲染其到现在的时间(单位: 毫秒)。

\subsubsection*{\texttt{useLoader}}

\texttt{useLoader} 用于加载外部资源,该钩子接收三个参数: 加载器,资源对应 URL,回调函数(用于自定义加载器)。

\subsubsection*{\texttt{useGraph}}

该钩子用于获取 \texttt{Object3D} 对象的几何形状,材质等。