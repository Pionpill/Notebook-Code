\section{基础对象}

\subsection{\texttt{Object3D}}

在 three.js 中,绝大部分对象都继承自 \texttt{THREE.Object3D} 对象。\texttt{THREE.Object3D} 对象十分复杂,这里只讲常用的。
\begin{itemize}
  \item \link{https://threejs.org/docs/index.html\#api/zh/core/Object3D}{查看文档}。
  \item \link{https://github.com/mrdoob/three.js/blob/master/src/core/Object3D.js}{查看 \texttt{Object3D} 源码}。
\end{itemize}

\subsubsection*{构造函数}

\texttt{Object3D} 的构造函数无需传入任何参数,在对象被构造时,会生成自己独特的的 uuid,为属性赋默认值。

\subsubsection*{属性}

\texttt{Object3D} 的属性非常多,常用的实例属性有:

\begin{table}[H]
  \small
  \centering
  \caption{\texttt{Object} 属性}
  \setlength{\tabcolsep}{4mm}
  \begin{tabular}{l|l|l|l}
    \toprule
    \textbf{属性} & \textbf{类型} & \textbf{默认值} & \textbf{含义} \\
    \midrule
    castShadow & Boolean & false & 是否被渲染到阴影贴图中 \\
    receiveShadow & Boolean & false & 材质是否接收阴影 \\
    renderOrder & Number & 0 & 渲染顺序 \\
    parent & Object3D & 父对象 & 只能有一个 \\
    children & Array & 子对象数组 & 很少用,一般用 Group 代替 \\
    id & Integer & - & 唯一标识,具体实现看源码 \\
    uuid & String & - & 自生成的 uuid \\
    isObject3D  & Boolean & true & 只读,用于判断类型 \\
    layers & Layers & new Layers() & 层级关系 \\
    matrix & Matrix4 & new Matrix4() & 局部变化矩阵 \\
    matrixWorld  & Matrix4 & new Matrix4() & 世界变化矩阵 \\
    name & String & ``'' & 自行设置,默认为空 \\
    onAfterRender & Function & 空 & 可选的回调函数 \\
    onBeforeRender & Function & 空 & 可选的回调函数 \\
    position & Vector3 & (0,0,0) & 默认位置 \\
    rotation & Euler & new Euler() & 物体局部旋转 \\
    scale & Vector3 & (1,1,1) & 物体局部缩放 \\
    up & Vector3 & (0,1,0) & 物体朝向 \\
    userData & Object & \{\} & 物体自身携带的数据 \\
    visible & Boolean & true & 是否被渲染 \\
    \bottomrule
  \end{tabular}
\end{table}

在 three.js 对象中有一些通用的属性方法。例如: \texttt{type}, \texttt{isXXX},\texttt{name} 用于判断对象类型。再例如 \texttt{copy} 进行深拷贝,\texttt{set/getXXX} 进行属性设置。下文不再提及这些通用属性。

此外有三个静态属性:

\begin{table}[H]
  \small
  \centering
  \caption{\texttt{Object3D} 静态属性}
  \setlength{\tabcolsep}{4mm}
  \begin{tabular}{l|l|l|l}
    \toprule
    \textbf{属性} & \textbf{类型} & \textbf{默认值} & \textbf{含义} \\
    \midrule
    DEFAULT\_UP & Vector3 & (0,1,0) & 默认朝向 \\
    DEFAULT\_MATRIX\_AUTO\_UPDATE & Boolean & false & 是否进行相关更新 \\
    DEFAULT\_MATRIX\_WORLD\_AUTO\_UPDATE   & Boolean & false & 是否进行相关更新 \\
    \bottomrule
  \end{tabular}
\end{table}

\subsubsection*{方法}

\begin{itemize}
  \item \texttt{.add ( object : Object3D, ... ) : this}: 添加对象到这个对象的子级,可以添加任意个。如果被添加的对象已有父对象,则该父对象被删除。
  \item \texttt{.attach ( object : Object3D ) : this}: 将object作为子级来添加到该对象中,同时保持该object的世界变换。
  \item \texttt{.clone ( recursive : Boolean ) : Object3D}: 如果传入的值为 \texttt{true},则该对象的子对象也会被克隆,默认为 \texttt{true}。
  \item \texttt{.copy ( object : Object3D, recursive : Boolean ) : this}: 复制给定的对象到这个对象中。 事件监听器和用户定义的回调函数不会被复制。
  \item \texttt{.remove ( object : Object3D, ... ) : this}: 从当前对象的子级中移除对象。可以移除任意数量的对象。
  \item \texttt{.rotateOnAxis ( axis : Vector3, angle : Float ) : this}: 在局部空间中绕着该物体的轴来旋转一个物体。
  \item \texttt{.rotateOnWorldAxis ( axis : Vector3, angle : Float ) : this}: 在全局空间中旋转。
  \item \texttt{.rotateX/Y/Z ( rad : Float ) : this}: 绕局部空间的轴旋转。
  \item \texttt{.translateX ( distance : Float ) : this}: 绕局部空间的轴平移。
  \item \texttt{.toJSON ( meta : Object ) : Object}: 转换为 three.js 特有的 JSON 格式。
\end{itemize}

此外,还有一个常规的 \texttt{get,set} 方法用于操作属性。

three.js 还提供了一个专门用于管理对象结构的类: \texttt{Group}, 实现非常简单:

\begin{JavaScript}
class Group extends Object3D {
	constructor() {
		super();
		this.isGroup = true;
		this.type = 'Group';
	}
}
\end{JavaScript}

\subsection{\texttt{Scene} 相关}

\begin{JavaScript}
class Scene extends Object3D
\end{JavaScript}

\texttt{THREE.Scene} 对象是所有不同对象的容器。它用来保存所有图形场景的必要信息。

\begin{itemize}
  \item \link{https://threejs.org/docs/index.html\#api/zh/scenes/Scene}{查看文档}。
  \item \link{https://github.com/mrdoob/three.js/blob/master/src/scenes/Scene.js}{查看 \texttt{Object3D} 源码}。
\end{itemize}

\texttt{Scene} 的构造函数也无需传入参数,场景添加了几个重要属性:

\begin{table}[H]
  \small
  \centering
  \caption{\texttt{Scene} 属性}
  \setlength{\tabcolsep}{4mm}
  \begin{tabular}{l|l|l|p{8cm}}
    \toprule
    \textbf{属性} & \textbf{类型} & \textbf{默认值} & \textbf{含义} \\
    \midrule
    background & Object & null & 可以是 \texttt{Color}, \texttt{Texture},\texttt{CubeTexture} \\
    backgroundBlurriness & Float & 0 & 背景模糊程度 \\
    environment & Texture & null & 该纹理贴图将会被设为场景中所有物理材质的环境贴图。 \\
    fog & Fog & null & 影响场景中的每个物体的雾的类型。 \\
    overrideMaterial & Material & null & 若不为空,强制使用该材质进行渲染 \\
    \bottomrule
  \end{tabular}
\end{table}

同属于场景对象的类还有两个: \texttt{Fog}, \texttt{FogExp2}。

\subsubsection*{\texttt{Fog}}

\begin{JavaScript}
class Fog
\end{JavaScript}

\texttt{Fog} 表示线性变换的雾效,它的构造函数函数需要传入三个参数:

\begin{JavaScript}
constructor( color, near = 1, far = 1000 )
\end{JavaScript}

这些属性的作用如下:

\begin{table}[H]
  \centering
  \caption{\texttt{Fog} 属性}
  \setlength{\tabcolsep}{4mm}
  \begin{tabular}{l|l|l|p{8cm}}
    \toprule
    \textbf{属性} & \textbf{类型} & \textbf{默认值} & \textbf{含义} \\
    \midrule
    color & Color & 构造函数赋值 & 雾的颜色 \\
    near & Float & 1 & 应用雾的最小距离 \\
    far & Float & 1000 & 应用雾的最大距离 \\
    \bottomrule
  \end{tabular}
\end{table}

\subsubsection*{\texttt{FogExp2}}

\begin{JavaScript}
class FogExp2
\end{JavaScript}

\texttt{FogExp} 表示指数变换的雾效,它的构造函数函数需要传入两个参数:

\begin{JavaScript}
constructor( color, density = 0.00025 )
\end{JavaScript}

其中 \texttt{density} 表示雾密度增长速度。

\subsection{\texttt{Camera} 相关}

three.js 为我们提供了很多 \texttt{Camera} 类型,它们都继承自抽象基类: \texttt{Camera}。

\begin{itemize}
  \item \link{https://threejs.org/docs/index.html\#api/zh/cameras/Camera}{查看文档}
  \item \link{https://github.com/mrdoob/three.js/blob/master/src/cameras/Camera.js}{查看源码}
\end{itemize}

\begin{JavaScript}
class Camera extends Object3D
\end{JavaScript}

由于是抽象基类,我们一般不会直接构造 \texttt{Camera} 对象。\texttt{Camera} 本省并没有实现实际的功能,仅仅是添加了相机公有的一些属性与标识。

我们常用的相机有两类: \texttt{PerspectiveCamera} 透视摄像机,\texttt{OrthographicCamera} 正交摄像机。

\subsubsection*{\texttt{PerspectiveCamera}}

透视摄像机用于模拟人眼看到的景象。

\begin{JavaScript}
class PerspectiveCamera extends Camera
\end{JavaScript}

在构造函数中我们需要传入以下几个值:

\begin{JavaScript}
constructor( fov = 50, aspect = 1, near = 0.1, far = 2000 )
\end{JavaScript}

其中,\texttt{fov} 表示摄像机视锥体垂直视野角度,\texttt{aspect} 表示摄像机视锥体长宽比。

除此之外,还有几个重要的属性:

\begin{table}[H]
  \centering
  \small
  \caption{\texttt{PerspectiveCamera} 属性}
  \setlength{\tabcolsep}{4mm}
  \begin{tabular}{l|l|l|l}
    \toprule
    \textbf{属性} & \textbf{类型} & \textbf{默认值} & \textbf{含义} \\
    \midrule
    focus & Focus & 10 & 焦点,用于设置景深 \\
    view & Object & null &  \\
    zoom & number & 1 & 设置摄像机缩放倍数 \\
    \bottomrule
  \end{tabular}
\end{table}

\texttt{PerspectiveCamera} 基本都是 \texttt{get,set} 方法,没什么好讲的。

正交摄像机的投影大小与物品距离相机远近无关。

\subsubsection*{\texttt{OrthographicCamera}}

\begin{JavaScript}
class OrthographicCamera extends Camera
\end{JavaScript}

它的构造函数也不存在 fov 值:

\begin{JavaScript}
constructor( left = - 1, right = 1, top = 1, bottom = - 1, near = 0.1, far = 2000 )
\end{JavaScript}

其中上下左右分别达标锥体侧边与中心的距离。其属性与方法和透视相机类型。

此外,还有几个不常用的相机类,做过摄影相关工作的人应该了解,不了解也没关系,不常用:
\begin{itemize}
  \item \texttt{ArrayCamera}: 用于包含一组子相机,继承自 \texttt{PerspectiveCamera}。
  \item \texttt{CubeCamera}: 用于同时创建 6 个正交相机。
  \item \texttt{StereoCamera}: 用于同时创建 2 个正交相机。
\end{itemize}

\newpage