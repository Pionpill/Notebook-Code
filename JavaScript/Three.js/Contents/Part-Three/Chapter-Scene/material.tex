\section{材质}
\subsection{基础材质}

three.js 的材质基类是 \texttt{Material}:

\begin{JavaScript}
class Material extends EventDispatcher
\end{JavaScript}

\texttt{Material} 继承自 \texttt{EventDispatch},实现了基本的事件发布模式。

\texttt{Material} 及其子类的属性比较多,除了通用的 \texttt{uuid} 属性,可以分为以下几类:
\begin{itemize}
  \item 基础属性: 最常用的,控制材质不透明度等显示方式。
  \item 融合属性: 决定物体与背景如何融合。
  \item 高级属性: 控制底层 WebGL 上下文对象渲染物体的方式。
\end{itemize}

\begin{table}[H]
  \centering
  \small
  \caption{材质属性}
  \label{table:材质属性}
  \setlength{\tabcolsep}{4mm}
  \begin{tabular}{l|l|l|l|l}
    \toprule
    \textbf{类别} & \textbf{属性} & \textbf{类型} & \textbf{默认值} & \textbf{含义} \\
    \midrule
    \multirow{9}{*}{基础} & opacity & Float & 1 & 不透明度,影响颜色或材质本身 \\
     & transparent & Boolean & false & 材质是否透明 \\
     & visible & Boolean & false & 是否可见 \\
     & override & Boolean & false & 是否渲染得大一些 \\
     & side & Integer & THREE.FrontSide & 指定渲染哪一面 \\
     & colorWrite & Boolean & true & 如果 false, 则不渲染 \\
     & flatShading & Boolean & false & 是否开启平滑作色 \\
     & lights & Boolean & true & 是否接收光照 \\
     & dithering & Boolean & false & 是否开启颜色抖动 \\
    \midrule
    \multirow{2}{*}{融合} & blending & Blending & NormalBlending &  材质如何进行融合 \\
     & blendSrc & Integer & SrcAlphaFactor & 融合源 \\
    \bottomrule
  \end{tabular}
\end{table}

融合属性还有很多,这里不一一列举。

\texttt{Material} 有一个重要的方法 \texttt{setValues},接受一个对象作为参数,该方法会遍历该对象并将可用的属性加载到自身。

\subsubsection*{\texttt{MeshBasicMaterial}}

\texttt{MeshBasicMaterial} 不考虑场景中光照的影响。使用该材质的网格会被渲染成简单的平面多边形,或者几何体线框。

\begin{JavaScript}
class MeshBasicMaterial extends Material {
	constructor( parameters )
  ... 
}
\end{JavaScript}

其中 \texttt{parameters} 是一个对象,传入可用的参数。

除了公有属性,\texttt{MeshBasicMaterial} 还有以下特殊属性:

\begin{table}[H]
  \centering
  \small
  \caption{\texttt{MeshBasicMaterial} 属性}
  \setlength{\tabcolsep}{4mm}
  \begin{tabular}{l|l|l|l}
    \toprule
    \textbf{属性} & \textbf{类型} & \textbf{默认值} & \textbf{含义} \\
    \midrule
    color & Color & 0xffffff & 颜色 \\
    wireframe & Boolean & false & 渲染成线框 \\
    wireframeLinewidth & Float & 1 & 线框宽度 \\
    wireframeLinecap & String & "round" & 线两端样式, 可选 "butt", "square" \\
    wireframeLinejoin & String & "round" & 线连接点样式,可选 "bevel", "miter" \\
    \bottomrule
  \end{tabular}
\end{table}

\subsubsection*{\texttt{MeshDepthMaterial}}

\texttt{MeshDepthMaterial} 的外观不是由光照或某个材质属性决定,而是由物体到摄像机的距离决定。

\texttt{MeshDepthMaterial} 构造函数和 \texttt{MeshBasicMaterial} 相同。存在两个特殊属性: \texttt{wireframe}, \texttt{wireframeLinewidth}。

在场景中,相机的 \texttt{near} 和 \texttt{far} 属性决定了使用该材质物体的显示效果。

\texttt{MeshDepthMaterial} 没有属性来设置方块颜色,这时候可以使用联合材质创建新效果。

\begin{JavaScript}
const cubeMaterial = new THREE.MeshDepthMaterial();
const colorMaterial = new THREE.MeshBasicMaterial({color: 0x00ff00, transparent: true, bleeding: THREE.MultiplyBlending});
const cube = new THREE.SceneUtils.createMultiMaterialObject(cubeGeometry, [colorMaterial, cubeMaterial]);
\end{JavaScript}

联合材质对 \texttt{MeshDepthMaterial} 没什么要求,对于 \texttt{MeshBasicMaterial} 则需要将 \texttt{transparent} 属性设置为 \texttt{true} 来促使 three.js 检查 \texttt{blending} 属性。这样才能让材质融合。

\subsubsection*{\texttt{MeshNormalMaterial}}

构造方法和上面两个类似,每个面显示的颜色与面法向量相关。

\subsection{高级材质}

\subsubsection*{\texttt{MeshLambertMaterial}}

用来创建暗淡的并不光亮的表面。该材质非常易用,而且会对场景中的光源产生反应。该材质有一个属性 \texttt{emissive : Color} 表示材质自发光的颜色。

\subsubsection*{\texttt{MeshPhongMaterial}}

用来创建具有镜面高光效果的材质。该材质有以下几个特殊属性:

\begin{table}[H]
  \centering
  \small
  \caption{\texttt{MeshPhongMaterial} 属性}
  \setlength{\tabcolsep}{4mm}
  \begin{tabular}{l|l|l|l}
    \toprule
    \textbf{属性} & \textbf{类型} & \textbf{默认值} & \textbf{含义} \\
    \midrule
    emissive & Color & 0x000000 & 自发光颜色 \\
    specular & Color & 0x111111 & 高光颜色 \\
    shininess & Float & 30 & 高光程度 \\
    \bottomrule
  \end{tabular}
\end{table}

\subsubsection*{\texttt{MeshStandardMaterial}}

这是新版 three.js 提供的材质,用于更加正确地计算物体表面与光源的互动。这种材质不但能够更好地表现塑料质感和金属质感,而且提供了两个全新的属性:

\begin{table}[H]
  \centering
  \small
  \caption{\texttt{MeshStandardMaterial} 属性}
  \setlength{\tabcolsep}{4mm}
  \begin{tabular}{l|l|l|l}
    \toprule
    \textbf{属性} & \textbf{类型} & \textbf{默认值} & \textbf{含义} \\
    \midrule
    metalness & Float & 0 & 金属感程度,默认非金属使用0,金属使用1。中间值模拟生锈金属。 \\
    roughness & Float & 1 & 粗糙程度,0 表示完全镜面反射,1表示完全漫反射 \\
    \bottomrule
  \end{tabular}
\end{table}

\subsubsection*{\texttt{MeshPhysicalMaterial}}

该材质继承自 \texttt{MeshStandardMaterial}:

\begin{JavaScript}
class MeshPhysicalMaterial extends MeshStandardMaterial
\end{JavaScript}

在 \texttt{MeshStandardMaterial} 基础上增添了如下属性:

\begin{table}[H]
  \centering
  \small
  \caption{\texttt{MeshPhysicalMaterial} 属性}
  \setlength{\tabcolsep}{4mm}
  \begin{tabular}{l|l|l|l}
    \toprule
    \textbf{属性} & \textbf{类型} & \textbf{默认值} & \textbf{含义} \\
    \midrule
    clearCoat & Float & 0 & 控制表面涂层明显程度 \\
    clearCoatRoughness & Float & 0 & 表面涂层粗糙程度 \\
    reflectivity & Float & 0.5 & 表面材质反射率 \\
    \bottomrule
  \end{tabular}
\end{table}

\subsection{线性几何体材质}

线性几何体材质只能适用于特比的几何体: \texttt{THREE.Line}。顾名思义,这只是一条线,没有面。three.js 提供了两种可用于线段的不同材质。

\subsubsection*{\texttt{LineBasicMaterial}}

\texttt{LineBasicMaterial} 提供了以下可用的材质属性:

\begin{table}[H]
  \centering
  \small
  \caption{\texttt{MeshPhysicalMaterial} 属性}
  \setlength{\tabcolsep}{4mm}
  \begin{tabular}{l|l|l|l}
    \toprule
    \textbf{属性} & \textbf{类型} & \textbf{默认值} & \textbf{含义} \\
    \midrule
    color & Color & 0xffffff & 材质颜色 \\
    linewidth & Float & 1 & 线宽 \\
    linecap & String & `round' & 线两端的样式 \\
    linejoin & String & `round' & 线连接点的样式 \\
    \bottomrule
  \end{tabular}
\end{table}

\subsubsection*{\texttt{LineDashedMaterial}}

\texttt{LineDashedMaterial} 继承自 \texttt{LineBasicMaterial},增添了如下属性:

\begin{table}[H]
  \centering
  \small
  \caption{\texttt{LineDashedMaterial} 新增属性}
  \setlength{\tabcolsep}{4mm}
  \begin{tabular}{c|ccc}
    \toprule
    \textbf{属性} & \textbf{类型} & \textbf{默认值} & \textbf{含义} \\
    \midrule
    scale & number & 1 & 虚线占比 \\
    dashSize & number & 3 & 虚线大小 \\
    gapSize & number & 1 & 虚线间隙大小 \\
    \bottomrule
  \end{tabular}
\end{table}

\newpage