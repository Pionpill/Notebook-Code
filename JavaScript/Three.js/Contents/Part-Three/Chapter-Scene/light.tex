\section{灯光}
\subsection{前置知识}
\subsubsection*{\texttt{Light}}

\texttt{Light} 类是所有光源的基类,其他所有光源都继承了该类的属性和方法。
\begin{itemize}
  \item \link{https://threejs.org/docs/index.html\#api/zh/lights/Light}{查看文档}
  \item \link{https://github.com/mrdoob/three.js/blob/master/src/lights/Light.js}{查看源码}
\end{itemize}

\begin{JavaScript}
class Light extends Object3D
\end{JavaScript}

\texttt{Light} 实现非常简单,作为抽象基类,他几乎不会被直接使用:

\begin{JavaScript}
constructor( color, intensity = 1 )
\end{JavaScript}

构造函数包含两个参数:
\begin{itemize}
  \item \texttt{color: Integer}: 16 进制颜色,默认为 0xffffff(白色)。
  \item \texttt{intensity: Float}: 关照强度,默认为 1。
\end{itemize}

增加了一个方法 \texttt{dispose()},不过并没有提供任何实现,由子类负责实现。

three.js 中其他的光源大都继承自 \texttt{light}:

\begin{table}[H]
  \centering
  \small
  \caption{three.js 中的光源}
  \label{table:three.js 中的光源}
  \setlength{\tabcolsep}{4mm}
  \begin{tabular}{l|l|l}
    \toprule
    \textbf{名字} & \textbf{描述} & \textbf{备注} \\
    \midrule
    AmbientLight & 环境光 & 无法创建阴影,颜色会叠加到物体上 \\
    PointLight & 点光源 & 向空间各方向发射光 \\
    SpotLight & 聚光源 & 向空间指定范围发射光 \\
    DirectionalLight & 平行光 & 类似于太阳光,没有源头,向某个方向发射光 \\
    \midrule
    HemisphereLight & 半球光 & 模拟自然光,无法创建阴影 \\
    RectAreaLight & 区域光 & 指定发射光的平面,无法创建阴影 \\
    \bottomrule
  \end{tabular}
\end{table}

\subsubsection*{\texttt{Color}}

\texttt{Color} 是 three.js 为我们提供的颜色工具类。一共可以通过以下几种构造方式创建颜色对象:
\begin{itemize}
  \item 传入 rgb: \texttt{new THREE.Color( 1, 0, 0 )}
  \item 传入数字: \texttt{new THREE.Color( 0xff0000 )}
  \item 传入 rgb 字符串: \texttt{new THREE.Color("rgb(255, 0, 0)"), new THREE.Color("rgb(100\%, 0\%, 0\%)")}
  \item 传入 hsl 字符串: \texttt{new THREE.Color("hsl(0, 100\%, 50\%)")}
  \item 传入颜色名: \texttt{new THREE.Color( 'skyblue' )}
  \item 传入 \texttt{Color} 对象。
\end{itemize}

此外也可以传入 alpha 值。\texttt{Color} 对象对颜色进行了封装,我们可以通过该对象快速获取 r,g,b 等值,也可以使用 \texttt{addColor} 等方法混合颜色。

\subsection{光源}

\subsubsection*{\texttt{AmbientLight}}

\texttt{AmbientLight} 指环境光,无法创建投影,因为它没有方向,一般用于给场景增添颜色,或者弱化阴影。

\texttt{AmbientLight} 源码非常简单:

\begin{JavaScript}
class AmbientLight extends Light {
	constructor( color, intensity ) {
		super( color, intensity );
		this.isAmbientLight = true;
		this.type = 'AmbientLight';
	}
}
\end{JavaScript}

由于没有方向,\texttt{AmbientLight} 只需要创建然后加入场景即可。

\subsubsection*{\texttt{PointLight}}

点光源,即对空间各个方向均发射的光源,对应现实生活中的灯泡。

\begin{JavaScript}
class PointLight extends Light {
  constructor( color, intensity, distance = 0, decay = 2 )
  ...
}
\end{JavaScript}

这里有两个新的参数:
\begin{itemize}
  \item \texttt{distance}: 表示从光源到光照强度为0的位置, 当设置为0时,光永远不会消失。
  \item \texttt{decay}: 沿着光照距离的衰退量。
\end{itemize}

\texttt{PointLight} 引入了一个新的可读写的属性 \texttt{power}。\texttt{power} 与 \texttt{intensity} 之间存在如下关系: \texttt{this.intensity = power / ( 4 * Math.PI )}。

旧版的 \texttt{PointLight} 是无法产生阴影的,新版中可以。

\subsubsection*{\texttt{SpotLight}}

聚光灯是最常用的光源之一, 特别用于创建阴影。

\begin{JavaScript}
class SpotLight extends Light {
  constructor( color, intensity, distance = 0, angle = Math.PI / 3, penumbra = 0, decay = 2 ) 
  ...
}
\end{JavaScript}

有两个新的属性:
\begin{itemize}
  \item \texttt{angle}: 管线的散射角度,最大为 $\pi/2$。
  \item \texttt{penumbra}: 聚光灯从中间到四周光线缩减比,默认为0(不缩减)。
\end{itemize}

聚光灯默认 \texttt{target} 为新的 \texttt{Object} 对象,即 (0,0,0) 原点。

在构造聚光灯对象时,会创建 \texttt{this.shadow = new SpotLightShadow()}。

\subsubsection*{\texttt{DirectionalLight}}

平行光是沿着特定方向发射的光。

\begin{JavaScript}
class DirectionalLight extends Light {
	constructor( color, intensity )
}
\end{JavaScript}

平行光也可以创建投影,具体方法和聚光灯类似。

\subsubsection*{\texttt{HemisphereLight}}

HemisphereLight 是指半球光,光照颜色从天空光线颜色渐变到地面光线颜色。可以将其简单理解为线性变化的光。

\begin{JavaScript}
class HemisphereLight extends Light {
	constructor( skyColor, groundColor, intensity )
  ...
}
\end{JavaScript}

\subsubsection*{\texttt{RectAreaLight}}

平面光光源从一个矩形平面上均匀地发射光线。

\begin{JavaScript}
class RectAreaLight extends Light {
  constructor( color, intensity, width = 10, height = 10 )
}
\end{JavaScript}

它包含两个特殊属性: 宽高,用于设置平面光的大小。

\subsection{阴影}

\subsubsection*{\texttt{LightShadow}}

three.js 中的阴影都继承自抽象基类: \texttt{LightShadow}:

\begin{JavaScript}
class LightShadow {
  constructor(camera)
  ...
}
\end{JavaScript}

\texttt{LightShadow} 并不继承自 \texttt{Object},它的构造函数需要传入一个 \texttt{camera} 对象。

\texttt{LightShadow} 有以下重要属性:

\begin{table}[H]
  \centering
  \small
  \caption{\texttt{LightShadow} 重要属性}
  \setlength{\tabcolsep}{4mm}
  \begin{tabular}{l|l|l|l}
    \toprule
    \textbf{属性} & \textbf{类型} & \textbf{默认值} & \textbf{含义} \\
    \midrule
    autoUpdate & Boolean & true & 是否动态更新阴影 \\
    needsUpdate & Boolean & false & 通过设置该值更新阴影 \\
    camera & Camera & 传入 &  \\
    bias & Float & 0 & 阴影贴图的偏差 \\
    blurSamples & Integer & 8 & 模糊采样值 \\
    mapSize & Vector2 & new Vector2( 512, 512 ) & 阴影贴图精度 \\
    radius & Float & 1 & 模糊阴影边缘 \\
    \bottomrule
  \end{tabular}
\end{table}

\subsubsection*{\texttt{PointLightShadow}}

点光源阴影,构造函数传入透视相机:

\begin{JavaScript}
class PointLightShadow extends LightShadow {
	constructor() {
    super( new PerspectiveCamera( 90, 1, 0.5, 500 ) );
    ...
  }
  ...
}
\end{JavaScript}

\subsubsection*{\texttt{DirectionalLightShadow}}

平行光阴影,构造函数传入平视相机:

\begin{JavaScript}
class DirectionalLightShadow extends LightShadow {
	constructor() {
		super( new OrthographicCamera( -5,5,5,-5,0.5,500 ) );
    ...
	}
}
\end{JavaScript}

\subsubsection*{\texttt{SpotLightShadow}}

聚光灯光源,构造函数传入透视相机:

\begin{JavaScript}
class SpotLightShadow extends LightShadow {
  constructor() {
    super( new PerspectiveCamera( 50, 1, 0.5, 500 ) );
    ...
  }
  ...
}
\end{JavaScript}