\section{几何体}

\subsection{二维几何体}

Three.js 提供了两个几何体基类,分别是旧版的 \texttt{Geometry} 与新版的 \texttt{BufferGeometry}。旧版基类易用性强,便于开发。新版基类内部数据组织形式与 GPU 期待的数据结构一致,效率更高,但易用性差一点。现在基本只用新版 \texttt{BufferGeometry}。

\begin{table}[H]
  \centering
  \small
  \caption{\texttt{BufferGeometry} 重要属性}
  \setlength{\tabcolsep}{4mm}
  \begin{tabular}{l|l|l|l}
    \toprule
    \textbf{属性} & \textbf{类型} & \textbf{默认值} & \textbf{含义} \\
    \midrule
    attributes & Object & {} & 直接被 GPU 处理的分量数据 \\
    index & BufferAttribute & null & 允许顶点在多个三角面片间可以重用 \\
    \bottomrule
  \end{tabular}
\end{table}

二维集合体的源码都非常简单,这里只介绍一下构造函数:

\subsubsection*{\texttt{PlaneGeometry}}

\texttt{PlaneGeometry} 用于构建平面:

\begin{JavaScript}
class PlaneGeometry extends BufferGeometry {
	constructor( width = 1, height = 1, widthSegments = 1, heightSegments = 1 )
  ...
}
\end{JavaScript}

其中 \texttt{widthSegments} 和 \texttt{heightSegments} 表示长宽方向上的分段数。

\subsubsection*{\texttt{CircleGeometry}}

\texttt{CircleGeometry} 用于构建圆形:

\begin{JavaScript}
class CircleGeometry extends BufferGeometry {
	constructor( radius = 1, segments = 32, thetaStart = 0, thetaLength = Math.PI * 2 )
  ...
}
\end{JavaScript}

其中 \texttt{thetaStart} 和 \texttt{thetaLength} 分别表示圆开始的角度与结束角度。

\subsubsection*{\texttt{RingGeometry}}

\texttt{RingGeometry} 用于构建圆环。

\begin{JavaScript}
class RingGeometry extends BufferGeometry {
  constructor( innerRadius = 0.5, outerRadius = 1, thetaSegments = 32, phiSegments = 1, thetaStart = 0, thetaLength = Math.PI * 2 )
  ... 
}
\end{JavaScript}

其中 \texttt{phiSegments} 表示沿着圆环长度的切分数。

\subsection{三维几何体}

\subsubsection*{\texttt{BoxGeometry}}

\texttt{BoxGeometry} 用于构建长方体:

\begin{JavaScript}
class BoxGeometry extends BufferGeometry {
  constructor( width = 1, height = 1, depth = 1, widthSegments = 1, heightSegments = 1, depthSegments = 1 )
  ... 
}
\end{JavaScript}

\subsubsection*{\texttt{SphereGeometry}}

\texttt{SphereGeometry} 用于构建圆球:

\begin{JavaScript}
class SphereGeometry extends BufferGeometry {
  constructor( radius = 1, widthSegments = 32, heightSegments = 16, phiStart = 0, phiLength = Math.PI * 2, thetaStart = 0, thetaLength = Math.PI )
  ... 
} 
\end{JavaScript}

其中 \texttt{phi} 和 \texttt{theta} 分别代表 x 轴和 y 轴。

\subsubsection*{\texttt{CylinderGeometry}}

\texttt{CylinderGeometry} 用于创建圆柱体。

\begin{JavaScript}
class CylinderGeometry extends BufferGeometry {
	constructor( radiusTop = 1, radiusBottom = 1, height = 1, radialSegments = 32, heightSegments = 1, openEnded = false, thetaStart = 0, thetaLength = Math.PI * 2 )
  ... 
}
\end{JavaScript}

\subsubsection*{\texttt{ConeGeometry}}

\texttt{ConeGeometry} 用于创建圆锥体。

\begin{JavaScript}
class ConeGeometry extends CylinderGeometry {
	constructor( radius = 1, height = 1, radialSegments = 32, heightSegments = 1, openEnded = false, thetaStart = 0, thetaLength = Math.PI * 2 )
  ... 
}
\end{JavaScript}

\subsubsection*{\texttt{TorusGeometry}}

\texttt{TorusGeometry} 用于创建圆环:

\begin{JavaScript}
class TorusGeometry extends BufferGeometry {
	constructor( radius = 1, tube = 0.4, radialSegments = 12, tubularSegments = 48, arc = Math.PI * 2 )
  ... 
}
\end{JavaScript}

除此以外,Three.js 还提供了很多三维几何体,例如正n面体,钻石等,这里不一一介绍。

\subsection{高级几何体}

所谓高级几何体,就是在 Blender 等三维软件中,通过旋转,挤压获取到的几何体。自这节开始,内容非常抽象,建议先在 Blender 中试试类似的效果。

\subsubsection*{\texttt{ConvexGeometry}}

\texttt{ConvexGeometry} 通过一组点创建一组凸包,即包围这组点的最小图形。因此它的构造函数需要接收点坐标组成的数组:

\begin{JavaScript}
class ConvexGeometry extends BufferGeometry {
	constructor( points = [] )
  ... 
}
\end{JavaScript}

\subsubsection*{\texttt{LatheGeometry}}

\texttt{LatheGeometry} 通过一条光滑的曲线创建图形。绕 z 轴旋转获取类似花瓶的几何体。

\begin{JavaScript}
class LatheGeometry extends BufferGeometry {
	constructor( points = [ new Vector2( 0, - 0.5 ), new Vector2( 0.5, 0 ), new Vector2( 0, 0.5 ) ], segments = 12, phiStart = 0, phiLength = Math.PI * 2 )
  ... 
}
\end{JavaScript}

\subsubsection*{\texttt{ExtrudeGeometry}}

\texttt{ExtrudeGeometry} 允许我们通过挤压获取三维几何体。

\begin{JavaScript}
class ExtrudeGeometry extends BufferGeometry {
  constructor( shapes = new Shape( [ new Vector2( 0.5, 0.5 ), new Vector2( - 0.5, 0.5 ), new Vector2( - 0.5, - 0.5 ), new Vector2( 0.5, - 0.5 ) ] ), options = {} )
  ...
} 
\end{JavaScript}

\subsubsection*{\texttt{TubeGeometry}}

\texttt{ExtrudeGeometry} 创建一个沿着三维曲线延伸的管道。

\begin{JavaScript}
class TubeGeometry extends BufferGeometry {
  constructor( path = new Curves[ 'QuadraticBezierCurve3' ]( new Vector3( - 1, - 1, 0 ), new Vector3( - 1, 1, 0 ), new Vector3( 1, 1, 0 ) ), tubularSegments = 64, radius = 1, radialSegments = 8, closed = false )
  ... 
} 
\end{JavaScript}

\subsubsection*{\texttt{TextGeometry}}

\texttt{TextGeometry} 用于将文本生成为单一的几何体的类。他必须接获取两个关键数据: 字体,传入值。此外,\texttt{TextGeometry} 是一个附加组件,必须显式导入。参考以下方式创建 3D 字体:
\begin{itemize}
  \item \link{https://threejs.org/docs/index.html\#examples/zh/geometries/TextGeometry}{官方文档}
\end{itemize}

\subsection{二元操作}

在 Three.js 中使用几何体的二元操作需要引入一种新的技术: 构造实时几何体技术(Constructive Solid Geometry,CSG)。对应的库是 ThreeBSP。ThreeBSP 提供了三个二元函数: 相交,联合,相减。

这三个函数在计算时使用的是网络的绝对位置,因此尽量不要在使用这些函数之前使用多种材质,这会导致非常奇怪的结果。

\begin{itemize}
  \item \texttt{intersect}: 相交,重叠部分定义为新的几何体。
  \item \texttt{union}: 联合, 将两个结合体联合起来创建新的几何体。
  \item \texttt{subtract}: 相减,与 \texttt{intersect} 相反。
\end{itemize}

\newpage