\section{Redux 构建应用}

\subsection{应用结构}

\subsubsection{Toolkit 方法}

\subsubsection*{\texttt{configureStore()}}

构建一个  Redux 容器,首先需要创建 \texttt{store},而 \texttt{store} 又必须获取 \texttt{reducer} 和 \texttt{state}。下面是一个常见的 \texttt{store} 构建方式。

\begin{JavaScript}
import { configureStore } from '@reduxjs/toolkit'
import counterReducer from '../features/counter/counterSlice'

export default configureStore({
  reducer: {
    counter: counterReducer
  }
})
\end{JavaScript}


\texttt{configureStore} 要求我们传入一个 \texttt{reducer} 对象。对象中的键名 key 将定义最终状态树中的键名 key。
\begin{itemize}
  \item 键: 对应状态名,例如 \texttt{counter} 对应有一个 \texttt{state.counter}。
  \item 值: 负责更新对应的 \texttt{state}。
\end{itemize}

\texttt{configureStore} 默认会自动在 store setup 中添加几个中间件以提供良好的开发者体验。

\subsubsection*{Redux Slice}

“slice” 是应用中单个功能的 Redux reducer 逻辑和 action 的集合, 通常一起定义在一个文件中。该名称来自于将根 Redux 状态对象拆分为多个状态 “slice”。

\begin{JavaScript}
import { configureStore } from '@reduxjs/toolkit'
import usersReducer from '../features/users/usersSlice'
import postsReducer from '../features/posts/postsSlice'
import commentsReducer from '../features/comments/commentsSlice'

export default configureStore({
  reducer: {
    users: usersReducer,
    posts: postsReducer,
    comments: commentsReducer
  }
})
\end{JavaScript}

例如上面代码中: \texttt{state.users, state.posts...} 均为独立的 ``slice''.由于 \texttt{usersReducer} 负责更新 \texttt{state.users} slice,我们将其称为 “slice reducer” 函数。

Redux store 需要在创建时传入一个 “root reducer” 函数。因此,如果我们有许多不同的 slice reducer 函数,理论上我们需要手动将其余 reducer 并入 root reducer。

\begin{JavaScript}
function rootReducer(state = {}, action) {
  return {
    users: usersReducer(state.users, action),
    posts: postsReducer(state.posts, action),
    comments: commentsReducer(state.comments, action)
  }
}
\end{JavaScript}

Redux 有一个名为 \texttt{combineReducers} 的函数,它会自动为我们执行此操作。它接受一个全是 slice reducer 的对象作为其参数,并返回一个函数,该函数在调度操作时调用每个 slice reducer。

\begin{JavaScript}
const rootReducer = combineReducers({
  users: usersReducer,
  posts: postsReducer,
  comments: commentsReducer
})
\end{JavaScript}

当我们将 slice reducer 的对象传递给 \texttt{configureStore} 时,它会将这些对象自动传递给 \texttt{combineReducers} 以便我们生成根 reducer。

\subsubsection*{\texttt{createSlice()}}

Redux Toolkit 的 \texttt{createSlice()} 函数简化了 reducer 的书写过程,它负责生成 action 类型字符串、action creator 函数和 action 对象的工作。你所要做的就是为这个 slice 定义一个名称,编写一个包含 reducer 函数的对象,它会自动生成相应的 action 代码。

\texttt{name} 选项的字符串用作每个 action 类型的第一部分,每个 reducer 函数的键名用作第二部分。因此, action 类型即为 ``counter'' 名称 + ``increment'',例如: \texttt{{type: "counter/increment"}}。

除此之外,我们还需要传入初始状态值,\texttt{createSlice} 会自动生成与我们编写的 reducer 函数同名的 action creator。

\begin{JavaScript}
import { createSlice } from '@reduxjs/toolkit'

export const counterSlice = createSlice({
  name: 'counter',
  initialState: {
    value: 0
  },
  reducers: {
    increment: state => {
      // Redux Toolkit 允许我们在 reducers 写 "可变" 逻辑。
      // 并不是真正的改变 state 因为它使用了 immer 库
      // 当 immer 检测到 "draft state" 改变时,会基于这些改变去创建一个新的
      // 不可变的 state
      state.value += 1
    },
    decrement: state => {
      state.value -= 1
    },
    incrementByAmount: (state, action) => {
      state.value += action.payload
    }
  }
})

export const { increment, decrement, incrementByAmount } = counterSlice.actions
export default counterSlice.reducer
\end{JavaScript}

前面我们说过,不要再 \texttt{reducer} 中更改 \texttt{state} 原始对象,但是上面代码中,我们改变了(slice 肯定做了什么)。例如下面代码:

\begin{JavaScript}
state.value = 123;
\end{JavaScript}

这是非法的,我们直接修改了 \texttt{state},正确的做法是:

\begin{JavaScript}
return {
  ...state,
  value: 123
}
\end{JavaScript}

产生副本,并覆盖原值,但是这样的逻辑非常繁琐,而且反直觉。所以 Toolkit 帮助我们简化了这一过程: \texttt{createSlice} 内部使用了一个名为 \texttt{Immer} 的库。 \texttt{Immer} 使用一种称为 “Proxy” 的特殊 JS 工具来包装你提供的数据,当你尝试 ”mutate“ 这些数据的时候,奇迹发生了,\texttt{Immer} 会跟踪你尝试进行的所有更改,然后使用该更改列表返回一个安全的、不可变的更新值,就好像你手动编写了所有不可变的更新逻辑一样。

但是,这只有在 Redux Toolkit 的 \texttt{createSlice} 和 \texttt{createReducer} 中可以用,其他函数中没有提供类似的方案。

\subsubsection*{Thunk 异步逻辑}

thunk 是一种特定类型的 Redux 函数,可以包含异步逻辑,Thunk 使用两个函数编写:

\begin{itemize}
  \item 一个内部 thunk 函数,以 \texttt{dispatch} 和 \texttt{getState} 作为参数。
  \item 外部创建者函数,创建并返回 thunk 函数。
\end{itemize}

\begin{JavaScript}
// 下面这个函数就是一个 thunk ,它使我们可以执行异步逻辑
// 你可以 dispatched 异步 action `dispatch(incrementAsync(10))` 就像一个常规的 action
// 调用 thunk 时接受 `dispatch` 函数作为第一个参数
// 当异步代码执行完毕时,可以 dispatched actions
export const incrementAsync = amount => dispatch => {
  setTimeout(() => {
    dispatch(incrementByAmount(amount))
  }, 1000)
}
\end{JavaScript}

我们可以像使用普通 Redux action creator 一样使用它们:

\begin{JavaScript}
store.dispatch(incrementAsync(5))
\end{JavaScript}

但是,使用 thunk 需要在创建时将 \texttt{redux-thunk} middleware(一种 Redux 插件)添加到 Redux store 中。幸运的是,Redux Toolkit 的 \texttt{configureStore} 函数已经自动为我们配置好了,所以我们可以继续在这里使用 thunk。

当你需要进行 AJAX 调用以从服务器获取数据时,你可以将该调用放入 thunk 中。

\begin{JavaScript}
// 外部的 thunk creator 函数
const fetchUserById = userId => {
  // 内部的 thunk 函数
  return async (dispatch, getState) => {
    try {
      // thunk 内发起异步数据请求
      const user = await userAPI.fetchById(userId)
      // 但数据响应完成后 dispatch 一个 action
      dispatch(userLoaded(user))
    } catch (err) {
      // 如果过程出错,在这里处理
    }
  }
}
\end{JavaScript}

\subsubsection{React 整合 Redux}

来看看怎么将 Redux 嵌入到 React 组件中(省略不必要的代码):

\begin{JavaScript}
import React from 'react'
import { useSelector, useDispatch } from 'react-redux'
import { decrement, increment, incrementByAmount, incrementAsync, selectCount } from './counterSlice'

export function Counter() {
  const count = useSelector(selectCount)
  const dispatch = useDispatch()
  const [incrementAmount, setIncrementAmount] = React.useState('2')

  return (
    <div>
      <button
        aria-label="Increment value"
        onClick={() => dispatch(increment())}
      >
        +
      </button>
      <span>{count}</span>
      <button
        aria-label="Decrement value"
        onClick={() => dispatch(decrement())}
      >
        -
      </button>
    </div>
  )
}
\end{JavaScript}

\subsubsection*{useSelector()}

我们只使用 \texttt{useState} 存储了 \texttt{incrementAmount},\texttt{count} 使用了 \texttt{useSelector} 存储。他是 Redux 的钩子函数。

\texttt{useSelector} 这个 hooks 让我们的组件从 Redux 的 \texttt{store} 状态树中提取它需要的任何数据。它接受一个函数,用于获取store 的 state 值:

\begin{JavaScript}
export const selectCount = state => state.counter.value
\end{JavaScript}

这样做显得很麻烦,为什么不直接导出 \texttt{value}。Redux 禁止 store 与 React 直接交互,只提供了有限且必要的钩子。

\subsubsection*{useDispatch()}

类似的,\texttt{useDispatch} 用于访问具体的方法。

\begin{JavaScript}
<button onClick={() => dispatch(increment())}>
  +
</button>
\end{JavaScript}

这两个钩子提供了向 Redux 容器交互的能力,但是我们并灭有导入 store,hooks 如何与 Redux store 对话呢:

\begin{JavaScript}
ReactDOM.render(
  <Provider store={store}>
    <App />
  </Provider>,
  document.getElementById('root')
)
\end{JavaScript}

\subsubsection{什么时候使用 Redux}

React 原生的 \texttt{useState} 钩子与 Redux 提供的 \texttt{store} 都可以管理状态,这两者的功能是否重合? 回到 Redux 要解决的问题上。Redux 用于管理全局状态,如果仅是组件局部的状态,不应该又 Redux 进行管理,因为这样没有必要,除了某个组件,并没有其他组件会用到这个状态。因此是否使用 Redux 的关键取决于是否有多个``距离远''的组件需要共享同一状态。

可以看看官网的建议:
\begin{itemize}
  \item 应用程序的其他部分是否关心这些数据?
  \item 你是否需要能够基于这些原始数据创建进一步的派生数据?
  \item 是否使用相同的数据来驱动多个组件?
  \item 能够将这种状态恢复到给定的时间点(即时间旅行调试)对你是否有价值?
  \item 是否要缓存数据(即,如果数据已经存在,则使用它的状态而不是重新请求它)?
  \item 你是否希望在热重载视图组件(交换时可能会丢失其内部状态)时保持此数据一致?
\end{itemize}

此外,用 Redux 进行状态管理,是一件比较厚重的事,会带来性能上以及逻辑上的开销。例如我们要修改表单,临时修改的数据是否应该由 Redux 管理,更好的方式应该是由表单组件 \texttt{useState} 来管理那些没有提交的数据,只有在表单提交时用 \texttt{dispatch} 更新 store。