\section{NodeFetch}

node-fetch 即在 node.js 上使用 fetch api(http 基础请看 Axios 第一节):

\begin{bash}
npm i node-fetch
\end{bash}

前端用 fetch 作请求响应是现在的主要趋势,xhr 正在被淘汰的路上。

\subsection{Fetch API}

\subsubsection{Fetch 基础}

fetch 是被浏览器原生支持的,这是他流行的主要原因。fetch 使用上和 axios 类似,但是 fetch 没有使用 XHR(使用 XHR 被成为传统 ajax),而是另辟新道,但它仍然遵守 ajax 规则。

\begin{figure}[H]
  \small
  \centering
  \begin{tikzpicture}[font=\small]
    \node[block, fill=red!20, minimum height=3cm, minimum width=6cm] at (0,0.25) {};
    \node at (0, 1.25) {Ajax};
    \node[block, fill=blue!20, minimum height=2cm, minimum width=2cm] at (-1.5,0) {Fetch};
    \node[block, fill=orange!20, minimum height=2cm, minimum width=2cm] at (1.5,0) {};
    \node at (1.5, 0.6) {XHR};
    \node[block, fill=orange!40, minimum height=1cm, minimum width=1cm] at (1.5,-0.25) {Axios};
  \end{tikzpicture}
  \caption{fetch 与 ajax}
  \label{fig:fetch 与 ajax}
\end{figure}

fetch 规范与 传统 ajax 有几下主要不同:
\begin{itemize}
  \item 非 2XX 响应码不会被标记为 \texttt{reject}, 仅设置 \texttt{resolve} 返回值的 \texttt{ok} 属性为 \texttt{false}。只有网路故障,CORS 配置错误或请求被阻止时,才会被标记为 \texttt{reject}。
  \item \texttt{fetch} 默认不会发送跨域 cookie。
\end{itemize}

与 axios 相同的是,fetch 也采用了 Promise 对象,关于 Promise 的部分不再说明。

\subsubsection*{\texttt{fetch()}}

\begin{JavaScript}
Promise<Response> fetch(input[, init]);
\end{JavaScript}

\texttt{input} 可以是 \texttt{USVString} 字符串(常用形式为 URL) 或者是一个 \texttt{Request} 对象。 \texttt{init} 为配置项,常见配置项如下:

\begin{table}[H]
  \small
  \centering
  \caption{fetch init 配置项}
  \label{table:fetch init 配置项}
  \setlength{\tabcolsep}{4mm}
  \begin{tabular}{c|ccc}
    \toprule
    \textbf{键} & \textbf{一般类型} & \textbf{说明} \\
    \midrule
    method & string & GET, POST... \\
    headers & Headers/ByteString & 请求头 \\
    body & Blob/URLSearchParams...  & 请求体 \\
    \midrule
    mode & string & 跨域类型,cors, no-cors, same-origin \\
    credentials & string & 凭据配置, omit, same-origin...\\
    cache & string & 缓存配置 \\
    redirect & string & 重定向模式 \\
    referrer & USVString & client/"URL"/no-referrer \\
    \bottomrule
  \end{tabular}
\end{table}

为了方便开发,还提供了三个便捷对象: \texttt{Headers}, \texttt{Request}, \texttt{Response}。

\newpage