\section{Hook,函数组件,高级用法}

最早支持 Hook 的 React 版本是 16.8。Hook 是完全可选的一项技术,Hook 技术使用与否不影响项目功能。但在本文撰写的2022年,函数组件配合 Hook 技术已经成为了主流写法; 同时这章会和借函数组件讲解一些高级用法,这些用法(除了 Hook)在类组件也同样适用。

在 React 官网对 Hook 的介绍中,有一句: \textit{没有计划从 React 中移除 class}。由此可以看出 Hook 可以让函数组件实现所有的类组件功能,并且更为便捷高效。

\subsection{基础 Hook}

Hook 的本质就是 JavaScript 函数,它可以让你“钩入” React 的特性。\texttt{State Hook} 和 \texttt{Effect Hook} 是 Hook 的两个核心函数,分别让函数组件获取了 \texttt{state} 与生命周期。

\subsubsection{State Hook}

下面是使用Hook技术的一个函数组件:
\begin{JavaScript}
import React, { useState } from 'react';

function Example() {
  // 声明一个叫 “count” 的 state 变量。
  const [count, setCount] = useState(0);

  return (
    <div>
      <p>You clicked {count} times</p>
      <button onClick={() => setCount(count + 1)}>
        Click me
      </button>
    </div>
  );
}
\end{JavaScript}

在函数组件中,我们没有 \texttt{this},所以我们不能分配或读取 \texttt{this.state}。我们直接在组件中调用 \texttt{useState} Hook。

\texttt{useState} 会返回一个有两个元素的数组:当前状态和一个让你更新它的函数。它类似类组件的 \texttt{this.setState},但是它不会把新的 \texttt{state} 和旧的 \texttt{state} 进行合并(因此它是 \texttt{const} 的,也可以直接操作 \texttt{state})。\texttt{useState} 唯一的参数就是初始 \texttt{state}。

\subsubsection*{\texttt{state} 特性}

前面我们说过,\texttt{state} 会 \texttt{props} 的改变通常会需要重新 \texttt{render} 组件。因此我们对 \texttt{state} 的操作必须十分小心,只有必要的,关联组件状态的数据才需要放在 \texttt{state} 中。

在上面的代码中,我们使用 \texttt{const} 修饰的数据接受了 \texttt{useState} 的返回值,这说明 \texttt{state} 数据本身是不可修改的,为此类组件中 React 专门提供了一个 \texttt{setState} 方法修改,Hook 也提供了对应的方法用于更新 \texttt{state}。

\texttt{state} 的更新是异步的,组件的\texttt{state}并不会立即改变,\texttt{setState}只是把要修改的状态放入一个队列中,React会优化真正的执行时机,并且出于性能原因,可能会将多次\texttt{setState}的状态修改合并成一次状态修改。因此不要依赖当前的 \texttt{state} 值计算下一个 \texttt{state} 值。

由于 \texttt{state} 本身应该是不可修改的对象,因此尽可能地使用 \texttt{string,number} 等数据类型,如果要用到数组或其他可变类型,则需要考虑是应该修改这些类型的数据,才是创建一个新的对应类型传给 \texttt{state}。最好的方法自然是使用 \texttt{const} 关键字。

\subsubsection{Effect Hook}

Effect Hook 对应的函数是 \texttt{useEffect} ,它给函数组件增加了操作副作用的能力。它跟类组件中的 \texttt{componentDidMount、componentDidUpdate} 和 \texttt{componentWillUnmount} 具有相同的用途,只不过被合并成了一个 API。

\begin{JavaScript}
// 相当于 componentDidMount 和 componentDidUpdate:
useEffect(() => {
  // 使用浏览器的 API 更新页面标题
  document.title = `You clicked ${count} times ${count}`;
});
\end{JavaScript}

当你调用 useEffect 时,就是在告诉 React 在完成对 DOM 的更改后(对应生命周期:\texttt{compone ntDidMount、componentDidUpdate})运行你的“副作用”函数。由于副作用函数是在组件内声明的,所以它们可以访问到组件的 \texttt{props} 和 \texttt{state}。与类组件生命周期不同的是,\texttt{useEffect} 调度的 \texttt{effect} 不会阻塞浏览器更新屏幕。

\texttt{useEffect} 可以返回一个函数,React 将会在执行清除操作时调用它(对应生命周期: \texttt{compo nentWillUnmount}),因此常被命名为 \texttt{cleanup}。

在某些情况下,每次渲染后都执行清理或者执行 \texttt{effect} 可能会导致性能问题。在类组件中,我们可以通过在 \texttt{componentDidUpdate} 中添加对 \texttt{prevProps} 或 \texttt{prevState} 的比较逻辑解决,在 \texttt{useEffect} 中可以通过添加第二个可选参数实现相同效果:

\begin{JavaScript}
useEffect(() => {
  document.title = `You clicked ${count} times ${count}`;
}, [count]); // 仅在 count 更改时更新
\end{JavaScript}

\subsubsection{自定义 Hook}

自定义 Hook 主要用于解决组件之间逻辑共享问题。当我们想在两个函数之间共享逻辑时,我们会把它提取到第三个函数中。而组件和 Hook 都是函数,所以也同样适用这种方式。

自定义 Hook 是一个函数,其名称以 “use” 开头,函数内部可以调用其他的 Hook:

\begin{JavaScript}
import { useState, useEffect } from 'react';

function useFriendStatus(friendID) {
  const [isOnline, setIsOnline] = useState(null);

  useEffect(() => {
    function handleStatusChange(status) {
      setIsOnline(status.isOnline);
    }

    ChatAPI.subscribeToFriendStatus(friendID, handleStatusChange);
    return () => {
      ChatAPI.unsubscribeFromFriendStatus(friendID, handleStatusChange);
    };
  });

  return isOnline;
}
\end{JavaScript}

与 React 组件不同的是,自定义 Hook 不需要具有特殊的标识。我们可以自由的决定它的参数是什么,以及它应该返回什么(如果需要的话)。换句话说,它就像一个正常的函数。但是它的名字应该始终以 \texttt{use} 开头,这样可以一眼看出其符合 Hook 的规则。

此处 \texttt{useFriendStatus} 的 Hook 目的是订阅某个好友的在线状态。这就是我们需要将 \texttt{friendID} 作为参数,并返回这位好友的在线状态的原因。

使用自定义 Hook 只需要在函数组件中调用即可。

\begin{JavaScript}
function FriendStatus(props) {
  const isOnline = useFriendStatus(props.friend.id);

  if (isOnline === null) {
    return 'Loading...';
  }
  return isOnline ? 'Online' : 'Offline';
}
\end{JavaScript}

\subsection{进阶 Hook}

\subsubsection{Context Hook}

Context 提供了一个无需为每层组件手动添加 \texttt{props},就能在组件树间进行数据传递的方法。例如当前认证的用户、主题或首选语言。

\begin{JavaScript}
// 调用 API 创建一个 Context
const ThemeContext = React.createContext('light');
function MyElement() {
  const theme = useContext(ThemeContext); // 使用 Context Hook
  return (
    <button style={{ background: theme.background, color: theme.foreground }}>
      I am styled by theme context!
    </button>
  );
}
\end{JavaScript}

React 提供的关于 Context 的 API 有这些(不包含 Context Hook):
\begin{itemize}
  \item \textbf{\texttt{React.createContext}}
  
\begin{JavaScript}
  const MyContext = React.createContext(defaultValue);
\end{JavaScript}
  
  创建一个 Context 对象。当 React 渲染一个订阅了这个 Context 对象的组件,这个组件会从组件树中离自身最近的那个匹配的 \texttt{Provider} 中读取到当前的 \texttt{context} 值。只有当组件所处的树中没有匹配到 Provider 时,其 \texttt{defaultValue} 参数才会生效。这有助于在不使用 Provider 包装组件的情况下对组件进行测试。

  \item \textbf{\texttt{Context.Provider}}

\begin{HTML}
<MyContext.Provider value={/* 某个值 */}>
\end{HTML}

每个 Context 对象都会返回一个 Provider React 组件,它允许消费组件订阅 context 的变化。Provider 接收一个 \texttt{value} 属性,传递给消费组件。一个 Provider 可以和多个消费组件有对应关系。多个 Provider 也可以嵌套使用,里层的会覆盖外层的数据。

当 Provider 的 \texttt{value} 值发生变化时,它内部的所有消费组件都会重新渲染。Provider 及其内部 consumer 组件都不受制于 \texttt{shouldComponentUpdate} 函数,因此当 consumer 组件在其祖先组件退出更新的情况下也能更新。

  \item \textbf{\texttt{Class.contextType}}
  
  挂载在 class 上的 \texttt{contextType} 属性会被重赋值为一个由 React.createContext() 创建的 Context 对象。这能让你使用 \texttt{this.context} 来消费最近 Context 上的那个值。你可以在任何生命周期中访问到它,包括 render 函数中。

  \item \textbf{\texttt{Context.Consumer}}
  
\begin{HTML}
<MyContext.Consumer>
  {value => /* 基于 context 值进行渲染*/}
</MyContext.Consumer>
\end{HTML}

  这里,React 组件也可以订阅到 context 变更。这能让你在函数式组件中完成订阅 context。

  \item \textbf{\texttt{Context.displayName}}
  
\begin{HTML}
const MyContext = React.createContext(/* some value */);
MyContext.displayName = 'MyDisplayName';
<MyContext.Provider> // "MyDisplayName.Provider" 在 DevTools 中
<MyContext.Consumer> // "MyDisplayName.Consumer" 在 DevTools 中
\end{HTML}

  context 对象接受一个名为 displayName 的 property,类型为字符串。React DevTools 使用该字符串来确定 context 要显示的内容。
\end{itemize}

看看例子:

\begin{JavaScript}
// theme-context.js
export const themes = {
  light: {
    foreground: '#000000',
    background: '#eeeeee',
  },
  dark: {
    foreground: '#ffffff',
    background: '#222222',
  },
};
export const ThemeContext = React.createContext(
  themes.dark // 默认值
);
// themed-button.js
import {ThemeContext} from './theme-context';

class ThemedButton extends React.Component {
  render() {
    let props = this.props;
    let theme = this.context;
    return (
      <button
        {...props}
        style={{backgroundColor: theme.background}}
      />
    );
  }
}
ThemedButton.contextType = ThemeContext;
export default ThemedButton;
\end{JavaScript}


\texttt{useHook} 本质上就是替换了类组件中的 \texttt{static contextType = MyContext} 或者 \texttt{<MyCon text.Consumer>}。仍然需要在上层组件树中使用 \texttt{<MyContext.Provider>} 来为下层组件提供 context。

\subsubsection{Ref Hook}

Ref 转发是一项将 ref 自动地通过组件传递到其一子组件的技巧。

看下面这个常规组件:

\begin{JavaScript}
function FancyButton(props) {
  return (
    <button className="FancyButton">
      {props.children}
    </button>
  );
}
\end{JavaScript}

React 组件隐藏其实现细节,包括其渲染结果。其他使用 \texttt{FancyButton} 的组件通常不需要获取内部的 DOM 元素 \texttt{button} 的 \texttt{ref}。这很好,因为这防止组件过度依赖其他组件的 DOM 结构。

但是一旦组件变得复杂,当我们需要操作组件中的子元素时,由于 React 对组建的封装,如果直接操作原生 DOM 往往会无法轻易获取某个元素。Ref 转发是一个可选特性,其允许某些组件接收 ref,并将其向下传递(换句话说,“转发”它)给子组件。

在下面的示例中,\texttt{FancyButton} 使用 \texttt{React.forwardRef} 来获取传递给它的 \texttt{ref},然后转发到它渲染的 DOM \texttt{button}:

\begin{JavaScript}
const FancyButton = React.forwardRef((props, ref) => (
  <button ref={ref} className="FancyButton">
    {props.children}
  </button>
));

// 你可以直接获取 DOM button 的 ref:
const ref = React.createRef();
<FancyButton ref={ref}>Click me!</FancyButton>;
\end{JavaScript}

这样,使用 \texttt{FancyButton} 的组件可以获取底层 DOM 节点 \texttt{button} 的 ref ,并在必要时访问,就像其直接使用 DOM \texttt{button} 一样。使用 Ref 转发后发生了如下事件:

\begin{itemize}
  \item 我们通过调用 \texttt{React.createRef} 创建了一个 React ref 并将其赋值给 \texttt{ref} 变量。
  \item 我们通过指定 \texttt{ref} 为 JSX 属性,将其向下传递给 \texttt{<FancyButton ref={ref}>}。
  \item React 传递 \texttt{ref} 给 forwardRef 内函数 \texttt{(props, ref) => ...},作为其第二个参数。
  \item 我们向下转发该 \texttt{ref} 参数到 \texttt{<button ref={ref}>},将其指定为 JSX 属性。
  \item 当 \texttt{ref} 挂载完成,\texttt{ref.current} 将指向 <button> DOM 节点。
\end{itemize}

第二个参数 \texttt{ref} 只在使用 \texttt{React.forwardRef} 定义组件时存在。常规函数和 class 组件不接收 \texttt{ref} 参数,且 \texttt{props} 中也不存在 \texttt{ref}。

\texttt{useRef} 返回一个可变的 ref 对象,其 \texttt{.current} 属性被初始化为传入的参数(\texttt{initialValue})。返回的 ref 对象在组件的整个生命周期内保持不变。

一个常见的用例便是命令式地访问子组件:

\begin{JavaScript}
const TextInputWithFocusButton: React.FC = () => {
  const inputEl: MutableRefObject<any> = useRef(null)
  const handleFocus = () => {
    // `current` 指向已挂载到 DOM 上的文本输入元素
    inputEl.current.focus()
  }
  return (
    <p>
      <input ref={inputEl} type="text" />
      <button onClick={handleFocus}>Focus the input</button>
    </p>
  )
}
\end{JavaScript}

\begin{itemize}
  \item 使用 \texttt{useRef} 创建一个 ref 并保存在 \texttt{inputEl} 中。
  \item 在 \texttt{return} 语句中,将 \texttt{inputEl} 通过 \texttt{ref} 标签属性绑定在 \texttt{<input>} 元素中。
  \item 在 \texttt{<button>} 点击事件中使用 \texttt{inputEl} 进而获得 \texttt{<input>} 元素进行控制。
\end{itemize}

本质上,\texttt{useRef} 就像是可以在其 \texttt{.current} 属性中保存一个可变值的“盒子”。

如果你将 ref 对象以 \texttt{<div ref={myRef} />} 形式传入组件,则无论该节点如何改变,React 都会将 ref 对象的 \texttt{.current} 属性设置为相应的 DOM 节点。然而,\texttt{useRef()} 比 ref 属性更有用。它可以很方便地保存任何可变值。