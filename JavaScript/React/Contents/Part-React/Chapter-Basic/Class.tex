\section{React 类组件}

类组件已经完成了它的使命,自 React16 给出 Hooks 后,React 主流的方向是使用函数组件。不过一些老项目仍在使用类组件,类组件的用法了解即可。如果有一定基础,也可以跳过类组件这一小节。

\subsection{类组件}

\subsubsection{定义组件}

组件是React的核心概念,是React应用程序的基石。组件将应用的U拆分成独立的、可复用的模块,React应用程序正是由一个一个组件搭建而成的。

\subsubsection*{类组件}

使用 \texttt{class} 定义类组件有两个条件:
\begin{itemize}
    \item \texttt{class} 继承自 \texttt{React.Component}。
    \item \texttt{class} 内部定义 \texttt{render} 方法,用于返回该组件的 UI 元素(一般用 JSX 语法)。
\end{itemize}

\begin{JavaScript}
import React, { Component } from "react";

class PostList extends Component {
  render() {
    return (
      <div>
        <span>Learn</span>
        <ul>
          <li>JSX</li>
          <li>React-DOM</li>
          <li>Redux</li>
        </ul>
      </div>
    );
  }
}

export default PostList;
\end{JavaScript}

这样我们就定义了一个 PostList 组件,只要引入它,就可以使用 \texttt{<PostList>} 标签。 

将其挂载到 DOM 节点上:

\begin{JavaScript}
import React from "react";
import ReactDOM from "react-dom";
import PostList from "./PostList";

ReactDOM.render(<PostList />, document.getElementById("root"));
\end{JavaScript}

这样,我们可以将一份复杂的 HTML 节点数分解成部分可重用的React组件。

\subsubsection*{函数组件}

由于 JS 定义函数的方法特别多,这里只写最主流的方案:

\begin{JavaScript}
const PostList = (props) => {
  return (
    <div>
      <span>Learn</span>
      <ul>
        <li>JSX</li>
        <li>React-DOM</li>
        <li>Redux</li>
      </ul>
    </div>
  );
}
\end{JavaScript}

函数组件无法获取 state 和自身的生命周期,需要通过 React v16.8 给出的 Hooks,下文均以类组件为例,后面会单独介绍 Hooks。

\subsubsection*{组件与元素}

React 组件可以看作一个 Html 标签,React 元素则是一个普通的 JavaScript 对象。在 JSX 中可以这样写:

\begin{JavaScript}
// 组件
<div>
  <CustomButton/>
</div>
// 元素
<div>
  {CustomButton}
</div>
\end{JavaScript}

\subsubsection{组件的 \texttt{props}}

就前面的例子而言,如果我们要重用 \texttt{PostList} 组件,需要修改里面的内容怎么办,重新定义一个 \texttt{PostList2} 组件,继续用硬编码的方式写入内容?这显然违背了重用的理念。

React 定义的组件允许我们自定义标签属性(HTML 中标签的属性,为了区分对象属性下文称为标签属性)。组件的\texttt{props}属性用于把父组件中的数据或方法传递给子组件。在类组件和函数组件中 \texttt{props} 调用方式不同:
\begin{itemize}
    \item 在类组件中,继承自 \texttt{React.Component} 的组件会有一个 \texttt{this.props} 调用属性。
    \item 在函数组件中,函数的唯一参数 \texttt{props} 代表了属性。
\end{itemize}

我们重写上面的代码(有部分改动):

\begin{JavaScript}
class PostList extends Component {
  render() {
    const {title, author, date} = this.props;   // 解构
    return (
      <div>
        <span>{title}</span>
        <ul>
          <li>{author}</li>
          <li>{date}</li>
        </ul>
      </div>
    );
  }
}
\end{JavaScript}

在调用时,我们只需要通过标签属性就可以指定 \texttt{props} 的值,React 会自动将自定义组件的标签属性装入 \texttt{props} 属性中:

\begin{HTML}
<PostList title="React" author="Pionpill" date="2022-12-21">
\end{HTML}

不过,无论是函数组件还是方法组件都不能修改 \texttt{props} 的值。

React 提供了 \texttt{PropTypes} 这个对象,用于校验组件属性的类型。\texttt{PropTypes}包含组件属性所有可能的类型,我们通过定义一个映射对象实现组件属性类型的校验。

\begin{JavaScript}
import PropTypes from 'prop-types';

class PostItem extends React. Component{}

PostItem.propTypes = {
  post: PropTypes.object,
  onVote: PropTypes.func
};
\end{JavaScript}

如果属性值是对象或者数值,我们仍然无法知道其内部具体的数据,这时可以使用 \texttt{PropTy pes.shape} 或 \texttt{PropTypes.arrayOf} 方法:

\begin{JavaScript}
style: PropTypes.shape ({
  color: PropTypes.string,
  fontSize: PropTypes.number
}),
sequence: PropTypes.arrayOf(PropTypes.number) 
\end{JavaScript}

\subsubsection{组件的 \texttt{state}}

\texttt{props} 代表组件的外部状态,标签属性是外部传输进来的,只能调用不能修改。\texttt{state} 则表示内部状态,可以通过 \texttt{setState()} 方法对其进行修改。

在类组件中使用 \texttt{state} 的唯一方法是在构造方法 \texttt{constructor} 中通过 \texttt{this.state} 定义组件的初始状态并调用。

\begin{JavaScript}
class PostList extends Component {
  constructor(props) {
    super(props);   // 这一句强制要求必须有
    this.state = {
      vote: 0
    };
  }

  handle() {
    let vote = this.state.vote; // 无法直接操作 state 值
    vote++;
    this.setState({
      vote: vote,
    });
  }

  render() {
    const {title, author, date} = this.props;   // 解构
    return (
      <div>
        <button onClick={()=>this.handle()}><button>  // 箭头函数
      </div>
    );
  }
}
\end{JavaScript}

操作 \texttt{state} 一般可分为以下三个步骤:
\begin{itemize}
  \item 在构造函数中,通过 \texttt{this.state} 定义 \texttt{state} 的数据。
  \item 将对 \texttt{state} 的操作封装在函数中,且在函数中只能通过 \texttt{setState} 方法修改 \texttt{state} 值。
  \item 在JSX中调用函数,只能用箭头函数方式。
\end{itemize}

\subsubsection{组件样式}

React 可以将样式表当作一个模块,在组件中导入样式表并且使用:

\begin{JavaScript}
import '.style.css’;
function Welcome(props) { 
  return <h1 className= 'foo' >Hello, {props.name}<hl>;
} 
\end{JavaScript}

React 的核心理念之一是: all-in-js 因此,更推荐直接而在组件中定义样式:

\begin{JavaScript}
function Welcome(props) {
  const style = {
    width: "100%",
    height: "50px",
    backgroundColor:"blue",
  };
  return <h1 style = {style}>Hello World!</h1>
}
\end{JavaScript}

React 原生定义内联样式属性必须采用小驼峰法命名。如果采用这种方案更推荐使用支持React 的 CSS框架,比如 \texttt{\texttt{styled-component}}。

\subsubsection{组件的生命周期}

通常,组件的生命周期可以被分为三个阶段:挂载阶段、更新阶段、卸载阶段。

挂载阶段组件被创建,执行初始化,并被挂载到 ODM 中,完成组件的第一次渲染。依次调用的生命周期方法有:
\begin{itemize}
  \item \textbf{\texttt{constructor}}
  
  \texttt{class} 的构造方法,接收一个\texttt{props}参数,必须在这个方法中首先调用\texttt{super(prop s)} 才能保证\texttt{props}被传入组件中。\texttt{constructor}通常用于初始化组件的\texttt{state}以及绑定事件处理方法等工作。

  \item \textbf{\texttt{componentWillMount}}
  
  在组件被挂载到DOM前调用,且只会被调用一次。这个方法在实际项目中很少会用到,因为可以在该方法中执行的工作都可以提前到\texttt{constructor}中。在这个方法中调用\texttt{this.setState}不会引起组件的重新渲染。

  \item \textbf{\texttt{render}}
  
  唯一必要的方法(其他方法可以省略),根据组件的\texttt{props}和\texttt{state}返回一个React元素,用于描述组件的UI。render并不负责组件的实际渲染工作,它只是返回一个UI的描述,真正的渲染出页面DOM的工作由React自身负责。\texttt{render}是一个纯函数,在这个方法中不能执行任何有副作用的操作,所以不能在\texttt{render}中调用\texttt{this.setState},这会改变组件的状态。

  \item \textbf{\texttt{componentDidMount}}
  
  在组件被挂载到DOM后调用,且只会被调用一次。这时候已经可以获取到DOM结构,因此依赖DOM节点的操作可以放到这个方法中。这个方法通常还会用于向服务器端请求数据。在这个方法中调用\texttt{this.setState}会引起组件的重新渲染。
\end{itemize}

组件被挂载到DOM后,组件的\texttt{props}或\texttt{state}可以引起组件更新。\texttt{props}引起的组件更新,本质上是由渲染该组件的父组件引起的,也就是当父组件的\texttt{render}方法被调用时,组件会发生更新过程;不过无论 \texttt{props} 是否改变,只要调用 \texttt{render} 就会引起组件更新。\texttt{state} 引起的组件更新,是通过调用 \texttt{this.setState} 修改组件 \texttt{state} 触发的。组件更新阶段调用生命周期方法有:
\begin{itemize}
  \item \textbf{\texttt{componentWillReceiveProps(nextProps)}}
  
  只在\texttt{props}引起的组件更新过程中,才会被调用。方法的参数\texttt{nextProps}是父组件传递给当前组件的新的\texttt{props}。往往需要比较\texttt{nextProps}和\texttt{this.props}来决定是否执行\texttt{props}发生变化后的逻辑,比如根据新的\texttt{props}调用\texttt{this.setState}触发组件的重新渲染。

  组件更新过程中,只有在组件\texttt{render}及其之后的方法中,\texttt{this.state} 指向的才是更新后的 \texttt{state}。在 \texttt{render} 之前的方法,\texttt{this.state} 依然指向更新前的 \texttt{state}。

  \texttt{setState} 方法不会触发该方法,否则可能会进入死循环,毕竟该方法更新机制之一就是调用 \texttt{setState}。

  \item \textbf{\texttt{shouldComponentUpdate(nextProps, nextState)}}
  
  该方法通过比较 \texttt{nextProps, nextState} 决定是否要执行更新过程,返回布尔值。当方法返回\texttt{false}时,组件的更新过程停止,后续的方法也不会再被调用。该方法可以减少不必要的渲染,从而优化组件性能。

  \item \textbf{\texttt{componentWillUpdate(nextProps, nextState)}}
  
  作为组件更新发生前执行某些工作的地方,一般很少用到。这里不能调用 \texttt{setState},否则可能引起循环问题,下一个也是。

  \item \textbf{\texttt{Render}}
  \item \textbf{\texttt{componentDidUpdate(prevProps, prevState)}}
  
  组件更新后被调用,可以作为操作更新后的DOM的地方。两个参数代表更新前的 \texttt{props} 和 \texttt{state}。

\end{itemize}

卸载阶段,只有一个生命周期方法:
\begin{itemize}
  \item \textbf{\texttt{componentWillUnmount}}
  
  在组件被卸载前调用,可以在这里执行一些清理工作,比如清除组件中使用的定时器,清除\texttt{componentDidMount}中手动创建的DOM元素等,以避免引起内存泄漏。
\end{itemize}

只有类组件才具有生命周期方法,函数组件是没有生命周期方法的。

\subsubsection{React16 新特性}

\subsubsection*{\texttt{render} 新的返回类型}

React 16 之前 \texttt{render} 方法必须返回单个元素,现在 \texttt{render} 方法支持两种新的返回类型: 数组和字符串:

\begin{JavaScript}
// 返回数组
class ListComponent extends Component{ 
  render() {
    return [ 
      <li key= ”A” >First item</li>, 
      <li key=”B”>Second item</li>, 
      <li key=”C”>Third item</li> 
    ];
  }
}
// 返回字符串
class StringComponent extends Component { 
  render () { 
    return "Just a strings"; 
  }
}
\end{JavaScript}

\subsubsection*{错误处理}

React 16之前,组件在运行期间如果执行出错,就会阻塞整个应用的渲染,这时候只能刷新页面才能恢复应用。React16引入了新的错误处理机制,默认情况下,当组件中抛出错误时,这个组件会从组件树中卸载,从而避免整个应用的崩溃。

这种方式比起之前的处理方式有所进步,但用户体验依然不够友好。React16还提供了一种更加友好的错误处理方式一一错误边界(ErrorBoundaries)。错误边界是能够捕获子组件的错误井对其做优雅处理的组件。

定义了\texttt{componentDidCatch(error, info)}这个方法的组件将成为一个错误边界:

\begin{JavaScript}
class ErrorBoundary extends React.Component {
  // ......
  componentDidCatch(error, info) { 
    // 输出错误日志
    console.log(error, info);
  }
}
\end{JavaScript}

\subsubsection*{Portals}

React 16的Portals特性让我们可以把组件渲染到当前组件树以外的DOM节点上,这个特性典型的应用场景是渲染应用的全局弹框,使用Portals后,任意组件都可以将弹框组件渲染到根节点上,以方便弹框的显示。Portals的实现依赖ReactDOM的一个新的API:

\begin{JavaScript}
ReactDOM.createPortal(child, container) 
\end{JavaScript}

第一个参数\texttt{child}是可以被渲染的React节点,例如React元素、由React元素组成的数组、字符串等,\texttt{container}是一个DOM元素,\texttt{child}将被挂载到这个DOM节点。

\subsubsection*{自定义 DOM 属性}

React 16之前会忽略不识别的HTML和SVG属性,现在React会把不识别的属性传递给DOM元素。

\begin{JavaScript}
// React16 之前
<div /> 
// React16 之后
<div custom-attribute=”something" /> 
\end{JavaScript}

\subsection{组件技巧}

\subsubsection{列表与 \texttt{Keys}}

看一下 JavaScript 的 \texttt{map()} 方法,它接受三个参数: \texttt{value, index, arr} 分别代表数据值,数据索引,数据所属的列表:
\begin{JavaScript}
const numbers = [1, 2, 3, 4, 5];
const doubled = numbers.map((number) => number * 2);
// [2, 4, 6, 8, 10]
\end{JavaScript}

在 React 中把数组转换为元素的过程也是类似的:

\begin{JavaScript}
function NumberList(props) {
  const numbers = props.numbers;
  const listItems = numbers.map((number) =>
    <li key={number.toString()}>
      {number}
    </li>
  );
  return (
    <ul>{listItems}</ul>
  );
}
\end{JavaScript}

为什么要给 \texttt{<li>} 标签添加 \texttt{key} 属性呢? 因为 React 使用 \texttt{key} 属性来标记列表中的每个元素,当列表数据发生变化时,React就可以通过\texttt{key}知道哪些元素发生了变化,从而只重新渲染发生变化的元素来提高渲染效率。

不要将 \texttt{index} 作为 \texttt{key},这在列表重排时会引起性能问题。此外,\texttt{key} 只有在数组上下文才有含义。

列表可以直接通过 \texttt{map()} 嵌入带 JSX 中,利用前面提到的React元素语法:

\begin{JavaScript}
function NumberList(props) {
  const numbers = props.numbers;
  return (
    <ul>
      {numbers.map((number) =>
        <ListItem key={number.toString()}
                  value={number} />
      )}
    </ul>
  );
}
\end{JavaScript}

\subsubsection{事件处理}

使用 HTML 标签绑定事件这样写:

\begin{HTML}
<button onclick="handle()"> XXX </button>
\end{HTML}

使用 React 则需要这样写:

\begin{JavaScript}
<button onclick={handle}> XXX </button>
\end{JavaScript}

此外,在 HTML 中绑定的事件可以通过返回 \texttt{false} 阻止事件发生,例如:

\begin{HTML}
<a href="#" onclick="console.log('The link was clicked.'); return false">
  Click me
</a>
\end{HTML}

而 React 则定义了一个专用的方法:

\begin{JavaScript}
function handleClick(e) {
  e.preventDefault();
  console.log('The link was clicked.');
}
\end{JavaScript}

有一个地方需要注意,在 JavaScript 中,class 的方法默认不会绑定 this(注意,只是方法,属性还是会绑定),因此在 JSX 中不能直接调用函数,但是可以将函数视作属性调用。

针对 JavaScript 复杂的 \texttt{this} 指向问题,React 定义了三种处理函数书写方案:

\begin{itemize}
  \item \textbf{箭头函数}
  
  箭头函数中的\texttt{this}指向的是函数定义时的对象,所以可以保证\texttt{this}总是指向当前组件的实例对象。但如果直接在箭头函数中写逻辑会让代码变得很乱,因此通常利用箭头函数\texttt{this}的特性传递真正的函数:

\begin{JavaScript}
render (
  <button onClick={ (event)=>{this.handleClick(event);} }> Button </button>
)
\end{JavaScript}

直接在\texttt{render}方法中为元素事件定义事件处理函数,最大的问题是,每次\texttt{render}调用时,都会重新创建一个新的事件处理函数,带来额外的性能开销,组件所处层级越低,这种开销就越大,因为任何一个上层组件的变化都可能会触发这个组件的\texttt{render}方法。一般情况下这种开销不必在意。

  \item \textbf{组件方法}
  
  直接将组件的方法赋值给元素的事件属性,同时在类的构造函数中,将这个方法的\texttt{this}绑定到当前对象。

\begin{JavaScript}
constructor(props) {
  super(props);
  this.handleClick = this.handleClick.bind(this);
}
\end{JavaScript}

这种方式的好处是每次\texttt{render}不会重新创建一个回调函数,没有额外的性能损失。但在构造函数中,为事件处理函数绑定\texttt{this},尤其是存在多个事件处理函数需要绑定时,这种模板式的代码坯是会显得烦琐。

  \item \textbf{属性初始化语法}
  
使用ES7的\texttt{propertyinitializers}会自动为\texttt{class}中定义的方法绑定\texttt{this}。

\begin{JavaScript}
handleClick = (event) => {
  console.log("111");
}

render {
  return (
    <button onClick={this.handleClick}> Button </button>
  )
}
\end{JavaScript}

\end{itemize}

\subsubsection{表单}

在 HTML 中,有些元素如表单元素自身维护着一些状态(输入的内容),这些状态默认情况下是不受 React 控制的。我们称这类状态不受React控制的表单元素为非受控组件。在React中,状态的修改必须通过组件的\texttt{state},非受控组件的行为显然有悖于这一原则。为了让表单元素状态的变更也能通过组件的\texttt{state}管理,React采用受控组件的技术达到这一目的。

如果一个表单元素的值是由React来管理的,那么它就是一个受控组件。React组件渲染表单元素,并在用户和表单元素发生交互时控制表单元素的行为,从而保证组件的\texttt{state}成为界面上所有元素状态的唯一来源:

\begin{JavaScript}
class NameForm extends React.Component {
  constructor(props) {
    super(props);
    this.state = {value: ''};

    this.handleChange = this.handleChange.bind(this);
    this.handleSubmit = this.handleSubmit.bind(this);
  }

  handleChange(event) {
    this.setState({value: event.target.value});
  }

  handleSubmit(event) {
    alert('提交的名字: ' + this.state.value);
    event.preventDefault();
  }

  render() {
    return (
      <form onSubmit={this.handleSubmit}>
        <label>
          名字:
          <input type="text" value={this.state.value} onChange={this.handleChange} />
        </label>
        <input type="submit" value="提交" />
      </form>
    );
  }
}
\end{JavaScript}

由于在表单元素上设置了 \texttt{value} 属性,因此显示的值将始终为 \texttt{this.state.value},这使得 React 的 \texttt{state} 成为唯一数据源。

常见的受控组件包括: \texttt{textarea,select,input}。

使用受控组件虽然保证了表单元素的状态也由React统一管理,但需要为每个表单元素定义\texttt{onChange}事件的处理函数,然后把表单状态的更改同步到React组件的\texttt{state},这一过程是比较烦琐的,一种可替代的解决方案是使用非受控组件。

使用非受控组件需要有一种方式可以获取到表单元素的值,React中提供了一个特殊的属性ref,用来引用React组件或DOM元素的实例。

\begin{JavaScript}
class NameForm extends React.Component {
  constructor(props) {
    super(props);
    this.handleSubmit = this.handleSubmit.bind(this);
    this.input = React.createRef();
  }

  handleSubmit(event) {
    alert('A name was submitted: ' + this.input.current.value);
    event.preventDefault();
  }

  render() {
    return (
      <form onSubmit={this.handleSubmit}>
        <label>
          Name:
          <input type="text" ref={this.input} />
        </label>
        <input type="submit" value="Submit" />
      </form>
    );
  }
}
\end{JavaScript}

\subsubsection*{状态提升}

在 React 中,将多个组件中需要共享的 \texttt{state} 向上移动到它们的最近共同父组件中,便可实现共享 \texttt{state}。这就是所谓的“状态提升”。

实现状态提升的方案之一是在父组件中 \texttt{state} 存储数据。再将父组件的 \texttt{state} 传递给子组件的 \texttt{props}。很明显由于 \texttt{props} 被改变,子组件会从新 \texttt{render}。

\newpage