\section{React 运行机制与 JSX 语法}

\subsection{React 运行机制}

在浏览器中\footnote{第一部分所有内容均在浏览器环境中,下文不再说明。}使用 React 需要引入两个核心库: React 和 ReactDOM。前者负责创建视图,后者负责在浏览器中渲染UI。

\subsubsection*{React 与 ReactDOM 库}

我们知道,HTML本质上是一系列指令,让浏览器构建 DOM。浏览器加载 HTML 并渲染用户界面时,构成 HTML 文档的元素变成 DOM 元素。我们通过 JavaScript 调用 DOM API 可以修改 DOM。React 是代我们更新浏览器 DOM 的一个库。有了 React 库,我们不再直接与 DOM API 交互,则是让 React 收到指令后帮助我们渲染和协调元素。

React DOM 由 React 元素组成。React 元素是对真正 DOM 元素的描述,换句话说,React 元素是如何创建浏览器 DOM 的指令。

我们可以通过 \texttt{React.createElement} 创建一个表示 \texttt{h1} 的 React 元素:

\begin{JavaScript}
React.createElement("h1", {id: "recipe-o"}, "Baked Salmon");
\end{JavaScript}

渲染时(渲染由 ReactDOM 库负责),React 把这个元素转换成真正的 DOM 元素:

\begin{HTML}
<h1 id = "recipe-o">Baked Salmon</h1>
\end{HTML}

如果在控制台输出这个元素,会看到如下内容:

\begin{JavaScript}
{
    $$typeof: Symbol(React.element),    // $$ 对象类型,一般都是 React.element 或者 HTML 元素
    "type": "h1",   // HTML 或 SVG 元素
    "key": null,    // 类似索引,用于快速获取元素,提高效率
    "ref": null,    
    "props": {id: "recipe-o", children: "Baked Salmon"},    // 元素属性,children 表示嵌套的文本
    "_owner": null,
    "_store": {},
}
\end{JavaScript}

创建 React 组件后,由 ReactDOM 负责渲染,渲染所需的 \texttt{render} 方法就在 ReactDOM 中。

\begin{JavaScript}
const dish = React.createElement("h1", null, "Baked Salmon");
ReactDOM.render(dish, document.getElementById("root"));
\end{JavaScript}

在 DOM 中与渲染有关的一些共都在 ReactDOM 包中,React16 之后,\texttt{render} 方法可以接受一个数组以渲染多个元素。

比较特殊的,\texttt{props.children} 的值可以是 React 元素对象或元的对象组成的数组:

\begin{JavaScript}
{
    "type": "ul"
    "props": {
        "children": [
            {"type": "li", "props":{...} ...}
            {"type": "li", "props":{...} ...}
            {"type": "li", "props":{...} ...}
            {"type": "li", "props":{...} ...}
        ]
    }
}
\end{JavaScript}

\subsection{JSX 语法}

上一节我们使用了 React 包中的函数直接操作 React DOM,这样做可以清晰地看出 React 是如何运作的,但是实际开发中,这样写比较复杂,可读性不高,因此有了 JSX 语法。

JSX 是一种用于描述UI的JavaScript扩展语法,React使用这种语法描述组件的 UI, JSX 本质上还是返回一个 React 元素。

JSX 的语法和 XML 类似,都是使用成对的标签构成一个树状结构的数据,例如:

\begin{JavaScript}
const element = (
    <div>
        <h1>Hello, World!</h1>
    </div>
)
\end{JavaScript}

JSX 有以下语法规范:
\begin{itemize}
    \item 节点树必须仅有一个根标签,通常是 \texttt{<div>} 或用 React 定义的 \texttt{<Wrapper>}。
    \item Html 原生标签用小写表示,React 定义的标签用大写表示。除此以外,React 定义的标签与原生标签使用上没有区别。
    \item 在 JSX 中使用 JavaScript 表达式需要用大括号包起来。且至多包括一条语句(可以是箭头函数)。如果是纯字符串则正常使用。
    \item 格式上,如果只有一个标签可以不写 (), 否则 JSX 需要被包含在 () 中。
\end{itemize}

由于 js 语法和 Html 语法关键字有重合,例如 \texttt{class}。因此部分属性的名称会有所改变,主要的变化有:HTML 中的 \texttt{class} 要写成 \texttt{className};事件属性要用小驼峰命名法,例如 \texttt{onclick -> onClick}。此外 JSX 中的注释需要用大括号包起来: \texttt{\{/**/\}}。

\subsubsection*{Fragment}

JSX 语法规定创建的 React 元素必须有且仅有一个根标签,通常情况下我们可以用 \texttt{<div>} 标签作为父标签,再做一些对应的样式调整。但有时候我们不需要多余的父标签,这会创建很多非必要的标签,导致 DOM 层级过深,此时可以使用 \texttt{<React.Fragment>} 标签表示一个 React 片段,且不会创建新标签。更简洁地,我们可以使用 \texttt{<>} 代替 \texttt{<React.Fragment>}:

\begin{JavaScript}
return (
    <>
        <h1> Hello </h1>
        <h1> World </h1>
    </>
)
\end{JavaScript}
 

\subsubsection*{JSX 的本质}

JSX 本质上只是一种语法糖,对于部分编译器,用 .js 文件写 JSX 语法也不会有问题。习惯上,我们将含有 JSX 语法的文件命名为 \texttt{.jsx} 以表示存在 UI 组件,其他脚本逻辑则保留在 \texttt{.jx} 文件中。

下面是 JSX 语法转换后的语句:

\begin{JavaScript}
// JSX
const element = <div className= 'foo' >Hello, React</div> 
// 转换后
const element = React.createElement('div',{className:'foo'},'Hello, React'); 
\end{JavaScript}

\subsubsection*{JavaScript 表达式}

在 JSX 语法中可以使用 JavaScript 表达式,一堆花括号之间的 JavaScript 代码会做求值,比如:

\begin{JavaScript}
<ul>
    {props.values.map((value, index) => (<li key="{i}">{value}</li>))}
<ul>
\end{JavaScript}







\newpage