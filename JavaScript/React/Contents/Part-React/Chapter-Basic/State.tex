\section{React 状态管理}

这一节开始,所介绍的内容都是函数组件与 Hooks。本节及以后只会用到函数组件。

\subsection{\texttt{useState} 钩子}

\subsubsection{\texttt{useState}}

在函数组件中,我们没有 \texttt{this}(更准确地说,不会用\texttt{this}),所以我们不能分配或读取 \texttt{this.state}。我们直接在组件中调用 \texttt{useState} Hook 让组件通过 state 保存状态数据:

\begin{JavaScript}
import React, { useState } from 'react';

function Example() {
  // 声明一个叫 “count” 的 state 变量。
  const [count, setCount] = useState(0);    // 数组解构

  return (
    <div>
      <p>You clicked {count} times</p>
      <button onClick={() => setCount(count + 1)}>
        Click me
      </button>
    </div>
  );
}
\end{JavaScript}

\texttt{useState} 会返回一个有两个元素的数组:当前状态和一个让你更新它的函数。它类似类组件的 \texttt{this.setState},但是它不会把新的 \texttt{state} 和旧的 \texttt{state} 进行合并(因此它是 \texttt{const} 的,也可以直接操作 \texttt{state})。\texttt{useState} 唯一的参数就是初始 \texttt{state}。

\subsubsection*{\texttt{state} 特性}

前面我们说过,\texttt{state} 会 \texttt{props} 的改变通常会需要重新 \texttt{render} 组件。因此我们对 \texttt{state} 的操作必须十分小心,只有必要的,关联组件状态的数据才需要放在 \texttt{state} 中。

在上面的代码中,我们使用 \texttt{const} 修饰的数据接受了 \texttt{useState} 的返回值,这说明 \texttt{state} 数据本身是不可修改的,为此类组件中 React 专门提供了一个 \texttt{setState} 方法修改,Hook 也提供了对应的方法用于更新 \texttt{state}(下文也叫做 \texttt{setState})。

\texttt{state} 的更新是异步的,组件的\texttt{state}并不会立即改变,\texttt{setState}只是把要修改的状态放入一个队列中,React会优化真正的执行时机,并且出于性能原因,可能会将多次\texttt{setState}的状态修改合并成一次状态修改。因此不要依赖当前的 \texttt{state} 值计算下一个 \texttt{state} 值。

由于 \texttt{state} 本身应该是不可修改的对象,因此尽可能地使用 \texttt{string,number} 等数据类型,如果要用到数组或其他可变类型,则需要考虑是应该修改这些类型的数据,还是创建一个新的对应类型传给 \texttt{state}。

\subsubsection*{组件树的状态}

为了避免状态乱用与更好地管理状态,衍生出了两种组件树之间传递状态的方案,分别是沿组件树向下发送状态和向上发送交互。

沿组件树向下发送状态很好理解,子元素接受父元素的状态作为参数。向上发送交互也很好理解,在子组件中调用 \texttt{setState} 方法即可,这里主要看一下向上发送交互:

\begin{JavaScript}
export default function StarRating({ style = {}, totalStars = 5, ...props }) {
  const [selectedStars, setSelectedStars] = useState(0);
  return (
    <div style={{ padding: 5, ...style }}>
      {createArray(totalStars).map((n, i) => (
        <Star
          key={i}
          selected={selectedStars > i}
          onSelect={() => setSelectedStars(i + 1)}
          {...props}
        />
      ))}
      <p>
        {selectedStars} of {totalStars} stars
      </p>
    </div>
  );
}
\end{JavaScript}

向上发送交互的过程:
\begin{itemize}
    \item 子标签 \texttt{<Star>} 通过 \texttt{onSelect} 函数调用 \texttt{setSelectedStars} 方法。
    \item 父组件 \texttt{selectedStar} 被改变。
    \item ReactDOM 识别到组件 state 发生改变,尝试重绘。
\end{itemize}

\subsubsection*{异步更新}

前面说过,\texttt{state} 的更新是异步的,多次\texttt{setState}的状态修改会合并成一次状态修改,如果我们需要避免异步更新导致数据不准确可以这样写:

\begin{JavaScript}
const [count, setCount] = useState(0);
function handleClickFn() {
  setTimeout(() => {
    setCount((prevCount) => {
      return prevCount + 1
    })
  }, 3000);
}
\end{JavaScript}

当使用上述方法更新(也叫函数式更新) state 的时候,就不会出现异步问题,因为它可以获取之前的 state 值,也就是代码中的 prevCount 每次都是最新的值。

\subsubsection{\texttt{useReducer}}

\texttt{useReducer} 是 \texttt{useState} 的替代方案:

\begin{JavaScript}
const [state, dispatch] = useReducer(reducer, initialArg, init);
\end{JavaScript}

\begin{itemize}
    \item \texttt{reducer}: 处理函数。
    \item \texttt{initialArg}: 初始值。
\end{itemize}

如果 state 逻辑比较复杂或者一个 state 依赖之前的 state,使用 \texttt{useReducer} 更合适。

\texttt{reducer} 函数接受两个参数,一个当前 state 值,一个 \texttt{action} 表示 \texttt{dispatch} 传入的值:

\begin{JavaScript}
/* 当state需要维护多个数据且它们互相依赖时,推荐使用useReducer
组件内部只是写dispatch({...})
处理逻辑的在useReducer函数中。获取action传过来的值,进行逻辑操作
*/

function reducer(state, action) {
  const { type, nextName } = action;
  switch (type) {
    case "ADD":
      return {
        ...state,
        age: state.age + 1
      };
    case "NAME":
      return {
        ...state,
        name: nextName
      };
  }
  throw Error("Unknown action: " + action.type);
}

export default function ReducerTest() {
  const [state, dispatch] = useReducer(reducer, { name: "qingying", age: 12 });
  function handleInputChange(e) {
    dispatch({
      type: "NAME",
      nextName: e.target.value
    });
  }
  function handleAdd() {
    dispatch({
      type: "ADD"
    });
  }
  const { name, age } = state;
  return (
    <>
      <input value={name} onChange={handleInputChange} />
      <br />
      <button onClick={handleAdd}>添加1</button>
      <p>
        Hello,{name}, your age is {age}
      </p>
    </>
  );
}
\end{JavaScript}

相当于将数据处理逻辑抽象到了 \texttt{reducer} 中处理,\texttt{setState} 改为 \texttt{dispatch} 只负责传递参数。

本质上 \texttt{useReducer} 和 \texttt{useState} 实现的效果是相同的,因此 \texttt{useReducer} 也会导致 state 改变,需要重新 render。可以将 \texttt{useReducer} 看作一种设计模式,将具体的逻辑抽象出去,更符合 React 声明式理念。 

\subsection{\texttt{useRef} 钩子}

\subsubsection{\texttt{useRef} 基础}

\texttt{useRef} 和 \texttt{useState} 一样,可以在函数组件中直接创建:

\begin{JavaScript}
const refContainer = useRef(initialValue);
\end{JavaScript}

\begin{itemize}
    \item 返回一个\texttt{可变的} \texttt{ref} 对象,该对象仅有一个 \texttt{current} 属性,初始值为传入的参数。
    \item 返回的 \texttt{ref} 对象在组建的整个生命周期保持不变。
    \item 更新 \texttt{current} 值不会导致 render。
    \item 更新 \texttt{useRef} 是副作用,应该放在 \texttt{useEffect} 里。
\end{itemize}

在 React 运行机制一章里,我们通过输出 React 元素对象,可以看到所有元素对象都有一个 \texttt{ref} 成员。可以直接使用标签属性绑定 \texttt{ref} 值。

\begin{JavaScript}
const TextInputWithFocusButton = () => {
   const inputEl = useRef(null);
   const handleFocus = () => {
       inputEl.current.focus();
   }
   return (
       <p>
           <input ref={inputEl} type="text" />
           <button onClick={handleFocus}>Focus the input</button>
       </p>
   )
}
\end{JavaScript}

这样,我们就将 \texttt{inputEl} 绑定到了 \texttt{input} 标签上,以后可以通过 \texttt{inputEl.current} 直接操作 \texttt{<input>} 元素。

除了 \texttt{useRef} 还有个 \texttt{createRef},\texttt{crateRef} 唯一的区别是每次组件更新都会重新返回新的引用,除此之外没有区别。

\subsubsection{\texttt{useRef} 共享数据}

\texttt{useState} 有一个特性,不同渲染之间无法共享 state 状态值。如果我们在函数组件中写函数并使用 \texttt{state},当我们更改状态的时候,React会重新渲染组件,每次的渲染都会拿到独立的 \texttt{state} 值,并重新定义对应的函数,每个函数体里的 state 值也是它自己的,这会导致函数内的 state 值无法更新\footnote{这个情况用的比较少,所以没有贴源码,有兴趣可以通过链接看原文}。

如果我们要在函数中获取实时状态,有两种方案,一是在函数外定义一个全局变量,由于变量是定义在组件外,所以不同渲染间是可以共用该变量,所以每次都会获取最新的变量值。这种方案的一个坏处是会定义全局变量,如果多个组件都有全局变量可能会让代码变得比较难维护,尤其是多个组件用同一个全局变量。

第二种方案是使用 \texttt{useRef}:

\begin{JavaScript}
const LikeButton: React.FC = () => {
  let like = useRef(0);
  function handleAlertClick() {
    setTimeout(() => {
      alert(`you clicked on ${like.current} ${like.current}`);
    }, 3000);
  }
  return (
    <>
      <button onClick={() => { like.current = like.current + 1; }}>
        {like.current}赞
      </button>
      <button onClick={handleAlertClick}>Alert</button>
    </>
  );
};
export default LikeButton;

\end{JavaScript}

这样的好处是数据在组件内,和其他组件没有关系。同时也说明 \texttt{useRef} 可以返回任意类型,不仅仅是绑定React 元素。

\subsubsection{\texttt{forwardRef}}

有的时候,我们需要获取React组件的某个 dom 元素,但是子组件已经被我们封装了,无法直接通过 React 组件获取内部 dom 元素节点。传统的方法是通过 Web API 一层层向下找 dom 节点。这有两个缺陷,一是如果封装的很深,获取 dom 节点的代码会很长; 二是回流可能会导致dom结构改变,用硬编码的方式会找不到 dom 节点。

于是 React 提供了 React.forwardRef 函数:

\begin{JavaScript}
React.forwardRef((props, ref) => {}) 
\end{JavaScript}

\begin{itemize}
    \item 该函数返回一个 React 组件。
    \item 这个组件会接受 \texttt{ref} 参数,用于绑定内部标签的 \texttt{ref}。
    \item 父组件通过传入 \texttt{ref} 实参与子组件对应标签的 \texttt{ref} 绑定。
    \item 父组件绑定后,可以通过\texttt{ref} 获取到对应的 dom 元素。
\end{itemize}

\begin{JavaScript}
const FancyButton = React.forwardRef((props, ref) => (  
  <button ref={ref} className="FancyButton">    
    {props.children}
  </button>
));
const ref = React.useRef();
<FancyButton ref={ref}>Click me!</FancyButton>;
\end{JavaScript}

\subsubsection{\texttt{useImperativeHandle}}

直接暴露给父组件带来的问题是某些情况的不可控,父组件可以拿到DOM后进行任意的操作,但多数情况下我们希望限定只做某些操作。通过\texttt{useImperativeHandle}可以值暴露固定的操作。

\begin{JavaScript}
useImperativeHandle(ref, createHandle, [deps])
\end{JavaScript}

\begin{JavaScript}
function FancyInput(props, ref) {
  const inputRef = useRef();
  useImperativeHandle(ref, () => ({
    focus: () => {
      inputRef.current.focus();
    }
  }));
  return <input ref={inputRef} ... />;
}
FancyInput = forwardRef(FancyInput);
\end{JavaScript}

这样,渲染 <FancyInput ref={fancyInputRef} /> 的父组件就只可以调用 fancyInputRef.current.focus()。

\texttt{useImperativeHandle} 一般与 \texttt{forwardRef} 配合使用。

\subsection{\texttt{useContext} 钩子}

Context 提供了一个无需为每层组件手动添加 \texttt{props},就能在组件树间进行数据传递的方法。例如当前认证的用户、主题或首选语言。

\texttt{useContext} 不同于 \texttt{useState} 与 \texttt{useRef},他需要先通过 \texttt{React.createContext} 创建数据。再通过 \texttt{useContext()} 或者 \texttt{Context.Provider} 使用数据。

最常用的使用方式如下:

\begin{JavaScript}
// 调用 API 创建一个 Context
const ThemeContext = React.createContext('light');
function MyElement() {
  const theme = useContext(ThemeContext); // 使用 Context Hook
  return (
    <button style={{ background: theme.background, color: theme.foreground }}>
      I am styled by theme context!
    </button>
  );
}
\end{JavaScript}

React 提供的关于 Context 的 API 有这些(不包含 Context Hook):
\begin{itemize}
  \item \textbf{\texttt{React.createContext}}
  
\begin{JavaScript}
  const MyContext = React.createContext(defaultValue);
\end{JavaScript}
  
  创建一个 Context 对象。当 React 渲染一个订阅了这个 Context 对象的组件,这个组件会从组件树中离自身最近的那个匹配的 \texttt{Provider} 中读取到当前的 \texttt{context} 值。只有当组件所处的树中没有匹配到 Provider 时,其 \texttt{defaultValue} 参数才会生效。这有助于在不使用 Provider 包装组件的情况下对组件进行测试。

  \item \textbf{\texttt{Context.Provider}}

\begin{HTML}
<MyContext.Provider value={/* 某个值 */}>
\end{HTML}

每个 Context 对象都会返回一个 Provider React 组件,它允许消费组件订阅 context 的变化。Provider 接收一个 \texttt{value} 属性,传递给消费组件。一个 Provider 可以和多个消费组件有对应关系。多个 Provider 也可以嵌套使用,里层的会覆盖外层的数据。

当 Provider 的 \texttt{value} 值发生变化时,它内部的所有消费组件都会重新渲染。Provider 及其内部 consumer 组件都不受制于 \texttt{shouldComponentUpdate} 函数,因此当 consumer 组件在其祖先组件退出更新的情况下也能更新。

  \item \textbf{\texttt{Class.contextType}}
  
  挂载在 class 上的 \texttt{contextType} 属性会被重赋值为一个由 React.createContext() 创建的 Context 对象。这能让你使用 \texttt{this.context} 来消费最近 Context 上的那个值。你可以在任何生命周期中访问到它,包括 render 函数中。

  \item \textbf{\texttt{Context.Consumer}}
  
\begin{HTML}
<MyContext.Consumer>
  {value => /* 基于 context 值进行渲染*/}
</MyContext.Consumer>
\end{HTML}

  这里,React 组件也可以订阅到 context 变更。这能让你在函数式组件中完成订阅 context。

  \item \textbf{\texttt{Context.displayName}}
  
\begin{HTML}
const MyContext = React.createContext(/* some value */);
MyContext.displayName = 'MyDisplayName';
<MyContext.Provider> // "MyDisplayName.Provider" 在 DevTools 中
<MyContext.Consumer> // "MyDisplayName.Consumer" 在 DevTools 中
\end{HTML}

  context 对象接受一个名为 displayName 的 property,类型为字符串。React DevTools 使用该字符串来确定 context 要显示的内容。
\end{itemize}

比较少用的,结合 \texttt{<Context.Provider>} 与 \texttt{<Context.Consumer>}:

\begin{JavaScript}
const ThemeContext = createContext()

class App extends React.Component {
  render () {
    return (
      // 使用 Context.Provider 包裹后续组件,value指定值 
      <ThemeContext.Provider value={'red'}>
        <Bottom></Bottom>
      </ThemeContext.Provider>
    )
  }
}

class Bottom extends React.Component {
  render () {
    return (
      // Context.Consumer Consumer消费者使用Context的值
      // 但子组件不能是其他组件,必须渲染一个函数,函数的参数就是Context的值
      <ThemeContext.Consumer>
        {
          theme => <h1>ThemeContext的值为{theme}</h1>
        }
      </ThemeContext.Consumer>
    )
  }
}
\end{JavaScript}

\newpage