\section{React 简介}

前端UI的本质问题是如何将来源于服务器端的动态数据和用户的交互行为高效地反映到复杂的用户界面上。React另辟蹊径,通过引入虚拟DOM、状态、单向数据流等设计理念,形成以组件为核心,用组件搭建U的开发模式。

React 的特点可以归结为以下 4 点:
\begin{itemize}
    \item \textbf{声明式视图层}: React 采用 JSX 语法来声明视图层,可以在视图中绑定各种状态数据以及相关操作。
    \item \textbf{简易更新流}: 声明式的视图定义方式有助于简化视图层的更新流程。你只需要定义u状态,React便会负责把它渲染成最终的UI。
    \item \textbf{灵活渲染实现}: React并不是把视图直接渲染成最终的终端界面,而是先把它们渲染成虚拟DOM。虚拟DOM再在各个平台渲染(react-dom 对应浏览器;Node 对应服务端;React Native 对应移动端)。
    \item \textbf{高效的 DOM 操作}: 基于React优异的差异比较算法,React可以尽量减少虚拟DOM到真实DOM的渲染次数,以及每次渲染需要改变的真实DOM节点数。
\end{itemize}

\subsection{React 开发环境}

React 应用开发有两个必要环境:
\begin{itemize}
    \item \textbf{Node.js}: React 在本地开发调试需要使用到 Node.js 环境中的 NPM,Webpack 等依赖。
    \item \textbf{NPM}: 模块管理工具,用来管理模块之间的依赖关系。也可以用 \textbf{Yarn} 代理,安装 Node。
    \item \textbf{Babel}: Babel是一个JavaScript编译器,为了浏览器兼容性考虑,需要把 ES6 或以后的语法编译成 ES5 及之前的语法达到向前兼容的目的,同时也负责编译 JSX 语法。
    \item \textbf{ESLint}: JavaScript 代码检查工具,由于 JS 的语法非常乱,同一种实现有多种写法,风格不尽相同,为了团队同一管理,会用 ESLint 进行风格检测。
\end{itemize}

这些工具的使用方法比较繁琐,可以直接用 React 提供的脚手架工具构建工程。在官方文档中提供了以下几个命名用于快速构建工程:

\begin{bash}
# 在当前目录下创建 my-app 项目
npx create-react-app my-app
# 运行项目
cd my-app
npm start
\end{bash}

通过这种方式创建的 React 项目结构如下(仅重要文件/文件夹):

\begin{bash}
my-app
|- README.md            # react 相关的指令介绍(可删除)
|- .gitignore           # 版本控制
|- package.json         # 项目信息
|- package-lock.json    # 项目绑定信息
|- node_modules         # 工程依赖的模块,会被 .gitignore 忽视
|- public               # 外部访问文件
    |- index.html       # 应用入口界面
    |- manifest.json    # 应用注册信息
|- src              # 项目源代码,主要工作区
    |- index.js     # 源代码入口
    |- react-app-env    # 应用变量环境
\end{bash}

放入 public 文件夹下的资源可以被直接引用。

还有很多其他文件,但主要的,启动一个项目会进入 \texttt{public/index.html} 界面,而这个界面一般加载了 \texttt{src/index.js} 脚本。开发者一般在 \texttt{src} 文件夹中写入功能。

此外,如果使用 typescript开发,创建项目指令如下:

\begin{bash}
npx create-react-app my-app --template typescript
\end{bash}

会新增几个 ts 管理文件。

\subsubsection*{\texttt{create-react-app}}

\texttt{create-react-app} 是一个单独的包,用于快速构建 React 项目,开源地址: \url{https://github.com/facebook/create-react-app}。

但实际上我们使用改包创建完 React 项目后,\texttt{package} 文件中并没有对应的包信息,这与 \texttt{npx} 命名的执行逻辑有关。使用 \texttt{npx} 命令会自动查找当前依赖包中的可执行文件,如果没有就在 \texttt{PATH} 中寻找,再找不到就会帮助我们安装,且安装的包在使用后会被卸载。

也就是说我们使用上述指令后,\texttt{npx} 帮我们安装了 \texttt{create-react-app} 包,然后构建了项目,最后将 \texttt{create-react-app} 包删除。渣男行为属于是。

非常遗憾的是 create-react-app 项目已经很久没有更新了,社会活力低下。而且由于采用 webpack 打包方式,创建的项目体量一旦上去,热加载时间很慢。

\subsubsection{Vite 开发环境}

Vite 是新一代前端工具链,现在大型项目逐渐由 Webpack 转向 Vite。这两者有如下几点不同:

\begin{table}[H]
    \centering
    \caption{Vite 与 Webpack 对比}
    \label{table:Vite 与 Webpack 对比}
    \setlength{\tabcolsep}{4mm}
    \begin{tabular}{c|ccc}
        \toprule
        \textbf{} & \textbf{Vite} & \textbf{Webpack} & \textbf{备注} \\
        \midrule
        编写语言 & Go & JavaScript & Go是纳秒级响应,理论上速度快 10-100 倍 \\
        加载模式 & 懒加载 & 勤加载 & Vite 利用 ESModule 按需加载必要模块 \\
        定位 & 打包+工具 & 纯打包 &  \\
        使用 & 需下载 & node 集成 &  \\
        \bottomrule
    \end{tabular}
\end{table}

Webpack 项目开始时,ES6 尚未推出,因此 Webpack 并没有引入 ESModule。Webpack 的打包策略是加载所有的文件进行编译,最终转换为原生 js 再让浏览器显示,随着项目体积增加,打包时间也线性增加,一个复杂的项目,使用 Webpack 热加载往往需要几秒钟的时间才能看到界面。

Vite 利用 ESModule 的特性,只加载必要的文件,所编译的组件会自动导入其他模块并完成编译,也即按需加载。因此 Vite 加载时间始终是 O(1),界面几乎是瞬间加载完成。

Vite 不能完全替代 Webpack,但现在绝大多数新项目都是用 Vite 打包。

继续上文,如果使用 vite 创建项目就不能使用 create-react-app:

\begin{bash}
npm create vite@latest app --template react-ts
\end{bash}

Vite 官方地址: \url{https://cn.vitejs.dev/}

\subsection{React 编程风格}

个人总结的 React 三个直观的编程风格: 声明式,all-in-js,vDOM。
\begin{itemize}
    \item \textbf{声明式}: 编程风格。
    
    声明式是函数式编程的一种风格,与之对应的是命令式。命令式很好理解,就是具体的命令,逻辑处理;命令式函数会显式地表明内部逻辑。声明式可以简单理解为只调用相关操作,不考虑内部具体逻辑。在实际项目中表现为将具体功能封装后,主要函数只负责调用。因此在 React 项目中很少直接看到函数该怎么做,而是关注于做什么。
    \item \textbf{all-in-js}: 组件化。
    
    前端三剑客中 HTML 负责结构,CSS 负责样式,JavaScript 负责逻辑。这样的分化存在一个问题,当项目过大后,一点小改动会使得整个前端牵一发而动全身。可能因为 HTML 的改变,DOM 结构发生变化;因为 CSS 的改变,预料之外的元素样式发生了变化...
    
    React 认为这三者天然存在耦合性,因此用 JSX 控制 HTML,将三者统一由js控制,而将整个界面分为零散的小组件进行管理。如果使用 \texttt{styled-component} 或 \texttt{TailWindCss},js 可以进一步控制 css,真正实现 all-in-js。

    \item \textbf{vDOM}: 性能优化。
    
    在 React 中可以创建 React 元素,即可以“自定义”新标签(虽然这些标签实际上还是原生 DOM),React 标签可以进一步封装。同时 vDOM 可以更加智能地让原生界面进行重绘与回流,React 内部做了大量优化让开发者既能快速开发,又不用担心性能。
\end{itemize}

此外,React 比 Vue 的一大优势是强大的生态,除了 react 项目本身,还有 react-native,next.js,redux.js 等众多优秀的项目。

\newpage