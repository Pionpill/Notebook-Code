\section{React 渲染管理}
\subsection{\texttt{useEffect} 钩子}

\subsubsection{\texttt{useEffect}}

Effect Hook 对应的函数是 \texttt{useEffect} ,它给函数组件增加了操作副作用的能力。它跟类组件中的 \texttt{componentDidMount、componentDidUpdate} 和 \texttt{componentWillUnmount} 具有相同的用途,只不过被合并成了一个 API。

\begin{JavaScript}
// 相当于 componentDidMount 和 componentDidUpdate:
useEffect(() => {
  // 使用浏览器的 API 更新页面标题
  document.title = `You clicked ${count} times ${count}`;
});
\end{JavaScript}

当你调用 useEffect 时,就是在告诉 React 在完成对 DOM 的更改后(render 之后)运行你的“副作用”函数。由于副作用函数是在组件内声明的,所以它们可以访问到组件的 \texttt{props} 和 \texttt{state}。与类组件生命周期不同的是,\texttt{useEffect} 调度的 \texttt{effect} 不会阻塞浏览器更新屏幕。

\texttt{useEffect} 可以返回一个函数,React 将会在执行清除操作时调用它(对应生命周期: \texttt{compo nentWillUnmount}),因此常被命名为 \texttt{cleanup}。

\begin{JavaScript}
useEffect(() => {
  return () => "bye";   // 清理时调用
})
\end{JavaScript}

在某些情况下,每次渲染后都执行清理或者执行 \texttt{effect} 可能会导致性能问题。在类组件中,我们可以通过在 \texttt{componentDidUpdate} 中添加对 \texttt{prevProps} 或 \texttt{prevState} 的比较逻辑解决,在 \texttt{useEffect} 中可以通过添加第二个可选参数(类型是数组)实现相同效果:

\begin{JavaScript}
useEffect(() => {
  document.title = `You clicked ${count} times ${count}`;
}, [count]); // 仅在 count 更改时更新
\end{JavaScript}

\subsubsection{\texttt{useLayoutEffect}}

\texttt{useLayoutEffect} 与 \texttt{useEffect} 作用相同: 处理 render 之后的副作用,但在组件生命周期调用顺序不同:

\begin{itemize}
  \item 渲染(render);
  \item 调用 \texttt{useLayoutEffect};
  \item 浏览器绘制,将组件元素添加到DOM;
  \item 调用 \texttt{useEffect};
\end{itemize}

\subsection{\texttt{useMemo} 钩子}

本节参考文献: \url{https://blog.csdn.net/sinat_17775997/article/details/94453167}

\texttt{useMemo} 用来判读一个函数组件是否要重新渲染,主要目的是减少不必要的重绘来提升性能。作用和类组件的 \texttt{shouldComponentUpdate} 函数相同。

\begin{JavaScript}
useMemo(fn, arr)
\end{JavaScript}

函数组件只要 props 或 state 发生改变就会重绘,不会判断改变后的值是否相同。这样会导致性能下降,\texttt{useMemo} 会比较传入的第二个参数是否发生了变化再判断是否要执行\texttt{fn}。

只有第二个参数匹配,并且其值发生改变,才会多次执行执行,否则只执行一次,如果为空数组,\texttt{fn}只执行一次。

\begin{JavaScript}
export default function WithoutMemo() {
    const [count, setCount] = useState(1);
    const [val, setValue] = useState('');
 
    function expensive() {
        console.log('compute');
        let sum = 0;
        for (let i = 0; i < count * 100; i++) {
            sum += i;
        }
        return sum;
    }
 
    return <div>
        <h4>{count}-{val}-{expensive()}</h4>
        <div>
            <button onClick={() => setCount(count + 1)}>+c1</button>
            <input value={val} onChange={event => setValue(event.target.value)}/>
        </div>
    </div>;
}
\end{JavaScript}

上面这个反例中,\texttt{expensive} 函数值依赖于 \texttt{count},但由于 \texttt{val} 也是 state,它的改变会让整个组件重绘,\texttt{expensive} 也会随之重新计算。这时候我们可以使用 \texttt{useMemo} 优化:

\begin{JavaScript}
export default function WithMemo() {
  const [count, setCount] = useState(1);
  const [val, setValue] = useState('');
  const expensive = useMemo(() => {
      console.log('compute');
      let sum = 0;
      for (let i = 0; i < count * 100; i++) {
          sum += i;
      }
      return sum;
  }, [count]);

  return <div>
      <h4>{count}-{expensive}</h4>
      {val}
      <div>
          <button onClick={() => setCount(count + 1)}>+c1</button>
          <input value={val} onChange={event => setValue(event.target.value)}/>
      </div>
  </div>;
}
\end{JavaScript}

这样 \texttt{expensive} 就只依赖 \texttt{count},只有依赖项改变才会触发 \texttt{expensive} 重新计算值,否则返回之前的值。

\subsection{\texttt{useCallback} 钩子}

本节参考文献: \url{https://blog.csdn.net/sinat_17775997/article/details/94453167}

\texttt{useCallback} 和 \texttt{useMemo} 类似,都用于优化渲染,不同的是,它但会的是缓存的函数。

\begin{JavaScript}
const fnA = useCallback(fnB, [a])
\end{JavaScript}

上面的\texttt{useCallback}会将我们传递给它的函数\texttt{fnB}返回,并且将这个结果缓存;当依赖\texttt{a}变更时,会返回新的函数。

使用场景是:有一个父组件,其中包含子组件,子组件接收一个函数作为props;通常而言,如果父组件更新了,子组件也会执行更新;但是大多数场景下,更新是没有必要的,我们可以借助\texttt{useCallback}来返回函数,然后把这个函数作为props传递给子组件;这样,子组件就能避免不必要的更新。

\begin{JavaScript}
import React, { useState, useCallback, useEffect } from 'react';
function Parent() {
    const [count, setCount] = useState(1);
    const [val, setVal] = useState('');
 
    const callback = useCallback(() => {
        return count;
    }, [count]);
    return <div>
        <h4>{count}</h4>
        <Child callback={callback}/>
        <div>
            <button onClick={() => setCount(count + 1)}>+</button>
            <input value={val} onChange={event => setVal(event.target.value)}/>
        </div>
    </div>;
}
 
function Child({ callback }) {
    const [count, setCount] = useState(() => callback());
    useEffect(() => {
        setCount(callback());
    }, [callback]);
    return <div>
        {count}
    </div>
}
\end{JavaScript}

useEffect、useMemo、useCallback都是自带闭包的。也就是说,每一次组件的渲染,其都会捕获当前组件函数上下文中的状态(state, props),所以每一次这三种hooks的执行,反映的也都是当前的状态,你无法使用它们来捕获上一次的状态。对于这种情况,我们应该使用ref来访问。


\subsection{自定义 Hook}

自定义 Hook 主要用于解决组件之间逻辑共享问题。当我们想在两个函数之间共享逻辑时,我们会把它提取到第三个函数中。而组件和 Hook 都是函数,所以也同样适用这种方式。

自定义 Hook 是一个函数,其名称以 “use” 开头,函数内部可以调用其他的 Hook:

\begin{JavaScript}
import { useState, useEffect } from 'react';

function useFriendStatus(friendID) {
  const [isOnline, setIsOnline] = useState(null);

  useEffect(() => {
    function handleStatusChange(status) {
      setIsOnline(status.isOnline);
    }

    ChatAPI.subscribeToFriendStatus(friendID, handleStatusChange);
    return () => {
      ChatAPI.unsubscribeFromFriendStatus(friendID, handleStatusChange);
    };
  });

  return isOnline;
}
\end{JavaScript}

与 React 组件不同的是,自定义 Hook 不需要具有特殊的标识。我们可以自由的决定它的参数是什么,以及它应该返回什么(如果需要的话)。换句话说,它就像一个正常的函数。但是它的名字应该始终以 \texttt{use} 开头,这样可以一眼看出其符合 Hook 的规则。

此处 \texttt{useFriendStatus} 的 Hook 目的是订阅某个好友的在线状态。这就是我们需要将 \texttt{friendID} 作为参数,并返回这位好友的在线状态的原因。

使用自定义 Hook 只需要在函数组件中调用即可。

\begin{JavaScript}
function FriendStatus(props) {
  const isOnline = useFriendStatus(props.friend.id);

  if (isOnline === null) {
    return 'Loading...';
  }
  return isOnline ? 'Online' : 'Offline';
}
\end{JavaScript}

\subsubsection{\texttt{useDebugValue()}}

\begin{JavaScript}
useDebugValue(value)
\end{JavaScript}

用于在 React 开发者工具中显示自定义 hook 的标签。

\texttt{useDebugValue} 接受一个格式化函数作为可选的第二个参数。该函数只有在 Hook 被检查时才会被调用。它接受 debug 值作为参数,并且会返回一个格式化的显示值。

\newpage