\documentclass{PionpillNote-book}

\title{React 笔记}
\author{
    Pionpill \footnote{笔名:北岸,电子邮件:673486387@qq.com,Github:\url{https://github.com/Pionpill}} \\
    本文档为作者学习 React 及相关技术的笔记。\\
}

\date{\today}

\begin{document}

\pagestyle{plain}
\maketitle

\noindent\textbf{前言:}

React 是 Facebook 开源的目前最主流的前端框架,其核心理念是 all-in-js。React 本身并不复杂,需要的前置知识如下:
\begin{itemize}
    \item HTML5
    \item CSS3
    \item JavaScript(ES6+)
\end{itemize}

最好拥有 TypeScript 基础,文中靠后的一些章节会使用到。

本文的主要参考文献如下:
\begin{itemize}
    \item React 中文官网: \url{https://react.docschina.org/}
    \item React 学习手册: [美]Alex Banks 安道译 中国电力出版社 2021 (第二版)
    \item React 进阶之路: 徐超 清华大学出版社 2018
\end{itemize}

本文撰写环境:

\begin{itemize}
    \item React: 18.2.0
    \item IDE: VSCode 1.72 
    \item Chrome: 91.0
    \item Node.js: 18.12
    \item OS: Window11
\end{itemize}

\date{\today}
\newpage

\tableofcontents

\newpage

\setcounter{page}{1} 
\pagestyle{fancy}

\part{React}
\chapter{React 基础篇}
\import{Contents/Part-React/Chapter-Basic}{Introduction.tex}
\import{Contents/Part-React/Chapter-Basic}{JSX.tex}
\import{Contents/Part-React/Chapter-Basic}{Class.tex}
\import{Contents/Part-React/Chapter-Basic}{State.tex}
\import{Contents/Part-React/Chapter-Basic}{Render.tex}
\import{Contents/Part-React/Chapter-Basic}{Fetch.tex}
\chapter{React 模块}
\import{Contents/Part-React/Chapter-Module}{ReactRouter.tex}

\end{document}

