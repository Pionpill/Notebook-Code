\section{函数}

我们知道,JavaScript 对待函数是十分宽裕的,可以有任意个参数,对参数只做匹配不检查。但 TypeScript 不同,TypeScript 要求函数参数必须有类型,且标注返回值类型,即使没有显示指定类型,编译器也会自动推导类型。

\begin{TypeScript}
function add(x: number, y: number): number {
    return x + y;
}
\end{TypeScript}

TypeScript里的每个函数参数都是必须的。编译器检查用户是否为每个参数都传入了值。编译器还会假设只有这些参数会被传递进函数。 简短地说,传递给一个函数的参数个数必须与函数期望的参数个数一致。

\begin{TypeScript}
function buildName(firstName: string, lastName: string) {
    return firstName + " " + lastName;
}
let result1 = buildName("Bob");                     // error, too few parameters
let result2 = buildName("Bob", "Adams", "Sr.");     // error, too many parameters
let result3 = buildName("Bob", "Adams");            // ah, just right
\end{TypeScript}

JavaScript 中每个函数参数都是可选的,不传的时候就是 \texttt{undefined}。在 TypeScript 中可选参数需要使用 \texttt{?} 关键字, 同时可选参数必须跟在必选参数后。

\begin{TypeScript}
function buildName(firstName: string, lastName?: string) {
    if (lastName)
        return firstName + " " + lastName;
    else
        return firstName;
}
let result1 = buildName("Bob");                     // works correctly now
let result2 = buildName("Bob", "Adams", "Sr.");     // error, too many parameters
let result3 = buildName("Bob", "Adams");            // ah, just right
\end{TypeScript}

当然 TypeScript 也支持默认参数值,用法和 JavaScript 一致。

TypeScript 拾取剩余参数的方式和 JavaScript 一致,只不过只要指定类型为 \texttt{string[]}。

\begin{TypeScript}
function buildName(firstName: string, ...restOfName: string[]) {
  return firstName + " " + restOfName.join(" ");
}
let employeeName = buildName("Joseph", "Samuel", "Lucas", "MacKinzie");
\end{TypeScript}

\subsubsection*{\texttt{this}}

JavaScript里,\texttt{this}的值在函数(非箭头函数)被调用的时候才会指定,从而确定上下文。在 TypeScript 中,我们可以通过将函数的第一个参数设置为 \texttt{this} 用于表示返回的 \texttt{this} 指代函数本身。

\begin{TypeScript}
function f(this: void) {
    // make sure `this` is unusable in this standalone function
}
\end{TypeScript}

\subsubsection*{函数重载}

JavaScript本身是个动态语言。 JavaScript里函数根据传入不同的参数而返回不同类型的数据是很常见的。由于 TypeScript 是静态语言,也就失去了动态处理函数参数的能力,但同时,TypeScript 允许函数重载。

\newpage