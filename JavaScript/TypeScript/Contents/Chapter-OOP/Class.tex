\section{类}

传统的JavaScript程序使用函数和基于原型的继承来创建可重用的组件,但对于熟悉使用面向对象方式的程序员来讲就有些棘手,因为他们用的是基于类的继承并且对象是由类构建出来的。

\subsection{类基础}

在TypeScript里,我们可以使用常用的面向对象模式。 基于类的程序设计中一种最基本的模式是允许使用继承来扩展现有的类

\begin{TypeScript}
class Animal {
    move(distanceInMeters: number = 0) {
        console.log(`Animal moved ${distanceInMeters}${distanceInMeters}m.`);
    }
}
class Dog extends Animal {
    bark() {
        console.log('Woof! Woof!');
    }
}
const dog = new Dog();
dog.bark();
dog.move(10);
\end{TypeScript}

在功能上,TypeScript 的类与 Java 类似。

\subsubsection*{成员作用域}

是的,TypeScript 支持类成员作用域: \texttt{public, protected, private}。默认为 \texttt{public}, 且这三种作用域功能和 Java 中的类似。

TypeScript使用的是结构性类型系统。 当我们比较两种不同的类型时,并不在乎它们从何处而来,如果所有成员的类型都是兼容的,我们就认为它们的类型是兼容的。

然而,当我们比较带有 \texttt{private}或 \texttt{protected}成员的类型的时候,情况就不同了。 如果其中一个类型里包含一个 \texttt{private}成员,那么只有当另外一个类型中也存在这样一个 \texttt{private}成员, 并且它们都是来自同一处声明时,我们才认为这两个类型是兼容的。 对于 \texttt{protected}成员也使用这个规则。

\begin{TypeScript}
class Animal {
    private name: string;
    constructor(theName: string) { this.name = theName; }
}
class Rhino extends Animal {
    constructor() { super("Rhino"); }
}
class Employee {
    private name: string;
    constructor(theName: string) { this.name = theName; }
}
let animal = new Animal("Goat");
let rhino = new Rhino();
let employee = new Employee("Bob");

animal = rhino;
animal = employee; // 错误: Animal 与 Employee 不兼容.
\end{TypeScript}

\subsubsection*{参数属性}

参数属性可以方便地让我们在一个地方定义并初始化一个成员。 

\begin{TypeScript}
class Octopus {
    readonly numberOfLegs: number = 8;
    constructor(readonly name: string) {
    }
}
\end{TypeScript}

仅在构造函数里使用 \texttt{readonly name: string} 参数来创建和初始化 \texttt{name}成员。 我们把声明和赋值合并至一处。

\subsubsection*{存取器}

TypeScript支持通过 \texttt{getters/setters} 来截取对对象成员的访问。 它能帮助你有效的控制对对象成员的访问。

\begin{TypeScript}
class Employee {
    fullName: string;
}

let employee = new Employee();
employee.fullName = "Bob Smith";
if (employee.fullName) {
    console.log(employee.fullName);
}
\end{TypeScript}

我们可以随意的设置 \texttt{fullName},这是非常方便的,但是这也可能会带来麻烦。

下面这个版本里,我们先检查用户密码是否正确,然后再允许其修改员工信息。 我们把对 \texttt{fullName}的直接访问改成了可以检查密码的 \texttt{set}方法。 我们也加了一个 \texttt{get}方法,让上面的例子仍然可以工作。

\begin{TypeScript}
let passcode = "secret passcode";
class Employee {
    private _fullName: string;
    get fullName(): string {
        return this._fullName;
    }
    set fullName(newName: string) {
        if (passcode && passcode == "secret passcode") {
            this._fullName = newName;
        }
        else {
            console.log("Error: Unauthorized update of employee!");
        }
    }
}
let employee = new Employee();
employee.fullName = "Bob Smith";
if (employee.fullName) {
    alert(employee.fullName);
}
\end{TypeScript}

只带有 \texttt{get}不带有 \texttt{set}的存取器自动被推断为 \texttt{readonly}。

\subsubsection*{静态属性}

TypeScript 的类支持静态属性,只需要通过类名访问即可:

\begin{TypeScript}
class Grid {
    static origin = {x: 0, y: 0};
    calculateDistanceFromOrigin(point: {x: number; y: number;}) {
        let xDist = (point.x - Grid.origin.x);
        let yDist = (point.y - Grid.origin.y);
        return Math.sqrt(xDist * xDist + yDist * yDist) / this.scale;
    }
    constructor (public scale: number) { }
}
let grid1 = new Grid(1.0);  // 1x scale
let grid2 = new Grid(5.0);  // 5x scale
console.log(grid1.calculateDistanceFromOrigin({x: 10, y: 10}));
console.log(grid2.calculateDistanceFromOrigin({x: 10, y: 10}));
\end{TypeScript}

\subsubsection*{抽象类}

抽象类做为其它派生类的基类使用。 它们一般不会直接被实例化。 不同于接口,抽象类可以包含成员的实现细节。 abstract关键字是用于定义抽象类和在抽象类内部定义抽象方法。

\begin{TypeScript}
abstract class Animal {
    abstract makeSound(): void;
    move(): void {
        console.log('roaming the earch...');
    }
}
\end{TypeScript}

\subsection{类的本质}

当一个 TypeScript 类被编译成 JavaScript 后,他会变成什么样子?

\begin{TypeScript}
// TypeScript
class Greeter {
    greeting: string;
    constructor(message: string) {
        this.greeting = message;
    }
    greet() {
        return "Hello, " + this.greeting;
    }
}
let greeter: Greeter;
greeter = new Greeter("world");
console.log(greeter.greet());

// JavaScript
let Greeter = (function () {
    function Greeter(message) {
        this.greeting = message;
    }
    Greeter.prototype.greet = function () {
        return "Hello, " + this.greeting;
    };
    return Greeter;
})();
let greeter;
greeter = new Greeter("world");
console.log(greeter.greet());
\end{TypeScript}

上面的代码里, \texttt{let Greeter}将被赋值为构造函数。 当我们调用 \texttt{new}并执行了这个函数后,便会得到一个类的实例。 这个构造函数也包含了类的所有静态属性。 换个角度说,我们可以认为类具有 实例部分与 静态部分这两个部分。

类定义会创建两个东西:类的实例类型和一个构造函数。 因为类可以创建出类型,所以你能够在允许使用接口的地方使用类。

\newpage