\documentclass{PionpillNote-book}

\title{TypeScript Notebook}
\author{
    Pionpill \footnote{笔名:北岸,电子邮件:673486387@qq.com,Github:\url{https://github.com/Pionpill}} \\
    本文档为作者学习 TypeScript 的笔记。\\
}

\date{\today}

\begin{document}

\pagestyle{plain}
\maketitle

\noindent\textbf{前言:}

本位介绍 TypeScript 相关语法。TypeScript 是 JavaScript 的超集,是有类型的 JavaScript,他比 JavaScript 更加严格,有更多的 OOP 理念,适用于大型项目,对后端程序员更加友好。

TypeScript 是 JavaScript 的超集。在实际应用中先用 TypeScript 语法编写脚本,再使用 \texttt{tsc} 命令将 ts 文件编译成 js 文件,在运行。因此,JavaScript 能用的语法,TypeScript 绝大部分都能用,即使报错不建议用,也可以强制编译。

本文默认读者会 JavaScript,因此相关内容不再说明。主要参考资料为微软官方的 TypeScript HandBook。

环境如下:
\begin{itemize}
    \item OS: Window11
    \item TypeScript: 3
\end{itemize}

\date{\today}
\newpage

\tableofcontents

\newpage

\setcounter{page}{1} 
\pagestyle{fancy}

\part{ECMAScript}
\chapter{基础语法}
\import{Contents/Chapter-Type}{Basic.tex}
\import{Contents/Chapter-Type}{Function.tex}
\chapter{面向对象}
\import{Contents/Chapter-OOP}{Interface.tex}
\import{Contents/Chapter-OOP}{Class.tex}


\end{document}

