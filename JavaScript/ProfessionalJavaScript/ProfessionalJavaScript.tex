\documentclass{PionpillNote-book}

\title{JavaScript 高级程序设计 笔记}
\author{
    Pionpill \footnote{笔名:北岸,电子邮件:673486387@qq.com,Github:\url{https://github.com/Pionpill}} \\
    本文档为作者学习《JavaScript 高级程序设计》\footnote{《Professional JavaScript for Web Developers》:Matt Frisbie 2020年9月第2版}一书时的笔记。\\
}

\date{\today}

\begin{document}

\pagestyle{plain}
\maketitle

\noindent\textbf{前言:}

笔者为软件工程系在校本科生,主要用 JavaScript 做一些前端项目。本文默认读者有一定的编程基础(至少会一门C语言系语言),基础语法不再说明。

JavaScript 语法非常灵活(乱), 有很多没必要记的东西,也有很多不明所以的设计方案,这和他的历史有关。现在主流的做法是使用 TypeScript 编写前端脚本,因此本人增加了 TypeScript 这一部分。

<<JavaScript 高级程序设计>> 是 JavaScript 学习的一本入门及进阶书,原书中文版多达117万字,笔者将其分为如下几个部分:
\begin{itemize}
    \item ECMAScript:介绍ECMAScript(JavaScript)的语法。
    \item TypeScript: 介绍TypeScript的语法。 
    \item DOM 与 BOM:与 Web 开发相关的内容。
    \item API:处理各类事务文件的接口。
\end{itemize}

本笔记不能代替原书,仅是对原书的一个总结归纳,笔记上只有知识点的总结,并没有详细的理解性语句,如有需要,还请购买原书。本文引用了一些 CSDN 或其他论讨的文章,有标注来源,如果原作者觉得不合适,请联系本人。

关于 Web 开发,不得不考虑浏览器兼容问题,令人欣慰的是,微软已经弃用 IE 浏览器,改用 chromium 核心的 Edge 浏览器,本文以 Chrome 浏览器为准,不再考虑任何IE(或其他浏览器)适配问题。

本人的编写及开发环境如下:
\begin{itemize}
    \item IDE: VSCode 1.72 
    \item Chrome: 91.0
    \item Node.js: 18.12
    \item OS: Window11
\end{itemize}

此外,JavaScript 是一门语法十分宽松的语言,同一种实现有非常多的写法,笔记中只强调推荐的写法,更多写法请读者用用现代编辑器代码提示功能查询。

\date{\today}
\newpage

\tableofcontents

\newpage

\setcounter{page}{1} 
\pagestyle{fancy}

\part{ECMAScript}
\chapter{基础语法}
\import{Parts/Part-ECMAScript/Chapter-Syntax}{Introduction.tex}
\import{Parts/Part-ECMAScript/Chapter-Syntax}{Syntax.tex}
\import{Parts/Part-ECMAScript/Chapter-Syntax}{Variable.tex}
\chapter{数据类型}
\import{Parts/Part-ECMAScript/Chapter-Type}{Simple.tex}
\import{Parts/Part-ECMAScript/Chapter-Type}{Reference.tex}
\import{Parts/Part-ECMAScript/Chapter-Type}{Collection.tex}
\chapter{函数与类}
\import{Parts/Part-ECMAScript/Chapter-Function}{Function.tex}
\import{Parts/Part-ECMAScript/Chapter-Function}{Iterator.tex}
\import{Parts/Part-ECMAScript/Chapter-Function}{Class.tex}


\end{document}

