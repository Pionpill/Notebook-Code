\section{集合引用类型}
\subsection{\texttt{Array} 数组}

数组是最常见的数据结构,ECMAScript 的数组比其他语言有几个不同:
\begin{itemize}
    \item ECMAScript 的数组是对象。
    \item 数组每个槽位可以存储不同的值。
    \item 数组会动态扩容。
\end{itemize}

\subsubsection{数组表现}

创建数组的方式非常多:

\begin{JavaScript}
// new 创建空数组
let colors = new Array();
// new 创建指定大小的数组
let colors = new Array(3);
// new 创建指定元素的数组
let colors = new Array("red", "blue", "yellow");
// 字面量创建数组
let colors = ["red", "blue", "yellow"];
// 字面量创建有空位的数组, 空位被视作 undefined
let colors = ["red",,,,,"blue"];
\end{JavaScript}

此外,ES6 还提供了两个创建数组的静态方法: \texttt{from()} 与 \texttt{of()}。\texttt{from()}用于将类数组结构转换为数组实例,而\texttt{of()}用于将一组参数转换为数组实例,请读者自行尝试这两个方法。


\noindent\textbf{\texttt{isArray()}} 

同一个上下文中可以使用 \texttt{instanceof} 判断数组,但是不同上下文会有不同版本的 Array 构造函数,导致 \texttt{instanceof} 判断失败,这时候可以使用静态方法 \texttt{isArray()}

\begin{JavaScript}
if (Array.isArray(value)) {
    ......
}
\end{JavaScript}

\subsubsection{数组作为映射,栈,队列}

ECMAScript 的 Array 有如下方法,不得不怀疑 Array 到底是如何实现的:
\begin{JavaScript}
const a = ["foo", "bar", "baz", "qux"]; // 因为这些方法都返回迭代器,所以可以将它们的内容
// 通过Array.from()直接转换为数组实例
const aKeys = Array.from(a.keys());         // [0, 1, 2, 3]
const aValues = Array.from(a.values());     // ["foo", "bar", "baz", "qux"]
const aEntries = Array.from(a.entries());   // [[0, "foo"], [1, "bar"], [2, "baz"], [3, "qux"]] 
\end{JavaScript}

因此,可以这样写:

\begin{JavaScript}
const a = ["foo", "bar", "baz", "qux"]; 
for (const [idx, element]of a.entries()) {
    alert(idx);
    alert(element);
}
\end{JavaScript}

此外,数组还可以像栈,队列一样操作,ECMAScript 提供了对应的方法。

\begin{JavaScript}
// 栈操作
let colors = new Array();   // 创建一个数组
let count = colors.push("red", "green");  // 压入两项
let item = colors.pop();       // 弹出一项
// 队列操作
let item = colors.shift();  // 取得最后一项
\end{JavaScript}

\subsubsection{谓词,迭代与归并}

谓词(predicate)可以看作对数据的筛选,数组中的 \texttt{find()} 与 \texttt{findIndex()} 都可以使用谓词筛选数据。

\begin{JavaScript}
const people = [
    {
        name: "Matt",     
        age: 27   
    },   
    {     
        name: "Nicholas",     
        age: 29   
    } 
]; 
alert(people.find((element, index, array) => element.age < 28)); 
// {name: "Matt", age: 27} 
alert(people.findIndex((element, index, array) => element.age < 28)); 
// 0
\end{JavaScript}

ECMAScript 为数组定义了 5 个迭代方式:
\begin{itemize}
    \item \texttt{every()}: 对数组每一项都运行传入的函数,如果每一项函数都返回 \texttt{true},这个方法返回 \texttt{true}。
    \item \texttt{some()}: 对数组每一项都运行传入的函数,如果有一项函数返回\texttt{true},则这个方法返回\texttt{true}。这些方法都不改变调用它们的数组。
    \item \texttt{filter()}: 对数组每一项都运行传入的函数,函数返回 \texttt{true} 的项会组成数组之后返回。
    \item \texttt{forEach()}: 对数组每一项都运行传入的函数,没有返回值。
    \item \texttt{map()}: 对数组每一项都运行传入的函数,返回由每次函数调用的结果构成的数组。
\end{itemize}

ECMAScript为数组提供了两个归并方法:\texttt{reduce()}和\texttt{reduceRight()}。这两个方法都会迭代数组的所有项,并在此基础上构建一个最终返回值。

\subsubsection{\texttt{ArrayBuffer}}

定型数组(typed array)更像我们常规认知中的数组,是申请内存后不可变的,引入它的目的是提高性能。如果熟悉其他强类型语言(如Java),一定会对\texttt{Array}感到不可思议,它能同时作为列表,队列,栈,映射使用,可以存储任意类型;但方便总是有代价的:性能不高。

\texttt{ArrayBuffer} 是所有定型数组及视图引用的基本单位。\texttt{ArrayBuffer()}是可用于在内存中分配特定数量的字节空间。

\begin{JavaScript}
const buf = new ArrayBuffer(16);  // 在内存中分配16字节
alert(buf.byteLength);            // 16 
\end{JavaScript}

\texttt{ArrayBuffer}一经创建就不能再调整大小。不过,可以使用\texttt{slice()}复制其全部或部分到一个新实例中:

\begin{JavaScript}
const buf1 = new ArrayBuffer(16); 
const buf2 = buf1.slice(4, 12); 
alert(buf2.byteLength);  // 8
\end{JavaScript}

\texttt{ArrayBuffer} 在申请内存时有几个注意点: 
\begin{itemize}
    \item 分配失败时会抛出错误。
    \item 分配内存大小有限制 \texttt{Number.MAX\_SAFE\_INTEGER$2^{53}-1$}字节。
    \item 会将所有二进制位初始化为0。
    \item 自动 GC。
\end{itemize}

不能仅通过对\texttt{ArrayBuffer}的引用就读取或写入其内容。要读取或写入\texttt{ArrayBuffer},就必须通过视图。视图有不同的类型,但引用的都是\texttt{ArrayBuffer}中存储的二进制数据。

\subsection{\texttt{Map} 映射}

\subsubsection{\texttt{Map}}

在 ES6 以前,JavaScript 实现 "键/值" 式存储可以使用 \texttt{Object} 来方便高效地完成。这样做没什么问题,但 ES6 以后更推荐使用 \texttt{Map} 存储键值对。

Map 的构造方法有很多,比较常用的可以直接通过二维数组构造:

\begin{JavaScript}
const m = new Map([   
    ["key1", "val1"],    
    ["key2", "val2"],   
    ["key3", "val3"] 
]); 
\end{JavaScript}

JavaScript 的 \texttt{Map} 有一点与众不同:它是有序的。而 \texttt{Object} 作为 \texttt{Map} 使用时却是无序的,这两者之前有如下差异:
\begin{itemize}
    \item \texttt{Map} 有序,\texttt{Object} 无序。
    \item \texttt{Map} 在同等内存大小下可以比 \texttt{Object} 多存储 50\% 的数据。
    \item 插入性能: \texttt{Map} 比 \texttt{Object} 好一点。
    \item 查找性能: \texttt{Object} 比 \texttt{Map} 好一点。
    \item 删除性能: \texttt{Map} 比 \texttt{Object} 好一点。
\end{itemize}

\subsubsection{\texttt{WeakMap}}

\texttt{WeakMap} 时 \texttt{Map} 的子集。这里的 \texttt{weak} 指的是 GC 对待 ``弱映射'' 中键的方式。

弱映射中的键只能是\texttt{Object}或者继承自\texttt{Object}的类型,尝试使用非对象设置键会抛出\texttt{TypeError}。值的类型没有限制。

\texttt{WeakMap}中“weak”表示弱映射的键是“弱弱地拿着”的。意思就是,这些键不属于正式的引用,不会阻止垃圾回收。但要注意的是,弱映射中值的引用可不是“弱弱地拿着”的。只要键存在,键/值对就会存在于映射中,并被当作对值的引用,因此就不会被当作垃圾回收。

\begin{JavaScript}
const wm = new WeakMap(); 
wm.set({}, "val"); 
\end{JavaScript}

上述代码执行完后,由于空对象不存在引用,所以它会被 GC,同时键被 GC 后,值也失去了引用,也会被 GC。

因为 \texttt{WeakMap} 中的键/值对任何时候都可能被销毁,所以没必要提供迭代其键/值对的能力。因为\texttt{WeakMap}实例不会妨碍垃圾回收,所以非常适合保存关联元数据。

\subsection{\texttt{Set} 集合}

\texttt{Set} 也是 ES6 新增的类型,\texttt{Set}在很多方面都像是加强的\texttt{Map},这是因为它们的大多数API和行为都是共有的。

同样,可以使用数组初始化 \texttt{Set}。

\begin{JavaScript}
const s1 = new Set(["val1", "val2", "val3"]);
\end{JavaScript}

\texttt{Set} 的很多特性和 \texttt{Map} 一致,可以存储任意类型,维护插入顺序,用于对应的弱集合 \texttt{WeakSet}。

\newpage