\section{简单数据类型}

ECMAScript 有 6 中简单数据类型(也称原始类型): \texttt{Undefined,Null,Boolean,Number,String} 和 \texttt{Symbol}(ES6 新增)。还有一种复杂的数据类 \texttt{Object}。 \texttt{Object} 是一种无序名值对的集合。

此外,ECMAScript 不能定义自己的数据类型,所有值都可以用上述 7 中数据类型之一来表示。ECMAScript 的数据类型十分灵活,一种数据类型可以当作多种数据类型来使用。

\subsection{\texttt{typeof} 操作符}

对一个值使用 \texttt{typeof} 操作符会返回下列字符串之一:
\begin{itemize}
    \item \texttt{"undefined"}: 值未定义;
    \item \texttt{"boolean"}: 布尔值;
    \item \texttt{"string"}: 字符串;
    \item \texttt{"number"}: 数值;
    \item \texttt{"object"}: 对象(不是函数)或 \texttt{null} ;
    \item \texttt{"function"}: 函数;
    \item \texttt{"symbol"}: 符号;
\end{itemize}

\texttt{typeof} 在某些情况下返回的结果可能会让人费解。比如,调用 \texttt{typeof(null)} 返回的是 \texttt{"object"}。这是因为特殊值 \texttt{null} 被认为是一个对空对象的引用。

\fbox{
    \parbox{0.87\textwidth}{
        \begin{notice}
            严格来讲,函数在 ECMAScript 中被认为是对象,并不代表一种数据类型。可是,函数也有自己特殊的属性。为此,就有必要通过 \texttt{typeof} 操作符来区分函数和其他对象。
        \end{notice}
    }
}

\subsection{\texttt{Undefined} 类型}

\texttt{Undefined} 类型只有一个值,就是特殊值 \texttt{undefined}。当使用 \texttt{var} 或 \texttt{let} 声明了变量但没有初始化时,就相当于给变量赋予了 \texttt{undefined} 值。

\begin{JavaScript}
let message;
console.log(message == undefined);  // true
\end{JavaScript}

我们也可以指定变量值为 \texttt{undefined},与未初始化效果一样。

\begin{JavaScript}
let message = undefined;
console.log(message == undefined);  // true
\end{JavaScript}

\fbox{
    \parbox{0.87\textwidth}{
        \begin{notice}
            虽然可以显式地将 \texttt{undefined} 赋给某个值。但是不要这样做,字面量 \texttt{undefined} 主要是用于比较。而且在 ES3 之前没有  \texttt{undefined}。增加这个特殊值主要是用于和 \texttt{null} 区别。
        \end{notice}
    }
}

与 \texttt{undefined} 字面意义不同的是,\texttt{null} 并不表示未声明变量,而表示没有引用对象。

\begin{JavaScript}
let message;

console.log(message);   // "undefined"
console.log(name);      // 报错
\end{JavaScript}

对于未声明的变量,只能执行一个有用的操作,就是对它调用 \texttt{typeof}。需要注意的是对未声明的变量和未初始化的变量调用 \texttt{typeof} 运算都会返回 \texttt{"undefined"}。

\begin{JavaScript}
let message 

console.log(typeof message);   // "undefined"
console.log(typeof name);      // "undefined"
\end{JavaScript}

\fbox{
    \parbox{0.87\textwidth}{
        \begin{advise}
            都返回 \texttt{"undefined"} 往往会让人摸不着头脑,到底是那种情况。因此建议在声明变量的同时一定要进行初始化。这样,当返回 \texttt{"undefined"} 时,就会知道那是因为给定的变量尚未声明。
        \end{advise}
    }
}

\subsection{\texttt{Null} 类型}

\texttt{Null} 类型同样只有一个值,即 \texttt{null}。逻辑上讲, \texttt{null} 值表示一个空对象指针,这也是给 \texttt{typeof} 传一个 \texttt{null} 会返回 \texttt{"object"} 的原因。

\begin{JavaScript}
let car = null;
console.log(typeof car)     // "object"
\end{JavaScript}

在定义将来要保存对象值的变量时,建议使用 \texttt{null} 来初始化,不要使用其他值。这样,只要检查这个变量是不是 \texttt{null} 就可以直到这个变量是否在后来被重新赋予了一个对象的引用。

\texttt{undefined} 是由 \texttt{null} 派生而来的,因此 ECMA 将它们定义为表面上相等。

\begin{JavaScript}
console.log(undefined == null);     // true
\end{JavaScript}

有别于 \texttt{undefined} , 如果要保存未来的对象,可以用 \texttt{null} 来进行填充。

此外, \texttt{undefined} 和 \texttt{null} 都是假值\footnote{在判定语句中等价于 \texttt{false}。 }。

\subsection{\texttt{Boolean} 类型}

\texttt{Boolean} 类型有两个字面值: \texttt{true} 和 \texttt{false}。 不同于数值, \texttt{true} 不等于1, \texttt{false} 不等于0。

虽然布尔值只有两个,但是其他所有 ECMAScript 类型的值都有对应的布尔值的等价形式。要将一个其他类型的值转换为布尔值,可以调用特定的 \texttt{Boolean()} 转换函数。

\begin{table}[H]
    \centering
    \small
    \caption{基本数据类型与布尔型之间的转换}
    \label{table:基本数据类型与布尔型之间的转换}
    \setlength{\tabcolsep}{18mm}
    \begin{tabular}{c|cc}
        \toprule
        \textbf{数据类型} & \textbf{转换为 \texttt{true}} & \textbf{转换为 \texttt{false}} \\
        \midrule
        Boolean & true & false\\
        String & 非空串 & 空串 \\
        Number & 非零数值 & 0,NAN \\
        Object & 任意对象 & null \\
        Undefined & N/A & undefined \\
        \bottomrule
    \end{tabular}
\end{table}

\subsection{\texttt{Number} 类型}

\texttt{Number} 类型使用 IEEE 754 格式表示整数和浮点数(双精度浮点数)。

\noindent\textbf{整型}

其中整数常用的有有三种写法:十进制,八进制\footnote{严格模式下无效。},十六进制。其中八进制和十六进制表示法在数字前分别加 0 和 0x:

\begin{JavaScript}
let intNum = 55;    // 十进制
let octNum = 070;   // 八进制 56
let hexNum = 0xA;   // 十六进制 10
\end{JavaScript}

如果字面量中包含的数字超出了应有的范围,会忽略前缀的零。

\begin{JavaScript}
let octNum = 079;   // 无效的八进制,当成 79 处理
\end{JavaScript}

\fbox{
    \parbox{0.87\textwidth}{
        \begin{notice}
            JavaScript 中的正零和负零是等同的。
        \end{notice}
    }
}

\noindent\textbf{浮点型}

要定义浮点型,数值中必须包含小数点,且小数点后至少有一个数字。虽然小数点前的数字不是必须的,但还是建议加上。

\begin{JavaScript}
let floatNum1 = 1.1;
let floatNum2 = 0.1;
let floatNum2 = .1;     // 有效,但不建议这样用
\end{JavaScript}

出于优化考虑, ECMAScript 会想方设法把值转换为整数。如果值本省是整数,那么它会被自动转换为整型值。

\begin{JavaScript}
let floatNum1 = 1.;     // 小数点后没有数字,当作整型1
let floatNum2 = 10.0;   // 小数点后是零,当作整型10
\end{JavaScript}

此外,ECMAScript 也可以使用科学计数法,在一定情况下会将小数转换为科学计数法,其精度最高可达小数点后 17 位。

\begin{JavaScript}
let floatNum1 = 3e-7;   // 0.0000007
\end{JavaScript}

此外,和大多数高级语言一样,为了避免精度误差,请不要对小数运用比较运算。

\noindent\textbf{值的范围}

ECMAScript 可表示的最大最小值分别存储在了以下两个变量中(大多数浏览器中值与下列相同):

\begin{itemize}
    \item \texttt{Number.MIN\_VALUE}: 5e-324
    \item \texttt{Number.MAX\_VALUE}: 1.797e+308
\end{itemize}

如果计算的值超出了这个范围,会自动被转换为特殊的 \texttt{Infinity}(无穷值),负值则为 \texttt{-Infinity}。如果计算返回无穷,则该值将不能再进一步用于任何计算。可以用 \texttt{isFinite()} 函数判断是否为无穷值。

\begin{JavaScript}
let result = Number.MAX_VALUE + Number.MAX_VALUE;
console.log(isFinite(result));      // false
\end{JavaScript}

此外,可以使用 \texttt{Number.NEGATIVE\_INFINITY} 和 \texttt{Number.POSITIVE\_INFINITY} 获取正负 \texttt{Infinity}。

\noindent\textbf{\texttt{NaN}}

\texttt{NaN} 用于表示不是数值(Not a Number),一般用于表示本要返回数值得操作失败了。

造成 \texttt{NaN} 得原因有很多,比如分子为零得运算。此外任何涉及 \texttt{NaN} 得操作始终返回 \texttt{NaN}。其次, \texttt{NaN} 不等于包括 \texttt{NaN} 在内得任何值。

\begin{JavaScript}
console.log(NaN == NaN);    // false
\end{JavaScript}

为了判断 \texttt{NaN},JavaScript 提供了 \texttt{isNaN()} 函数。这个函数非常灵活,任何可以转化为数值类型得值都会返回 \texttt{true}:

\begin{JavaScript}
console.log(isNaN(NaN));        // true
console.log(isNaN(10));         // false,10是数值
console.log(isNaN("10"));       // false,"10"可以转换为数值
console.log(isNaN("blue"));     // true,不可以转换为数值
console.log(isNaN(true));       // false,可以转换为数值1
\end{JavaScript}

\noindent\textbf{数值转换}

有3个函数可以将非数值转换为数值: \texttt{Number(),parseInt(),parseFloat()}。 其中 \texttt{Number()} 是转型函数,可用于任何数据类型。后两个函数主要用于将字符串转换为数值。

\texttt{Number()} 函数基于如下规则执行转换:

\begin{itemize}
    \item 布尔值: \texttt{true} 转为1, \texttt{false} 转为 0。
    \item 数值: 直接返回。
    \item \texttt{null}: 返回0。
    \item \texttt{undefined}: 返回 \texttt{NaN}。
    \item 字符串,有如下规则:
    \begin{itemize}
        \item 整数字符串,包含字符前有加减号得情况,转换为一个对应的十进制数值。忽略最前面的零。
        \item 浮点格式:转换为浮点数。
        \item 十六进制:例如 \texttt{"0xf"},转换为对应的十进制数值。
        \item 空字符串:转为 0。
        \item 其他情况:\texttt{NaN}。
    \end{itemize}
    \item 对象: 调用 \texttt{valueOf()} 方法,并按照上述规则转换返回的值。如果转换结果为 \texttt{NaN},则调用 \texttt{toString()} 方法,再按照转换字符串的规则转换。
\end{itemize}

考虑到 \texttt{Number()} 函数转换字符串时相对复杂且有点反常规,通常需要在得到整数时可以优先使用 \texttt{parseInt()} 函数。

\texttt{parseInt()} 函数更专注于字符串是否包含数值模式。字符串最前面的空格会被忽略,从第一个非空格字符串开始转换。如果第一个字符串不是数值字符,加号,减号。 \texttt{parseInt()} 立即返回 \texttt{NaN}。这意味着空字符串也返回 \texttt{NaN}。如果第一个字符是数值字符,加号,减号;则继续依次检测每个字符,直到字符串末尾,或碰到非数值字符。

\begin{JavaScript}
parseInt("1234blue");   // 1234
parseInt("22.5");       // 22
\end{JavaScript}

假设字符串第一个字符是数值字符,\texttt {parseInt()} 函数也能识别不同的整数格式(十六进制,八进制,十进制)。如果字符以 ``0'' 开头,且紧跟着数值字符,在非严格模式下会被某些实现解释为八进制整数。

\begin{JavaScript}
parseInt("0xA");        // 10
parseInt("070");        // 56(视浏览器而定)
\end{JavaScript}

不同的数值格式很容易混淆,因此 \texttt{parseInt()} 函数也接收第二个参数,用于指定进制数,例如:

\begin{JavaScript}
parseInt("0xAF",16);    // 175
parseInt("AF",16);      // 175
parseInt("AF");         // NaN
\end{JavaScript}

通过第二个参数,可以极大扩展转换后获得的结果类型:

\begin{JavaScript}
parseInt("10",2);       // 二进制
parseInt("10",8);       // 八进制
parseInt("10",10);      // 十进制
parseInt("10",16);      // 十六进制
\end{JavaScript}

因为不传底数参数相当于让 \texttt{parseInt()} 函数自己判断如何解释,所以应始终传第二个参数,这个参数多数情况下是 10。

\fbox{
    \parbox{0.87\textwidth}{
        \begin{advise}
            \texttt{parseInt()} 函数与 \texttt{Number()} 函数的主要区别在于:1.对空字符串的处理。 2.字符串后面非数值字符的处理。3.八进制处理(视浏览器决定)。
        \end{advise}
    }
}

\subsection{\texttt{String} 类型}

\texttt{String} 数据类型表示零个或多个 16 为 Unicode 字符序号。字符串可以用双引号(\texttt{"}),单引号(\texttt{'}),反引号(\texttt{`}) 标志。但引号必须同队出现,不同类型引号不可以随意匹配。

注意,ECMAScript 中的字符串是不可变的。

\noindent\textbf{转换为字符串}

JavaScript 提供了两种方式转换字符串。首先是几乎所有值都有的 \texttt{toString()} 方法。这个方法唯一的用途就是返回当前值的字符串等价物。

\texttt{toString()} 方法可用于数值,布尔值,对象和字符串值(字符串值返回自身的一个副本)。 \texttt{null} 和 \texttt{undefined} 值没有 \texttt{toString()} 方法。

\begin{JavaScript}
let age = 11;
let ageString = age.toString();         // 字符串 "11"
let found = true;
let foundString = found.toString();     // 字符串 "true"
\end{JavaScript}

多数情况下, \texttt{toString()} 不接收任何参数。不过,在对数值调用这个方法时,\texttt{toString()} 可以接收一个底数参数,即以什么底数来输出数值的字符串表示。默认为十进制(即参数为10)。

\begin{JavaScript}
let num = 10;
console.log(num.toString());    // "10"
console.log(num.toString(2));   // "1010"
console.log(num.toString(8));   // "12"
console.log(num.toString(16));  // "a"
\end{JavaScript}

如果不确定一个值是不是 \texttt{null} 或 \texttt{undefined} ,可以使用 \texttt{String()} 转型函数,它始终会返回表示相应类型值的字符串,其处理规则如下:

\begin{itemize}
    \item 如果只有 \texttt{toString()} 方法,则调用该方法并返回结果。
    \item 如果值是 \texttt{null},则返回 \texttt{null}。
    \item 如果值是 \texttt{undefined},则返回 \texttt{undefined}。
\end{itemize}

\noindent\textbf{模板字面量}

ES6 新增了使用模板字面量定义字符串的能力。与使用单引号或双引号不同,模板字面量保留换行字符,可以跨行定义字符串。

\begin{JavaScript}
let multiLine1 = "first lin\nsecond line";
let multiLine1 = "first line
second line";

console.log(multiline1);
// first line
// second line

console.log(multiline2);
// first line
// second line
\end{JavaScript}

\noindent\textbf{字符串插值}

模板字面量最常用的一个特性是支持字符串插值,也就是可以在一个连续定义中插入一个或多个值。技术上讲,模板字面量不是字符串,而是一种特殊的 JavaScript 语法表达式,只不过求值后得到的是字符串。模板字面量在定义时立即求值并转换为字符串实例,任何插入的变量也会从它们最接近的作用域中取值。

字符串插值通常在 \texttt{\$\{\}} 中使用一个 JavaScript 表达式实现:

\begin{JavaScript}
let value = 5;
let exponent = 'second';
// 之前的写法
let interpolatedString = value + ' to the ' + exponent + ' power is ' + (value * value);
// 现在的写法
let interpolatedString = '${value} to the ${exponent} power is ${value*value}'; // $ 写法
\end{JavaScript}

所有插入的值都会使用 \texttt{toString()} 强制转型为字符串,而且任何 JavaScript 表达式都可以用于插值。嵌套的模板字符串无需转义:

\begin{JavaScript}
function capitalize(word) {
    return '${word[0].toUpperCase()} ${word.slice[1]}'
}
console.log('${capitalize('hello')}, ${capitalize('world')}!')  // Hello, World!
\end{JavaScript}

\noindent\textbf{模板字面量标签函数}

模板字面量也支持定义标签函数,通过标签函数可以自定义插值行为。标签函数会接收被插值记号分隔后的模板和对每个表达式求值的结果。

标签函数本身是一个常规函数,通过前缀到模板字面量来应用自定义行为。标签函数接收到的参数依次是原始字符串数组和对每个表达式求值的结果。这个函数的返回值是对模板字面量求值得到的字符串。

\begin{JavaScript}
let a = 6;
let b = 9;

function simpleTag(strings, aValExpression, bValExpression, sumExpression) {
    console.log(strings);
    console.log(aValExpression);
    console.log(bValExpression);
    console.log(sumExpression);
    return 'foobar';
}

let untaggedResult = '${a} + ${b} == ${a+b}';
let taggedResult = simpleTag'${a} + ${b} == ${a+b}';
// ["","+","=",""]
// 6
// 9
// 15

console.log(untaggedResult);    // "6 + 9 = 15"
console.log(taggedResult);      // "foobar"
\end{JavaScript}

因为表达式参数的数量是可变的,所以通常应该使用剩余操作符将它们收集到一组数组中:

\begin{JavaScript}
let a = 6;
let b = 9;
    
function simpleTag(strings, ...expressions) {
    console.log(strings);
    for (const expression of expressions) {
        console.log(expression);
    }
    return 'foobar';
}
    
let taggedResult = simpleTag'${a} + ${b} == ${a+b}';
// ["","+","=",""]
// 6
// 9
// 15
    
console.log(taggedResult);      // "foobar"
\end{JavaScript}

对于有 n 个插值的模板字面量,传给标签函数的表达式参数的个数始终是 n。而传给标签函数的第一个参数所包含的字符串个数则始终是 n+1。因此,如果你想把这些字符串和对表达式求值的结果拼接起来作为默认返回的字符串,可以这样做:

\begin{JavaScript}
    let a = 6;
    let b = 9;
        
    function zipTag(strings, ...expressions) {
        return strings[0] + expressions.map((e,i) => '${e}${strings[i+1]}').join('');
    }
        
let taggedResult = zipTag'${a} + ${b} == ${a+b}';
        
console.log(taggedResult);      // "6 + 9 = 15"
\end{JavaScript}

\noindent\textbf{原始字符串}

使用模板字面量也可以直接获取原始的模板字面量内容(如换行符或 Unicode字符),而不是被转换后的字符表示。为此,可以使用默认的 \texttt{String.raw} 标签函数:

\begin{JavaScript}
console.log('\u00A9');              // ©
console.log(String.raw'\u00A9');    // \u00A9

console.log(String.raw'first line
second line')     
// first line
// second line 对实际的换行符不起作用
\end{JavaScript}

\subsection{\texttt{Symbol} 符号类型}

\texttt{Symbol} 是 ES6 新增的数据类型。符号是原始值,是唯一的,不可变的。符号的用途是确保对象属性使用唯一标识符,不会发生属性冲突的危险。

\noindent\textbf{符号的基本用法}

符号要使用 \texttt{Symbol()} 函数初始化,也可以传入一个字符串参数作为符号的描述,将来可以通过这个字符串来调试代码。但是,这个字符串参数与符号定义或标识完全无关:

\begin{JavaScript}
let firstSymbol = Symbol();
let secondSymbol = Symbol();

let thirdSymbol = Symbol("foo");
let forthSymbol = Symbol("foo");

console.log(firstSymbol == secondSymbol);   // false
console.log(thirdSymbol == forthSymbol);    // false

console.log(firstSymbol);    // Symbol()
console.log(thirdSymbol);    // Symbol(foo)
\end{JavaScript}

\texttt{Symbol} 函数不能与 \texttt{new} 关键字一起作为构造函数使用。这样做是为了避免创建符号包装对象,如果一定要做可以使用 \texttt{Object()} 函数:

\begin{JavaScript}
let mySymbol = Symbol;
let myWrappedSymbol = Object(mySymbol);
console.log(typeof myWrappedSymbol)
\end{JavaScript}

\noindent\textbf{使用全局符号注册表}

如果运行时的不同部分需要共享和重用符号实例,那么可以用一个字符串作为键,在全局符号注册表中创建并重用符号。为此,需要使用 \texttt{Symbol.for()} 方法:

\begin{JavaScript}
let fooGlobalSymbol = Symbol.for('foo');
console.log(typeof fooGlobalSymbol);    // symbol
\end{JavaScript}

\texttt{Symbol.for()} 对每个字符串健都执行等幂操作。第一次使用某个字符串调用时,它会检查全局运行时注册表,发现不存在对应的符号,于是就会生成一个新符号实例并添加到注册表中。后续使用相同字符串的调用同样会检查注册表,大仙存在与该字符串对应的符号,然后就会返回该符号实例。

\begin{JavaScript}
let fooGlobalSymbol = Symbol.for('foo');        // 创建新符号
let otherFooGlobalSymbol = Symbol.for('foo');   // 重用已有符号

console.log(fooGlobalSymbol === otherFooGlobalSymbol);  // true
\end{JavaScript}

在全局注册表中定义的符号和使用 \texttt{Symbol()} 定义的符号并不等同:

\begin{JavaScript}
let localSymbol = Symbol('foo');
let globalSymbol = Symbol.for('foo');

console.log(localSymbol === globalSymbol);  // false
\end{JavaScript}

全局注册表中的符号必须使用字符串键来创建,因此作为参数传给 \texttt{Symbol.for()} 的任何值都会被转换为字符串。此外,注册表中使用的键同时也会被用作符号描述。

\begin{JavaScript}
let emptyGlobalSymbol = Symbol.for();
console.log(emptyGlobalSymbol);     // Symbol(undefined)
\end{JavaScript}

还可以使用 \texttt{Symbol.keyFor()} 来查询全局注册表,这个方法接收符号,返回该全局符号对应的字符串键。如果查询的不是全局符号,则返回 \texttt{undefined} 。

\begin{JavaScript}
let s = Symbol.for('foo');
console.log(Symbol.keyFor(s));      // foo

let s2 = Symbol('bar');
console.log(Symbol.keyFor(s2));     // undefined

Symbol.keyFor(123);                 // TypeError: 123 is not a symbol
\end{JavaScript}

\noindent\textbf{使用符号作为属性}

凡是可以使用字符串或数值作为属性的地方,都可以使用符号。这就包括了对象字面量属性 \texttt{Object.defineProperty()/Object.defineProperties()} 定义的属性。对象字面量只能在计算性语法中使用符号作为属性。

\fbox{
    \parbox{0.87\textwidth}{
        \begin{notice}
            原文 P47-55 页还有更多关于 Symbol 的介绍,但本人觉得,Symbol 作为新的数据类型并不值得这么大篇幅的详细介绍,尤其是在基础篇中,故省略,读者可自行查看原文,但不查看也不会影响后续阅读。
        \end{notice}
    }
}

\newpage