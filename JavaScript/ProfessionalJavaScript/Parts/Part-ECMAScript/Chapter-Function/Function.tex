\section{函数}

\subsection{定义函数的多种方式}

ECMAScript 中,每个函数都是 \texttt{Function} 类型的实例\footnote{与 Python 一样。}。而 \texttt{Function} 也有属性和方法,跟其他引用类型一样。函数名就是指向函数对象的指针,而且不一定与函数本身密切绑定。函数通常以函数声明的方式定义:

\begin{JavaScript}
function sum(num1, num2) {
    return num1 + num2;
}
\end{JavaScript}

另一种定义函数的语法是函数表达式。函数表达式与函数声明几乎是等价的,但是函数表达式不会提升,也即必须先声明后调用:

\begin{JavaScript}
let sum = function(num1, num2) {
    return num1 + num2;
};
\end{JavaScript}

函数表达式可以通过这种方式立刻被调用,块级作用域的语句会被立即执行:

\begin{JavaScript}
(function() {   
    // 块级作用域
})(); 
\end{JavaScript}

注意函数表达式最后是要带分号的,而函数声明则没有必要。

还有一种定义函数的方式与函数表达式很像,叫``箭头函数''(类似 lambda 表达式):

\begin{JavaScript}
let sum = (num1, num2) => {
    return num1 + num2;
};
\end{JavaScript}

最后一种定义函数的方式是使用 \texttt{Function} 构造函数。这个构造函数收集任意多个字符串参数,最后一个参数为函数体,之前的参数都是新函数的参数:

\begin{JavaScript}
let sum = new Function("num1", "num2", "return num1 + num2");
\end{JavaScript}

这种方式并不推荐,会影响内部处理的性能。不过要记作 ECMAScript 中的函数是对象,函数名是指针。

箭头函数是 ES6 新增的语法。箭头函数和函数表达式十分相像,任何可以使用函数表达式的地方,都可以使用箭头函数。

箭头函数的语法非常适合嵌入函数的场景:

\begin{JavaScript}
let ints = [1,2,3];
console.log(ints.map(function(i) {return i+1;}));   // [2,3,4]
console.log(ints.map((i) => {return i+1;}));   // [2,3,4]
\end{JavaScript}

如果只有一个参数,那可以不用括号。

\begin{JavaScript}
let double = (x) => {return 2*x;};
let triple = x => {return 3*x;};
\end{JavaScript}

如果函数语句十分精简,只需要一行代码,那么大括号也可以省略。省略大括号会隐式返回这些代码的值(不需要额外的 \texttt{return} 语句)。

\begin{JavaScript}
let double = x => 2 * x;
\end{JavaScript}

箭头函数虽然语法简洁,但也有很多场合不适合。箭头函数不能使用 \texttt{arguments , super} 和 \texttt{new.target},也不能用作构造函数。此外,箭头函数也没有 \texttt{prototype} 属性。

\subsection{ECMAScript 的函数}

\subsubsection*{函数名}

因为函数名是指向函数的指针,所以它们跟其他包含对象指针的变量具有相同行为。这意味着一个函数可以拥有多个名称。

ECMAScript6 的所有函数对象都会暴露一个和只读的 \texttt{name} 属性,其中包含关于函数的信息。多数情况下,这个属性保存的就是一个函数标识符,或者说是一个字符串化的变量名。如果它是使用 \texttt{Function} 构造函数创建的,则会标识为 ``anonymous''。

\begin{JavaScript}
function foo() {}
let bar = function() {};
let baz = () => {};

console.log(foo.name);      // foo
console.log(bar.name);      // bar
console.log(baz.name);      // baz
console.log(() => {}.name);      // 空字符串
console.log((new Function()).name);      // anonymous
\end{JavaScript}

如果一个函数是获取函数,设置函数,或者使用 \texttt{bind()} 实例化,那么标识符前面会加上一个前缀\footnote{后文会详细说明}。

\subsubsection*{函数参数}

ECMAScript 的函数去其他大多数语言不同,它既不关心参数的类型也不关心参数的个数,甚至传入参数的个数可以超过定义函数时参数的个数。

之所以会这样,是因为 ECMAScript 函数的参数在内部表现为一个数数组。函数被调用时总会接收一个数组,但函数并不关系这个数组中有什么(甚至什么都没有)。

事实上,在使用 \texttt{function} 关键字定义(非箭头)函数时,可以在函数内部访问 \texttt{arguments} 对象,从中取得传进来的每个参数值。我们可以通过中括号语法访问 \texttt{arguments} 对象中的元素(第一个是 \texttt{arguments[0]})。而要确定传进来多少个参数,可以访问 \texttt{arguments.length} 属性。

\begin{JavaScript}
function sayHi(name, message) {
    console.log("Hello" + name + "," + message);
    }
    
// 等价写法
function sayHi() {
    console.log("Hello" + arguments[0] + "," + arguments[1]);
}
\end{JavaScript}

上面例子表明,ECMAScript 函数的参数只是为了方便才写出来的,并不是必须写出来。与此同时,ECMAScript 中的命名参数不会创建让之后的调用必须匹配的函数签名。这是因为根本不存在验证命名参数的机制。

如前所述,ECMAScript 的函数参数本质上是一个数组,且并不对这个数组进行校验,因此函数无法重载。

ES6 起,可以显示定义默认参数。

\begin{JavaScript}
function makeKing(name = 'Henry') {   
    return name;  
} 
\end{JavaScript}

ECMAScript 支持参数扩展,传入参数时可以通过 ... 将可迭代对象拆分为独立的参数:

\begin{JavaScript}
function getSum() {   
    let sum = 0;   
    for (let i = 0; i < arguments.length; ++i) {     
        sum += arguments[i];   
    }   
    return sum; 
} 
let values = [1,2,3,4];
console.log(getSum(...values)); // 10
\end{JavaScript}

同时,定义函数时 ... 表示多个任意个参数组成的数组:

\begin{JavaScript}
function ignoreFirst(firstValue, ...values) {   
    console.log(values); 
}
ignoreFirst(1,2,3);  // [2, 3]
\end{JavaScript}

\subsection{函数内部}

我们知道,JavaScript 中的函数本质上是 \texttt{Function} 类的实例,因此将 JavaScript 的函数当作对象来看,既可以在参数中使用函数,也可以定义函数闭包(类似于Java内部类),也可以拥有属性,总而言之,\texttt{Object} 能做的,他都能做。

在 ES6 后,函数内部有三个特殊的对象: \texttt{arguments, this, new.target}。

\begin{itemize}
    \item \texttt{arguments}: 前面出现过多次,本质上是一个数组对象,包含调用函数时传入的所有参数。这个对象只有以 \texttt{function} 关键字定义函数时有用(箭头函数没用)。\texttt{arguments} 有一个特殊的属性: \texttt{callee},它可以指向 \texttt{arguments} 对象所在的函数,因此可以在函数内部这样调用函数本身: \texttt{arguments.callee}(严格模式下不准这样用)。
    \item \texttt{this}: 在标准函数和箭头函数中不同
    \begin{itemize}
        \item 标准函数: \texttt{this}引用的是把函数当成方法调用的上下文对象,这时候通常称其为\texttt{this}值(在网页的全局上下文中调用函数时,\texttt{this}指向\texttt{windows})
\begin{JavaScript}
window.color = 'red';  
let o = { color: 'blue' }; 
function sayColor() {   
    console.log(this.color); 
} 
sayColor();    // 'red' 
o.sayColor = sayColor; 
o.sayColor();  // 'blue'
\end{JavaScript}
        \item 在箭头函数中,\texttt{this} 引用的是定义箭头函数的上下文(更符合多数语言中 \texttt{this} 的作用)。
    \end{itemize}
    \item \texttt{caller}: ES5 新增的函数属性,这个属性引用的是调用当前函数的函数。
    \item \texttt{new.target}: ECMAScript中的函数始终可以作为构造函数实例化一个新对象,也可以作为普通函数被调用。ECMAScript 6新增了检测函数是否使用\texttt{new}关键字调用的\texttt{new.target}属性。如果函数是正常调用的,则\texttt{new.target}的值是\texttt{undefined}; 如果是使用new关键字调用的,则\texttt{new.target}将引用被调用的构造函数。
\begin{JavaScript}
function King() {
    if (!new.target) {
        throw 'King must be instantiated using "new"';
    }
    console.log('King instantiated using "new"');
}
new King(); // King instantiated using "new" 
King();     // Error: King must be instantiated using "new" 
\end{JavaScript}
\end{itemize}

函数有两个特殊属性:
\begin{itemize}
    \item \texttt{length}: 保存函数定义的命名参数的个数。
    \item \texttt{prototype}: 保存引用类型所有实例方法的地方。
\end{itemize}

函数有两个特殊方法:
\begin{itemize}
    \item \texttt{apply()}: 接收两个参数:函数内\texttt{this}的值和一个参数数组。第二个参数可以是\texttt{Array}的实例,也可以是\texttt{arguments}对象。
\begin{JavaScript}
function sum(num1, num2) {
    return num1 + num2;
}
function callSum1(num1, num2) {
    return sum.apply(this, arguments);
    // 传入arguments对象
}
function callSum2(num1, num2) {
    return sum.apply(this, [num1, num2]); // 传入数组
}
console.log(callSum1(10, 10));  // 20 
console.log(callSum2(10, 10));  // 20 
\end{JavaScript}
    \item \texttt{call()}: 和 \texttt{apply()} 方法作用一样,不过第二及以后参数是单个对象,而不是数组。
\end{itemize}




\newpage