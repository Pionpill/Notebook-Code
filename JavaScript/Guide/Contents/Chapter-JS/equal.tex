\section{相等判断}

JavaScript 有两种等于判断,相等(弱等于) == 与全等(强等于) ===。限制一般都用全等,若等于基本被弃用了。

先说一下全等:
\begin{itemize}
  \item 简单类型: 判断值。
  \item 引用类型: 判断栈指针指向的堆地址。
  \item 基本与引用: false。
\end{itemize}

然后是弱等于,弱等于不分简单类型与引用类型,因为若等于会进行类型转换。如果类型相同,则按照全等逻辑判断。类型转换的逻辑如下:
\begin{itemize}
  \item \texttt{null == undefined}: 返回 \texttt{true}。
  \item \texttt{null undefined}: 不会进行转换,仅有上面一种特殊情况。
  \item 一个为数字,一个为字符串: 转换为数字。
  \item 一个为布尔值,一个为非布尔值: 布尔值转换为数字。
  \item 一个为对象,一个为非对象: 将对象转换为原始值。
  \item 一个为 NaN, 则一定返回 \texttt{false}, 包括 \texttt{NaN == NaN}。
\end{itemize}

字符串转为数字比较特殊,如果不能转换,则转换为 NaN,如果是空字符串(包括都是空元素的情况),转换为0。

有一点需要注意,虽然在弱等运算的时候,字符串会转为数值进行判断,但是在进行加法运算的时候,数值会转化为字符串。

如果加法运算时双方任意一方为字符串,或者双方均不能进行加法运算,则转换为字符串拼接。