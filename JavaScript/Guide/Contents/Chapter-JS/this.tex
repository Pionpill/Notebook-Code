\section{\texttt{this}}

\subsection{普通函数的中的 \texttt{this}}

JavaScript 中非箭头函数的 \texttt{this} 有四种绑定规则:
\begin{itemize}
  \item 默认绑定: 即默认绑定到全局作用域。
\begin{JavaScript}
// 浏览器环境
(function() {
  console.log(this === window); // true
})()
\end{JavaScript}
  \item 隐式绑定: 函数作为对象的方法被调用时,\texttt{this} 指向方法运行时所在的当前对象。
\begin{JavaScript}
const obj = {
  a: 1,
  foo: function () {
    console.log(this);  // {a: 1, foo: ƒ}
  }
}
\end{JavaScript}
隐式绑定存在优先级和隐式丢失的问题:
\begin{JavaScript}
const obj1 = {
  a: "a1",
  b: "b",
  obj2: {
    a: "a2",
    foo: function () {
      console.log(this.a);  // a2
      console.log(this.b);  // undefined
    }
  }
}
\end{JavaScript}
这里调用 \texttt{foo()},\texttt{this} 只会绑定到上一级对象上,获取不到更外层的变量。

  \item 显示绑定: 使用 \texttt{call, apply, bind} 改变函数的调用对象。
\begin{JavaScript}
  var name = "window";
  var outer = "outer";
  
  const foo = function () {
    var mark = "mark";
    console.log(this.name);
    console.log(this.outer);
    console.log(this.mark);
  }
  const obj = {
    name: "obj",
    mark: "mark",
  }
  
  foo();
  foo.call(obj);
// window outer, undefined
// obj, undefined, mark
\end{JavaScript}
这里第一次调用时,\texttt{this} 默认绑定到 \texttt{window} 上,第二次调用时,显示绑定到 \texttt{obj} 对象上。
\item \texttt{new} 绑定: 函数通过 \texttt{new} 调用后,会返回一个新对象,并将新对象绑定到函数调用的 \texttt{this} 上。
\end{itemize}

总的来说,普通函数的 \texttt{this} 指向会出现以下几种情况:
\begin{itemize}
  \item 以函数形式调用: 永远指向 \texttt{window}。
  \item 以对象方法形式调用: 指向调用的那个对象。
  \item 以构造函数形式调用: 指向构造函数创建的实例对象。
  \item 以事件绑定函数形式调用: 指向绑定事件的对象。
  \item 显示绑定: 指向绑定的对象。
\end{itemize}

此外,显示绑定有例外,对于基本类型,浏览器会自动转化为引用类型进行绑定,但如果是 \texttt{undefined,null},使用默认绑定规则。

绑定优先级: \texttt{new}/显示绑定 > 隐式绑定 > 默认绑定。

\subsection{箭头函数中的 \texttt{this}}

箭头函数自身没有 \texttt{this}。箭头函数中的 \texttt{this} 拿的是定义函数是外部环境的 \texttt{this}。

\begin{JavaScript}
var obj = {
  data: [],
  getData: function() {
    return (function() {
      var result = ["abc","cba","nba"]
      this.data = result
    })()
  }
}
 
obj.getData()
console.log(obj.data);  // Array(0)
\end{JavaScript}

上面调用方法时,该方法本身的 \texttt{this} 指向 \texttt{obj}。但是方法内 回调的方法在执行时 \texttt{this} 指向的是 \texttt{window}。此时 \texttt{this} 没有 \texttt{data} 属性,因此没有任何影响。

如果要达到给数组赋值的效果,可以这样改进:

\begin{JavaScript}
var obj = {
  data: [],
  getData: function () {
    let self = this;
    return (function() {
      var result = ["abc","cba","nba"]
      self.data = result
    })()
  }
}
 
obj.getData()
console.log(obj.data);  // Array(0)
\end{JavaScript}

这样用的是闭包思想,在函数指向上下文中创建变量,这个变量会被加入到内部函数的作用域链中。

或者用箭头函数:

\begin{JavaScript}
var obj = {
  data: [],
  getData: function () {
    return (() => {
      var result = ["abc","cba","nba"]
      this.data = result
    })()
  }
}
 
obj.getData()
console.log(obj.data);  // Array(0)
\end{JavaScript}

箭头函数中的 \texttt{this} 是外部函数的 \texttt{this},外部函数的 \texttt{this} 被隐式绑定到了 \texttt{obj} 对象上。

\subsection{手写显式绑定}

\subsubsection*{手写 \texttt{call, apply}}

我们知道绑定 \texttt{this} 只有四种方法,如果需要手写显式绑定,自然需要借助其他三种方法中的某一种,默认绑定和\texttt{new}绑定自然不行,三个显式绑定都需要传入一个对象作为参数,因此自然是使用隐式绑定。

我们需要将函数放到对象中。这样就可以通过隐式绑定改变 \texttt{this} 指向。有几点需要注意:
\begin{itemize}
  \item \texttt{call} 是原型上的方法,所以我们也需要写在原型上。
  \item 调用 \texttt{newCall} 时,是使用函数调用的,函数本身也是对象,因此调用时的 \texttt{this} 指向我们要绑定的函数本身。
  \item 传入参数为 \texttt{undefined,null} 时,需要默认绑定到 \texttt{window} 对象上。
\end{itemize}

\begin{JavaScript}
Function.prototype.newApply = function (obj, ...args) {
  const realObj = obj || window;
  realObj.p = this;
  let result = realObj.p(...args);
  delete realObj.p;
  return result;
}
\end{JavaScript}

有了 \texttt{call},\texttt{apply} 思路也类似:

\begin{JavaScript}
Function.prototype.newApply = function (obj, ...args) {
  const realObj = obj || window;
  realObj.p = this;
  let result = realObj.p(...args[0]);
  delete realObj.p;
  return result;
}
\end{JavaScript}

\subsubsection*{手写 \texttt{bind}}

首先最基础的,返回的一个函数:

\begin{JavaScript}
Function.prototype.newBind = function (obj) {
  const self = this;
  return function () {
    return self.apply(obj || window);
  }
}
\end{JavaScript}

其次,我们知道,原生 \texttt{bind} 方法可以接受多个参数作为函数参数,且需要柯里化:

\begin{JavaScript}
Function.prototype.newBind = function (obj, ...params1) {
  const self = this;
  return function (...params2) {
    return self.apply(obj || window, [...params1, ...params2]);
  }
}
\end{JavaScript}

另外,原生 \texttt{bind} 方法可以 \texttt{new} 对象,因此需要绑定原型:

\begin{JavaScript}
Function.prototype.customBind = (obj, ...args1) => {
  const realObj = obj || window;
  const innerFunc = (...args2) => {
    this.apply(realObj, [...args1, ...args2]);
  };
  innerFunc.prototype = this.prototype;
  innerFunc.prototype.constructor = innerFunc;
  return innerFunc;
};
\end{JavaScript}