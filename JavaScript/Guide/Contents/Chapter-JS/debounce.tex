\section{防抖与节流}

\subsection{防抖}

防抖是指事件触发时,相应的函数不会立即执行,而是会被推迟执行。多数情况下,触发的多个任务仅执行一次。

防抖最常见的应用场景是搜索框输入内容,只在最后等待一会后做响应。

\begin{JavaScript}
const debounce = (func, ms) => {
  let process = null;
  return function() {
    clearTimeout(process);
    process = setTimeout(func, ms);
  }
}
\end{JavaScript}

这是最简单的防抖写法,还有两个问题没有解决:
\begin{itemize}
  \item \texttt{this} 应该指向事件对应的元素本身。
  \item 函数应该具备传参功能。
\end{itemize}

在上述代码中,我们会直接返回函数本身,这个函数的 \texttt{this} 会动态绑定指向 \texttt{元素},但是\texttt{setTimeout} 回调函数不会(这里回调的 \texttt{func} 不能是箭头函数,否则无法动态绑定到元素上)。

\begin{JavaScript}
const debounce = (func, ms) => {
  let process = null;
  return function (...args) {
    clearTimeout(process);
    process = setTimeout(() => {
      func.call(this, ...args);
    }, ms);
  }
}
\end{JavaScript}

\subsection{节流}

节流是指当事件触发时,会执行这个事件的响应函数,但如果事件被频繁触发,则间隔一段频率执行函数。

节流和防抖实现差不多:
\begin{JavaScript}
const throttle = (func, delay) => {
  let process = null;
  return function (...args) {
    if (process)
    return;
    process = setTimeout(() => {
      func.call(this, ...args);
      process = null;
    }, delay);
  }
}
\end{JavaScript}

也可以不用 \texttt{setTimeout} 直接判断时间间隔。

\begin{JavaScript}
const throttle = (func, delay) => {
  let pre = new Date();
  return function (...args) {
    let now = new Date();
    if (now - pre > delay) {
      func.call(this, args);
      pre = now;
    }
  }
}
\end{JavaScript}