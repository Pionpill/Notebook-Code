\section{事件}

\subsection{事件处理程序}

DOM 事件流分为三个阶段: 事件捕获,事件执行,事件冒泡。也即先深度搜索,再找到节点,再向外冒泡。

DOM 事件何时触发分为捕获阶段触发与冒泡阶段触发,下文简称事件捕获,事件冒泡。

事件处理程序分为几种:
\begin{itemize}
  \item HTML 事件处理: 事件冒泡。下面代码结果为 3 2 1。
\begin{HTML}
<div onclick="console.log('1')">
  <button onclick="console.log('2')">
    <span onclick="console.log('3')">
      text
    </span>
  </button>
</div>
\end{HTML}
  \item DOM0 级: 实现了 HTML 与 JS 分离,事件冒泡,结果为 3 2 1。
\begin{HTML}
const div = document.getElementsByTagName("div")[0];
const button = document.getElementsByTagName("button")[0];
const span = document.getElementsByTagName("span")[0];

div.onclick = () => console.log("1");
button.onclick = () => console.log("2");
span.onclick = () => console.log("3");
\end{HTML}
  \item DOM2 级: 默认事件冒泡,但可以通过 \texttt{addEventListener} 改为事件捕获,下面代码结果为 DIV, SPAN, BUTTON。
  
\begin{JavaScript}
const div = document.getElementsByTagName("div")[0];
const button = document.getElementsByTagName("button")[0];
const span = document.getElementsByTagName("span")[0];

const clickFunc = (element) => {
  return () => {
    console.log(element.nodeName);
  }
}

div.addEventListener("click", clickFunc(div), true);
button.addEventListener("click", clickFunc(button), false);
span.addEventListener("click", clickFunc(span), true);
\end{JavaScript}

\end{itemize}