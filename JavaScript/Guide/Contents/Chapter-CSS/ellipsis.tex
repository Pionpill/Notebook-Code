\section{ellipsis}

\subsection{单行省略号}

这个比较简单,直接上代码:

\begin{HTML}
.ellipsis {
  white-space: nowrap;
  width: 100px;       <!--设置一个宽度-->
  overflow: hidden;   <!--超出隐藏-->
  text-overflow: ellipsis;  <!--隐藏部分使用省略号-->
}
\end{HTML}

\subsection{多行省略号}

有两种方案,一是采用 webkit 提供的扩展属性:

\begin{HTML}
.ellipse {
  width: 50px;
  height: 65px;
  word-break: break-all;
  overflow: hidden;
  display: -webkit-box;
  -webkit-box-orient: vertical;
  -webkit-line-clamp: 3;
}
\end{HTML}

这种方案有两个缺陷: 一是并非所有浏览器支持 webkit(虽然绝大部分支持)。而是后续内容其实并没有被隐藏,如果设置足够高,会发现后续内容依然存在,我们只是让第3行末尾添加了省略号,仅此而已。

除此以外,我们可以创建为元素:

\begin{HTML}
.ellipse {
  width: 50px;
  height: 65px;
  position: relative;
  word-break: break-all;
  overflow: hidden;
  background-color: beige;
}

.ellipse::before {
  content: "...";
  position: absolute;
  bottom: 0;
  right:0;
}
\end{HTML}

但这样存在一个问题,适用性差,如果文本没有超出范围,省略号依然存在。一个解决方案是再添加一个遮蔽块伪元素放在文档后面,背景颜色与容器颜色相同:

\begin{HTML}
.ellipse::before {
  content: "...";
  position: absolute;
  bottom: 0;
  right:0;
}

.ellipse::after {
  content: "";
  background-color: blue;
  width: 1em;
  height: 1em;
  position: absolute;
  right: 0;
  margin-top: 5px;
}
\end{HTML}

但这样还是存在问题,我们设置的遮蔽块会跟在文本后面,但如果文本只有两行,遮蔽块会出现在第二行末尾,而省略号出现在第三行末尾,并没有起到遮蔽效果。

此外,如果背景颜色不是固定的,比如是图片,或者渐变。那么遮蔽块就无效了。

终极解决方案是:用 js 控制。不过这就不是这里要讨论的内容了。