\section{水平居中}

\subsubsection*{文档流}

HTML 的元素类型大致可以分为三类:
\begin{itemize}
  \item 块级元素: 元素创建独立的块,宽度为容器(父元素)的整个宽度。常见的块级元素有 \texttt{h1,h2...,p,div,ul...}
  \item 行内元素: 元素不会创建新行,只会在文本中嵌入并与文本同行显示元素。它们只会占据内容所需的宽度,常见的有 \texttt{span, img, input, button, strong}
  \item 行内块级元素: 像行内元素一样显示在同一行上,但可以设置宽度高度属性。行内快元素也经常被归为行内元素。如 \texttt{button, input, img}。
\end{itemize}

\subsubsection*{简单居中}

关于居中要分多种情况讨论:
\begin{itemize}
  \item 行内元素: 行内元素由于没有宽高,因此需要改变父级属性达到居中效果。
  \begin{itemize}
    \item 水平居中(父属性): \texttt{text-align: center}。
    \item 垂直居中(父属性): \texttt{line-height: xxx}。
  \end{itemize}
  \item 块级元素: 块级元素本身可以调整位置:
  \begin{itemize}
    \item 水平居中: \texttt{margin: 0 auto}。
    \item 垂直居中: 看下文。
  \end{itemize}
\end{itemize}

水平居中我们解决了,垂直居中就比较难搞了。一种方案是硬编码:
\begin{itemize}
  \item \texttt{margin-left: xxx}: 优点是实现简单,缺点是硬编码,适用性不高。
  \item 改为 \texttt{absolute} 布局,硬编码调整位置(容器改为 \texttt{relative} 布局): 没什么优点,需要看需求。缺点是硬编码,另外会导致元素脱离文档流。
\end{itemize}

\subsubsection*{Flex 布局}

Flex 布局现在已经推广开来,且有良好的浏览器兼容性。使用 Flex 布局居中非常简单:

\begin{HTML}
.flex {
  justify-content: center;
  align-items: center;
}
\end{HTML}

这样就可以达到主轴和副轴居中的效果。

\subsubsection*{\texttt{translate}}

\texttt{translate} 方法可以调整元素位置,注意这里使用 \% 调整是指元素本身宽高的百分比。该方法常常用在绝对布局,SVG 标签中。

\begin{HTML}
  .translate {
    position: absolute;
    left: 50\%;
    top: 50\%;
    transform: translate(-50\%, -50\%);
    }
  \end{HTML}
  
\subsubsection*{\texttt{vertical-align}}

这个方法用的非常少,用来表示普通流垂直方向的基线,即对齐线,Flex 布局也有对应的属性,这里不做过多介绍。