\section{BFC}

BFC, 块级作用域上下文。主要有两个功能:
\begin{itemize}
  \item 对外,去除外边距折叠。
  \item 对内,形成独立的内部区域,所有元素均在 BFC 内。
\end{itemize}

BFC 创建后由于对内会将内部所有元素包含进来,因此可以清除浮动,清除高度塌陷效果。

建立一个 BFC 区域的方法有:
\begin{itemize}
  \item 根元素:根元素即 HTML 元素,它默认就是一个 BFC。
  \item 浮动元素:如果一个元素被设置为浮动(float),那么它会形成一个 BFC。
  \item 绝对定位元素:如果一个元素被设置为绝对定位(position: absolute/fixed),那么它会形成一个 BFC。
  \item display 属性值为 inline-block、table-cell、table-caption 的元素:这些元素会形成一个 BFC。
  \item overflow 属性值不为 visible 的元素:如果一个元素的 overflow 属性值设置为 auto、scroll 或 hidden,那么它会形成一个 BFC。
\end{itemize}

主动建立 BFC 常常用 \texttt{overflow: auto}。因为这样没什么实际影响(影响很小)。或者使用 Flex 布局。