\section{绘制三角形}

\subsubsection*{\texttt{border}}

最简单的方法,使用 \texttt{border} 属性:
\begin{itemize}
  \item 确保内部 \texttt{width,height} 为 0。
  \item 设置 \texttt{border} 颜色为透明。
  \item 设置某一边 \texttt{border} 存在颜色。
\end{itemize}

\begin{HTML}
.triangle {
  width: 100px;
  height: 100px;
  border: 100px solid transparent;
  border-bottom: 100px solid #000;
  transform: translateY(-25\%);   <!--调整位置-->
}
\end{HTML}

这样有个缺点,是等腰直角三角形,也可以绘制正三角形,无需调整位置:

\begin{HTML}
.triangle {
  width: 120px;
  height: 120px;
  border-left: 69px solid transparent;  
	border-right: 69px solid transparent;  
	border-bottom: 120px solid skyblue;
}
\end{HTML}

\subsubsection*{\texttt{clip-path}}

最推荐的写法,但部分浏览器不支持。利用多边形函数,进行裁剪:

\begin{HTML}
.triangle {
  width: 100px;
  height: 80px;
  clip-path: polygon(0 0, 100\% 0, 50\% 100\%);
  background-color: aquamarine;
}
\end{HTML}

\subsubsection*{\texttt{linear-gradient}}

原理是使用两张背景图片叠在一起。

\begin{HTML}
.triangle {
  width: 80px;
  height: 100px;
  outline: 2px solid skyblue;
  background-repeat: no-repeat;
	background-image: linear-gradient(32deg, orangered 50\%, rgba(255, 255, 255, 0) 50\%), linear-gradient(148deg, orangered 50\%, rgba(255, 255, 255, 0) 50\%);
	background-size: 100\% 50\%;
	background-position: top left, bottom left;
}
\end{HTML}

实际上中间部分存在一条透明线,不推荐使用。

类似的,还有绘制圆形,也有上述三种思路。