\documentclass{PionpillNote-book}

\title{前端八股文}
\author{
    Pionpill \footnote{笔名:北岸,电子邮件:673486387@qq.com,Github:\url{https://github.com/Pionpill}} \\
    前端面经\\
}

\date{\today}

\begin{document}

\pagestyle{plain}
\maketitle

\noindent\textbf{前言:}

前端面经,包含 HTML,CSS,JavaScript,TypeScript,React,Redux。

简单的内容只给出代码,不画图。这里只给出结果,至于比较复杂的一些原理与深入解析,请参考我的其他文章。

本文撰写环境:

\begin{itemize}
    \item React: 18.2.0
    \item IDE: VSCode 1.72 
    \item Chrome: 91.0
    \item Node.js: 18.12
    \item OS: Window11
\end{itemize}

\date{\today}
\newpage

\tableofcontents

\newpage

\setcounter{page}{1} 
\pagestyle{fancy}

\import{Contents}{style.tex}
\chapter{CSS}
\import{Contents/Chapter-CSS}{abstract.tex}
\import{Contents/Chapter-CSS}{triangle.tex}
\import{Contents/Chapter-CSS}{ellipsis.tex}
\import{Contents/Chapter-CSS}{center.tex}
\import{Contents/Chapter-CSS}{BFC.tex}
\newpage
\chapter{JavaScript}
\import{Contents/Chapter-JS}{equal.tex}
\import{Contents/Chapter-JS}{type.tex}
\import{Contents/Chapter-JS}{scope.tex}
\import{Contents/Chapter-JS}{prototype.tex}
\import{Contents/Chapter-JS}{event.tex}
\import{Contents/Chapter-JS}{this.tex}
\import{Contents/Chapter-JS}{setTimeout.tex}
\import{Contents/Chapter-JS}{debounce.tex}
\import{Contents/Chapter-JS}{lazyLoading.tex}
\import{Contents/Chapter-JS}{deepCopy.tex}
\import{Contents/Chapter-JS}{promise.tex}

\end{document}