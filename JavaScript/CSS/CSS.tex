\documentclass{PionpillNote-book}
\usetikzlibrary {intersections,through,arrows.meta,graphs,shapes.misc,positioning,shapes.misc,positioning,calc}
\usetikzlibrary{animations}
\usetikzlibrary {shapes.geometric}
\usetikzlibrary {animations}
\usetikzlibrary {shapes.multipart}
\usetikzlibrary {positioning}
\usetikzlibrary {fit,shapes.geometric}
\usetikzlibrary {automata}
\usetikzlibrary {quotes}
\usetikzlibrary {matrix}
\usetikzlibrary {backgrounds}
\usetikzlibrary {scopes}
\usetikzlibrary {calc}
\usetikzlibrary {intersections}
\usetikzlibrary {svg.path}
\usetikzlibrary {decorations}
\usetikzlibrary {patterns}
\usetikzlibrary {decorations.pathmorphing}
\usetikzlibrary {shadows}
\usetikzlibrary {bending}
\usetikzlibrary{decorations.pathreplacing}

\title{CSS 笔记}
\author{
    Pionpill \footnote{笔名:北岸,电子邮件:673486387@qq.com,Github:\url{https://github.com/Pionpill}} \\
    本文档为作者系统学习CSS时的笔记。\\
}

\date{\today}

\begin{document}

\pagestyle{plain}
\maketitle

\noindent\textbf{前言:}

本文主要参考书籍: <<深入解析 CSS>>\footnote{<<CSS IN DEPTH>>: [美] Keith J. Grant 黄小璐译 2020年4月 第一版}。顾名思义,主要参考书籍是一本进阶书,因此本文不适合完全不了解 CSS 的入门读者。在阅读本文之前,确保有以下前置知识:
\begin{itemize}
    \item 掌握 HTML。
    \item 熟悉 CSS 的部分属性以及一些选择器。
\end{itemize}

HTML 与 CSS 的诸多标签/属性不一定要全部知道,不了解的直接查 MDN 文档即可。

这本笔记完全不能代替原书,仅是对原书的一个要点记录,只有对 CSS 有经验但需要查阅资料的读者适合本书。也可以将本书当作一个概要,根据不了的内容另行查找资料。

本人的编写及开发环境如下:
\begin{itemize}
    \item IDE: VSCode 1.72 
    \item Chrome: 91.0
    \item Node.js: 18.12
    \item OS: Window11
\end{itemize}

\date{\today}
\newpage

\tableofcontents

\newpage

\setcounter{page}{1} 
\pagestyle{fancy}

\chapter{基础}
\import{Contents/Chapter-Basic}{Basic.tex}
\import{Contents/Chapter-Basic}{Box.tex}
\chapter{布局}
\import{Contents/Chapter-Layout}{Float.tex}
\import{Contents/Chapter-Layout}{Flexbox.tex}
\import{Contents/Chapter-Layout}{Grid.tex}
\import{Contents/Chapter-Layout}{Position.tex}
\import{Contents/Chapter-Layout}{Responsive.tex}
\chapter{动画}
\import{Contents/Chapter-Animation}{Style.tex}
\import{Contents/Chapter-Animation}{Transitions.tex}
\import{Contents/Chapter-Animation}{Animation.tex}
\end{document}

