\section{动画}

CSS中的动画包括两部分:用来定义动画的@keyframes规则和为元素添加动画的animation属性。

关键帧定义了某个时间段内,不同时间刻的状态,具体的状态使用变换以及常规属性指定;至于具体的时间以及时间变换函数由调用关键帧的属性给出。关键帧语法如下:

\begin{HTML}
@keyframes over-and-back {
    0% {
      background-color: hsl(0, 50%, 50%);
      transform: translate(0);
    }

    50% {
      transform: translate(50px);
    }

    100% {
      background-color: hsl(270, 50%, 90%);
      transform: translate(0);
    }
}
\end{HTML}

调用关键帧需要 \texttt{animation} 属性,这个属性的具体用法和 \texttt{transition} 类似(毕竟都是动态的)。

\begin{HTML}
animation: over-and  -back 1.5s linear 
\end{HTML}

\texttt{animation} 也是一个简写,它的值依次对应如下属性:
\begin{itemize}
    \item \texttt{animation-name}: 动画名称,对应关键帧名。
    \item \texttt{animation-duration}: 动画持续的时间。
    \item \texttt{animation-timing-function}: 定时函数,前面讲过。
    \item \texttt{animation-}: 动画重复次数。
\end{itemize}

主要的动画技术就这些,动画需要实战经验,比较主流的动画框架是 \texttt{tree.js}。

\newpage