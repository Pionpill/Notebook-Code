\section{过渡与变换}

\subsection{过渡}

过渡是通过一系列\texttt{transition-*}属性来实现的。如果某个元素设置了过渡,那么当它的属性值发生变化时,并不是直接变成新值,而是使用过渡效果。\texttt{transition} 是该系列属性的简写,包括如下主要属性:
\begin{itemize}
    \item \texttt{transition-property}: 过度对象,值为 CSS 属性。
    \item \texttt{transition-duration}: 过度经历的时间。
    \item \texttt{transition-timing-function}: 过度速度曲线。
    \item \texttt{transition-delay}: 过渡动画延迟时间。
\end{itemize}

使用 \texttt{transition} 比较简单,也是常用做法,比如在 \texttt{:hover} 伪类选择器上写入:

\begin{HTML}
color: hsl(180,0.5,0.5);
transition: color .5s;
\end{HTML}

就会在 0.5s 内改变 \texttt{color} 属性的值。也可以单独设置每个属性,\texttt{transition-property} 可以设置为 \texttt{all},代表改变所有属性。

\texttt{transition-timing-function} 有以下几个常见值:
\begin{itemize}
    \item \texttt{liner}: 线性变化。
    \item \texttt{ease}: 慢-快-满
    \item \texttt{ease-in}: 慢-快
    \item \texttt{ease-out}: 快-满
    \item \texttt{ease-in-out}: 前两者组合
\end{itemize}

这个值本质上是调用贝塞尔函数: \texttt{cubic-bezier(n1,n2,n3,n4)},这四个值依次分为两队,代表了两个坐标 \texttt{(n1,n2)}, \texttt{(n3,n4)}。如果有类似 Illustrator 矢量作图软件经验,就知道这两个坐标决定了曲线控制柄的位置,贝塞尔曲线随之变换。

还有一种离散过度函数: \texttt{steps()},他会让元素突然过度,没有渐变效果,比如 \texttt{width} 属性从 100 到 200,会在相同时间间隔内经历这几个值变化 100,125,150,175,200。\texttt{step()} 有两个参数,阶跃次数和阶跃位置,次数即变化几次,位置只有两个值 \texttt{start,end},即什么时候发生节约。这个比较抽象,建议实战看看就知道了。

并非所有的属性都可以过渡,比如 \texttt{display};也并不是所有属性的任意值都可以过渡,比如只有\texttt{background-color} 仅有一个值时可以过度。一般值为数值,颜色,可以使用 \texttt{calc()}计算的的属性值可以过度,而非连续性属性不可以过度。

有两个属性可以替代 \texttt{display: none} 的效果;一个是 \texttt{opacity},代表不透明度;另一个是 \texttt{visibility} 有两个值 \texttt{visible, hidden}。不过它们最多做到看不到元素,但其实还存在于 DOM 树中。 

\subsection{变换}

变换通过 \texttt{transform} 属性实现。它的值为变换函数,可以有一个或多个值,常用的如下:
\begin{itemize}
    \item \texttt{rotate(deg)}: 顺时针旋转。
    \item \texttt{translate(x,y)}: 平面位移。 
    \item \texttt{scale(scale)}: 缩放。
    \item \texttt{skew(deg)}: 倾斜。
\end{itemize}

使用 \texttt{transform-origin} 属性可以改变基点,默认自然是盒模型中心。

CSS 还支持以上四类函数的变种: 三维版。三维的三个轴与屏幕的对应关系为: 
\begin{itemize}
    \item x轴: 屏幕宽度方向。
    \item y轴: 屏幕高度方向。
    \item z轴: 屏幕内方向,即人眼的方向。
\end{itemize}

三维属性命名也十分简单,例如 \texttt{rotate}的三维版: \texttt{rotateX, rotateY, rotateZ}。此外还有一个 \texttt{perspective()} 函数,表示透视距离,越小效果越明显。视距也可以设置基点,对应属性为 \texttt{perspective-origin}。

此外还有几个常用属性:
\begin{itemize}
    \item \texttt{backface-visibility}: 当旋转到一定角度后,我们可能看不到元素正面,如果想穿透显示背面元素,使用该属性。
    \item \texttt{ transform-style(preserve-3d)}: 看3D?没用过。
\end{itemize}

过渡与变换的区别在于发生在时间段还是时间刻: 过渡是一个过程,需要一段时间去完成, 而变换只是一个状态,它和常规样式属性没有本质区别。

\newpage