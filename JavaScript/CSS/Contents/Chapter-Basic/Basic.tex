\section{层叠样式表}

\subsection{层叠}

层叠决定了如何解决冲突,当声明冲突时,层叠会依次根据三种条件解决冲突:
\begin{itemize}
    \item 样式表的来源: 内联 > 外部 > 默认
    \item 选择器优先级: 例如 \texttt{id} > \texttt{class}
    \item 源码顺序: 样式在样式表里的声明顺序。
\end{itemize}

\subsubsection{样式表的来源}

开发者编写的样式表属于作者样式表,除此之外还有用户代理样式表(浏览器默认样式)。有的浏览的允许用户定义一个用户样式表,优先级介于作者样式表的用户代理样式表之间。

样式表的优先级为: 作者样式表 > 用户样式表 > 用户代理样式表。

\subsubsection*{\texttt{!important} 声明}

样式来源规则有一个例外:标记为重要(\texttt{important})的声明:

\begin{HTML}
color: red !important
\end{HTML}

标记了 \texttt{!important} 的声明会被当作更高优先级的来源。

\subsubsection{理解优先级}

浏览器将优先级分为两部分:HTML的行内样式和选择器的样式。

\subsubsection*{行内样式}

如果用HTML的\texttt{style}属性写样式,这个声明只会作用于当前元素。实际上行内元素属于“带作用域的”声明,它会覆盖任何来自样式表或者\texttt{<style>}标签的样式。行内样式没有选择器,因为它们直接作用于所在的元素。

如果要覆盖样式表里的行内声明,需要为声明添加\texttt{!important},这样能将它提升到一个更高优先级的来源。但如果行内样式也被标记为\texttt{!important},就无法覆盖它了。最好是只在样式表内用\texttt{!important}。

\subsubsection*{选择器优先级}

一般的,选择器优先级如下: ID > Class > Tag。伪类选择器与属性选择器和类选择器优先级相同; 通用选择器和组合器对优先级没有影响。

有时候我们会混用多个选择器,一个常用的表示优先级的方式是用数值形式来标记,通常用逗号隔开每个数。如,“1,2,2”表示选择器由1个ID、2个类、2个标签组成。优先级最高的ID列为第一位,紧接着是类,最后是标签。

比如 \texttt{\#page-header \#page-title} 有两个 Id, 表示为 ``2,0,0'' \texttt{ul li}有2个标签,表示为 ``0,0,2'',很明显 200 > 002,前者优先级更高。

如果涉及到行内样式,可以在最前面加一个数字表示是否为行内样式,此时,行内样式的优先级为“1,0,0,0”。不过这并没有实际意义,因为行内样式总是高于外部样式,而行内样式又不会写(一般不写)得过于复杂。

至于 \texttt{!important},所有使用该关键字的样式都会提升到最高的优先级来源(升维),因此他总是最高级的,多个 \texttt{!important} 会让一切回到起点,即撇去 \texttt{!important} 比较优先级。

\subsubsection{源码顺序}

如果两个声明的来源和优先级相同,其中一个声明在样式表中出现较晚,或者位于页面较晚引入的样式表中,则该声明胜出。

\begin{HTML}
.nav a {        /* 0,1,1 */
  color: white;
  background-color: #13a4a4;
  padding: 5px;
  border-radius: 2px;
  text-decoration: none;
}

a.featured {    /* 0,1,1 */
  background-color: orange;
}
\end{HTML}

上面两个声明块优先级都是 ``0,1,1'' 但下面的被选中使用。

由于这个特性多个状态的书写顺序就显得尤其重要,比如对按钮有悬停和激活两种状态,如果悬停状态在激活状态之后,激活后将会显示悬停的样式。因此状态的书写顺序应该为: \texttt{a:linked > a:visited > a:hover > a:active}。

\subsubsection*{层叠值}

浏览器遵循三个步骤,即来源、优先级、源码顺序,来解析网页上每个元素的每个属性。如果一个声明在层叠中“胜出”,它就被称作一个层叠值。元素的每个属性最多只有一个层叠值。

\subsubsection*{两条经验法则}

处理层叠时有两条通用的经验法则:
\begin{itemize}
    \item 非必要不要使用 \texttt{ID}: 如果不是要通过脚本控制节点树,应不用 ID 选择器。
    \item 不要使用 \texttt{!important}: 使用 \texttt{!important} 将带来一个新的维度,这让维护变得非常困难。
\end{itemize}

\subsection{继承}

如果一个元素的某个属性没有层叠值,则可能会继承某个祖先元素的值。比如通常会给\texttt{<body>}元素加上\texttt{font-family},里面的所有后代元素都会继承这个字体,就不必给页面的每个元素明确指定字体了。

但不是所有的属性都能被继承。默认情况下,只有特定的一些属性能被继承,通常是我们希望被继承的那些。它们主要是跟文本相关的属性:\texttt{color、font、font-family、font-size、font-weight、font-variant、font-style、line-height、letter-spacing、text-align、text-indent、text-transform、white-space、ord-spacing}。

此外,列表的一些属性也会被继承。

\subsubsection*{\texttt{inherit} 关键字}

有时,我们想用继承代替一个层叠值。这时候可以用\texttt{inherit}关键字。可以用它来覆盖另一个值,这样该元素就会继承其父元素的值。

\begin{HTML}
a {
    color: inherit;
    text-decoration: underline;
}
\end{HTML}

这么做的好处是如果父样式颜色改变,子样式也会随之改变。

\subsubsection*{\texttt{initial} 关键字}

\texttt{initial}关键字用于撤销作用于某个元素的样式,恢复为默认值。

\begin{HTML}
a {
    color: initial;
    text-decoration: underline;
}
\end{HTML}

和 \texttt{initial} 很像的有 \texttt{auto} 值,但 \texttt{auto} 并不是所有属性的默认值。

\subsection{相对单位}

CSS为网页带来了后期绑定(late-binding)的样式:直到内容和样式都完成了,二者才会结合起来。这会给设计流程增加复杂性,而这在其他类型的图形设计中是不存在的。不过这也带来了好处,即一个样式表可以作用于成百上千个网页。此外,用户还能直接改变最终的渲染效果,比如用户可以改变默认字号或者缩放浏览器窗口。

\subsubsection*{\texttt{em} 和 \texttt{rem}}

\texttt{em}是最常见的相对长度单位,适合基于特定的字号进行排版。在CSS中,1\texttt{em}等于当前元素的字号,其准确值取决于作用的元素。浏览器会根据相对单位的值计算出绝对值,称作计算值(computed value)。

由于 \texttt{em} 指代当前元素的字号,因此 \texttt{font-size} 不可以指定为 \texttt{em},比较字号不能是自己的多少倍,不过如果父级可以得出字号大小,这样做是可以做的。默认的字号是 16px,也即 \texttt{medium} 关键字的值。

\texttt{em} 相信绝大多数人都用过,\texttt{rem} 就不一定了。\texttt{rem} 是 \texttt{root em} 的缩写,表示根元素的单位。除此之外还有个 \texttt{ex} 表示 \texttt{x} 字体的大小,通常是 \texttt{em} 的一半。

在实际开发中,通常用用\texttt{rem}设置字号,用\texttt{px}设置边框,用\texttt{em}设置其他大部分属性。

\subsubsection*{视口相对单位}

前面介绍的\texttt{em}和\texttt{rem}都是相对于\texttt{font-size}定义的,但CSS里不止有这一种相对单位。还有相对于浏览器视口定义长度的视口的相对单位:
\begin{itemize}
    \item \texttt{vh}: 视口高度的 1/100;
    \item \texttt{vw}: 视口宽度的 1/100;
    \item \texttt{vmin}: 视口高度和宽度中较小一方的 1/100;
    \item \texttt{vmax}: 视口高度和宽度中较大一方的 1/100。
\end{itemize}

灵活使用这些相对单位可以帮助我们有效解决跨端问题。

此外,CSS 提供了一个 \texttt{calc()} 函数,可以对值进行基本运算,支持包括加减乘除运算,运算符两边都需要留出一个空格。比如 \texttt{calc(1em + 10px)}。

\subsubsection*{无单位值}

当一个元素的值定义为长度(\texttt{px、em、rem},等等)时,子元素会继承它的计算值。当使用\texttt{em}等单位定义行高时,它们的值是计算值。使用无单位的数值时,继承的是声明值,即在每个继承子元素上会重新算它的计算值。这样得到的结果几乎总是我们想要的。

\subsubsection*{自定义属性}

自定义属性(全称: 层叠变量的自定义属性) 给CSS引进了变量的概念,开启了一种全新的基于上下文的动态样式。你可以声明一个变量,为它赋一个值,然后在样式表的其他地方引用这个值。

比如我们定义一个自定义属性:

\begin{HTML}
:root {  
    --main-font: Helvetica, Arial, sans-serif;
} 
\end{HTML}

调用函数var()就能使用该变量。

\begin{HTML}
p {  
    font-family: var(--main-font);
} 
\end{HTML}

在样式表某处为自定义属性定义一个值,作为“单一数据源”,然后在其他地方复用它。这种方式特别适合反复出现的值,比如颜色值。

如果\texttt{var()}函数算出来的是一个非法值,对应的属性就会设置为其初始值。

\newpage