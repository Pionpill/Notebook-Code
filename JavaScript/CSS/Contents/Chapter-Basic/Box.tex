\section{盒模型}

\subsection{元素宽度问题}

元素宽度包括基础的盒模型概念,多个盒模型占整个网页的宽度。

\subsubsection*{盒模型}

盒模型的宽度有两种计算方式,使用 \texttt{box-sizing} 控制,有两个属性值:
\begin{itemize}
    \item \texttt{content-box}: 高宽仅包含内部元素,默认值。
    \item \texttt{border-box}: 高宽包括 \texttt{padding} 和 \texttt{border}。
\end{itemize}

\begin{figure}[H]
    \small
    \centering
    \begin{minipage}[t]{0.4\linewidth}
        \begin{tikzpicture}[scale = 1]
            \draw[line width=0, color = black!10, fill=black!10] (-1.6,-1.6) rectangle (5.6,4.6);
            \draw[line width=0, color = black!50, fill=black!50] (-1,-1) rectangle (5,4);
            \draw[line width=0, color = black!20, fill=black!20] (-0.5,-0.5) rectangle (4.5,3.5);
            \draw[line width=0, color = black!1, fill=black!1] (0,0) rectangle (4,3);
            \begin{scope} [font = \footnotesize]
                \node at (-1,-1.3) {margin};
                \node at (-0.5,-0.75) {border};
                \node at (0.2,-0.25) {padding};
                \draw [Stealth-Stealth] (0,2) -- (4,2) node [midway, below] {width};
                \draw [Stealth-Stealth] (3,0) -- (3,3) node [midway, right] {height};
            \end{scope}
        \end{tikzpicture}
        \texttt{box-sizing: content-box}
    \end{minipage}\qquad
    \begin{minipage}[t]{0.4\linewidth}
        \begin{tikzpicture}[scale = 1]
            \draw[line width=0, color = black!10, fill=black!10] (-1.6,-1.6) rectangle (5.6,4.6);
            \draw[line width=0, color = black!50, fill=black!50] (-1,-1) rectangle (5,4);
            \draw[line width=0, color = black!20, fill=black!20] (-0.5,-0.5) rectangle (4.5,3.5);
            \draw[line width=0, color = black!1, fill=black!1] (0,0) rectangle (4,3);
            \begin{scope} [font = \footnotesize]
                \node at (-1,-1.3) {margin};
                \node at (-0.5,-0.75) {border};
                \node at (0.2,-0.25) {padding};
                \draw [Stealth-Stealth] (-1,2) -- (5,2) node [midway, below] {width};
                \draw [Stealth-Stealth] (3,-1) -- (3,4) node [midway, right] {height};
            \end{scope}
        \end{tikzpicture}
        \texttt{box-sizing: border-box}
    \end{minipage}
    \caption{box-sizing}
\end{figure}

在实际开发中,我们更常用 \texttt{border-box} 值,因为这样有助于我们对其视口。

如我们使用默认值,在指定高宽时由于 \texttt{padding, border, margin} 都会影响整个盒模型在布局中的占用情况,如果我们缩放界面或者使用其他平台的设备查看网页,则很难确认最终的呈现效果。

当然这个缺陷有几个解决方案,一是  \texttt{padding, border, margin} 都设置为 0,属于是自损800的方案。而是情况慢慢调整 \texttt{width, height} 的值,这在一个界面上也许有用,但如果在手机等其他设备上效果往往不尽人意。

\subsubsection*{\texttt{box-sizing} 的继承问题}

前面说过,只有内容相关和表格的一些属性会被继承,很遗憾 \texttt{box-sizing} 是不能继承的,这会给开发带来很多繁琐的问题,下面有几个解决方案:

\begin{itemize}
    \item \textbf{全局修改盒模型}: 这是最常用的方案,将所有盒模型均设置为 \texttt{border-box},一般情况下没有问题。但如果我们使用第三方组件,如果有些组件使用的是 \texttt{content-box},这会改变这些组建的样式。
\begin{HTML}
*, ::before, ::after {   
    box-sizing: border-box;
} 
\end{HTML}
    \item \textbf{根节点继承}: 为了避免组件样式被修改,我们可以使用 \texttt{inherit} 强制 \texttt{box-sizing} 继承,再在需要继承的父节点(通常是根节点,也即 \texttt{<html>}) 节点设置盒模型方案。
\begin{HTML}
:root {  
    box-sizing: border-box;
} 
*, ::before, ::after {   
    box-sizing: inherit;
} 
\end{HTML}
\end{itemize}

\subsection{元素高度问题}

高度与宽度不同,由于网页高度是无限的,可以上下滑动,因此在高度方面我们更加关注元素的内容。为什么不用宽度来控制内容呢,因为宽度更加注重设备缩放以及分辨率问题。

\subsubsection*{控制溢出行为}

当明确设置一个元素的高度时,内容可能会溢出容器。用\texttt{overflow}属性可以控制溢出内容的行为,该属性支持以下4个值:
\begin{itemize}
    \item \texttt{visible}: 所有内容可见,即使溢出容器边缘。
    \item \texttt{hidden}: 溢出容器内边距边缘的内容被裁剪,无法看见。
    \item \texttt{scroll}: 容器出现滚动条,用户可以通过滚动查看剩余内容。在某些情况下,会出现水平和数值两种滚动条。
    \item \texttt{auto}: 只有内容溢出时容器才会出现滚动条,多数情况下使用 \texttt{auto} 而不是 \texttt{scroll}。
\end{itemize}

虽然不推荐,但是少数情况下宽度也是可以控制溢出行为的,使用 \texttt{overflow-y, overflow-x} 可分别控制高宽的溢出行为。

和高度相关的很多问题会在后面布局章节中说明,高度往往是随着布局的改变而改变的,除非特别需要,否则不要设置元素的高度,这会带来更棘手的问题。

\subsubsection*{\texttt{min-height} 与 \texttt{max-height}}

这两个属性可以指定最小或最大值,而不是明确定义高度,这样元素就可以在这些界限内自动决定高度。同样的还有 \texttt{min-width, max-width}。

\subsection{外边距问题}

\subsubsection*{负外边距}

不同于 \texttt{border, padding}, 外边距 \texttt{margin} 是可以设置为负值得。负外边距有一些特殊用途,比如让元素重叠或者拉伸到比容器还宽。

负外边距的具体行为取决于设置在元素的哪边:
\begin{itemize}
    \item 如果设置左边或顶部的负外边距,元素就会相应地向左或向上移动,导致元素与它前面的元素重叠。
    \item 如果设置右边或者底部的负外边距,并不会移动元素,而是将它后面的元素拉过来。
\end{itemize}

\fbox{
    \parbox{0.87\textwidth}{
        \begin{warning}
            如果元素被别的元素遮挡,利用负外边距让元素重叠的做法可能导致元素不可点击。
        \end{warning}
    }
}

\subsubsection*{外边距折叠}

有一个奇怪的现象,有时候我们设置外边距为 1em,但是多个盒模型顶部/底部相邻时,理论上应该是 2em,但实际上只有 1em 距离,被吃了一半。这种现象叫做外边距折叠。

外边距折叠的主要原因与包含文字的块之间的间隔相关。段落(\texttt{<p>})默认有1em的上外边距和1em的下外边距。这是用户代理的样式表添加的,但当前后叠放两个段落时,它们的外边距不会相加产生一个2em的间距,而会折叠,只产生1em的间隔。

折叠外边距的大小等于相邻外边距(两个及以上)中的最大值。

有几个方式可以防止外边距折叠:
\begin{itemize}
    \item 对容器使用\texttt{overflow: auto}(或者非\texttt{visible}的值),防止内部元素的外边距跟容器外部的外边距折叠。这种方式副作用最小。
    \item 在两个外边距之间加上边框或者内边距,防止它们折叠。
    \item 如果容器为浮动元素、内联块、绝对定位或固定定位时,外边距不会在它外面折叠。
    \item 当使用\texttt{Flexbox}布局或网络布局时,弹性布局内的元素之间不会发生外边距折叠。
    \item 当元素显示为\texttt{table-cell}时不具备外边距属性,因此它们不会折叠。此外还有\texttt{table-row}和大部分其他表格显示类型
\end{itemize}