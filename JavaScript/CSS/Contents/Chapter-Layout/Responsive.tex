\section{响应式设计}

\begin{center}
    \textit{这节讲的都是抽象概念,粒子请查看原书。如果只做桌面端设计也可以跳过这章。}
\end{center}

响应式设计主要用于解决多设备屏幕分辨率不同,响应式设计的主要目的是只需要创建一个网站,就可以在智能手机、平板,或者其他任何设备上运行,它有三大原则。

\begin{itemize}
    \item \textbf{移动优先}: 实现桌面布局之前先构建移动版的布局。
    \item \textbf{\texttt{@media} 规则}: 使用这个样式规则,可以为不同大小的视口定制样式。用这一语法,通常叫作媒体查询(media queries),写的样式只在特定条件下才会生效。
    \item \textbf{流式布局}: 这种方式允许容器根据视口宽度缩放尺寸。
\end{itemize}

\subsubsection*{移动优先}

考虑用户使用移动端的场景,一般通过手机/平板打开网站都处于快速浏览的状态,而桌面端多处于工作/学习等状态。因此移动端需要的提供更直观简要的信息。

移动端的需求和网页设计的逻辑是相符的,网页设计也应该先完成主要的核心功能,再考虑添加其他边缘功能。

如果你失陪了小屏设备,需要在 \texttt{meta} 加上这么一句,不然浏览器会假定你的界面不是响应式的:

\begin{HTML}
<meta name="viewport" content="width=device-width, initial-scale=1.0">
\end{HTML}

\texttt{meta}标签的\texttt{content}属性里包含两个选项。首先,它告诉浏览器当解析CSS时将设备的宽度作为假定宽度,而不是一个全屏的桌面浏览器的宽度。其次当页面加载时,它使用\texttt{initial-scale}将缩放比设置为100\%。

此外\texttt{content}属性还有第三个选项\texttt{user-scalable=no},阻止用户在移动设备上用两个手指缩放。通常这个设置在实践中并不友好,不推荐使用。

\subsubsection*{媒体查询}

响应式设计的第二个原则是使用媒体查询。媒体查询允许某些样式只在页面满足特定条件时才生效。这样就可以根据屏幕大小定制样式。可以针对小屏设备定义一套样式,针对中等屏幕设备定义另一套样式,针对大屏设备再定义一套样式,这样就可以让页面的内容拥有多种布局。

媒体查询使用\texttt{@media}规则选择满足特定条件的设备。简单的媒体查询如下代码所示:

\begin{HTML}
@media (min-width: 560px) {  
    .title > h1 {
        font-size: 2.25rem;  
    } 
} 
\end{HTML}

\texttt{@media}规则会进行条件检查,只有满足所有的条件时,才会将这些样式应用到页面上。本例中浏览器会检查\texttt{min-width: 560px}。只有当设备的视口宽度大于等于\texttt{560px}的时候,才会给标题设置\texttt{2.25rem}的字号。如果视口宽度小于\texttt{560px},那么里面的所有规则都会被忽略。

在媒体查询断点中推荐使用em单位。在各大主流浏览器中,当用户缩放页面或者改变默认的字号时,只有em单位表现一致。

可以整复杂一点,设置多个条件:

\begin{HTML}
@media (max-width: 20em), (min-width: 35em) { ... }
\end{HTML}

媒体查询的所有条件如下:
\begin{itemize}
    \item \texttt{min-height}: 高度大于等于。
    \item \texttt{max-height}: 高度小于等于。
    \item \texttt{max-width}: 宽度小于等于。
    \item \texttt{max-width}: 宽度小于等于。
    \item \texttt{orientation: landscape}: 宽度大于高度。
    \item \texttt{orientation: portrait}: 高度大于宽度。
    \item \texttt{min-resolution: 2dppx}: 匹配屏幕分辨率大于等于2dppx(dppx指每个CSS像素里包含的物理像素点数)的设备。
    \item \texttt{max-resolution: 2dppx}: 匹配屏幕分辨率小等于2dppx的设备。
\end{itemize}

比较少见的,还可以使用 \texttt{@media screen} 控制打印时的网页布局。

\subsubsection*{流式布局}

流式布局,有时被称作液体布局(liquid layout),指的是使用的容器随视口宽度而变化。它跟固定布局相反,固定布局的列都是用px或者em单位定义。固定容器(比如,设定了\texttt{width: 800px}的元素)在小屏上会超出视口范围,导致需要水平滚动条,而流式容器会自动缩小以适应视口。

在流式布局中,主页面容器通常不会有明确宽度,也不会给百分比宽度,但可能会设置左右内边距,或者设置左右外边距为auto,让其与视口边缘之间产生留白。也就是说容器可能比视口略窄,但永远不会比视口宽。

\subsubsection*{响应式图片}

图片传输往往要消耗大量流量,在不同大小的设备下可以放置不同分辨率的图片,比如:

\begin{HTML}
@media (min-width: 35em) {
    .hero {
        padding         : 5em 3em;
        font-size       : 1.2rem;
        background-image: url(coffee-beans-medium.jpg);
    }
}

@media (min-width: 50em) {
    .hero {
        padding         : 7em 6em;
        background-image: url(coffee-beans.jpg);
    }
}
\end{HTML}

在不同屏幕的浏览器上加载这样的网页,根本看不出有什么区别。

HTML里的\texttt{<img>}标签有一个 \texttt{srcset} 属性可以根据条件指定不同的图片URL:

\begin{HTML}
<img alt="A white coffee mug on a bed of coffee beans"
    src="coffee-beans-small.jpg"
    srcset="coffee-beans-small.jpg 560w,
        coffee-beans-medium.jpg 800w,
        coffee-beans.jpg 1280w"
/> 
\end{HTML}

这种 方式允许针对不同的屏幕尺寸优化图片。更棒的是,浏览器会针对高分辨率的屏幕做出调整。如果设备的屏幕像素密度是2倍,浏览器就会相应地加载更高分辨率的图片。