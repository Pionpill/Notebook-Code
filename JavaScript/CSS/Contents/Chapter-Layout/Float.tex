\section{浮动布局}

浮动布局,顾名思义,它会让元素浮动在界面上,类似于 word 文档里面的图片浮动。

浮动布局是一种很老的布局方案,现在基本已经不用了,Flexbox 正在取代浮动布局,一般只有做旧版兼容的时候会用到浮动布局。但要实现将图片移动到网页一侧,并且让文字围绕图片的效果,浮动仍然是唯一的方法。

\subsection*{容器折叠}

浮动布局存在一些问题(特性),浮动元素不同于普通文档流的元素,它们的高度不会加到父元素上。类似 word 文章中图片的高度由自己确定,对其他元素没有影响。

在网页中,这个特性存在一些问题,有时候浮动元素需要被父级元素包围,最简单的方式是增加一个 \texttt{<div>} 标签,并添加 \texttt{clear: both} 属性,\texttt{clear} 属性指定一个元素是否必须移动(清除浮动后)到在它之前的浮动元素下面。这样直到有 \texttt{clear} 属性标签的内容都会被包括在内:

\begin{HTML}
<div style="clear: both"></div>
\end{HTML}

但这样存在一个问题,我们为了添加 CSS 属性而创建了新的 HTML 标签。

进一步的解决方案是使用 \texttt{::after} 伪元素,这样就不用构建新的标签了:

\begin{HTML}
.float::after {
    clear: both;
}
\end{HTML}

注意,要给包含浮动的元素清除浮动,而不是给别的元素,比如浮动元素本身,或包含浮动的元素的后面的兄弟元素。

这个清除浮动还有个一致性问题没有解决:浮动元素的外边距不会折叠到清除浮动容器的外部,非浮动元素的外边距则会正常折叠。要解决这个问题可以使用 \texttt{display: table} 使用表格布局。

\subsubsection*{BFC}

BFC(block formatting context) 即 块级格式化上下文, BFC是网页的一块区域,元素基于这块区域布局。虽然BFC本身是环绕文档流的一部分,但它将内部的内容与外部的上下文隔离开。这种隔离为创建BFC的元素做出了以下3件事情。

\begin{itemize}
    \item 包含了内部所有元素的上下外边距。它们不会跟BFC外面的元素产生外边距折叠。
    \item 包含了内部所有的浮动元素。
    \item 不会跟BFC外面的浮动元素重叠。
\end{itemize}

简而言之,BFC里的内容不会跟外部的元素重叠或者相互影响。如果给元素增加\texttt{clear}属性,它只会清除自身所在BFC内的浮动。如果强制给一个元素生成一个新的BFC,它不会跟其他BFC重叠。给元素添加以下的任意属性值都会创建BFC。

\begin{itemize}
    \item \texttt{float}: \texttt{left} 或 \texttt{right}。
    \item \texttt{overflow}: \texttt{hidden、auto、scroll}。
    \item \texttt{display}: \texttt{inline-block、table-cell、table-caption、flex、inline-flex、grid、inline-grid}。拥有这些属性的元素称为块级容器(block container)。
    \item \texttt{position}:\texttt{absolute, fixed}。
\end{itemize}

最常见的设置 BFC 的方式是: \texttt{overflow: auto},这并不会实际上改变什么,但是设置了 BFC。

\subsubsection*{网格系统}

网络系统是一种定义样式的方案,而不是具体的技术。

网格系统可以提高代码的可复用性。网格系统提供了一系列的类名,可添加到标记中,将网页的一部分构造成行和列。它应该只给容器设置宽度和定位,不给网页提供视觉样式,比如颜色和边框。需要在每个容器内部添加新的元素来实现想要的视觉样式。

要构建一个网格系统,首先要定义它的行为。通常网格系统的每行被划分为特定数量的列,一般是12个,但也可以是其他数。每行子元素的宽度可能等于1~12个列的宽度。选取12作为列数是因为它能够被2、3、4、6整除,组合起来足够灵活。

在实际操作中,我们往往给标签添加 \texttt{column-x} 这样的属性构建网络,在 CSS 中使用 \texttt{[class*="column-"]} 这类属性选择器批量控制网络。再通过 \texttt{column-x} 详细控制单个元素属性。

\newpage