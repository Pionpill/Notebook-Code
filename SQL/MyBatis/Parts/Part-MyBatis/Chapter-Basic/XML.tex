\section{MyBatis 的 XML 使用方式}

\subsection{使用纯 XML 方式}

使用纯 XML 方式开发需要经历以下几个步骤:
\begin{itemize}
    \item 配置: 编写 mybatis-config.xml 配置文件。
    \item 映射: 编写 mapper 文件,统一管理 sql 语句。
    \item 编码: 在程序中获取并执行 SqlSession 语句。
\end{itemize}
K)
\subsubsection*{mybatis-config 配置文件}

使用 XML 形式进行配置,需要在 src/main/resources 下创建 mybatis-config.xml 文件\footnote{该配置来自MyBatis官网,包含了必要的配置信息,详细标签: \url{https://mybatis.net.cn/configuration.html}}:

\begin{xml}
<?xml version="1.0" encoding="UTF-8" ?>
<!DOCTYPE configuration
        PUBLIC "-//mybatis.org//DTD Config 3.0//EN"
        "http://mybatis.org/dtd/mybatis-3-config.dtd">
<configuration>
    <environments default="development">
        <environment id="development">
            <transactionManager type="JDBC"/>
            <dataSource type="POOLED">
                <property name="driver" value="${driver}"/>
                <property name="url" value="${url}"/>
                <property name="username" value="${username}"/>
                <property name="password" value="${password}"/>
            </dataSource>
        </environment>
    </environments>
    <mappers>
        <mapper resource="org/mybatis/example/BlogMapper.xml"/>
    </mappers>
</configuration>
\end{xml}

其中,<environments> 配置了数据库连接,<mappers> 中包含了一个 MyBatis 的 SQL 语句和映射配置文件,我们需要在对应的路径下创建映射文件,默认的,使用 Mapper 作为结尾(下文 mapper 文件就指的是 mybatis 的映射 xml 文件)。开发者可以根据需求修改或增添配置文件内容。

除此之外,我们还可以在 <configuration> 内添加下面这些元素:

\begin{xml}
<!-- 包别名,可以避免繁琐的完全限定名 -->
<typeAliases>
    <package name="learn.mybatis.simple.model"/>
</typeAliases>

<!-- 单独配置 sql 并引入 -->
<properties resource="sql.properties"/>

<!-- 使用日志 -->
<settings>
    <setting name="logImpl" value="LOG4J"/>
</settings>
\end{xml}

\subsubsection*{mapper 映射文件}

在对应的映射文件中,我们可以添加自定义 sql 语句, 下面是一个例子:

\begin{xml}
<?xml version="1.0" encoding="UTF-8" ?>
<!DOCTYPE mapper PUBLIC "-//mybatis.org//DTD Mapper 3.0//EN"
        "http://mybatis.org/dtd/mybatis-3-mapper.dtd">
<mapper namespace="org.mybatis.example.BlogMapper">
    <select id="selectAll" resultType="Country">
        select id, country_name, country_code from country
    </select>
</mapper>
\end{xml}

其中,<mapper> 是根元素,后面紧跟当前 xml 的命名空间。<select> 代表这是一个查询操作,id 是唯一标识,resultType 对应我们自己写的实体类。相应的,我们还可以使用 update, delete 等其他元素,这里不一一介绍。

有了这些配置,我们就可以在代码中进行操作了:

\begin{Java}
// 通过 xml 获取 SessionFactory
Reader reader = Resources.getResourcesAsReader("mybatis-config.xml")
SqlSessionFactory sqlSessionFactory = new SqlSessionFactoryBuilder().build(reader);
reader.close();
// 通过 SessionFactory 使用已书写的 sql 语句
SqlSession sqlSession = sqlSessionFactory.openSession();
List<Country> countryList = sqlSession.selectList("selectAll");
sqlSession.close();
\end{Java}

上述方式是完全使用 xml 来配置 Mybatis, 这种方式已经不常用了,下文不再说明, MyBatis3 提供了接口来调用方法。

\subsection{Mapper 代理开发}

Mapper 代理开发是指结合映射文件与接口的开发方式,它不需要通过字面值(上文的 selectAll)这种硬编码方式执行,安全性更高,更加面向接口编程,可以看作是 MyBatis3 对 XML 方式的一种改进,是目前主流的开发方式之一。

如果映射文件过多,我们还可以在 <mappers> 中使用 <package> 元素包含对应包下的所有映射文件。下面两种写法效果相同:

使用这种方式开发需要经历以下几步:
\begin{itemize}
    \item 配置: 编写 mybatis-config.xml 配置文件,定义与映射文件名相同的 Mapper 接口,并且映射文件和接口文件必须在同一目录下。
    \item 映射: 设置映射文件的名称空间为 Mapper 接口的完全限定名。
    \item 接口: 在接口中定义方法,方法名就是映射文件中的 id, 保持参数类型和返回值类型一致。
    \item 编码: 通过 SqlSession 获取 Mapper 接口的代理对象并执行 sql 操作。
\end{itemize}

在配置过程中,使用代理开发的多个同目录的 mapper 映射可以改用 package 元素同一加载:

\begin{xml}
<mappers>
    <mapper resource="learn/mybatis/simple/mapper/UserMapper.xml"/>
    <mapper resource="learn/mybatis/simple/mapper/RoleMapper.xml"/>
    <mapper resource="learn/mybatis/simple/mapper/PrivilegeMapper.xml"/>
    <mapper resource="learn/mybatis/simple/mapper/UserRoleMapper.xml"/>
    <mapper resource="learn/mybatis/simple/mapper/RolePrivilegeMapper.xml"/>
</mappers>
<!-- 等效写法 -->
<mappers>
    <package name="learn/mybatis/simple/mapper"/>
</mappers>
\end{xml}

这种配置方法会先查找对应 java 包下的所有接口,并加载接口对应的 XML 映射文件。

同样,mapper 文件的 namespace 也需要修改:

\begin{xml}
<?xml version="1.0" encoding="UTF-8" ?>
<!DOCTYPE mapper PUBLIC "-//mybatis.org//DTD Mapper 3.0//EN"
        "http://mybatis.org/dtd/mybatis-3-mapper.dtd">
<mapper namespace="learn.mybatis.simple.mapper.PrivilegeMapper"/>
\end{xml}

\subsection{SELECT 用法}

\subsubsection*{查询单个结果}

首先的,我们需要在 UserMapper.java 接口中声明 SELECT 方法,例如:
\begin{Java}
public interface UserMapper {
    SysUser selectById(long id);
}
\end{Java}

然后在对应的 UserMapper.xml 中添加下面代码:

\begin{xml}
<?xml version="1.0" encoding="UTF-8" ?>
<!DOCTYPE mapper PUBLIC "-//mybatis.org//DTD Mapper 3.0//EN"
        "http://mybatis.org/dtd/mybatis-3-mapper.dtd">
<mapper namespace="learn.mybatis.simple.mapper.UserMapper">
    <resultMap id="userMap" type="learn.mybatis.simple.mapper.SysUser">
        <id property="id" column="id"/>
        <result property="userName" column="user_name"/>
        <result property="userPassword" column="user_password"/>
        <result property="userEmail" column="user_email"/>
        <result property="userInfo" column="user_info"/>
        <result property="headImg" column="head_img" jdbcType="BLOB"/>
        <result property="createTime" column="create_time" jdbcType="TIMESTAMP"/>
    </resultMap>
    
    <select id="selectById" resultMap="userMap">
        select * from sys_user where id = #{id}
    </select>
</mapper>
\end{xml}

那么映射和接口文件/方法是怎么样绑定的呢?
\begin{itemize}
    \item 文件: 通过映射文件的 namespace 设置为接口的全限定名进行关联。如果不使用接口,则 namespace 可以随便写。
    \item 方法: 通过 <select> 标签的 id 属性值绑定。
\end{itemize}

映射文件中重要的标签和属性:
\begin{itemize}
    \item <select>: 映射查询语句使用的标签,id 属性与接口中的方法名对应。
    \item <resultMap>: 设置返回值的类型和映射关系。
    \item #{id}: MyBatis SQL中使用预编译参数的一种方式,大括号中的id是传入的参数名。
\end{itemize}

其中 <resultMap> 是最为重要的标签,它所包含的标签如下:
\begin{itemize}
    \item <constructor>: 配置使用构造方法注入结果,包含以下两个子标签。
    \begin{itemize}
        \item <idArg>: id 参数,标记结果作为 id,可以帮助提高性能。
        \item <arg>: 注入到构造方法的一个普通结果。
    \end{itemize}
    \item <id>: 一个 id 结果,标记结果作为 id, 可以帮助提高性能。
    \item <result>: 注入到 Java 对象属性的普通结果。
    \item <association>: 一个复杂的类型关联,许多结果将包成这种类型。
    \item <collection>: 复杂类型的集合。
    \item <discriminator>: 根据结果值来决定使用哪个结果映射。
    \item <case>: 基于某些值的结果映射。
\end{itemize}

更多标签的作用请查看官网文档,这里不再说明。

在使用时,我们只需要获通过 sqlSession 的 getMapper 获取接口对应的对象即可。

\begin{Java}
UserMapper userMapper = sqlSession.getMapper(UserMapper.class);
userMapper.selectAll();
\end{Java}

这里我们明明没有手动实现 UserMapper 接口,为什么就能用了呢?当然是动态代理,具体实现方法另查资料。

\subsubsection*{查询多个结果}

加入我们查询的结果不是单个对象而是对象集,例如在 UserMapper.java 中添加如下方法:

\begin{Java}
List<SysUser> selectAll();
\end{Java}

对应的映射文件需要添加如下代码:

\begin{xml}
<select id="selectAll" resultType="learn.mybatis.simple.model.SysUser">
    SELECT id,
           user_name     userName,
           user_password userPassword,
           user_email    userEmail,
           user_info     userInfo,
           head_img      headImg,
           create_time   createTime
    FROM sys_user
</select>
\end{xml}

当返回结果大于1个时,必须使用 List<SysUser> 或 SysUser[] 作为返回类型,否则会抛 TooManyResultsException 异常。 

此外,在映射文件中我们使用了 resultType 返回结果类型,配合别名可以实现类型的自动映射,而不需要手动设置 <resultMap>

\paragraph*{名称映射规则} 如前文所述,MyBatis 有两种名称映射规则:
\begin{itemize}
    \item 通过 resultMap 手动设置结果类型。
    \item 通过 SQL 别名配合 resultType 实现类型自动映射。
\end{itemize}

注意,<result> 中的 property 要和对象中的属性名相同,而实际上,MyBatis 会将这两者都转换为大写进行判断。

我们知道,sql 语句是不区分大小写的,采用下划线命名,而 Java 属性一般采用小驼峰命名。MyBatis 十分贴心地为提供了一个全局属性 mapUnderscoreToCamelCase, 通过配置这个属性可以自动映射这两种命名。在 mybatis-config 文件中可以开启该属性:

\begin{xml}
<settings>
    <setting name="mapUnderscoreToCamelCase" value="true"/>
</settings>
\end{xml}

于是,我们就可以这样写了:
\begin{xml}
<select id="selectAll" resultType="learn.mybatis.simple.model.SysUser">
    SELECT id, user_name, user_password, user_email, user_info, head_img, create_time
    FROM sys_user
</select>
\end{xml}

当然,也可以直接 select *, 但出于性能考虑,最好不要这样做。

\subsubsection*{处理复杂返回值}

前面的例子返回的都是一个表中的单个或多个数据,如果返回的数据中包含不止一个表中的结果,那么应该如何处理 resultType 呢? 例如如下语句:

\begin{sql}
SELECT r.id, r.role_name, r.enabled, r.create_by, r.create_time, u.user_name, u.user_email
FROM sys_user AS u,
     sys_user_role AS ur,
     sys_role AS r
         INNER JOIN ur on u.id = ur.user_id
         INNER JOIN r on ur.role_id = r.id
WHERE u.id = #{userId}
\end{sql}

这里有两种处理方式,一种是在 SysRole 对象中直接添加 userName 属性,这样仍然可以返回,但破坏了原对象,更推荐的方法是继承一个新的对象:

\begin{Java}
public class SysRoleExtend extends SysRole{
    private String userName;
    // getter setter 方法
}
\end{Java}

这种方式比较适合在需要少量字段时使用,但如果需要大量字段,就不适用了。

第二种方法,是直接让 SysRole 包含一个 SysUser 对象,然后修改对应的 XML 配置:

\begin{Java}
public class SysRole {
    // 其他字段
    private SysUser user;
}
\end{Java}

\begin{xml}
<select id="selectRolesByUserId" resultType="learn.mybatis.simple.model.SysRole">
    SELECT r.id, r.role_name, r.enabled, r.create_by, r.create_time, u.user_name as "user.userName", u.user_email as "user.userEmail"
    FROM sys_user AS u,
        sys_user_role AS ur,
        sys_role AS r
            INNER JOIN ur on u.id = ur.user_id
            INNER JOIN r on ur.role_id = r.id
    WHERE u.id = #{userId}
</select>
\end{xml}

\subsection{INSERT 用法}

INSERT 的用法和 SELECT 类似:

\begin{Java}
int insert(SysUser sysUser);
\end{Java}

\begin{xml}
<insert id="insert">
    insert into sys_user(id, user_name, user_password, user_email, user_info, head_img, create_time)
    values (#{id}, #{userName}, #{userPassword}, #{userEmail}, #{userInfo}, #{headImg, jdbcType=BLOB}, #{createTime, jdbcType=TIMESTAMP})
</insert>
\end{xml}

\begin{Java}
@Test
public void testInsert() {
    SqlSession sqlSession = this.getSqlSession();
    try {
        UserMapper userMapper = sqlSession.getMapper(UserMapper.class);
        SysUser user = new SysUser();
        user.setUserName("test");
        user.setUserPassword("123456");
        user.setUserEmail("test@mybatis.com");
        user.setUserInfo("test info");
        user.setHeadImg(new byte[]{1,2,3});
        user.setCreateTime(new Date());
        int result = userMapper.insert(user);
        Assert.assertEquals(1, result);
        Assert.assertNull(user.getId());
    } finally {
        // 不影响其他测试,这里回滚
        sqlSession.rollback();
        sqlSession.close();
    }
}
\end{Java}

这里有两个要点,一是在映射文件中,为了防止类型错误,对于一些特殊的数据类型,建议指定具体的 jdbcType 值。例如 headImg 对应 BLOB, createTime 对应 TIMESTAMP。
\begin{itemize}
    \item BLOB对应的类型是ByteArrayInputStream,就是二进制数据流。
    \item date、time、datetime对应的JDBC类型分别为DATE、TIME、TIMESTAMP。
\end{itemize}

此外的,我们会发现 insert 的默认返回值类型为 int,也即插入的数据条数。

有些情况下,我们需要指定返回值以方便后续操作,例如主键自增的情况下,可以这样编写映射文件:
\begin{xml}
<insert id="insert" useGeneratedKeys="true" keyProperty="id">
    <!-- 由于主键自增,删除insert 中的 id -->
</insert>
\end{xml}

useGeneratedKeys设置为true后,MyBatis会使用JDBC的getGeneratedKeys方法来取出由数据库内部生成的主键。获得主键值后将其赋值给keyProperty配置的id属性。当需要设置多个属性时,使用逗号隔开,这种情况下通常还需要设置 keyColumn 属性,按顺序指定数据库的列,这里列的值会和keyProperty配置的属性一一对应。

但这种写法只适用于支持自增的数据库,比如 Oracle 就不支持主键自增,而是使用序列获取 id。对此可以使用另一种方法 <selectKey>,可以处理自增和序列赋值两种情况。

\begin{xml}
<insert id="insert3">
    insert into sys_user(user_name, user_password, user_email, user_info, head_img, create_time)
    values (#{userName}, #{userPassword}, #{userEmail}, #{userInfo}, #{headImg, jdbcType=BLOB}, #{createTime, jdbcType=TIMESTAMP}
    <selectKey keyColumn="id" resultType="long" keyProperty="id" order="AFTER">
        SELECT LAST_INSERT_ID()
    </selectKey>
</insert>
\end{xml}

其他属性不再说明,order属性的设置和使用的数据库有关。在MySQL数据库中,order属性设置的值是AFTER,因为当前记录的主键值在insert语句执行成功后才能获取到。而在Oracle数据库中,order的值要设置为BEFORE,这是因为Oracle中需要先从序列获取值,然后将值作为主键插入到数据库中。

MySQL中的SQL语句SELECT LAST\_INSERT\_ID()用于获取数据库中最后插入的数据的 ID 值。其他数据对应的语句需要自行查阅。

\subsection{UPDATE 与 DELETE 用法}

UPDATE 用法和前面类似,这里只给出映射文件片段:

\begin{xml}
<update id="updateById">
    update sys_user
    set user_name     = #{userName},
        user_password = #{userPassword},
        user_email    = #{userEmail},
        user_info     = #{userInfo},
        head_img      = #{headImg, jdbcType=BLOB},
        create_time=#{createTime, jdbcType=TIMESTAMP}
    where id = #{id}
</update>
\end{xml}

DELETE 更简单:

\begin{xml}
<delete id="deleteById">
    delete from sys_user where id = #{id}
</delete>
\end{xml}

\subsection{多个接口参数的用法}

参数的类型可以分为两种: 一种是基本数据类型,一种是 JavaBean。
\begin{itemize}
    \item 基本类型: 在 XML 文件中对应的 SQL 语句只会使用一个参数,例如 delete 方法。
    \item JavaBean: 在 XML 文件中对应的 SQL 语句会有多个参数,例如 insert, update 方法。
\end{itemize}

之前我们都是将多个参数合并到一个 JavaBean 中,并使用这个 JavaBean 作为接口方法的参数。参数较少的时候,为此创建一个新的 JavaBean 就显得浪费,有两种方式解决这个问题: 使用 Map 类型作为参数或者使用 @Param 注解。

使用Map类型作为参数的方法,就是在Map中通过key来映射XML中SQL使用的参数值名字,value用来存放参数值,需要多个参数时,通过Map的key-value方式传递参数值,由于这种方式还需要自己手动创建Map以及对参数进行赋值,其实并不简洁,所以对这种方式掠过。

首先对不加 @param 的代码进行测试:

\begin{Java}
List<SysRole> selectRolesByUserIdAndRoleEnabled(Long userId, Integer enabled);
\end{Java}

\begin{xml}
<select id="selectRolesByUserIdAndRoleEnabled" resultType="learn.mybatis.simple.model.SysRole">
    select r.id, r.role_name, r.enabled, r.create_by, r.reate_time
    from sys_user as u,
        sys_user_role as ur,
        sys_role as r
             inner join ur on u.id = ur.user_id
             inner join r on ur.role_id = r.id
    where u.id = #{userId}
        and r.enabled = #{enabled}
</select>
\end{xml}

对他进行测试会显示如下错误:
\begin{xml}
Parameter 'userId' not found
Available parameters are [0,1,param1,param2]
\end{xml}

这时如果将XML中的#{userId}改为#{0}或#{param1},将#{enabled}改为#{1}或#{param2},这个方法就可以被正常调用了。但并不推荐这样做

正确的方法应该是在接口中加入 @Param 注解:

\begin{Java}
List<SysRole> selectRolesByUserIdAndRoleEnabled(
    @Param("userId") Long userId, 
    @Param("enabled") Integer enabled);
\end{Java}

这时的XML文件中对应的SQL的可用参数变成了[userId,enabled,param1,param2],如果把#{userId}改为#{param1},把#{enabled}改为#{param2},测试也可以通过。

给参数配置@Param注解后,MyBatis就会自动将参数封装成Map类型,@Param注解值会作为Map中的key,因此在SQL部分就可以通过配置的注解值来使用参数。

当只有一个参数(基本类型或拥有TypeHandler配置的类型)的时候,为什么可以不使用注解?这是因为在这种情况下(除集合和数组外),MyBatis不关心这个参数叫什么名字就会直接把这个唯一的参数值拿来使用。

但!如果包含多个 JavaBean 对象,比如:
\begin{Java}
List<SysRole> selectRolesByUserIdAndRoleEnabled(
    @Param("user") SysUser user, 
    @Param("role") SysRole role);
\end{Java}

就不能直接在映射文件中使用 \#{userId} 和 \#{enabled},毕竟没法确定这些属性只有一个,需要使用 \#{user.id} 和 \#{role.enabled} 从两个 JavaBean 中去除指定属性的值。

\newpage