\section{MyBatis 高级查询}

\subsection{高级结果映射}

\subsubsection{一对一映射}

一对一映射因为不需要考虑是否存在重复数据,因此使用起来很简单,而且可以直接使用MyBatis的自动映射。

\paragraph*{使用自动映射处理一对一关系}

使用自动映射就是通过别名让MyBatis自动将值匹配到对应的字段上,简单的别名映射如user\_name 对应 userName。

除此之外 MyBatis 还支持复杂的属性映射,可以多层嵌套,例如将role.role\_name映射到role.roleName上。MyBatis会先查找role属性,如果存在role属性就创建role 对象,然后在role 对象中继续查找roleName,将role\_name的值绑定到role对象的roleName属性上。

在 SysUser 中添加属性:
\begin{Java}
public class SysUser {
    private SysRole role;
    public SysRole getRole() {
        return role;
    }
    public void setRole(SysRole role) {
        this.role = role;
    }
    // 其他代码
}
\end{Java}

\begin{xml}
<select id="selectUserAndRoleById" resultType="learn.mybatis.simple.model.SysUser">
    SELECT u.id,
           u.user_name,
           u.user_password,
           u.user_email,
           u.user_info,
           u.head_img,
           u.create_time,
           r.id "role.id",
           r.role_name "role.roleName",
           r.enabled "role.enabled",
           r.create_by "role.createBy",
           r.create_time "role.createBy"
    FROM sys_user AS u,
         sys_role AS r,
         sys_user_role AS ur
             INNER JOIN ur on u.id = ur.user_id
             INNER JOIN r on ur.role_id = r.id
    WHERE u.id = #{id}
</select>
\end{xml}

注意上述方法中sys\_role查询列的别名都是“role.”前缀,通过这种方式将role的属性都映射到了SysUser的role属性上。通过SQL日志可以看到已经查询出的一条数据,MyBatis将这条数据映射到了两个类中,像这种通过一次查询将结果映射到不同对象的方式,称之为关联的嵌套结果映射。

\paragraph*{使用 resultMap 配置一对一映射}

resultMap 方式,除了写起来繁,没其他缺点:

\begin{xml}
<resultMap id="userRoleMap" type="learn.mybatis.simple.model.SysUser">
    <id property="id" column="id"/>
    <result property="userName" column="user_name"/>
    <result property="userPassword" column="user_password"/>
    ......
    <result property="role.id" column="role_id">
    <result property="role.roleName" column="role_name">
    <result property="role.enabled" column="enabled">
    ......
</resultMap>
\end{xml}

通过生成器可以生成一些基础代码后进行扩充可以方便一点。

\begin{xml}
<resultMap id="userRoleMap" type="learn.mybatis.simple.model.SysUser" extends="userMap">
    <result property="role.id" column="role_id"/>
    <result property="role.roleName" column="role_roleName"/>
    <result property="role.enabled" column="role_enabled"/>
    <result property="role.createBy" column="role_createBy"/>
    <result property="role.createTime" column="role_create_ime" jdbcType="TIMESTAMP"/>
</resultMap>
\end{xml}

\paragraph*{使用 resultMap 的 association 标签配置一对一映射}

在resultMap中,association标签用于和一个复杂的类型进行关联,即用于一对一的关联配置。

在上面配置的基础上,再做修改,改成association标签的配置方式,代码如下:

\begin{xml}
<resultMap id="userRoleMap" type="learn.mybatis.simple.model.SysUser" extends="userMap">
    <association property="role" columnPrefix="role_" javaType="learn.mybatis.simple.model.SysRole">
        <result property="id" column="id"/>
        <result property="roleName" column="name"/>
        <result property="enabled" column="enabled"/>
        <result property="createBy" column="createBy"/>
        <result property="createTime" column="create_img" jdbcType="TIMESTAMP"/>
    </association>
</resultMap>
\end{xml}

<association> 标签包含如下属性:
\begin{itemize}
    \item property: 对应实体类你中的属性名,必填。
    \item javaType: 属性对应 java 类型。
    \item resultMap: 使用现有的 resultMap。
    \item columnPrefix: 配置前缀,下面就不用写了。
\end{itemize}

\paragraph*{association 标签的嵌套查询}

除了前面3种通过复杂的SQL查询获取结果,还可以利用简单的SQL通过多次查询转换为我们需要的结果,这种方式与根据业务逻辑手动执行多次SQL的方式相像,最后会将结果组合成一个对象。

association 标签的嵌套查询常用的属性如下:
\begin{itemize}
    \item select: 另一个映射查询的 id, MyBatis 会额外执行这个查询获取嵌套对象的结果。
    \item column: 列名,将主查询中列的结果作为嵌套查询的参数,配置方式如: column = {prop = col1m prop2 = col2} prop1 和 prop2 将作为嵌套查询的参数。
    \item fetchType: 数据加载方式,可选值为 lazy 和 eager,分别为延迟加载和积极加载,这个配置会覆盖全局的lazyLoadingEnabled配置。
\end{itemize}

使用嵌套查询的方式配置一个和前面功能一样的方法,首先在UserMapper.xml中创建如下的resultMap。

\begin{xml}
<resultMap id="userRoleMapSelect" type="learn.mybatis.simple.model.SysUser" extends="userMap">
    <association property="role" column="{id = role_id}" select="learn.mybatis.simple.mapper.RoleMapper.selectRoleById"/>
</resultMap>
\end{xml}

然后创建对应的 select 方法:

\begin{xml}
<select id="selectUserAndRoleByIdSelect" resultMap="userRoleMapSelect">
    SELECT u.id,
           u.user_name,
           u.user_password,
           u.user_email,
           u.user_info,
           u.head_img,
           u.create_time,
           ur.role.id
    FROM sys_user AS u,
         sys_user_role AS ur
             INNER JOIN ur on u.id = ur.user_id
    WHERE u.id = #{id}
</select>
\end{xml}

\subsubsection{一对多映射}

\paragraph*{collection 集合的嵌套结果映射}

集合的嵌套结果映射就是指通过一次SQL查询将所有的结果查询出来,然后通过配置的结果映射,将数据映射到不同的对象中去。在一对多的关系中,主表的一条数据会对应关联表中的多条数据,因此一般查询时会查询出多个结果,按照一对多的数据结构存储数据的时候,最终的结果数会小于等于查询的总记录数。

在 SysUser 中修改属性:

\begin{Java}
public class SysUser {
    private List<SysRole> roleList;
    public List<SysRole> getRoleList() {
        return roleList;
    }

    public void setRoleList(List<SysRole> roleList) {
        this.roleList = roleList;
    }
    // 其他成员
}
\end{Java}

association 和 collection 的属性以及属性的作用完全相同,不再解释。如果我们已经配置过 userMap 和 roleMap, 那么我们可以直接这样写:

\begin{xml}
<resultMap id="userRoleListMap" type="learn.mybatis.simple.model.SysUser">
    <collection property="roleList" columnPrefix="role_" resultMap="learn.mybatis.simple.mapper.RoleMapper.roleMapper"/>
</resultMap>
\end{xml}

对应的映射代码如下:

\begin{xml}
<select id="selectAllUserAndRoles" resultMap="userRoleListMap">
    SELECT u.id,
           u.user_name,
           u.user_password,
           u.user_email,
           u.user_info,
           u.head_img,
           u.create_time,
           r.id          role_id,
           r.role_name   role_role_name,
           r.enabled     role_enabled,
           r.create_by   role_create_by,
           r.create_time role_create_time
    FROM sys_user AS u,
         sys_user_role AS ur,
         sys_role AS r
             INNER JOIN ur ON u.id = ur.user_id
             INNER JOIN r ON ur.role_id = r.id
</select>
\end{xml}

下面说明 mybatis 如何处理一对多的配置:

MyBatis 在处理结果的时候,会判断结果是否相同,如果是相同的结果,则只会保留第一个结果,所以这个问题的关键点就是 MyBatis 如何判断结果是否相同。MyBatis 判断结果是否相同时,最简单的情况就是在映射配置中至少有一个id标签,在userMap中配置如下。

\begin{xml}
<id property = "id" column = "id"/>
\end{xml}

我们对 id(构造方法中为 idArg)的理解一般是,它配置的字段为表的主键(联合主键时可以配置多个id标签),因为MyBatis的resultMap只用于配置结果如何映射,并不知道这个表具体如何。id的唯一作用就是在嵌套的映射配置时判断数据是否相同,当配置id标签时,MyBatis只需要逐条比较所有数据中id标签配置的字段值是否相同即可。在配置嵌套结果查询时,配置id标签可以提高处理效率。

这样一来,上面的查询就不难理解了。因为前两条数据的userMap部分的id相同,所以它们属于同一个用户,因此这条数据会合并到同一个用户中。

很可能会出现一种没有配置id的情况。没有配置id时,MyBatis就会把resultMap中配置的所有字段进行比较,如果所有字段的值都相同就合并,只要有一个字段值不同,就不合并。

在嵌套结果配置 id 属性时,如果查询语句中没有查询 id 属性配置的列,就会导致 id对应的值为null。这种情况下,所有值的id都相同,因此会使嵌套的集合中只有一条数据。所以在配置id列时,查询语句中必须包含该列。

MyBatis会对嵌套查询的每一级对象都进行属性比较。MyBatis会首先比较顶层的对象,如果SysUser部分相同,就继续比较SysRole部分,如果SysRole不同,就会增加一个SysRole,两个SysRole相同就保留前一个。假设SysRole还有下一级,仍然按照该规则去比较。

\subsubsection{鉴别器映射}

有时一个单独的数据库查询会返回很多不同数据类型(希望有些关联)的结果集。discriminator 鉴别器标签就是用来处理这种情况的。鉴别器非常容易理解,因为它很像Java语言中的switch语句。

discriminator标签常用的两个属性如下。
\begin{itemize}
    \item column: 用于设置要进行鉴别比较值的列。
    \item javaType: 用于指定列的类型,保证使用相同的 Java 类型来比较。
\end{itemize}

discriminator标签可以有1个或多个case标签,case标签包含以下三个属性:
\begin{itemize}
    \item value: 该值为discriminator指定column用来匹配的值。
    \item resultMap, resultType。
\end{itemize}

\begin{xml}
<resultMap id = "..." type = "...">
    <discriminator column="enabled" javaType="int">
        <case value="1" resultMap="..."/>
        <case value="0" resultMap="..."/>
    </discriminator>
</resultMap>
\end{xml}

\subsection{存储过程}

\subsubsection{调用存储过程}

添加存储过程与对应的映射文件如下:

\begin{sql}
# 根据用户 id 查询用户其他信息
DROP PROCEDURE IF EXISTS `select_user_by_id`;
DELIMITER ;;
CREATE PROCEDURE `select_user_by_id`(
    IN userId BIGINT,
    OUT userName VARCHAR(50),
    OUT userPassword VARCHAR(50),
    OUT userEmail VARCHAR(50),
    OUT userInfo TEXT,
    OUT headImg BLOB,
    OUT createTime DATETIME)
BEGIN
    SELECT user_name, user_password, user_email, user_info, head_img, create_time
    INTO userName, userPassword, userEmail, userInfo, headImg, createTime
    FROM sys_user
    WHERE id = userId;
END ;;
DELIMITER ;
\end{sql}

\begin{xml}
<select id="selectUserById" statementType="CALLABLE" useCache="false">
    {call select_user_by_id(
        #{id, mode = IN},
        #{userName, mode = OUT, jdbcType = VARCHAR},
        #{userPassword, mode = OUT, jdbcType = VARCHAR},
        #{userEmail, mode = OUT, jdbcType = VARCHAR},
        #{userInfo, mode = OUT, jdbcType = VARCHAR},
        #{headImg, mode = OUT, jdbcType = BLOB, javaType = _byte[]},
        #{createTime, mode = OUT, jdbcType = TIMESTAMP}
    )}
</select>
\end{xml}

在调用存储过程的方法中,需要把statementType设置为CALLABLE,在使用select标签调用存储过程时,由于存储过程方式不支持MyBatis的二级缓存,因此为了避免缓存配置出错,直接将select标签的useCache属性设置为false。

OUT 模式的参数必须指定jdbcType。这是因为在IN模式下,MyBatis提供了默认的jdbcType,在OUT模式下没有提供。

headImg还特别设置了javaType。在MyBatis映射的Java类中,不推荐使用基本类型,数据库BLOB类型对应的Java类型通常都是写成byte[]字节数组的形式的,因为 byte[]数组不存在默认值的问题,所以不影响一般的使用。但是在不指定javaType 的情况下,MyBatis 默认使用 Byte 类型。由于 byte 是基本类型,所以设置javaType时要使用带下画线的方式,在这里就是\_byte[]。\_byte对应的是基本类型,byte对应的是Byte类型。

添加接口及对应的测试如下:

\begin{xml}
void selectUserById(SysUser user);
\end{xml}

\begin{Java}
@Test
public void testSelectUserById() {
    try(SqlSession sqlSession = this.getSqlSession()) {
        UserMapper userMapper = sqlSession.getMapper(UserMapper.class);
        SysUser user = new SysUser();
        user.setId(1L);
        userMapper.selectUserById(user);
        Assert.assertNotNull(user.getUserName());
    }
}
\end{Java}

运行测试,所有的 OUT 属性就被加载到对应的对象中了。

\subsubsection{多参数的存储过程}

添加存储过程与对应的映射文件如下:

\begin{sql}
# 简单的根据用户名和分页参数进行查询,返回总数和分页数据
DROP PROCEDURE IF EXISTS `select_user_page`;
DELIMITER ;;
CREATE PROCEDURE `select_user_page`(
    IN userName VARCHAR(50),
    IN _offset BIGINT,
    IN _limit BIGINT,
    OUT total BIGINT)
BEGIN
    SELECT count(*) INTO total FROM sys_user WHERE user_name LIKE CONCAT('%', userName, '%');
    SELECT * FROM sys_user WHERE user_name LIKE CONCAT('%', userName, '%') LIMIT _offset,_limit;
END ;;
DELIMITER ;
\end{sql}

\begin{xml}
<select id="selectUserPage" statementType="CALLABLE" useCache="false" resultMap="userMap">
    {call select_user_page(
        #{userName, mode = IN},
        #{offset, mode = IN},
        #{limit, mode = IN},
        #{total, mode = OUT, jdbcType = BIGINT}
    )}
</select>
\end{xml}

这个映射字段对应的接口需要这样写:
\begin{sql}
List<SysUser> selectUserPage(Map<String, Object> params);
\end{sql}

由于需要多个入参和一个出参,而入参中除了userName属性在SysUser中,其他3个参数都和SysUser无关,因此为了使用SysUser而增加3个属性也是可以的。这里演示使用 Map 容器作为参数类型。

\begin{Java}
@Test
public void testSelectUserPage() {
    try(SqlSession sqlSession = this.getSqlSession()) {
        UserMapper userMapper = sqlSession.getMapper(UserMapper.class);
        Map<String, Object> params = new HashMap<String, Object>();
        params.put("userName","ad");
        params.put("offset", 0);
        params.put("limit", 10);
        List<SysUser> userList = userMapper.selectUserPage(params);
        Long total = (Long)params.get("total");
        System.out.println(total);
        for(SysUser user:userList){
            System.out.println(user.getUserName());
        }
    }
}
\end{Java}

\subsection{使用枚举或其他对象}

\subsubsection{使用 MyBatis 的枚举类型处理器}

首先创建枚举类型:

\begin{Java}
public enum Enabled {
    disable, enable;
}
\end{Java}

添加到对应类中:

\begin{Java}
private Enabled enabled;
public Enabled getEnabled() {
    return enabled;
}

public void setEnabled(Enabled enabled) {
    this.enabled = enabled;
}
\end{Java}

在数据库中不存在一个和Enabled枚举对应的数据库类型,因此在和数据库交互的时候,不能直接使用枚举类型,在查询数据时,需要将数据库int类型的值转换为Java中的枚举值。在保存、更新数据或者作为查询条件时,需要将枚举值转换为数据库中的int类型。

MyBatis在处理Java类型和数据库类型时,使用TypeHandler(类型处理器)对这两者进行转换。Mybatis为Java 和数据库JDBC中的基本类型和常用的类型提供了TypeHandler接口的实现。MyBatis 在启动时会加载所有的 JDBC 对应的类型处理器,在处理枚举类型时默认使用org.apache.ibatis.type.EnumTypeHandler处理器,这个处理器会将枚举类型转换为字符串类型的字面值并使用,对于Enabled而言便是"disabled"和"enabled"字符串。

除了这个枚举类型处理器,MyBatis还提供了另一个 EnumOrdinalTypeHandler处理器,这个处理器使用枚举的索引进行处理,可以解决此处遇到的问题。想要使用这个处理器,需要在mybatis-config.xml中添加如下配置。

\begin{xml}
<typeHandlers>
    <typeHandler javaType = "learn.mybatis.simple.type.Enabled" handler = "org.apache.ibatis.type.EnumOrdinalTypeHandler"/>
</typeHandlers>
\end{xml}

\subsubsection{自定义枚举类型处理器}

如果值既不是枚举的字面值,也不是枚举的索引值,这种情况下就需要自己来实现类型处理器。

修改枚举类型如下:
\begin{Java}
public enum Enabled {
    enabled(1), disable(0);

    private final int value;

    Enabled(int value) {
        this.value = value;
    }

    public int getValue() {
        return value;
    }
}
\end{Java}

现在 Enabled 中的值和顺序无关,针对该类,添加新的枚举类型处理器。

\begin{Java}
public class EnabledTypeHandler implements TypeHandler<Enabled> {
    private final Map<Integer, Enabled> enabledMap = new HashMap<Integer, Enabled>();

    public EnabledTypeHandler() {
        for(Enabled enabled: Enabled.values()) {
            enabledMap.put(enabled.getValue(), enabled);
        }
    }

    @Override
    public void setParameter(PreparedStatement ps, int i, Enabled parameter, JdbcType jdbcType) throws SQLException {
        ps.setInt(i, parameter.getValue());
    }

    @Override
    public Enabled getResult(ResultSet rs, String columnName) throws SQLException {
        return enabledMap.get(rs.getInt(columnName));
    }

    @Override
    public Enabled getResult(ResultSet rs, int columnIndex) throws SQLException {
        return enabledMap.get(rs.getInt(columnIndex));
    }

    @Override
    public Enabled getResult(CallableStatement cs, int columnIndex) throws SQLException {
        return enabledMap.get(cs.getInt(columnIndex));
    }
}
\end{Java}

EnabledTypeHandler实现了TypeHandler接口,并且针对4个接口方法对Enabled类型进行了转换。在TypeHandler接口实现类中,除了默认无参的构造方法,还有一个隐含的带有一个Class参数的构造方法。

\begin{Java}
public EnabledTypeHandler(Class<?> type) {
    this();
}
\end{Java}

当针对特定的接口处理类型时,使用这个构造方法可以写出通用的类型处理器,就像MyBatis 提供的两个枚举类型处理器一样。有了自己的类型处理器后,还需要在mybatis-config.xml中进行如下配置。

\begin{xml}
<typeHandlers>
    <typeHandler javaType = "learn.mybatis.simple.type.Enabled" handler = "learn.mybatis.simple.type.EnabledTypeHandler"/>
</typeHandlers>
\end{xml}

\newpage