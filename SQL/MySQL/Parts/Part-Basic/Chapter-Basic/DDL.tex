\section{DDL 数据定义语言}

数据定义语言,主要是针对数据库,数据表,视图的创建,修改。它包括以下主要语句关键字:
\begin{itemize}
    \item CREATE: 创建数据库,数据表,视图。
    \item DROP: 删除表。
    \item ALTER: 修改表。
    \item RENAME: 对表进行重命名。
    \item TRUNCATE: 清空表内容。
\end{itemize}

\fbox{
    \parbox{0.87\textwidth}{
        \begin{remark}
            非常奇怪,《MySQL 必知必会》一书没有介绍数据库操作,数据类型。本人自行添加了这部分内容。
        \end{remark}
    }
}

\subsection{数据库操作}

数据库操作非常简单,查看;创建;删除以及很少用到的修改。

\subsubsection{创建数据库}

在创建数据库之前,我们可以使用以下语句查看当前拥有的数据库:

\begin{sql}
SHOW DATABASES;
\end{sql}

更详细的,我们可以使用如下指令查看数据库信息:

\begin{sql}
SHOW CREATE DATABASE <databaseName>;
\end{sql}

创建数据库只需要确定数据库名称即可:

\begin{sql}
CREATE DATABASE <databaseName>;
\end{sql}

创建之后,我们可以使用数据库,如果不使用,则需要完全限定名来指定数据库。

\begin{sql}
USE <databaseName>;
\end{sql}

\subsubsection{删除数据库}

就这:

\begin{sql}
DROP DATABASE <databaseName>;
\end{sql}

\subsection{数据类型}

在进入到数据表操作之前,先说明一下基本的 MySQL 数据类型。

如前文所述,MySQL 数据类型大致可分为四种: 串数据类型,数值数据类型,日期和时间数据类型,二进制数据类型。

\subsubsection{串数据类型}

串数据类型类比高级语言中的字符串。MySQL 的串数据类型大致分为两种:
\begin{itemize}
    \item 定长串: 串长度是确定的,占用的空间也是确定的,处理速度快,空间换时间。
    \item 变长串: 串长度及其分配的空间不确定,由具体值确定,处理数据满,时间换空间。
\end{itemize}

不管使用何种形式的串数据类型,串值都必须括在引号内(通常单引号更好)。

\begin{table}[H]
    \small
    \centering
    \caption{串数据类型}
    \label{table:串数据类型}
    \setlength{\tabcolsep}{4mm}
    \begin{tabular}{c|ccc}
        \toprule
        \textbf{数据类型} & \textbf{类别} & \textbf{说明} & \textbf{示例} \\
        \midrule
        CHAR & 定长 & 1-255个字符。长度必须在创建时指定,默认 1 & CHAR(10) \\
        VARCHAR & 变长 & 0-255个字符。最大长度可在创建时指定 & VARCHAR(255) \\
        \midrule
        TINYTEXT & 变长 & 与 TEXT 相同,最大长度为 255 字节 & TINYTEXT \\
        MEDIUMTEXT & 变长 & 与 TEXT 相同,最大长度为 16K & MEDIUMTEXT \\
        TEXT & 变长 & 最大 64K,创建时无需确定长度 & TEXT \\
        LONGTEXT & 变长 &  与 TEXT 相同,最大长度为 4G & LONGTEXT \\
        \midrule
        ENUM & 变长 & 接受最多 64K 个串组成的集合的某个串 & ENUM(`A',`B') \\
        SET & 变长 & 接受最多64个串组成的集合的零个或多个串 & SET(`A',`B') \\
        \bottomrule
    \end{tabular}
\end{table}

关于 ENUM 和 SET 类型可以看下面两篇文章有个初步了解:
\begin{itemize}
    \item ENUM: \url{https://blog.csdn.net/qq_41684621/article/details/123111915}
    \item SET: \url{https://blog.csdn.net/qq_32253969/article/details/121746458}
\end{itemize}

\subsubsection{数值数据类型}

所有数值数据类型(除BIT 和BOOLEAN 外)都可以有符号或无符号。默认情况为有符号,可以使用UNSIGNED 关键字,这样做将允许你存储两倍大小的值。

\begin{table}[H]
    \small
    \centering
    \caption{数值数据类型}
    \label{table:数值数据类型}
    \setlength{\tabcolsep}{4mm}
    \begin{tabular}{c|ccc}
        \toprule
        \textbf{数据类型} & \textbf{类别} & \textbf{说明} & \textbf{示例} \\
        \midrule
        BIT & 二进制 & 位字段,1-64位 & b'01010001', 0x0001 \\
        \midrule
        BOOLEAN / BOOL & 布尔 & 本质上是 TINYINT(1) & true,0 \\
        \midrule
        TINYINT & 整数 & 8位,默认最大支持 127&  \\
        SMALLINT & 整数 & 16位,默认最大支持 $ 3.2 \times 10^4$  &  \\
        MEDIUMINT & 整数 & 24位,默认最大支持 $ 8.3 \times 10^6$ &  \\
        INT / INTEGER & 整数 & 32位,默认最大支持 $ 2.1 \times 10^9$ &  \\
        BIGINT & 整数 & 64位,默认最大支持 $ 9.2 \times 10^18$ &  \\
        \midrule
        REAL & 浮点 & 4字节的浮点值 &  \\
        FLOAT & 浮点 & 单精度浮点值(4位) &  \\
        DOUBLE & 浮点 & 双精度浮点值(8位) &  \\
        DECIMAL / DEC & 浮点 & 精度可变浮点值 & decimal(5,2) \\
        \bottomrule
    \end{tabular}
\end{table}

关于几种数值数据类型的基础用法如下:
\begin{itemize}
    \item BIT: \url{https://blog.csdn.net/dreamyuzhou/article/details/125535450}
    \item DECIMAL: \url{https://blog.csdn.net/u013214212/article/details/103028775}
\end{itemize}

\subsubsection{日期和时间数据类型}

\begin{table}[H]
    \small
    \centering
    \caption{日期和时间数据类型}
    \label{table:日期和时间数据类型}
    \setlength{\tabcolsep}{4mm}
    \begin{tabular}{c|c|c}
        \toprule
        \textbf{数据类型} & \textbf{说明} & \textbf{格式} \\
        \midrule
        DATA & 表示1000-01-01~9999-12-31 的日期 & YYYY-MM-DD \\
        TIME & 时间 & HH:MM:SS \\
        DATETIME & DATE 和TIME 的组合 &  \\
        TIMESTAMP & 功能和DATETIME 相同(但范围较小) &  \\
        YEAR & 2位数字: 范围是1970~2069,4位数字: 范围是1901~2155 &  \\
        \bottomrule
    \end{tabular}
\end{table}

TIMESTAMP 相关内容: \url{https://blog.csdn.net/hyfsx/article/details/114101630}

\subsubsection{二进制数据类型}

二进制数据类型可以存储图像,多媒体等等文件。

\begin{table}[H]    
    \small
    \centering
    \caption{二进制数据类型}
    \label{table:二进制数据类型}
    \setlength{\tabcolsep}{4mm}
    \begin{tabular}{c|ccc}
        \toprule
        \textbf{数据类型} & \textbf{最大长度} & \textbf{数据类型} & \textbf{最大长度} \\
        \midrule
        TINYBLOB & 255B & BLOB & 64K \\
        MEDIUMBLOB & 16M & LONGBLOB & 4G \\
        \bottomrule
    \end{tabular}
\end{table}

\subsection{创建数据表}

\subsubsection{CREATE TABLE 创建表}

创建表必须给出两个主要信息: 表名和表列的名字及定义。

\begin{sql}
CREATE TABLE customers (
    cust_id         int         NOT NULL AUTO_INCREMENT,
    cust_name       char(50)    NOT NULL ,
    cust_address    char(50)    NULL ,
    cust_city       char(50)    NULL ,
    cust_state      char(5)     NULL ,
    cust_zip        char(10)    NULL ,
    cust_country    char(50)    NULL ,
    cust_contact    char(50)    NULL ,
    cust_email      char(255)   NULL ,
    PRIMARY KEY (cust_id)
) ENGINE=InnoDB;
\end{sql}

除了表名,列名,列类型。还包含三个重要的其他信息: 列规定关键字,主键约束,引擎类型。

\fbox{
    \parbox{0.87\textwidth}{
        \begin{notice}
            如果你仅想在一个表不存在时创建它,应该在表名后给出IF NOT EXISTS 。这样做不检查已有表的模式是否与你打算创建的表模式相匹配。它只是查看表名是否存在,并且仅在表名不存在时创建它。
        \end{notice}
    }
}

\subsubsection{NOT NULL 强制数据}

默认情况下,列都是 NULL 列,即可以空缺。如果要指定某列不能空缺,可以使用 NOT NULL 进行规定。

\begin{sql}
CREATE TABLE orders (
      order_num  int      NOT NULL AUTO_INCREMENT,
      order_date datetime NOT NULL ,
      cust_id    int      NOT NULL ,
      PRIMARY KEY (order_num)
    ) ENGINE=InnoDB;
\end{sql}

\subsubsection{AUTO\_INCREMENT 自增}

AUTO\_INCREMENT 很好理解,自增。有两种方式指定初始值,一种是在创建表时用 AUTO\_INCREMENT=n 指定,一种通过 ALTER 语句指定: 

\begin{sql}
ALTER TABLE table_name AUTO_INCREMENT=n。
\end{sql}

AUTO\_INCREMENT 具体的数值遵循以下算法:
\begin{itemize}
    \item 如果插入的值与已有的编号重复,报错。
    \item 如果插入的值大于已编号的值,则从这个新值开始递增。
\end{itemize}

AUTO\_INCREMENT 有以下注意点:
\begin{itemize}
    \item 值必须唯一, 且必须有唯一索引,以避免序号重复(是主键,或是主键的一部分)。
    \item 值必须具有 NOT NULL 属性。
    \item 序号的最大值受该列的数据类型约束,一旦达到上限,AUTO\_INCREMENT就会失效。
\end{itemize}

如果我们要获取上一个 AUTO\_INCREMENT 值,可以通过下列语法:
\begin{sql}
SELECT last_insert_id();
\end{sql}

\subsubsection{DEFAULT 指定默认值}

使用 DEFAULT 可以指定某列的默认值,许多数据库开发人员使用默认值而不是 NULL 列。

\begin{sql}
CREATE TABLE orderitems (
  order_num  int          NOT NULL ,
  order_item int          NOT NULL ,
  prod_id    char(10)     NOT NULL ,
  quantity   int          NOT NULL  DEFAULT 1,
  item_price decimal(8,2) NOT NULL ,
  PRIMARY KEY (order_num, order_item)
) ENGINE=InnoDB;
\end{sql}

\subsubsection{UNIQUE 唯一}

顾名思义,这列的值必须是唯一的,有两种方法规定:

\begin{sql}
create table department(
    id int,
    name char(10) unique
)
create table department(
    id int,
    name char(10),
    unique(name)
)
\end{sql}

\subsubsection{PRIMARY KEY 主键}

主键有的主要作用是确定该数据的唯一性,因此主键必须 UNIQUE 且 NOT NULL。

主键有两种方式进行规定:

\begin{sql}
-- 方式 1
CREATE TABLE orderitems
(
  order_num  int          NOT NULL,
  order_item int          NOT NULL,
  prod_id    char(10)     NOT NULL,
  quantity   int          NOT NULL,
  item_price decimal(8,2) NOT NULL,
  PRIMARY KEY (order_num, order_item)
) ENGINE=InnoDB;
-- 方式 2
CREATE TABLE orderitems
(
  order_num  int          NOT NULL  PRIMARY KEY,
  order_item int          NOT NULL  PRIMARY KEY,
  prod_id    char(10)     NOT NULL,
  quantity   int          NOT NULL,
  item_price decimal(8,2) NOT NULL
) ENGINE=InnoDB;
\end{sql}

\subsubsection{FOREIGN KEY 外键}

\fbox{
    \parbox{0.87\textwidth}{
        \begin{notice}
            阿里开发规范: 禁用外键。除非不得已或者老板要求,不要用外键。
        \end{notice}
    }
}

外键的定义:
\begin{itemize}
    \item 外键是某个表中的一列,它包含在另一个表的主键中。
    \item 外键也是索引的一种,是通过一张表中的一列指向另一张表中的主键,来对两张表进行关联。
    \item 一张表可以有一个外键,也可以存在多个外键,与多张表进行关联。
\end{itemize}

外键的主要作用是保证数据的一致性和完整性,并且减少数据冗余。简言之,外键是为了关联其他数据表,例如在学生表中加入课程 ID 外键,以此来关联对应的课程数据。

与主键不同,外键可以是 NULL 或重复(不要求 UNIQUE)。外键有如下要求:

\begin{itemize}
    \item 从表插入/修改/删除新行,其外键值不是主表的主键值便阻止插入/修改/删除。
    \item 主表删除行,其主键值在从表里存在便阻止删除(要想删除,必须先删除从表的相关行)。
    \item 主表修改主键值,旧值在从表里存在便阻止修改(要想修改,必须先删除从表的相关行)。
\end{itemize}

级联执行:
\begin{itemize}
    \item 主表删除行,连带从表的相关行一起删除。
    \item 主表修改主键值,连带从表相关行的外键值一起修改。
\end{itemize}

创建外键的子句如下:
\begin{sql}
FOREIGN KEY (<columnName>) REFERENCE <tableName> (<columnName>)
\end{sql}

两种创建外键的方式

\begin{sql}
-- 创建表时创建
CREATE TABLE Products (
	prod_id     INT             NOT NULL PRIMARY KEY AUTO_INCREMENT,
	vend_id     INT             NOT NULL COMMENT '供应商ID',
	prod_name   VARCHAR(30)     NOT NULL COMMENT '产品名',
	prod_price  DOUBLE          NOT NULL COMMENT '产品价格',
	prod_desc   VARCHAR(100)    COMMENT '产品描述',
	FOREIGN KEY (vend_id) REFERENCES Vendors (vend_id)CO
);
-- 已创建的表添加外键
ALTER TABLE Products 
    ADD FOREIGN KEY products_vendors_fk_1 (vend_id) REFERENCES Vendors (vend_id);
\end{sql}


总而言之,外键虽然能带来一定的好处,但操作外键却比较复杂,且影响数据库性能。如果可以,应尽量在应用层完成相关操作。下面有两个深入了解外键的链接:

\begin{itemize}
    \item 外键: \url{https://blog.csdn.net/Jerry_Chang31/article/details/105093881}
    \item 创建外键: \url{https://blog.csdn.net/Jack_PengPeng/article/details/87690101}
\end{itemize}

\subsubsection{引擎类型}

MySQL5 及后续版本默认使用 InnoDB 引擎,下面说一下几个常用引擎的区别:

\begin{itemize}
    \item InnoDB: MySQL5 及之后的默认引擎,具有事务处理能力,不支持全文本搜索,是最主流的引擎。
    \item MyISAM: MySQL5 之前的默认引擎,不具备事务处理能力,支持全文本搜索,逐渐被 InnoDB 取代。
    \item MEMORY: 功能等同于 MyISAM, 数据存储在内存中,不是磁盘,速度快,适用于临时表。
\end{itemize}

主要,只有 InnoDB 支持外键,其他引擎不支持外键。

\subsubsection{字符集与校对}

这部分了解即可。

MySQL支持众多的字符集。为查看所支持的字符集完整列表,使用以下语句:

\begin{sql}
SHOW CHARACTER SET;
\end{sql}

校对是为规定字符如何比较的指令,如是否区分大小写。为了查看所支持校对的完整列表,使用以下语句:

\begin{sql}
SHOW COLLATION;
\end{sql}

实际上,字符集很少是服务器范围(甚至数据库范围)的设置。不同的表,甚至不同的列都可能需要不同的字符集,而且两者都可以在创建表时指定。

\begin{sql}
CREATE TABLE mytable (
   columnn1   INT,
   columnn2   VARCHAR(10)
) DEFAULT CHARACTER SET hebrew
  COLLATE hebrew_general_ci;
\end{sql}

除了能指定字符集和校对的表范围外,MySQL还允许对每个列设置它们,如下所示:

\begin{sql}
CREATE TABLE mytable (
   columnn1   INT,
   columnn2   VARCHAR(10),
   column3    VARCHAR(10) CHARACTER SET latin1 COLLATE latin1_general_ci
) DEFAULT CHARACTER SET hebrew
  COLLATE hebrew_general_ci;
\end{sql}

校对在对用 ORDER BY 子句检索出来的数据排序时起重要的作用。可以显示指出 ORDER BY 所用的校对类型:

\begin{sql}
SELECT * FROM customers
    ORDER BY lastname, firstname COLLATE latin1_general_cs;
\end{sql}


\subsection{其他表操作}

\subsubsection{ALTER TABLE 更新表}

理想状态下,当表中存储数据以后,该表就不应该再被更新。更新表往往需要大量的时间处理。

ALTER TABLE 本身的语法十分简单:

\begin{sql}
ALTER TABLE vendors
    ADD vend_phone CHAR(20);    -- 增加列
    DROP old_vend_phone;        -- 删除列
\end{sql}

默认情况下,ADD/DROP 都是对列进行操作。也可以显示指定 COLUMN 关键字。

更常用的,ALTER TABLE 用来指定外键约束:

\begin{sql}
ALTER TABLE Products
    ADD FOREIGN KEY products_vendors_fk_1 (vend_id) REFERENCES Vendors (vend_id)
\end{sql}

\fbox{
    \parbox{0.87\textwidth}{
        \begin{notice}
            使用 ALTER TABLE 需要及其小心,因为 MySQL 没有回滚,不能撤销。修改表往往会对一整列进行修改。
        \end{notice}
    }
}

\subsubsection{DROP TABLE 删除表}

删除表是删除整个表,不同于行操作中的 DELETE FROM 删除所有行。

\begin{sql}
DROP TABLE customers2;
\end{sql}

除此之外,还可以用 TRUNCATE 删除表中所有数据:

\begin{sql}
TRUNCATE [TABLE] <tableName>;
\end{sql}

虽然 TRUNCATE 删除的是所有数据而不是表本省,但他属于 DDL 语言,速度更快。

\subsubsection{RENAME TABLE 重命名表}

可以重命名一个或多个表:

\begin{sql}
-- 重命名一个
RENAME TABLE customers2 TO customers;
-- 重命名多个
RENAME TABLE old_student TO student,
            old_major TO major,
            old_teacher TO teacher;
\end{sql}

\subsection{视图}

\subsubsection{视图的作用}

视图是虚拟的表,表面上它和表的显示形式一样,都是表结构。实际上是对表的数据显示。

视图从 MySQL5 开始引入,为什么要引入视图? 举个很简单的例子,学生选课,在学生表中只有课程的 ID,这就够了,因为可以根据 ID 再次查询到选课的相关信息,这样设计数据库避免了数据冗余。但是我要看数据的时候,选课 ID 对于人来说并不能提供任何信息,我需要将学生表和课程表组合起来。

再举个例子,我要从几个表中提取相关的数据信息,可以这样做:
\begin{sql}
SELECT cust_name, cust_contact
    FROM customers, orders, orderitems
    WHERE customers.cust_id = orders.cust_id
      AND orderitems.order_num = orders.order_num
      AND prod_id = 'TNT2';
\end{sql}

也不是不行,但似乎有点麻烦,假如可以把整个查询包装成一个名为productcustomers 的虚拟表,则可以如下轻松地检索出相同的数据:

\begin{sql}
SELECT cust_name, cust_contact
    FROM productcustomers
    WHERE prod_id = 'TNT2';
\end{sql}

显而易见,这样更方便。可以看出,设计数据表的时候我们更关注的是性能,后续维护。但在使用数据表的时候,我们却希望更加方便,视图就是为了解决这个问题而出现的。

使用视图有以下好处:
\begin{itemize}
    \item 简化复杂的SQL操作。
    \item 使用表的组成部分而不是整个表。
    \item 保护数据。可以给用户授予表的特定部分的访问权限而不是整个表的访问权限。
    \item 更改数据格式和表示。视图可返回与底层表的表示和格式不同的数据。
\end{itemize}

在视图创建之后,可以用与表基本相同的方式利用它们。可以对视图执行SELECT 操作,过滤和排序数据,将视图联结到其他视图或表,甚至能添加和更新数据。

重要的是知道视图仅仅是用来查看存储在别处的数据的一种设施。视图本身不包含数据,因此它们返回的数据是从其他表中检索出来的。在添加或更改这些表中的数据时,视图将返回改变过的数据。

 \fbox{
     \parbox{0.87\textwidth}{
         \begin{notice}
            试图本身并不包含数据,如果你用多个联结和过滤创建了复杂的视图或者嵌套了视图,可能会发现性能下降得很厉害。
         \end{notice}
     }
 }

 视图有如下规则和限制:

 \begin{itemize}
    \item 与表一样,视图必须唯一命名。
    \item 对于可以创建的视图数目没有限制。
    \item 为了创建视图,必须具有足够的访问权限。这些限制通常由数据库管理人员授予。
    \item 视图可以嵌套,即可以利用从其他视图中检索数据的查询来构造一个视图。
    \item ORDER BY 可以用在视图中,但如果从该视图检索数据SELECT 中也含有ORDER BY ,那么该视图中的ORDER BY 将被覆盖。
    \item 视图不能索引,也不能有关联的触发器或默认值。
    \item 视图可以和表一起使用。例如,编写一条联结表和视图的SELECT 语句。
 \end{itemize}

 \subsubsection{使用视图}

操作视图和操作表类似,常见语句如下:
 \begin{itemize}
    \item 创建视图: CREATE VIEW ... AS ...
    \item 显示视图: SHOW CREATE VIEW viewName
    \item 删除视图: DROP VIEW viewName
    \item 更新视图: 先删除后创建,或者一起: CREATE OR REPLACE VIEW
 \end{itemize}

视图的作用有很多,下面举几个例子:

\begin{sql}
-- 利用视图简化复杂的联结
CREATE VIEW productcustomers AS
    SELECT cust_name, cust_contact, prod_id
    FROM customers, orders, orderitems
    WHERE customers.cust_id = orders.cust_id
        AND orderitems.order_num = orders.order_num;
-- 用视图重新格式化检索出的数据
SELECT Concat(RTrim(vend_name), ' (', RTrim(vend_country), ')')
    AS vend_title
    FROM vendors
    ORDER BY vend_name;
-- 用视图过滤不想要的数据
CREATE VIEW customeremaillist AS
    SELECT cust_id, cust_name, cust_email
    FROM customers
    WHERE cust_email IS NOT NULL;
-- 使用视图与计算字段
CREATE VIEW orderitemsexpanded AS
    SELECT prod_id, quantity, item_price, quantity*item_price AS expanded_price
    FROM orderitems;
\end{sql}

一般,应该将视图用于检索(SELECT 语句)而不用于更新(INSERT 、UPDATE 和 DELETE)。前面提到过,视图本身并没有数据,只是对表数据的显示。如果视图定义中有如下操作,则不能对视图进行更新:

\begin{itemize}
    \item 分组: GROUP BY, HAVING
    \item 联结
    \item 子查询
    \item 并
    \item 聚集函数
    \item DISTINCT
    \item 导出/计算 列
\end{itemize}

由此可以看出,视图的限制还很很多的,没事别更新视图,看看就行了。

\newpage