\section{DML 数据操作语言(增删改查)}

数据库操作语言,主要针对表中数据操作。是最常用的数据库操作语言,它包括以下几个主要语句关键字:
\begin{itemize}
    \item SELECT: 查找/检索记录。
    \item INSERT: 插入记录。
    \item UPDATE: 修改记录。
    \item DELETE: 删除记录。
\end{itemize}

其中 SELECT 即查数据,高级一点叫检索数据。毋庸置疑 SELECT 语句是 SQL 语句中最复杂,最常用的。本章的 1-11 节都在介绍 SELECT 语句。

\subsection{SELECT 语句}
\subsubsection{SELECT 检索数据}

单表查询非常简单,这里只给出一些例子,不做解释:

\begin{sql}
SELECT * FROM mysql.user;       -- 检索全部列
SELECT user FROM mysql.user;    -- 检索单个指定列
SELECT user,host FROM mysql.user;   -- 检索多个指定列
\end{sql}

\subsubsection{DISTINCT 返回不同的值}

如果我们执行下面语句:
\begin{sql}
USE mysql;
SELECT host FROM user;
\end{sql}

会返回多个值相同的视图,如果我们只想让重复的值出现一次,可以加上关键字: DISTINCT

\begin{sql}
SELECT DISTINCT host from user;
\end{sql}

注意,DISTINCT 关键字会作用于所有的列,可以尝试下列语句观察结果。

\begin{sql}
SELECT DISTINCT host,user from user;
\end{sql}

\subsubsection{LIMIT 限制显示数量}

默认的 SELECT 语句会返回所有行,有时候数据太多会很卡,LIMIT 关键字将限制返回的行数量:

\begin{sql}
SELECT * FROM information_schema.plugins LIMIT 10;
\end{sql}

在终端中这只会显示前 10 行,如果需要返回指定的行数(11-20),可以这样写:

\begin{sql}
SELECT * FROM information_schema.plugins LIMIT 10,10;
\end{sql}

\fbox{
    \parbox{0.87\textwidth}{
        \begin{notice}
            这里 LIMIT 10,10 表示的是从第 10 行开始的 10 行,也即 11-20 行。MySQL 还支持另一种写法 4 OFFSET 3,表示从第 3 行开始的 4 行。和 LIMIT 3,4 作用一样。
        \end{notice}
    }
}

\subsubsection*{完全限定名}

下面语句的作用是一样的:
\begin{sql}
-- 语句 1
USE mysql;
SELECT host FROM user;
-- 语句 2
SELECT user.host FROM mysql.user;
\end{sql}

完全限定名更显示地指出了具体数据库,数据表,字段的信息。

\subsection{数据排序}

\fbox{
    \parbox{0.87\textwidth}{
        \begin{notice}
            子句: SQL 语句由子句构成,有些子句是必须的,有些的可选的。比如 SELECT 语句有一个必须的子句 FROM。
        \end{notice}
    }
}

\subsubsection{ORDER BY 排序}

ORDER BY 子句取一个或多个列的名字,并据此进行排序。

\begin{sql}
SELECT * FROM mysql.help_keyword ORDER BY help_keyword_id LIMIT 10;
\end{sql}

如果有必须要的话,也可以按多个列排序。

\fbox{
    \parbox{0.87\textwidth}{
        \begin{notice}
            通常,ORDER BY 子句中使用的列将是为显示所选择的列。但是,实际上并不一定要这样,用非检索的列排序数据是完全合法的。
        \end{notice}
    }
}

默认的排序方式为升序排序(A-Z), 同样可以指定降序排序。两种排序的关键字分别为:
\begin{itemize}
    \item ASC: 升序排序。
    \item DESC: 降序排序。
\end{itemize}

关键字跟在列名后面,且仅对改列有效:

\begin{sql}
SELECT prod_id, prod_price, prod_name
    FROM products
    ORDER BY prod_price DESC, prod_name;
\end{sql}

上述例子表示从 products 表中获取数据,按 prod\_price 降序排序,若相同,再按 prod\_name 升序排序。

\subsection{数据过滤}

\subsubsection{WHERE 过滤}

WHERE 子句可以对搜索条件进行过滤:

\begin{sql}
SELECT user,host FROM mysql.user WHERE user='root';
\end{sql}

基本的 WHERE 子句操作符有:

\begin{table}[H]
    \centering
    \caption{WHERE 子句操作符}
    \label{table:WHERE 子句操作符}
    \setlength{\tabcolsep}{4mm}
    \begin{tabular}{cc|cc|cc}
        \toprule
        \textbf{操作符} & \textbf{说明} & \textbf{操作符} & \textbf{说明} &\textbf{操作符} & \textbf{说明} \\
        \midrule
        = & 等于 & <>/!= & 不等于 & < & 小于 \\
        <= & 小于等于 & > & 大于 & >= & 大于等于 \\
        \bottomrule
    \end{tabular}
\end{table}

如果是对数值进行过滤,直接填写相应的值即可:
\begin{sql}
SELECT * FROM product WHERE price=10;
\end{sql}

如果是对字符串进行过滤,则需要加上单引号,以表示这是字符串:
\begin{sql}
SELECT * FROM product WHERE produce_name='rice';
\end{sql}

此外还有一个常用的范围比较操作符: BETWEEN, 用法如下:
\begin{sql}
SELECT prod_name, prod_price
    FROM products
    WHERE prod_price BETWEEN 5 AND 10;
\end{sql}

数据库中有一个特殊的值: NULL,对他进行过滤的时候,需要特殊的子句 IS NULL。

\begin{sql}
SELECT produce_name FROM produces WHERE prod_price IS NULL;
\end{sql}

\subsubsection{AND/OR 组合操作符}

上述 WHERE 语句比较简单,只能对单一条件进行限定,为了更强的过滤控制,MYSQL 允许给出多个 WHERE 子句: 通过 AND / OR 操作符。

这两个操作符意思非常简单,就和 JAVA 中的 \&\& 和 || 类似。

\begin{sql}
-- AND 操作符
SELECT prod_id, prod_price, prod_name
    FROM products
    WHERE vend_id = 1003 AND prod_price <= 10;
-- OR 操作符
SELECT prod_name, prod_price
    FROM products
    WHERE vend_id = 1002 OR vend_id = 1003;
\end{sql}

类似 JAVA,这两个操作符可以组合使用,组合过程中一定要使用 () 以表示清晰的组合逻辑。

\begin{sql}
SELECT prod_name, prod_price
    FROM products
    WHERE (vend_id = 1002 OR vend_id = 1003) AND prod_price >= 10;
\end{sql}

\subsubsection{IN/NOT 操作符}

IN 与圆括号的组合,可以更方便地指定离散条件:

\begin{sql}
SELECT prod_name, prod_price
    FROM products
    WHERE vend_id IN (1002,1003)
    ORDER BY prod_name;
\end{sql}

可以发现,IN 和 OR 操作符的作用是相同的,但 IN 有以下几个优点:
\begin{itemize}
    \item IN 操作符更直观,计算次序更容易管理。
    \item IN 操作符一般比 OR 操作符执行速度更快。
    \item IN 可以包含其他 SELECT 语句。
\end{itemize}

NOT 操作符只有一个功能,否定后面跟的任何条件,类似于 Java 中的 !。

\begin{sql}
SELECT prod_name, prod_price
    FROM products
    WHERE vend_id NOT IN (1002,1003)
    ORDER BY prod_name;
\end{sql}

\subsubsection{LIKE 操作符}

\textbf{通配符}: 用来匹配值的一部分的特殊字符。

前面介绍的所有操作符都是针对已知值进行过滤的。LIKE 通配符类似于简单的正则表达式,可以进行模糊匹配。

LIKE 操作符主要有两个常用的通配符:
\begin{itemize}
    \item \%: 匹配任何字符出现任意次数(0个也算)。
    \item \_: 匹配单个字符。
\end{itemize}

\% 通配符的使用:
\begin{sql}
-- 产品名以 jet 开头的产品数据
SELECT prod_id, prod_name
    FROM products
    WHERE prod_name LIKE 'jet%';
-- 产品名中包含 anvil 字符串的产品数据
SELECT prod_id, prod_name
    FROM products
    WHERE prod_name LIKE '%anvil%';
\end{sql}

虽然 \% 通配符可以匹配任何东西,但有一个例外,即NULL。

\_ 通配符的使用:

\begin{sql}
SELECT prod_id, prod_name
    FROM products
    WHERE prod_name LIKE '_ ton anvil';
\end{sql}

尽管通配符非常好用,但它的搜索速度远不及之前提到的几种过滤方式。而且将通配符放在最前面的搜索速度是极慢的。

\subsubsection{正则表达式}

上文说过,LIKE 类似于简单的正则表达式,这里介绍正则表达式在 MySQL 中的使用方式。正则表达式的语法比较复杂,MySQL 也只实现了一部分的正则匹配,完整的正则表达式语法请参考其他文章。

在 WHERE 子句中使用正则表达式的关键字为 REGEXP(Regular Expression)。注意 LIKE 和 REGEX 的区别,LIKE 匹配的是整个文本,REGEX 匹配的是文本中的值。

\begin{sql}
-- 文本包含 1000 的所有行
SELECT prod_name
    FROM products
    WHERE prod_name REGEXP '1000'
    ORDER BY prod_name;
-- 文本为 1000 的所有行
SELECT prod_name
    FROM products
    WHERE prod_name LIKE '1000'
    ORDER BY prod_name;
\end{sql}

当然,使用 \^  \$ 定位符,REGEX 可以实现 LIKE 的效果。

下面再给出几个例子:
\begin{sql}
-- OR 匹配: 文本中出现 1000 或 2000 的数据
SELECT prod_name
    FROM products
    WHERE prod_name REGEXP '1000|2000'
    ORDER BY prod_name;
-- [] 出现中括号内的字符
SELECT prod_name
    FROM products
    WHERE prod_name REGEXP '[123] Ton'
    ORDER BY prod_name;
\end{sql}

类似的用法还有很多,需要对正则表达式有一定了解。

此外,正则表达式会使用一些特定字符,比如 ?。如果我们要查询这些字符,使用两个反斜杠进行转义即可:

\begin{sql}
SELECT vend_name
    FROM vendors
    WHERE vend_name REGEXP '\\?'
    ORDER BY vend_name;
\end{sql}

\subsection{计算字段}

我们取出的数据往往不是数据库中单独的某些数据段,例如我们需要取出一个姓名+国家的数据字段,而数据库中只有这两个单独的数据段。一种方式是将数据字段发送给程序,让程序进行处理。但 MySQL 也为我们提供了字段拼接服务。

\subsubsection{Concat 拼接字段}

Concat() 函数可以将多列拼接起来,它的用法和 Java 函数十分相似,参数为列名或字符串。

\begin{sql}
SELECT Concat(user, ' (', host, ')') FROM mysql.user;
\end{sql}

观察上述语句的返回情况,会发现列名就是我们的 Concat 函数,我们可以用别名来指定我们所需要的名字。

\begin{sql}
SELECT Concat(user, ' (', host, ')') AS title
    FROM mysql.user;
\end{sql}

\subsubsection{算数运算}

MySQL 支持最基本的加减乘除运算(+,-,*,/), 下面是一个例子:

\begin{sql}
SELECT prod_id,
    quantity,
    item_price,
    quantity*item_price AS expanded_price
FROM orderitems
WHERE order_num = 20005;
\end{sql}

\subsection{数据处理函数}

函数可以帮我我们对数据进行处理,本章不深入讨论函数,仅给出一些常用的数据处理函数及例子。

SQL 中的函数按作用可以分为如下几类:
\begin{itemize}
    \item 处理文本串的文本函数,即处理字符串的。
    \item 处理数值数据,进行算数操作。
    \item 处理日期和时间并从这些值中提取特定部分。
    \item 返回 DBMS 正使用的特殊信息。
\end{itemize}

函数的使用非常简单,像 Java 一样传参即可。

\subsubsection{文本处理函数}

举一个简单的例子,Upper() 函数将字符串转换为大写:
\begin{sql}
SELECT host, Upper(host) FROM mysql.user;
\end{sql}

常见文本处理函数:

\begin{table}[H]
    \centering
    \caption{常见文本处理函数}
    \label{table:常见文本处理函数}
    \setlength{\tabcolsep}{4mm}
    \begin{tabular}{cc|cc}
        \toprule
        \textbf{函数} & \textbf{说明} & \textbf{函数} & \textbf{说明} \\
        \midrule
        Left() & 返回串左边的字符 & Right() & 返回串右边的字符 \\
        Length() & 返回串的长度 & Locate() & 找出串的一个字串 \\
        Lower() & 转换为小写 & Upper() & 转换为大写 \\
        LTrim() & 去掉串左边的空格 & RTrim() & 去掉串右边的空格 \\
        Soundex() & 返回 Soundex 值 & SubString() & 返回字串的字符 \\
        \bottomrule
    \end{tabular}
\end{table}

这些函数的具体用法,可以使用 help 语句查询。

\subsubsection{日期和时间处理函数}

日期和时间采用相应的数据类型和特殊的格式存储,以便能快速和有效地排序或过滤,并且节省物理存储空间。

\begin{sql}
SELECT Date(last_update) from mysql.server_cost;
\end{sql}

常见日期和时间处理函数:

\begin{table}[H]
    \centering
    \caption{常见日期和时间处理函数}
    \label{table:常见日期和时间处理函数}
    \setlength{\tabcolsep}{4mm}
    \begin{tabular}{cc|cc}
        \toprule
        \textbf{函数} & \textbf{说明} & \textbf{函数} & \textbf{说明} \\
        \midrule
        AddDate() & 增加一天 & AddTime() & 增加一个时间 \\
        CurDate() & 返回当前日期 & CurTime() & 返回当前时间 \\
        Date() & 返回日期时间的日期部分 & DateDiff() & 计算两个日期之差 \\
        Date\_Add() & 高度灵活的日期运算函数 & DayofWeek() & 返回日期对应的星期 \\
        Date\_Format() & 格式化的日期或时间串 & Now() & 返回当前日期和时间 \\
        \midrule
        \multicolumn{3}{l}{Day(), Hour(), Minute(), Month(), Second(), Time(), Year()} & 返回对应的日期或时间部分 \\
        \bottomrule
    \end{tabular}
\end{table}

\subsubsection{数值处理函数}

这个没什么好说的,和 Java 中的 Math 类似。

常见数值处理函数:

\begin{table}[H]
    \centering
    \caption{常见数值处理函数}
    \label{table:常见数值处理函数}
    \setlength{\tabcolsep}{4mm}
    \begin{tabular}{cc|cc}
        \toprule
        \textbf{函数} & \textbf{说明} & \textbf{函数} & \textbf{说明} \\
        \midrule
        Abs() & 返回一个数的绝对值 & Cos() & 返回余弦值 \\
        Exp() & 返回一个数的指数值 & Mod() & 返回取余操作 \\
        Pi() & 返回圆周率 & Rand() & 返回一个随机数 \\
        Sin() & 返回正弦值 & Sqrt() & 返回数的平方根 \\
        Tan() & 返回正切值 \\
        \bottomrule
    \end{tabular}
\end{table}

\subsubsection{聚集函数}

前面介绍了几个检索与处理函数,聚集函数用于汇总数据,便于分析和报表生成。聚集函数有以下几种:

\begin{itemize}
    \item 确定表中的行数。
    \item 获得表中行组的和。
    \item 找出表列的最大值,最小值,平均值。
\end{itemize}

聚集函数运行在行组上,计算和返回单个值的函数。常见聚集函数:

\begin{table}[H]
    \centering
    \caption{常见聚集函数}
    \label{table:常见聚集函数}
    \setlength{\tabcolsep}{4mm}
    \begin{tabular}{cc|cc}
        \toprule
        \textbf{函数} & \textbf{说明} & \textbf{函数} & \textbf{说明} \\
        \midrule
        AVG() & 返回某列平均值 & COUNT() & 返回某列行数 \\
        MAX() & 返回某列最大值 & MIN() & 返回某列最小值 \\
        SUM() & 返回某列之和 \\
        \bottomrule
    \end{tabular}
\end{table}

这些函数和 EXCEL 的函数类似,没什么好说的,部分函数可以加上 DISTINCT 等关键字,详情使用 help 语句查看。下面给出一个例子:

\begin{sql}
SELECT COUNT(*) AS num_items,
    MIN(prod_price) AS price_min,
    AVG(DISTINCT prod_price) AS price_avg
FROM produces;
\end{sql}

\subsection{数据分组}

分组允许把数据分为多个逻辑组,以便能对每个组进行聚集计算。

\subsubsection{GROUP BY 创建分组}

GROUP BY 的使用方式:
\begin{sql}
SELECT object_type , Count(*) 
    FROM performance_schema.setup_objects 
    GROUP BY object_type; 
\end{sql}

上述例子中,计算结果会按 object\_type 进行分组,也可以理解为计算各个 object\_type 包含的数据行数。

GROUP BY 子句有一些规定:
\begin{itemize}
    \item GROUP BY 子句可以包含任意数目的列。这使得能对分组进行嵌套,为数据分组提供更细致的控制。
    \item 如果在GROUP BY 子句中嵌套了分组,数据将在最后规定的分组上进行汇总。
    \item GROUP BY 子句中列出的每个列都必须是检索列或有效的表达式(但不能是聚集函数)。如果在SELECT 中使用表达式,则必须在GROUP BY 子句中指定相同的表达式。不能使用别名。
    \item 除聚集计算语句外,SELECT 语句中的每个列都必须在 GROUP BY 子句中给出。
    \item 如果分组列中具有 NULL 值,则 NULL 将作为一个分组返回。如果列中有多行NULL 值,它们将分为一组。
    \item GROUP BY 子句必须出现在WHERE 子句之后,ORDER BY 子句之前。
\end{itemize}

使用WITH ROLLUP 关键字,可以得到每个分组以及每个分组汇总级别(针对每个分组)的值。

\subsubsection{HAVING 过滤分组}

我们前面已经使用过 WHERE 进行过滤,但是 WHERE 子句只能对行进行过滤,事实上,WHERE 子句没有分组的概念。

HAVING 子句为我们提供了组过滤的作用,HAVING 非常类似于 WHERE,事实上 HAVING 完全可以替代 WHERE 子句。唯一的区别是 WHERE 用于过滤行,HAVING 用来过滤组。

\fbox{
    \parbox{0.87\textwidth}{
        \begin{notice}
            HAVING 支持所有 WHERE 操作符,它们的句法是相同的,只是关键字有差别。WHERE 在数据分组前进行过滤,HAVING 在分组后进行过滤。
        \end{notice}
    }
}

执行下面语句:
\begin{sql}
SELECT plugin_version as version,
    Count(*) as count
FROM information_schema.plugins
GROUP BY plugin_version
HAVING Count(*) > 3;
\end{sql}

我们使用 HAVING Count(*) > 3 子句仅返回了数量大于 3 的分组。

\subsection{阶段小结}

目前位置,SELECT 语句的基本语法就结束了,下面总结一下各个子句的顺序以及用处。

\begin{table}[H]
    \centering
    \caption{SELECT 子句及其顺序}
    \label{table:SELECT 子句及其顺序}
    \setlength{\tabcolsep}{4mm}
    \begin{tabular}{c|c|cc}
        \toprule
        \textbf{子句} & \textbf{说明} & \textbf{是否必须} \\
        \midrule
        SELECT & 要返回的列或表达式 & 是 \\
        FROM & 从中检索数据的表 & 仅在从表选择数据时使用 \\
        WHERE & 行级过滤器 & 否 \\
        GROUP BY & 分组说明 & 仅在按组计算聚集时使用 \\
        HAVING & 组级过滤 & 否 \\
        ORDER BY & 输出排序顺序 & 否 \\
        LIMIT & 要检索的行数 & 否 \\
        \bottomrule
    \end{tabular}
\end{table}

\subsection{子查询}

子查询,即嵌套在其他查询中的查询。将查询结果作为上一级查询的目标。在多表查询中经常使用。

\subsubsection{利用子查询进行过滤}

考虑下面一种情况: 需要列出订购物品 TNT2 的所有客户,按照之前的方法,需要有以下步骤:
\begin{itemize}
    \item 检索包含物品TNT2 的所有订单的编号;
    \item 检索具有前一步骤列出的订单编号的所有客户的ID;
    \item 检索前一步骤返回的所有客户ID的客户信息。
\end{itemize}

具体实现脚本为:
\begin{sql}
-- 查编号,然后记下
SELECT order_num FROM orderitems WHERE prod_id = 'TNT2';
-- 查编号对应的 id
SELECT cust_id FROM orders WHERE order_num IN (20005,20007);
\end{sql}

现在我们使用子查询,合并上面的步骤:

\begin{sql}
SELECT cust_id FROM orders
    WHERE order_num IN (SELECT order_num 
                        FROM orderitems
                        WHERE prod_id = 'TNT2');
\end{sql}

子查询总是从内向外处理。在处理上面的SELECT 语句时,MySQL实际上执行了两个操作: 先子查询,再查询。

MySQL 并没有限制子查询的嵌套数量,不过为了性能考虑,不建议使用过深的嵌套。

\subsubsection{作为计算字段使用子查询}

使用子查询的另一方法是创建计算字段。假如需要显示customers 表中每个客户的订单总数。订单与相应的客户ID存储在orders 表中。

为了执行这个操作,需要执行以下步骤:
\begin{itemize}
    \item 从 customers 表中检索客户列表。
    \item 对于检索出的每个客户,统计其在 orders 表中的订单数目。
\end{itemize}

\begin{sql}
SELECT cust_name, cust_state,
    (SELECT COUNT(*) FROM orders WHERE orders.cust_id = customers.cust_id) AS orders
    FROM customers
    ORDER BY cust_name;
\end{sql}

这条语句中 orders 是一个计算字段,它是由圆括号中的子查询建立的。该子查询对检索出的每个客户执行一次。在此例子中,该子查询执行了5次,因为检索出了5个客户。

\fbox{
    \parbox{0.87\textwidth}{
        \begin{notice}
            \textbf{相关子查询}: 涉及外部查询的子查询。任何时候只要列名可能有多义性,就必须使用这种语法(表名和列名由一个句点分隔)。
        \end{notice}
    }
}

\subsection{联结表}

自本节开始,将讨论多表查询/检索的语法。

\subsubsection{理论}

在数据库设计中,为了避免数据重复,我们往往将不同类型的数据区分在不同的表中。关系表的设计就是要保证把信息分解成多个表,一类数据一个表。各表通过某些常用的值(即关系设计中的关系 (relational))互相关联。

外键(foreignkey)为某个表中的一列,它包含另一个表的主键值,定义了两个表之间的关系。

总之,关系数据可以有效地存储和方便地处理。因此,关系数据库的可伸缩性远比非关系数据库要好。

如果数据存储在多个表中,怎样用单条SELECT语句检索出数据?

答案是使用联结。简单地说,联结是一种机制,用来在一条SELECT 语句中关联表,因此称之为联结。使用特殊的语法,可以联结多个表返回一组输出,联结在运行时关联表中正确的行。

\subsubsection{创建联结}

联结的创建非常简单,规定要联结的所有表以及它们如何关联即可。

\begin{sql}
SELECT vend_name, prod_name, prod_price
    FROM vendors, products
    WHERE vendors.vend_id = products.vend_id
    ORDER BY vend_name, prod_name;
\end{sql}

上述代码中,所指定的两个列(prod\_name 和prod\_price )在一个表中,而另一个列(vend\_name )在另一个表中。在 FROM 语句中出现了两个表,分别是vendors 和products 。它们就是这条SELECT 语句联结的两个表的名字。这两个表用WHERE 子句正确联结,WHERE 子句指示MySQL匹配vendors 表中的vend\_id 和products 表中的vend\_id 。

在一条 SELECT 语句中联结几个表时,相应的关系是在运行中构造的。在联结两个表时,你实际上做的是将第一个表中的每一行与第二个表中的每一行配对。WHERE 子句作为过滤条件,它只包含那些匹配给定条件(这里是联结条件)的行。没有WHERE 子句,第一个表中的每个行将与第二个表中的每个行配对,而不管它们逻辑上是否可以配在一起。

如果没有 WHERE 子句,将返回笛卡尔积: 检索出的行的数目将是第一个表中的行数乘以第二个表中的行数。可以尝试下面语句:

\begin{sql}
SELECT vend_name, prod_name, prod_price
    FROM vendors, products
    ORDER BY vend_name, prod_name;
\end{sql}

应该保证所有联结都有WHERE 子句,否则MySQL将返回比想要的数据多得多的数据。同理,应该保证WHERE 子句的正确性。不正确的过滤条件将导致MySQL返回不正确的数据。

\subsubsection{INNER JOIN 等值/内部联结}

目前为止所用的联结称为等值联结 (equijoin),它基于两个表之间的相等测试(可以理解为两个集合的交集)。这种联结也称为内部联结。其实,对于这种联结可以使用稍微不同的语法来明确指定联结的类型。下面的SELECT 语句返回与前面例子完全相同的数据:

\begin{sql}
SELECT vend_name, prod_name, prod_price
    FROM vendors INNER JOIN products
    ON vendors.vend_id = products.vend_id;
\end{sql}

此语句使用 INNER JOIN 指定内部联结,ON 指定联结条件。本质上和 WHERE 子句效果相同。

与自联结相对应的是交叉联结 CROSS JOIN, 他将返回笛卡尔积。

\fbox{
    \parbox{0.87\textwidth}{
        \begin{notice}
            虽然两种写法达到的目的一致,但 ANSI SQL 规范首选 INNER JOIN 语法。此外,尽管使用WHERE 子句定义联结的确比较简单,但是使用明确的联结语法能够确保不会忘记联结条件,有时候这样做也能影响性能。
        \end{notice}
    }
}

\subsubsection{NATURAL JOIN 自然联结}

自然连接是一种特殊的等值连接,它要求两个关系中进行比较的分量必须是相同的属性组,并且在结果中把重复的属性列去掉。可以理解为等值联结 + 删除重复列。

\subsubsection{联结多个表}

SQL对一条SELECT 语句中可以联结的表的数目没有限制。创建联结的基本规则也相同。首先列出所有表,然后定义表之间的关系。例如:

\begin{sql}
SELECT prod_name, vend_name, prod_price, quantity
    FROM orderitems, products, vendors
    WHERE products.vend_id = vendors.vend_id
        AND orderitems.prod_id = products.prod_id
        AND order_num = 20005;
\end{sql}

\fbox{
    \parbox{0.87\textwidth}{
        \begin{notice}
            MySQL在运行时关联指定的每个表以处理联结。这种处理可能是非常耗费资源的,因此应该仔细,不要联结不必要的表。联结的表越多,性能下降越厉害。
        \end{notice}
    }
}

再看之前的子查询,下面两个语句作用相同,而多表联结显然可读性更高:

\begin{sql}
-- 子查询
SELECT cust_name, cust_contact
FROM customers
WHERE cust_id IN (SELECT cust_id
                  FROM orders
                  WHERE order_num IN (SELECT order_num
                                      FROM orderitems
                                      WHERE prod_id = 'TNT2'));
-- 多表联结
SELECT cust_name, cust_contact
FROM customers, orders, orderitems
WHERE customers.cust_id = orders.cust_id
    AND orderitems.order_num = orders.order_num
    AND prod_id = 'TNT2';
\end{sql}

\subsubsection*{表别名}

为了缩短 SQL 语句,可以对表使用表别名,句法和列一样:
\begin{sql}
SELECT cust_name, cust_contact
    FROM customers AS c, orders AS o, orderitems AS oi
    WHERE c.cust_id = o.cust_id;
\end{sql}

注意表别名不会返回到客户机上。

\subsubsection{自联结}

使用表别名的主要原因之一是能在单条SELECT 语句中不止一次引用相同的表。因此,我们可以对同一张表引用两次,看作不同的表。

现在考虑这样一种情况,我们要找到生产ID为 DTNTR 的物品的供应商,然后找出这个供应商生产的其他物品。下面两种写法等价:

\begin{sql}
-- 子查询
SELECT prod_id, prod_name
FROM products
WHERE vend_id = (SELECT vend_id
                 FROM products
                 WHERE prod_id = 'DTNTR');
-- 自联结
SELECT p1.prod_id, p1.prod_name
FROM products AS p1, products AS p2
WHERE p1.vend_id = p2.vend_id
  AND p2.prod_id = 'DTNTR';
\end{sql}

此查询中需要的两个表实际上是相同的表,因此 products 表在FROM 子句中出现了两次。虽然这是完全合法的,但对products 的引用具有二义性,因为MySQL不知道你引用的是products 表中的哪个实例。

为解决此问题,使用了表别名。products 的第一次出现为别名p1 ,第二次出现为别名p2 。现在可以将这些别名用作表名。例如,SELECT 语句使用p1 前缀明确地给出所需列的全名。如果不这样,MySQL将返回错误,因为分别存在两个名为prod\_id 、prod\_name 的列。MySQL不知道想要的是哪一个列(即使它们事实上是同一个列)。WHERE (通过匹配p1 中的vend\_id 和p2 中的vend\_id )首先联结两个表,然后按第二个表中的prod\_id 过滤数据,返回所需的数据。

在自联结中,将 p1, p2 看作不同的表,会更好理解。

\subsubsection{OUTER JOIN 外部联结}

外部联结的关键词是 OUTER JOIN, 使用外联结必须指定左或右外联结。左联结和右联结本质上是一样的,颠倒顺序而已。下面以左联结为例:

\begin{sql}
SELECT * FROM A LEFT JOIN B ON A.aID = B.bID;
\end{sql}

左外联结是以左表为基础的,即left join是以左表为基础的,在这个例子中,左表(表A)的记录全部会显示出来,而表B只显示符合过滤条件的那部分行。

\subsubsection{小结}

MySQL 必知必会讲联结有点乱,这里整理一下。我们使用 HELP JOIN; 语句可以查找出和表联结相关的语法:

\begin{sql}
HELP JOIN;
-- joined_table: {
--     table_reference {[INNER | CROSS] JOIN | STRAIGHT_JOIN} table_factor [join_specification]
--   | table_reference {LEFT|RIGHT} [OUTER] JOIN table_reference join_specification
--   | table_reference NATURAL [INNER | {LEFT|RIGHT} [OUTER]] JOIN table_factor
-- }
\end{sql}

其中 STRAIGHT\_JOIN 属于进阶语法,大概作用为使用左表驱动右表,提高检索性能,仅适用于内联结。

总的来说,表联结有以下几个注意点:
\begin{itemize}
    \item NATURAL 关键字用于删除重复列。
    \item INNER JOIN / CROSS JOIN: 一般无需显示说明 CROSS JOIN,如果没有联结条件,默认返回笛卡尔积,否则返回等值连接。
    \item INNER / OUTER JOIN: 等值联结取的是并集,外联结取的是本身 + 并集。
\end{itemize}

\subsection{组合查询}

\fbox{
    \parbox{0.87\textwidth}{
        \begin{notice}
            组合查询与使用 WHERE 条件的效果往往是一样的,使用那种技术取决于脚本可读性,性能等。
        \end{notice}
    }
}

\subsubsection{UNION 组合结果}

使用 UNION 可以将多个 SELECT 语句的结果合并,取交集并自动删除重复的行。下面条语句效果是一样的:

\begin{sql}
-- WHERE
SELECT vend_id, prod_id, prod_price
    FROM products
    WHERE prod_price <= 5
       OR vend_id IN (1001,1002);
-- UNION
SELECT vend_id, prod_id, prod_price
    FROM products
    WHERE prod_price <= 5
UNION
SELECT vend_id, prod_id, prod_price
    FROM products
    WHERE vend_id IN (1001,1002);
\end{sql}

如果我们不想删除重复的行,可以使用 UNION ALL 关键字。

UNION 关键字的使用规则如下:
\begin{itemize}
    \item UNION 必须由两条或以上 SELECT 语句组成,语句之间用关键字 UNION 分隔。
    \item UNION 中每个查询必须包括相同的列,表达式或聚集函数。不然没法合并。
    \item 列数据必须兼容,不然没法转换比较。
\end{itemize}

使用了 UNION 关键字之后,只能在最后使用一次 ORDER BY 进行排序。提示一下,没啥好解释的。

\subsection{全文本搜索}

\fbox{
    \parbox{0.87\textwidth}{
        \begin{notice}
            \textbf{并非所有引擎都支持全文本搜索}: MySQL 的两个最常用引擎: MyISAM 和 InnoDB。后者不支持全文本搜索。但一般都使用 InnoDB 引擎。
        \end{notice}
    }
}

\subsubsection{启用全文本搜索}

一般在创建表时启用全文本搜索,必须指定 FULLTEXT 子句和引擎:
\begin{sql}
CREATE TABLE productnotes (
  note_id    int           NOT NULL AUTO_INCREMENT,
  prod_id    char(10)      NOT NULL,
  note_date datetime       NOT NULL,
  note_text  text          NULL ,
  PRIMARY KEY(note_id),
  FULLTEXT(note_text)
) ENGINE=MyISAM;
\end{sql}

现在,只需知道这条CREATE TABLE 语句定义表productnotes 并列出它所包含的列即可。这些列中有一个名为note\_text 的列,为了进行全文本搜索,MySQL根据子句FULLTEXT(note\_text) 的指示对它进行索引。这里的FULLTEXT 索引单个列,如果需要也可以指定多个列。

\subsubsection{进行全文本搜索}

在索引之后,使用两个函数 Match() 和 Against() 执行全文本搜索:
\begin{itemize}
    \item Match() 指定被搜索的列, 传递给Match() 的值必须与FULLTEXT() 定义中的相同。如果指定多个列,则必须列出它们(而且次序正确)。
    \item Against() 指定要使用的搜索表达式
\end{itemize}

下面两个语句效果类似:

\begin{sql}
-- LIKE 语句
SELECT note_text
    FROM productnotes
    WHERE note_text LIKE '%rabbit%';
-- 全文本搜索
SELECT note_text
    FROM productnotes
    WHERE Match(note_text) Against('rabbit');
\end{sql}

此SELECT 语句检索单个列note\_text 。由于WHERE 子句,一个全文本搜索被执行。Match(note\_text) 指示MySQL针对指定的列进行搜索,Against('rabbit') 指定词rabbit 作为搜索文本。

上面两个语句返回结果内容相同,但顺序不一致,全文本搜索的一个重要部分就是对结果排序。具有较高等级的行先返回。且由于全文本搜索是索引的,速度相当快。

全文本搜索还有许多内容,但由于 InnoDB 引擎不支持,这里不做过多说明,详细参考这篇文章: \url{https://zhuanlan.zhihu.com/p/146361883}

\subsection{INSERT 语句}

\subsubsection{插入完整行}

INSERT 的常用语法结构如下:
\begin{sql}
INSERT INTO <tableName[(columnName)]>
VALUES <(value...)>, <(value...)>;
\end{sql}

最简单的 INSERT 向表中插入完整的一行:

\begin{sql}
INSERT INTO Customers
    VALUES(NULL, 'Pep E. LaPew', '100 Main Street', 'Los Angeles', 'CA', '90046', 'USA', NULL, NULL);
\end{sql}

虽然这种语法很简单,但并不安全,应该尽量避免使用。上面的SQL语句高度依赖于表中列的定义次序,并且还依赖于其次序容易获得的信息。

编写INSERT 语句的更安全(不过更烦琐)的方法如下:

\begin{sql}
INSERT INTO customers(cust_name, cust_address, cust_city, cust_state, cust_zip, cust_country, cust_contact, cust_email)
VALUES('Pep E. LaPew', '100 Main Street', 'Los Angeles', 'CA', '90046', 'USA', NULL, NULL);
\end{sql}

因为提供了列名,VALUES 必须以其指定的次序匹配指定的列名,不一定按各个列出现在实际表中的次序。

使用这种语法,还可以省略列。这表示可以只给某些列提供值,给其他列不提供值。省略的列必须满足以下某个条件。

\begin{itemize}
    \item 该列定义为允许NULL 值(无值或空值)。
    \item 在表定义中给出默认值。这表示如果不给出值,将使用默认值。
\end{itemize}

如果对表中不允许NULL 值且没有默认值的列不给出值,则MySQL将产生一条错误消息,并且相应的行插入不成功。

\subsubsection{插入多行}

插入多行可以用多个上述语句,或者在 VALUES 中使用 , 隔开不同行的数据:

\begin{sql}
INSERT INTO customers(cust_name, cust_address, cust_city, cust_state, cust_zip, cust_country)
VALUES ( 'Pep E. LaPew', '100 Main Street', 'Los Angeles', 'CA', '90046', 'USA'),
       ( 'M. Martian', '42 Galaxy Way', 'New York', 'NY', '11213', 'USA');
\end{sql}

\subsubsection{插入检索出的数据}

INSERT 还存在另一种形式,可以利用它将一条SELECT 语句的结果插入表中。这就是所谓的INSERT SELECT。

\begin{sql}
INSERT INTO customers(cust_id, cust_contact, cust_email, cust_name, cust_address, cust_city, cust_state, cust_zip, cust_country)
SELECT cust_id, cust_contact, cust_email, cust_name, cust_address, cust_city, cust_state, cust_zip, cust_country
FROM custnew;
\end{sql}

\subsection{UPDATE 语句}

\subsubsection{更新数据}

UPDATE 语句非常简单,由三部分组成:
\begin{itemize}
    \item 需要更新的表;
    \item 列名和它的新值;
    \item WHERE 过滤条件。
\end{itemize}

UPDATE 语句的常用语法如下:
\begin{sql}
UPDATE <tableName>
    SET <column = newValue>
    WHERE <condition>;
\end{sql}

一个简单的例子如下:
\begin{sql}
UPDATE customers
    SET cust_name = 'The Fudds', cust_email = 'elmer@fudd.com'
    WHERE cust_id = 10005;
\end{sql}

注意 WHERE 子句十分重要,如果缺失了 WHERE 语句,将对对应列的所有值进行修改。

\subsection{DELETE 语句}

\subsubsection{删除数据}

和更新数据类似,删除数据可以删除所有行或特定行,取决于 WHERE 子句。删除一行的例子如下:

\begin{sql}
DELETE FROM customers
    WHERE cust_id = 10006;
\end{sql}

注意,DELETE 是删除一整个行,因此不需要列名,如果需要删除某列的数据,应使用 UPDATE 语句。

\newpage