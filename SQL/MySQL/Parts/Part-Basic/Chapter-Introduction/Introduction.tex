\section{MySQL 简介}

\subsection{安装}

MySQL 有两种安装方式,一种是线上安装,一种是安装包安装。线上安装比较适合对 MySQL 熟悉的用户,可以自定义安装一些应用功能,但如果一次安装错了,后果很严重(亲身体会)。这里建议新手使用安装包安装的方式。MySQL8 社区版的安装可以参考这篇文章: \url{https://blog.csdn.net/weixin_43605266/article/details/110477391}。

在 Window 上安装好 MySQL8 后,我们通过终端 \texttt{mysql} 命令进入数据库,可能会报如下错误:
\begin{bash}
ERROR 1045 (28000): Access denied for user 'ODBC'@'localhost' (using password: NO)
\end{bash}

这是因为 Windows 默认进入 SQL 的用户为 ODBC,但我们没有创建 ODBC 用户,使用如下指令进入 MySQL 即可:

\begin{bash}
mysql -u <userName> -p
\end{bash}

注意安装好后,本地服务器名: localhost (基本所有数据库系统的本地数据库服务器都是这个名字),端口 3306 (MySQL 默认端口号)。

\subsection{常用命令}

为了更好的操作 MySQL,这里提前给出一些常用的命令。首先是在终端操作 MySQL 的一些命令:

\begin{itemize}
    \item mysql 服务器的启用与关闭(Windows 默认开启)。
\begin{bash}
net stop mysql  # 停止 mysql 服务
net start mysql  # 启用 mysql 服务
\end{bash}
    \item 登录 mysql 服务器。如果是远程链接,需要参数 -h。
\begin{bash}
mysql [-h <hostIP>] -u <userName> -p [<password>]
\end{bash}
\end{itemize}

其次是 MySQL 内部的几个常用命令(注意 mysql 所有命令都需要 ; 结尾,sql 语句不区分大小写,但关键字一般使用大写)。

\begin{itemize}
    \item 增加新用户。
\begin{sql}
CREATE USER '<userName>'@'LOCALHOST' IDENTIFIED BY '<password>';
\end{sql}
    \item 修改用户密码。
\begin{sql}
ALTER USER '<userName>'@'LOCALHOST' IDENTIFIED WITH MYSQL_NATIVE_PASSWORD BY '<newPassword>';
\end{sql}
    \item 显示所有数据库。
\begin{sql}
show databases;
\end{sql}
    \item 显示用户信息(注意 mysql 会对用户密码进行加密,所以无法获取密码,只能获取 user,host 等信息)。
\begin{sql}
SELECT <infos> FROM mysql.user;
\end{sql}
\end{itemize}

\subsection{GUI 应用}

MySQL 本体的 GUI 工具不是很好用,推荐两个好用的数据库 GUI 软件:

\begin{itemize}
    \item Navicat: 主流的 GUI 工具,公司一般都会用这个,费用昂贵。
    \begin{itemize}
        \item 官网: \url{https://navicat.com.cn/}
    \end{itemize}
    \item Dbeaver: 社区版免费,功能齐全,但有点卡。
    \begin{itemize}
        \item 官网: \url{https://dbeaver.io/}
    \end{itemize}
    \item VSCode-MySQL 插件: 全宇宙最好用的文本编辑器和最好的生态。
\end{itemize}

不使用这些辅助工具也完全没问题,只是在终端中写命令不是很方便。

\newpage