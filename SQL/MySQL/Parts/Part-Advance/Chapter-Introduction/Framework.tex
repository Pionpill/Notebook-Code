\section{MySQL 架构}
\subsection{MySQL逻辑架构}

MySQL 的基本逻辑架构如下,其中最上层的服务是大多数基于网络的客户端/服务器公有的。第二层包含了 MySQL 的核心服务功能。第三层则包含了存储引擎。:

\begin{figure}[H]
    \centering
    \begin{tikzpicture}[rectangle, text centered]
        \node (0) at (0,1.5) {客户端};
        \draw (-2,0.5) rectangle (2,-3.5);
        \node [draw, rectangle] (1) at (0,0) {连接/线程处理};
        \node [draw, rectangle, text width=2em] (2) at (-1,-1.5) {查询缓存};
        \node [draw, rectangle, text width=3em] (3) at (1,-1.5) {解析器};
        \node [draw, rectangle, text width=6em] (4) at (0,-3) {优化器};
        \node (5) at (0,-4) {存储引擎};
        \begin{scope}
            \draw[-Stealth] (0) -- (1);
            \draw[-Stealth] (1) -- (2);
            \draw[-Stealth] (1) -- (3);
            \draw[-Stealth] (3) -- (2);
            \draw[-Stealth] (3) -- (4);
        \end{scope}
    \end{tikzpicture}
    \caption{MySQL 服务器逻辑架构图}
    \label{fig:MySQL 服务器逻辑架构图}
\end{figure}

\subsubsection*{连接管理与安全性}

每个客户端连接都会在服务器进程中拥有一个线程,这个连接的查询只会在这个单独的线程中执行,该线程只能轮流在某个CPU核心或者CPU中运行。服务器会负责缓存线程。

当客户端(应用)连接到MySQL服务器时,服务器需要对其进行认证。认证基于用户名、原始主机信息和密码。如果使用了安全套接字(SSL)的方式连接,还可以使用X.509证书认证。一旦客户端连接成功,服务器会继续验证该客户端是否具有执行某个特定查询的权限。

\subsubsection*{优化与执行}

MySQL会解析查询,并创建内部数据结构(解析树),然后对其进行各种优化,包括重写查询、决定表的读取顺序,以及选择合适的索引等。

优化器并不关心表使用的是什么存储引擎,但存储引擎对于优化查询是有影响的。优化器会请求存储引擎提供容量或某个具体操作的开销信息,以及表数据的统计信息等。

对于SELECT语句,在解析查询之前,服务器会先检查查询缓存(Query Cache),如果能够在其中找到对应的查询,服务器就不必再执行查询解析、优化和执行的整个过程,而是直接返回查询缓存中的结果集。

\subsection{并发控制}

但如果两个进程在同一时刻对同一个邮箱投递邮件,会发生什么情况?显然,邮箱的数据会被破坏,两封邮件的内容会交叉地附加在邮箱文件的末尾。设计良好的邮箱投递系统会通过锁(lock)来防止数据损坏。如果客户试图投递邮件,而邮箱已经被其他客户锁住,那就必须等待,直到锁释放才能进行投递。

这种锁的方案在实际应用环境中虽然工作良好,但并不支持并发处理。因为在任意一个时刻,只有一个进程可以修改邮箱的数据,这在大容量的邮箱系统中是个问题。

\subsubsection*{读写锁}

在处理并发读或者写时,可以通过实现一个由两种类型的锁组成的锁系统来解决数据安全问题。这两种类型的锁通常被称为共享锁(shared lock)和排他锁(exclusive lock),也叫读锁(read lock)和写锁(write lock)。

读锁是共享的,或者说是相互不阻塞的。多个客户在同一时刻可以同时读取同一个资源,而互不干扰。写锁则是排他的,也就是说一个写锁会阻塞其他的写锁和读锁,这是出于安全策略的考虑,只有这样,才能确保在给定的时间里,只有一个用户能执行写入,并防止其他用户读取正在写入的同一资源。

\subsubsection*{粒度锁}

一种提高共享资源并发性的方式就是让锁定对象更有选择性。尽量只锁定需要修改的部分数据,而不是所有的资源。更理想的方式是,只对会修改的数据片进行精确的锁定。任何时候,在给定的资源上,锁定的数据量越少,则系统的并发程度越高,只要相互之间不发生冲突即可。

所谓的锁策略,就是在锁的开销和数据的安全性之间寻求平衡,这种平衡当然也会影响到性能。大多数商业数据库系统没有提供更多的选择,一般都是在表上施加行级锁(row-level lock),并以各种复杂的方式来实现,以便在锁比较多的情况下尽可能地提供更好的性能。

而MySQL则提供了多种选择。每种MySQL存储引擎都可以实现自己的锁策略和锁粒度。MySQL 有两种重要的锁策略:表锁,行级锁。
\begin{itemize}
    \item 表锁:表锁是MySQL中最基本的锁策略,并且是开销最小的策略。写锁也比读锁有更高的优先级,因此一个写锁请求可能会被插入到读锁队列的前面
    \item 行级锁:行级锁只在存储引擎层实现,而MySQL服务器层没有实现。
\end{itemize}

\subsection{事务}

事务就是一组原子性的SQL查询,或者说一个独立的工作单元。事务内的语句,要么全部执行成功,要么全部执行失败。

一个运行良好的事务处理系统,必须具备这些标准特征(ACIS)。
\begin{itemize}
    \item 原子性(atomicity): 一个事务必须被视为一个不可分割的最小工作单元,整个事务中的所有操作要么全部提交成功,要么全部失败回滚,对于一个事务来说,不可能只执行其中的一部分操作。
    \item 一致性(consistency): 数据库总是从一个一致性的状态转换到另外一个一致性的状态。例如某一部执行失败,只要事务没有提交,就没有改变对应的状态。
    \item 隔离性(isolation): 一个事务所做的修改在最终提交以前,对其他事务是不可见的。
    \item 持久性(durability): 一旦事务提交,则其所做的修改就会永久保存到数据库中。此时即使系统崩溃,修改的数据也不会丢失。
\end{itemize}

\subsubsection*{隔离级别}

在SQL标准中定义了四种隔离级别,每一种级别都规定了一个事务中所做的修改,哪些在事务内和事务间是可见的,哪些是不可见的。较低级别的隔离通常可以执行更高的并发,系统的开销也更低。

\begin{itemize}
    \item READ UNCOMMITTED(未提交): 事务中的修改,即使没有提交,对其他事务也都是可见的。事务可以读取未提交的数据,这也被称为脏读。这个级别会导致很多问题,从性能上来说,READ UNCOMMITTED不会比其他的级别好太多,但却缺乏其他级别的很多好处,除非真的有非常必要的理由,在实际应用中一般很少使用。
    \item READ COMMITTED(提交读): 大多数数据库系统的默认隔离级别都是READ COMMITTED(但MySQL不是)。READ COMMITTED满足前面提到的隔离性的简单定义:一个事务开始时,只能“看见”已经提交的事务所做的修改。换句话说,一个事务从开始直到提交之前,所做的任何修改对其他事务都是不可见的。
    \item REPEATABLE READ(可重复读): 解决了脏读的问题。该级别保证了在同一个事务中多次读取同样记录的结果是一致的。但是理论上,可重复读隔离级别还是无法解决另外一个幻读问题。所谓幻读,指的是当某个事务在读取某个范围内的记录时,另外一个事务又在该范围内插入了新的记录,当之前的事务再次读取该范围的记录时,会产生幻行。InnoDB和XtraDB存储引擎通过多版本并发控制解决了幻读问题。
    \item SERIALIZABLE(可串行化): SERIALIZABLE是最高的隔离级别。它通过强制事务串行执行,避免了前面说的幻读的问题。简单来说,SERIALIZABLE会在读取的每一行数据上都加锁,所以可能导致大量的超时和锁争用的问题。实际应用中也很少用到这个隔离级别,只有在非常需要确保数据的一致性而且可以接受没有并发的情况下,才考虑采用该级别。
\end{itemize}

\subsubsection*{死锁}

死锁是指两个或者多个事务在同一资源上相互占用,并请求锁定对方占用的资源,从而导致恶性循环的现象。当多个事务试图以不同的顺序锁定资源时,就可能会产生死锁。多个事务同时锁定同一个资源时,也会产生死锁。

\begin{sql}
START TRANSACTION;
UPDATE StockPrice SET close = 45.50 WHERE stock_id = 4 AND date = '2022-05-01';
UPDATE StockPrice SET close = 19.80 WHERE stock_id = 3 AND date = '2022-05-02';
COMMIT;
    
START TRANSACTION;
UPDATE StockPrice SET high = 20.12 WHERE stock_id = 3 AND date = '2022-05-02';
UPDATE StockPrice SET high = 47.20 WHERE stock_id = 4 AND date = '2022-05-01';
COMMIT;
\end{sql}

如果这两个事务都执行了第一条语句,那么就至少会锁定对应的行数据,接着执行第二条语句发现对应的资源被锁定,双方都无法获取资源,就陷入了死锁循环。

为了解决这种问题,数据库系统实现了各种死锁检测和死锁超时机制。越复杂的系统,比如InnoDB存储引擎,越能检测到死锁的循环依赖,并立即返回一个错误。这种解决方式很有效,否则死锁会导致出现非常慢的查询。还有一种解决方式,就是当查询的时间达到锁等待超时的设定后放弃锁请求,这种方式通常来说不太好。InnoDB目前处理死锁的方法是,将持有最少行级排他锁的事务进行回滚。

死锁发生以后,只有部分或者完全回滚其中一个事务,才能打破死锁。对于事务型的系统,这是无法避免的,所以应用程序在设计时必须考虑如何处理死锁。大多数情况下只需要重新执行因死锁回滚的事务即可。

\subsubsection*{事务日志}

事务日志可以帮助提高事务的效率。使用事务日志,存储引擎在修改表的数据时只需要修改其内存拷贝,再把该修改行为记录到持久在硬盘上的事务日志中,而不用每次都将修改的数据本身持久到磁盘。

\subsubsection*{MySQL 中的事务}

MySQL提供了两种事务型的存储引擎:InnoDB和NDB Cluster。另外还有一些第三方存储引擎也支持事务,比较知名的包括XtraDB和PBXT。

MySQL默认采用自动提交(AUTOCOMMIT)模式。也就是说,如果不是显式地开始一个事务,则每个查询都被当作一个事务执行提交操作。在当前连接中,可以通过设置AUTOCOMMIT变量来启用或者禁用自动提交模式

\begin{sql}
SET AUTOCOMMIT = 1;
\end{sql}

当AUTOCOMMIT=0时,所有的查询都是在一个事务中,直到显式地执行COMMIT提交或者ROLLBACK回滚,该事务结束,同时又开始了另一个新事务。

另外还有一些命令,在执行之前会强制执行COMMIT提交当前的活动事务。典型的例子,在数据定义语言(DDL)中,如果是会导致大量数据改变的操作,比如ALTER TABLE,就是如此。另外还有LOCK TABLES等其他语句也会导致同样的结果。

MySQL可以通过执行SET TRANSACTION ISOLATION LEVEL命令来设置隔离级别。

\subsection{多版本并发控制}

MySQL的大多数事务型存储引擎实现的都不是简单的行级锁。基于提升并发性能的考虑,它们一般都同时实现了多版本并发控制(MVCC)。

可以认为MVCC是行级锁的一个变种,但是它在很多情况下避免了加锁操作,因此开销更低。虽然实现机制有所不同,但大都实现了非阻塞的读操作,写操作也只锁定必要的行。

MVCC的实现,是通过保存数据在某个时间点的快照来实现的。也就是说,不管需要执行多长时间,每个事务看到的数据都是一致的。MySQL 通过保存 Undo Redo 日志的方式,使得多个事务均可以获取正确的数据。

前面说到不同存储引擎的MVCC实现是不同的,典型的有乐观(optimistic)并发控制和悲观(pessimistic)并发控制。

InnoDB的MVCC,是通过在每行记录后面保存两个隐藏的列来实现的。这两个列,一个保存了行的创建时间,一个保存行的过期时间(或删除时间)。当然存储的并不是实际的时间值,而是系统版本号(system version number)。每开始一个新的事务,系统版本号都会自动递增。事务开始时刻的系统版本号会作为事务的版本号,用来和查询到的每行记录的版本号进行比较。下面看一下在REPEATABLE READ隔离级别下,MVCC具体是如何操作的。

\begin{itemize}
    \item \textbf{SELECT} InnoDB会根据以下两个条件检查每行记录:
    \begin{itemize}
        \item InnoDB只查找版本早于当前事务版本的数据行,这样可以确保事务读取的行,要么是在事务开始前已经存在的,要么是事务自身插入或者修改过的。
        \item 行的删除版本要么未定义,要么大于当前事务版本号。这可以确保事务读取到的行,在事务开始之前未被删除。
    \end{itemize}
    \item \textbf{INSERT} InnoDB为新插入的每一行保存当前系统版本号作为行版本号。
    \item \textbf{DELETE} InnoDB为删除的每一行保存当前系统版本号作为行删除标识。
    \item \textbf{UPDATE} InnoDB为插入一行新记录,保存当前系统版本号作为行版本号,同时保存当前系统版本号到原来的行作为行删除标识。
\end{itemize}

MVCC只在REPEATABLE READ和READ COMMITTED两个隔离级别下工作。

\newpage