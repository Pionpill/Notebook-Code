\documentclass{PionpillNote-book}
\usetikzlibrary {intersections,through,arrows.meta,graphs,shapes.misc,positioning,shapes.misc,positioning,calc}
\usetikzlibrary{animations}
\usetikzlibrary {shapes.geometric}
\usetikzlibrary {animations}
\usetikzlibrary {shapes.multipart}
\usetikzlibrary {positioning}
\usetikzlibrary {fit,shapes.geometric}
\usetikzlibrary {automata}
\usetikzlibrary {quotes}
\usetikzlibrary {matrix}
\usetikzlibrary {backgrounds}
\usetikzlibrary {scopes}
\usetikzlibrary {calc}
\usetikzlibrary {intersections}
\usetikzlibrary {svg.path}
\usetikzlibrary {decorations}
\usetikzlibrary {patterns}
\usetikzlibrary {decorations.pathmorphing}
\usetikzlibrary {shadows}
\usetikzlibrary {bending}

\title{MySQL 笔记}
\author{
    Pionpill \footnote{笔名:北岸,电子邮件:673486387@qq.com,Github:\url{https://github.com/Pionpill}} \\
    本文档为作者学习 MySQL 时的笔记。\\
}

\date{\today}

\begin{document}

\pagestyle{plain}
\maketitle

\noindent\textbf{前言:}

笔者为软件工程系在校本科生,有计算机学科理论基础(操作系统,数据结构,计算机网络,编译原理等),本人在撰写此笔记时已经学过基础的数据库概念,类似于表,行,主键等概念不再赘述。

在指令中出现的 <> 表示必要参数,[] 表示可选参数。一般的,做前端开发或者小厂开发看完基础部分即可。进阶部分适合中大厂的开发人员以及高级开发人员。

本文分为以下几个部分:
\begin{itemize}
    \item MySQL 基础: 主要参考 《MySQL 必知必会》\footnote{《MySQL Crash Course》: [英] Ben Forta 2009 年第一版。}一书,快速过一遍 MySQL 基础语法,一些样例数据下载方式如下:
    \begin{itemize}
        \item \url{https://forta.com/wp-content/uploads/books/0672327120/mysql_scripts.zip}
    \end{itemize}
    \item MySQL 进阶: 主要参考 《高性能 MySQL》\footnote{High Performance MySQL(3rd): Baron Schwartz, Peter Zaitsev,Vadim Tkachenko. 2013年第三版}。讲解了内部原理及优化方法。
    \begin{itemize}
        \item 这是一本面向 DBA 的专业书籍,部分和后端无关的内容被我省略了。
    \end{itemize}
\end{itemize}

本人的编写及开发环境如下:
\begin{itemize}
    \item OS: Windows11 
    \item MySQL: 8.0.3
\end{itemize}

本文默认使用 MySQL5 及以上版本,远古版本不再提及。

\date{\today}
\newpage

\tableofcontents

\newpage

\setcounter{page}{1} 
\pagestyle{fancy}

\part{MySQL 基础}
\chapter{介绍与基本操作}
\import{Parts/Part-Basic/Chapter-Introduction}{Introduction.tex}
\import{Parts/Part-Basic/Chapter-Introduction}{Database.tex}
\chapter{基础数据操作语言}
\import{Parts/Part-Basic/Chapter-Basic}{DML.tex}
\import{Parts/Part-Basic/Chapter-Basic}{DDL.tex}
\import{Parts/Part-Basic/Chapter-Basic}{DCL.tex}

\part{MySQL 进阶}
\chapter{深入了解 MySQL}
\import{Parts/Part-Advance/Chapter-Introduction}{Framework.tex}
\import{Parts/Part-Advance/Chapter-Introduction}{PerformanceSchema.tex}
\chapter{优化 MySQL}
\import{Parts/Part-Advance/Chapter-Optimization}{Schema.tex}
\import{Parts/Part-Advance/Chapter-Optimization}{Index.tex}

\end{document}