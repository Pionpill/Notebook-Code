\documentclass{PionpillNote-book}

\import{chapters/style}{tikz-style.tex}

\title{Fluent Python 笔记}
\author{
    Pionpill \footnote{笔名:北岸,电子邮件:673486387@qq.com,Github:\url{https://github.com/Pionpill}} \\
    本文档为作者学习《Fluent Python》\footnote{《Fluent Python》:Luciano Ramalho 2017年中文第一版}一书时的笔记。\\
}

\date{\today}

\begin{document}

\pagestyle{plain}
\maketitle

\noindent\textbf{前言:}

笔者为软件工程系在校本科生,主要利用 Python 进行数据科学与机器学习使用,也有一定的开发经验。

<<Fluent Python>> 是  Python 学习的进阶书籍,在学习本书之前,本人已拜读过 <<Python 从入门到实践>>\footnote{俗称:蟒蛇书},<<Automate with Python>> 等书。

书上有大量的示例,在本文中也有给出,脚本位于 Scripts 文件下,但本人只书写了部分脚本,还有小部分本人觉得没有必要,或者是以命令行形式书写,这些并没有给出。已有的脚本推荐读者使用 VSCode 和 Python Preview 插件,便于查看脚本运行时的内部逻辑。此外,原文有大段对 Python3\footnote{原书支持到 Python3.4} 和 Python2 的对比,除非特别必要,本人不会再详述 Python2 \footnote{特指 Python2.7 ,再远古的版本不再提及}的相关内容,除非特殊说明,默认适用于 Python3 和 CPython 解释器。

本笔记不能代替原书,仅是对原书的一个总结归纳;原书大段精妙的解释均没有被记录在笔记中。本人极其推荐有一定 Python 基础的人购买原书阅读,直到截稿日期,我都认为本书是我在学习 Python 路线上阅读过最好的书籍之一。

本笔记只是对原书的马虎概括与整理,如有疑问或需求,还请购买原书; 本文引用了一些 CSDN 或其他论讨的文章,如果原作者觉得不合适,请联系本人。

语言环境:
\begin{itemize}
    \item OS: Window11
    \item Python: Python 3.8
\end{itemize}

\date{\today}
\tableofcontents
\newpage

\setcounter{page}{1} 
\pagestyle{fancy}

\chapter{序幕}
\import{Chapters/Chapter-Introduction}{Introduction.tex}

\chapter{数据结构}
\import{Chapters/Chapter-DataStruct}{List.tex}
\import{Chapters/Chapter-DataStruct}{Dictionary.tex}
\import{Chapters/Chapter-DataStruct}{Byte.tex}

\chapter{把函数视作对象}
\import{Chapters/Chapter-Function}{Function.tex}
\import{Chapters/Chapter-Function}{DesignPatterns.tex}
\import{Chapters/Chapter-Function}{Closure.tex}

\chapter{面向对象惯用法}
\import{Chapters/Chapter-Class}{Reference.tex}
\import{Chapters/Chapter-Class}{Style.tex}
\import{Chapters/Chapter-Class}{Vector.tex}
\import{Chapters/Chapter-Class}{Interface.tex}
\import{Chapters/Chapter-Class}{Inherit.tex}
\import{Chapters/Chapter-Class}{OperatorOverloading.tex}

\chapter{控制流程}
\import{Chapters/Chapter-ControlFlow}{Iterator.tex}
\import{Chapters/Chapter-ControlFlow}{ContextManager.tex}

\end{document}