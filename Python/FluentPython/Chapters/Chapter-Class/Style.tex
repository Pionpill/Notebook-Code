\section{符合 Python 风格的对象}
\subsection{对象表示形式}

每门面向对象的语言至少含有一种获取对象的字符串表示形式和标准方法。Python 提供了两种方法。
\begin{itemize}
    \item \texttt{repr()} \\
    便于开发者理解的方法,返回对象的字符串表示形式。
    \item \texttt{str()} \\
    便于用户理解的方法,返回对象的字符串表示形式。默认 \texttt{print()} 打印对象时调用,如果 \texttt{\_\_str\_\_} 没有实现,则调用 \texttt{\_\_repr\_\_} 代替。
\end{itemize}

为了给对象提供其他的表示形式,还会用到另外两个特殊方法:\texttt{\_\_bytes\_\_} 和 \texttt{\_\_format\_\_} 。\texttt{\_\_bytes\_\_} 与 \texttt{\_\_str\_\_} 类似:\texttt{bytes()} 函数调用它获取对象的字符序列表示形式。而 {\_\_format\_\_} 方法会被内置的 \texttt{format()} 函数和 \texttt{str.format()} 方法调用,使用特殊的格式代码显示对象的字符串表示形式。

\subsection{再谈向量类}
下面我们构建一个 \texttt{Vector2d} 类来说明对象表示形式,说明均在代码中给出。

\lstinputlisting[language=Python]{../../Scripts/Class/9-1.py}

\subsection{备选构造方法}

我们可以把 \texttt{Vector2d} 实例转换成字节序列,同理,也应能从字节序列转换成 \texttt{Vector2d} 实例。在标准库中 \texttt{array.array} 有个类方法 \texttt{.frombytes} 符合要求。下面为 \texttt{Vector2d} 定义一个同名类方法。

\begin{python}
# 例9-3:定义备选构造方法
@classmethod    # 类方法装饰器,下一节将说明
def frombytes(cls,octets):  # 不传入 self 参数,相反通过 cls 传入类本身
    typecode = char(octets[0])  # 从第一个字节中读取 typecode
    memv = memoryview(octets[1:]).cast(typecode)    # 使用传入的 octets 字节序列创建一个 memoryview,然后使用 typecode 转换
    return cls(*memv)       # 拆包转换后的 memoryview,得到构造方法所需的一对参数
\end{python}

\subsection{\texttt{classmethod} 和 \texttt{staticmethod}}

\texttt{classmethod} 用于定义操作类,而不是操作实例的方法。\texttt{classmethod} 改变了调用方法的方法,因此类方法的第一个参数是类本身(\texttt{cls}),而不是实例(\texttt{self})。\texttt{classmethod}最常见的用途是定义备选构造方法。例如上一个例子 \texttt{frombytes}。注意,\texttt{frombytes}的最后一行使用 \texttt{cls} 参数构建了一个新实例,即 \texttt{cls(*memv)}。按照约定,类方法的第一个参数名为 \texttt{cls}。

\texttt{staticmethod} 装饰器也会改变方法的调用方法,但是第一个参数不是特殊的值。其实,静态方法就是普通的方法,只是碰巧定义在类中,而不是在模块里。

\lstinputlisting[language=Python]{../../Scripts/Class/9-4.py}

\texttt{classmethod} 装饰器的作用毋庸置疑,而 \texttt{staticmethod} 装饰器的作用却显得相形见绌,部分人认为 \texttt{staticmethod} 并不是十分必要\footnote{原作者这样认为}。

\subsection{格式化显示}

内置的 \texttt{format()} 函数和 \texttt{str.format()} 方法把各个类型的格式化方法委托给相应的 \texttt{.\_\_format\_\_} 方法。 \texttt{format\_spec} 是格式说明符,它是:

\begin{itemize}
    \item \texttt{format(my\_obj,format\_spec)} 的第二个参数,或者
    \item \texttt{str.format()} 方法的格式字符串,\texttt{\{\}} 里代换字段中冒号后面的部分
\end{itemize}

\begin{python}
>>> brl = 1/2.43
>>> brl
0.4115226337448559
>>> format(brl,'0.4f')      # 说明符是 0.4f
'0.4115'
>>> '1BRL = {rate:0.2f} USD'.format(rate=brl)   # 说明符是 0.2f,代换字段中的 'rate' 子串是字段名称
'1BRL = 0.41 USD'
\end{python}

上例第 6 行指出了一个重要的知识点,'{0.mass:5.3e}' 这样的格式字符串其实包括两部分,冒号左边的 '0.mass' 在代换字段句法中式字段名,冒号后面的 '5.3e' 式格式说明符。格式说明符使用的表示法叫做格式规范微语言。

格式规范微语言为一些内置类型提供了专门的表示代码。比如,\texttt{b} 和 \texttt{x} 分别表示二进制和十六进制的 \texttt{int} 类型,\texttt{f} 表示小数形式的 \texttt{float} 类型。而 \% 表示百分数形式:

\begin{python}
>>> format(42,'b')
'101010'
>>> format(2/3,'.1%')
'66.7%'
\end{python}

格式规范微语言式可扩展的,因为各个类可以自行决定如何解释 \texttt{format\_spec} 参数。例如,\texttt{datetime} 模块中的类,它们的 \texttt{\_\_format\_\_} 方法使用的格式代码与 \texttt{strftime()} 函数一样。下面是几个例子。

\begin{python}
>>> from datetime import datetime
>>> now = datetime.now()
>>> format(now,'%H:%M:%S')
'18:49:05'
>>> "It is now {:%I:%M %p}".format(now)
"It's now 06:49 PM"
\end{python}

如果类没有定义 \texttt{\_\_format\_\_} 方法,从 \texttt{object} 继承的方法会返回 \texttt{str(my\_object)}。我们为 \texttt{Vector2d} 类定义了 \texttt{\_\_str\_\_} 方法,因此可以这样做:

\begin{python}
>>> v1 = Vector2d(3,4)
>>> format(v1)
'(3.0,4.0)'
\end{python}

然而,如果传入格式说明符,\texttt{object.\_\_format\_\_} 方法会抛出 TypeError:
\begin{python}
>>> format(v1,'.3f')
TypeError:......
\end{python}

我们将实现自己的微语言来解决这个问题。首先,假设用户提供的格式说明符式用于格式化向量中的各个浮点数分量的。我们想达到的效果是:

\begin{python}
>>> v1 = Vector2d(3,4)
>>> format(v1)
'(3.0, 4.0)'
>>> format(v1,'.2f')
'(3.00, 4.00)'
>>> format(v1,'.3e')
'(3.000e+00, 4.000e+00)'
\end{python}

实现这种输出的 \texttt{\_\_format\_\_} 方法如下例所示:
\begin{python}
def __format__(self, fmt_spec=''):
    components = (format(c,fmt_spec) for c in self)
    return '({},{})'.format(*component)
\end{python}

下面要在微语言中添加一个自定义的格式代码:如果格式说明符以 'p' 结尾,那么在极坐标中显示向量,即 <r,$\theta$>。

下面定义一个计算 \texttt{angle} 的方法。
\begin{python}
def angle(self):
    return math.atan2(self.y,self.x)
\end{python}

这样便可以增强 \texttt{\_\_format\_\_} 方法,计算极坐标。

\begin{python}
def __format__(self, fmt_spec='') -> str:
if fmt_spec.endswith('p'):      # 极坐标系坐标
    fmt_spec = fmt_spec[:-1]    # 删除 p 后缀
    coords = (abs(self),self.angle())
    outer_fmt = '<{},{}>'
else:                           # 平面直角坐标系坐标
    coords = self
    outer_fmt = '({},{})'
components = (format(c,fmt_spec) for c in coords)   # 使用各个分量生成可迭代的对象,构成格式化字符串
return outer_fmt.format(*components)    # 把格式化字符串带入外层格式
\end{python}

\subsection{可散列的 \texttt{Vector2d}}

按照目前的定义,\texttt{Vector2d} 是不可散列的,应为没有实现 \texttt{\_\_hash\_\_} 方法( \texttt{\_\_eq\_\_} 方法已实现 )。此外,还要让向量不可变。

为了让向量不可变,需要将变量变成私有的,其方法是在变量前加两个 \_。

\lstinputlisting[language=Python]{../../Scripts/Class/9-7.py}

让这些向量不可变才能实现 \texttt{\_\_hash\_\_} 方法。这个方法应该返回一个整数,理想情况下还要考虑对象属性的散列值(\texttt{\_\_eq\_\_} 方法也要使用),因为相等的对象应该具有相等的散列值。

根据特殊方法 \\texttt{\_\_hash\_\_} 的文档,最好使用位运算符异或混合各分量的散列值。

\begin{python}
def __hash__(self) -> int:
    return hash(self.x) ^ hash(self.y)
\end{python}

添加了 \texttt{\_\_hash\_\_} 方法之后,向量变成可散列的了。

\begin{python}
>>> v1 = Vector2d(3,4)
>>> v2 = Vector2d(3.1,4.2)
>>> hash(v1)
7
>>> hash(v2)
38430......
\end{python}

要创建可散列的类型,不一定要实现特性,也不一定要保护实例属性。只需正确地实现 \texttt{\_\_hash\_\_} 和 \texttt{\_\_eq\_\_} 方法。但是,实例的散列值绝不应该变化,因此我们借机提到了只读属性。

\subsection{私有属性和受保护属性}

Python 没有像 Java 那样使用 \texttt{private} 修饰符创建的私有属性,但它有个简单的机制能够避免子类意外覆盖 ``私有'' 属性。

首先,我们要将属性设置为私有属性的形式,即前面加两个下划线,尾部没有或至多一个下划线。Python 会将此类属性名存入实例的 \texttt{\_\_dict\_\_} 属性中,而且会在前面加上一个下划线和类名,用于区分父类和子类的属性。这个语言特性叫名称改写。

以之前定义的 \texttt{Vector2d} 为例:

\begin{python}
>>> v1 = Vector2d(3,4)
>>> v1.__dict__
{'_Vector2d__y':4.0, '_Vector2d__x':3.0}
>>> v1._Vector2d__x
3.0
\end{python}

名称改写是一种安全措施,不能保证万无一失:他的目的是避免意外访问,不能防止故意做错事。

例如上一个例子中,只要编写 \texttt{v1.\_Vector2d\_\_x = 7} 这样的代码,就能轻松地位 \texttt{Vector2d} 实例地私有分量直接赋值。

不是所有人都喜欢名称改写功能。也不是所有人都喜欢 \texttt{self.\_\_x} 这种名称。对于不喜欢这种句法的人,他们约定使用一个下划线前缀编写 ``受保护'' 的属性。而其他人认为应该使用命名约定来避免意外覆盖属性。

Python 解释器不会到使用单个下划线的属性名做特殊处理,不过这是很多 Python 程序员的约定,他们不会在类外部访问这种属性。Python 文档的一些角落将这种写法的属性称为``受保护''的属性,不过并不是所有人都这样叫。

\subsection{使用 \texttt{\_\_slots\_\_} 类属性节省空间}

默认情况下,Python在各个实例中名为 \texttt{\_\_dict\_\_} 的字典里存储实例属性。这会大大提高访问速度,但字典会消耗大量内存。如果要处理属性不多的大量实例,则可通过 \texttt{\_\_slots\_\_} 类属性,能节省大量内存空间,方法是让解释器在元组中存储实例属性,而不是字典。

定义 \texttt{\_\_slots\_\_} 的方式是:创建一个类属性,使用 \texttt{\_\_slots\_\_} 这个名字,并把它的值设为一个字符串构成的可迭代对象,其中各个元素表示各个实例属性。原作者倾向于使用元组,因为这样定义的 \texttt{\_\_slots\_\_} 中含有的信息不会变化。

\begin{python}
class Vector2d:
    __slots__ = ('__x','__y')

    # 各个方法的实现省略
\end{python}

在类中定义 \texttt{\_\_slots\_\_} 属性的目的是告诉解释器:这个类中所有的实例属性都在这了。这样 Python 会用类似元组的结构存储实例变量。此外,在类中定义 \texttt{\_\_slots\_\_} 属性后,实例不能再有 \texttt{\_\_slots\_\_} 中所列名称之外的其他属性。

在作者的实验中,创建了 10,000,000 个 \texttt{Vector2d} 实例,使用 \texttt{\_\_dict\_\_} 属性时,RAM 用量最高为 1.5GB。而使用 \texttt{\_\_slots\_\_} 则为 655 MB。

还有一种神奇的用法,如果将 \texttt{\_\_dict\_\_} 这个名称加到 \texttt{\_\_slots\_\_ } 中,实例会在元组中保存各个实例的属性,此外还支持动态创建属性,这些属性存储在常规的 \texttt{\_\_dict\_\_}。但是这违背了 \texttt{\_\_slots\_\_} 设计的初衷(节省内存)。

\noindent\textbf{\texttt{\_\_slots\_\_} 的问题}

如果使用得当,\texttt{\_\_slots\_\_} 能显著节省内存,不过有几点要注意。
\begin{itemize}
    \item 每个子类都要定义 \texttt{\_\_slots\_\_} 属性,因为解释器会忽略继承的 \texttt{\_\_slots\_\_} 属性。
    \item 实例只能拥有 \texttt{\_\_slots\_\_} 中列出的属性。
    \item 如果不把 `\texttt{\_\_weakref\_\_}' 加入 \texttt{\_\_slots\_\_} ,实例就不能作为弱引用的目标。
\end{itemize}

\subsection{覆盖类属性}

Python 有个独特的特性:类属性可用于为实例属性提供默认值。 \texttt{Vector2d} 中有个 \texttt{typecode} 类属性, \texttt{\_\_bytes\_\_} 方法两次用到了它,而且故意使用 \texttt{self.typecode} 读取它的值。因为 \texttt{Vector2d} 实例本身没有 \texttt{typecode} 属性,所以 \texttt{self.typecode} 默认获取的是 \texttt{Vector2d.typecode} 类属性的值。

但是,如果为不存在的实例属性赋值,会新建实例属性。假如我们为 \texttt{typecode} 实例属性赋值,那么同名类属性不受影响。然而,自此以后,实例读取的 \texttt{self.typecode} 是实例属性 \texttt{typecode},也就是把同名类属性遮盖了。借助这一特性,可以为各个实例的 \texttt{typecode} 属性定制不同的值。

如果想修改类属性的值,必须直接在类上修改,不能通过实例修改。如果要修改所有实例(没有 \texttt{typecode} 实例变量)的 \texttt{typecode} 属性的默认值,可以这样做:

\begin{python}
>>> Vector2d.typecode = 'f'
\end{python}

然而,有种修改方法更符合 Python 风格,而且效果持久,也更具针对性。类属性是公开的,因此会被子类继承,以实经常会创建一个子类,只用于定制类的数据属性。具体做法如下:

\begin{python}
class ShortVector2d(Vector2d):
    typecode = 'f'
\end{python}

这也说明了在 \texttt{Vector2d.\_\_repr\_\_} 方法中为什么没有硬编码 \texttt{class\_name} 的值,而是使用 \texttt{type(self).\_\_name\_\_} 获取。

\begin{python}
# 在 Vector2d 类中定义
def __repr__(self):
    class_name = type(self).__name__
    return '{}({!r},{!r})'.format(class_name. *self)
\end{python}

如果硬编码 \texttt{class\_name} 的值,那么 \texttt{Vector2d} 的子类要覆盖 \texttt{\_\_repr\_\_} 方法,只是为了修改 \texttt{class\_name} 的值。从实例的类型中读取类名,\texttt{\_\_repr\_\_} 方法就可以放心继承。

\newpage