\section{正确重载运算符}

本章只讨论一元运算符和中缀运算符。\footnote{其他运算符如函数调用,属性访问等大多不可重载,用户也不能修改用法。}

\subsection{运算符重载基础}

在某些圈子中,运算符重载的名声并不好。比如 Java 禁止用户重载运算符。对此 Python 施加了一些限制,做好了灵活性,可用性和安全性方面的平衡:

\begin{itemize}
    \item 不能重载内置类型的运算符
    \item 不能新建运算符,只能重载现有的运算符
    \item 某些运算符不能重载: \texttt{is, and, or, not}
\end{itemize}

\subsection{一元运算符}

在 Python 语言参考手册中,列出了三个一元运算符。下面是这三个运算符和对应的特殊方法:

\begin{itemize}
    \item \texttt{- (\_\_neq\_\_)}
    
    一元负数运算符。如果 \texttt{x} 是 \texttt{-2},那么 \texttt{-x == 2}
    \item \texttt{+ (\_\_pos\_\_)}
    
    一元取正算术运算符。通常\footnote{例外:\texttt{Decimal} 大整数运算,\texttt{collections.Counter} 计算运算。原书 P309 有说明。} \texttt{x == +2}
    \item \texttt{$\sim$(\_\_invert\_\_)}
    
    对整数按位取反,定义 \texttt{$\sim$x == -(x+1)}
\end{itemize}

除此之外,还有 \texttt{abs()} 函数也被列为了一元运算符。前面已多次提及。

支持一元运算符非常简单,只需实现相应的特殊方法。这些特殊方法只有一个参数:\texttt{self}。不过,要遵守运算符的一个基本规则:始终返回一个新对象。也就是说,不能修改 \texttt{self},要创建并且返回新的实例。

下面给出了之前定义的 \texttt{Vector} 类实现的几个新运算符:

\begin{python}
def __abs__(self):
    return math.sqrt(sum(x*x for x in self))

def __neg__(self):
    return Vector(-x for x in self)

def __pos__(self):
    return Vector(self)     # 注意:即使相同,也要构建新的实例。
\end{python}

\subsection{重载向量加法运算符 +}

两个欧几里得向量加在一起得到的是一个新的向量,它的各个分量是两个向量中相应的分量之和,如果长度不同,则应对部分元素进行相加:

下面在之前定义过的 \texttt{Vector} 类基础上添加加法运算:

\begin{python}
def __add__(self, other):
    pairs = itertools.zip_longest(self, other, fillvalue = 0.0) # 生成一个 (a,b) 形式的元组
    return Vector(a + b for a, b in pairs)  # 返回新的实例
\end{python}

这里我们使用了 \texttt{zip\_longest} 函数,它能处理任何可迭代对象,但仅当左操作数是 \texttt{Vector} 实例时才有效。

\begin{python}
>>> v1 = Vector([3,4,5])
>>> v1 + (10, 20, 30)
Vector([13.0, 24.0, 35.0])
>>> (10, 20, 30) + v1
Traceback (most recent call last):.....
\end{python}

为了支持涉及不同类型的运算,Python 为中缀运算符特殊方法提供了特殊的分派机制。对 \texttt{a+b} 来说,解释器会提供以下几步操作:

\begin{itemize}
    \item 如果 \texttt{a} 有 \texttt{\_\_add\_\_} 方法,而且返回值不是 \texttt{NotImplemented},调用 \texttt{a.\_\_add\_\_(b)},然后返回结果。
    \item 如果 \texttt{a} 没有 \texttt{\_\_add\_\_} 方法,或者返回值是 \texttt{NotImplemented},检查 \texttt{b} 有没有 \texttt{\_\_radd\_\_} 方法,如果有,而且返回值不是 \texttt{NotImplemented},调用 \texttt{b.\_\_radd\_\_(a)},然后返回结果。
    \item 如果 \texttt{b} 没有 \texttt{\_\_radd\_\_} 方法,或者返回值是 \texttt{NotImplemented},抛出 \texttt{TypeError}。
\end{itemize}

\texttt{\_\_radd\_\_} 是 \texttt{\_\_add\_\_} 的 ``反射'',``右向'' 版本。以 \texttt{r} 开头的内置方法多为类似功能。

因此,如果我们要实现 \texttt{Vecotr} 实例的 \texttt{(10, 20, 30) + v1} 方法,只需实现 \texttt{\_\_radd\_\_} 内置函数即可。

\begin{python}
def __add__(self, other):
    pairs = itertools.zip_longest(self, other, fillvalue = 0.0)
    return Vector(a + b for a, b in pairs)

def __radd__(self, other):
    return self + other     # 直接委托给 __add__
\end{python}

在实现了右向运算后,我们又遇到一个问题:如果提供的对象不可迭代,那么 \texttt{\_\_add\_\_} 就无法处理,此时如果我们强行运算,就会抛出一堆难以理解的错误。对此 Python 有一个规定:如果由于类型不兼容而导致运算符特殊方法无法返回有效的结果,那么应该返回 \texttt{NotImplemented},而不是抛出 \texttt{TypeError}。返回 \texttt{NotImplemented} 时,另一个操作数所属的类型还有机会执行运算,即 Python 会尝试调用反向方法。

下面是 \texttt{Vector} 加法运算的最终版:

\begin{python}
def __add__(self, other):
    try:
        pairs = itertools.zip_longest(self, other, fillvalue = 0.0)
        return Vector(a+b for a,b in pairs)
    except TypeError:
        return NotImplemented
\end{python}

\subsection{重载标量乘法运算符 *}

在 \texttt{Vector} 实例中使用乘法运算:\texttt{Vector([1, 2, 3]) * x}。如果 \texttt{x} 是数字,就是计算标量积,结果是一个新 Vector 实例,各个分量都会乘以 \texttt{x} ,这也叫元素级乘法。

涉及乘法运算的还有点积;Numpy 等库目前的做法是,不提供两种意义的 * 运算,而是单独使用 \texttt{dot()} 函数计算点积。

下面我们实现简单的乘法运算:

\begin{python}
def __mul__(self, scalar):
    return Vector(n * scalar for n in self)

def __rmul__(self, scalar):
    return self * scalar
\end{python}

这里有个问题, \texttt{scalar} 必须是实数,我们可以像加法运算那样使用 \texttt{try} 语句,但有个更易于理解的方法。我们将使用 \texttt{isinstance()} 检查 \texttt{scalar} 的类型,但是不硬编码具体的类型,而是检查 \texttt{numbers.Real} 抽象基类的真实子类或虚拟子类的数值类型。

\begin{python}
from array import array
import reprlib
import math
import functools
import operator
import itertools
import numbers      # 为了检查类型导入

class Vector:
    typecode = 'd'

    def __init__(self, components):
        self._components = array(self.typecode, components)

    def __mul__(self, scalar):
        if isinstance(scalar, numbers.Real):        # 检查是否是 numbers.Real 的子类实例
            return Vector(n * scalar for n in self)
        else:
            return NotImplemented

    def __rmul__(self, scalar):
        return self * scalar

    #......
\end{python}

这样实现后, \texttt{Vector} 实例就可以与各种实数进行乘法运算了。

通过 + 和 * 运算,我们了解了中缀表达式的常用模式,其相关的计数如下表所示:

\begin{table}[H]
    \centering
    \caption{中缀运算符方法名}
    \label{table:中缀运算符方法名}
    \setlength{\tabcolsep}{2mm}
    \begin{tabular}{l|llll}
        \toprule
        \textbf{运算符} & \textbf{正向方法} & \textbf{反向方法} & \textbf{就地方法} & \textbf{说明} \\
        \midrule
        \texttt{+}  & \texttt{\_\_add\_\_} & \texttt{\_\_radd\_\_} & \texttt{\_\_iadd\_\_} & 加法或拼接 \\
        \texttt{-}  & \texttt{\_\_sub\_\_} & \texttt{\_\_rsub\_\_} & \texttt{\_\_isub\_\_} & 减法\\
        \texttt{*}  & \texttt{\_\_mul\_\_} & \texttt{\_\_rmul\_\_} & \texttt{\_\_imul\_\_} & 乘法或重复复制\\
        \texttt{/}  & \texttt{\_\_truediv\_\_} & \texttt{\_\_rtruediv\_\_} & \texttt{\_\_itruediv\_\_} & 除法\\
        \texttt{//} & \texttt{\_\_floordiv\_\_} & \texttt{\_\_rfloordiv\_\_} & \texttt{\_\_ifloordiv\_\_} & 整除\\
        \texttt{\%} & \texttt{\_\_mod\_\_} & \texttt{\_\_rmod\_\_} & \texttt{\_\_imod\_\_} & 取余\\
        \texttt{divmod()} & \texttt{\_\_divmod\_\_} & \texttt{\_\_rdivmod\_\_} & \texttt{\_\_idivmod\_\_} & 商和模组成的元组\\
        \texttt{**, pow()} & \texttt{\_\_pow\_\_} & \texttt{\_\_rpow\_\_} & \texttt{\_\_ipow\_\_} & 幂运算\\
        \texttt{@} & \texttt{\_\_matmul\_\_} & \texttt{\_\_rmatmul\_\_} & \texttt{\_\_imatmul\_\_} & 矩阵乘法\\
        \texttt{\&} & \texttt{\_\_and\_\_} & \texttt{\_\_rand\_\_} & \texttt{\_\_iand\_\_} & 位与\\
        \texttt{|} & \texttt{\_\_or\_\_} & \texttt{\_\_ror\_\_} & \texttt{\_\_ior\_\_} & 位或\\
        \texttt{\^} & \texttt{\_\_xor\_\_} & \texttt{\_\_rxor\_\_} & \texttt{\_\_ixor\_\_} & 位异或\\
        \texttt{<<} & \texttt{\_\_lshift\_\_} & \texttt{\_\_rlshift\_\_} & \texttt{\_\_ilshift\_\_} & 按位左移\\
        \texttt{>>} & \texttt{\_\_rshift\_\_} & \texttt{\_\_rrshift\_\_} & \texttt{\_\_irshift\_\_} & 按位右移\\
        \bottomrule
    \end{tabular}

\end{table}

\subsection{众多比较运算符}

Python 解释器对众多比较运算符的处理与前文类似,不过在两个方面又重大区别:

\begin{itemize}
    \item 正向和反向调用使用的是同一系列方法。
    \item 对 \texttt{==} 和 \texttt{!=} 来说,如果反向调用失败,Python 会比较对象的 ID,而不是抛出 \texttt{TypeError}。
\end{itemize}

\begin{table}[H]
    \centering
    \caption{比较运算符方法名}
    \label{table:比较运算符方法名}
    \setlength{\tabcolsep}{4mm}
    \begin{tabular}{c|c|ccc}
        \toprule
        \textbf{分组} & \textbf{运算符} & \textbf{正向方法调用} & \textbf{反向方法调用} & \textbf{后备机制} \\
        \midrule
        相等性 & \texttt{a==b} & \texttt{a.\_\_eq\_\_(b)} & \texttt{b.\_\_eq\_\_(a)} & 返回 \texttt{id(a) == id(b)} \\
         & \texttt{a!=b} & \texttt{a.\_\_ne\_\_(b)} & \texttt{b.\_\_ne\_\_(a)} & 返回 \texttt{not(a == b)}\\
        \midrule
        排序 & \texttt{a>b} & \texttt{a.\_\_gt\_\_(b)} & \texttt{b.\_\_lt\_\_(a)} & 抛出 TypeError\\
         & \texttt{a<b} & \texttt{a.\_\_lt\_\_(b)} & \texttt{b.\_\_gt\_\_(a)} & 抛出 TypeError\\
         & \texttt{a>=b} & \texttt{a.\_\_ge\_\_(b)} & \texttt{b.\_\_le\_\_(a)} & 抛出 TypeError\\
         & \texttt{a<=b} & \texttt{a.\_\_le\_\_(b)} & \texttt{b.\_\_ge\_\_(a)} & 抛出 TypeError\\
        \bottomrule
    \end{tabular}
\end{table}

之前我们是这样实现 \texttt{Vector.\_\_eq\_\_} 方法的:
\begin{python}
def __eq__(self, other):
    return (len(self) == len(other) and all(a == b for a,b in zip(self, other)))
\end{python}

这样做会有一个缺陷:会导致 \texttt{Vector} 实例能与所有的可迭代对象进行比较,究竟是否要这样做,要视情况二定。但对操作数过度宽容会可能导致令人惊讶的结果,而程序员讨厌惊喜。

因此我们应该更保守一点:
\begin{python}
def __eq__(self, other):
    if isinstance(other, self):     # 如果是 Vector 或者其子类实例
        return (len(self) == len(other) and all(for a,b in zip(self, other)))
    else:
        return NotImplemented       # 如果不是,交给反向比较判断
\end{python}

值得说明的是,我们一般不需要自定义 \texttt{!=} 于是暖,因为从 \texttt{object} 继承的 \texttt{\_\_ne\_\_} 方法后备行为满足了我们的需求:定义了 \texttt{\_\_eq\_\_} 方法,而且它不返回 \texttt{NotImplemented},\texttt{\_\_ne\_\_} 方法会对 \texttt{\_\_eq\_\_} 方法返回的结果取反。当然如果对某些特定运算,需要用户自己实现两种方法,而不是简单取反。

\subsection{增量运算符}

在外面实现 \texttt{+,*} 运算后, \texttt{Vector} 类已经支持增量运算符 \texttt{+=,*=} 了:

\begin{python}
>>> v1 = Vector([1,2,3])
>>> v1_alias = v1           # 复制一份,后面审查啊
>>> id(v1)
4302860128
>>> v1 += Vector([4,5,6])
>>> v1                      # 增量加法运算,结果与预期相符
Vector([5.0,7.0,9.0])
>>> id(v1)                  # id 发生了变化,创建了新实例
4302859904
>>> v1_alias                # 原来的 Vector 没有变
Vector([1.0,2.0,3.0])
>>> v1*=11
>>> v1                      # 增量乘法运算,结果与预期相符
Vector([55.0,77.0,99.0])
>>> id(v1)                  # 创建了新实例
4302858336
\end{python}

如果一个类没有实现就地运算,增量赋值运算符只是语法糖: \texttt{a+=b} 的作用与 \texttt{a=a+b} 完全一样。对不可变类型来说,这是预期行为,而且,如果定义了 \texttt{\_\_add\_\_} 方法,就没有必要写格外的代码。

然而,如果实现了就地运算方法,计算的结果会调用对应方法。这种运算符的名称表明,它们会就地修改左操作数,而不会创建新的对象作为结果。对于不可变了类型,一定不能实现就地运算。

如果是可迭代对象,往往需要实现 \texttt{\_\_iadd\_\_} 这类的方法。并且,通过观察 \texttt{list} 方法的实现,我们可以知道,与 \texttt{+} 相比, \texttt{+=} 运算符对第二个操作数更宽容。 \texttt{+} 运算符的两个操作数必须是相同类型,如若不然,结果的类型可能让人摸不着头脑。而 \texttt{+=} 的情况更明确,因为就地修改左操作数,所以结果的类型是确定的。

下面我们使用 \texttt{BingoCage} 的子类,演示 \texttt{+=} 运算符的实现:

\begin{python}
class AddableBingoCage(BingoCage):

    def __add__(self, other):
        if isinstance(self, Tombola):   # 第二个操作数只能是 Tombola 及其子类
            return AddableBingoCage(self.inspect() + other.inspect())
        else:
            return NotImplemented

    def __iadd__(self, other):
        if isinstance(other, Tombola):
            other_iterable = other.inspect()
        else:
            try:
                other_iterable = iter(other)    # iter 将在下一章讨论
            except TypeError:                   # 若失败,告诉用户该怎么做
                self_cls = type(self).__name__
                msg = "right operand in += must be {!r} or an iterable"
                raise TypeError(msg.format(self_cls))
        self.load(other_iterable)   # 如果能执行到这里,载入
        return self     # 增量赋值必须返回 self
\end{python}

此外,我们也无需实现 \texttt{\_\_radd\_\_} 方法,因为不需要。如果右操作数是相同类型,那么正向方法 \texttt{\_\_add\_\_} 会处理。

\newpage