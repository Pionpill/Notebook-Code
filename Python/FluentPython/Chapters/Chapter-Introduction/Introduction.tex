\newpage
\section{Python 数据模型}
\subsection{内置函数}
\subsubsection{内置函数的概念}

Python 语言往往使用 len(object) 的方式获取对象的长度,而 C++,Java 等语言在获取类长度时,需要自行在类中定义相关函数,而后采用成员方法或调用成员变量(object.len())的方式获取相关值。

这是由于 python 编译器使用了名为内置函数的方式,这样有许多好处,例如内置函数的计算速度比成员函数快许多,许多常用的操作在定义内置函数后将变得极其容易使用。

内置函数的通常使用 \_\_function\_\_ 的形式。

\lstinputlisting[language=Python]{../../Scripts/Introduction/1-1.py}

上例中使用了三个内置函数,其中 \_\_init\_\_ 类似于 C++ 的构造函数,用于创建对象。\_\_len\_\_ 函数的定义使得我们可以调用 len() 函数,而 \_\_getitem\_\_ 的定义允许我们对 FrenchDeck 类创建的对象进行迭代,以及类似 C++ 中的数组操作\footnote{原文对这些操作做了精巧的解释,如果读者不清楚,还请阅读原文}。

\subsubsection{内置函数的进一步理解}

内置函数的用处远比一般的成员方法多得多。其左右在 Scripts/introduction 文件下的几个脚本中有所体现。下面归纳几个常用的内置函数的作用。

\begin{itemize}
    \item \textbf{\_\_getitem\_\_}
    
    实现了该方法的类将自带一个迭代器,也即我们可以使用 for x in object 这类的迭代器操作,同时也可以使用 object[i] 这样类似 C++ 通过下标取值的操作。甚至于可以调用标准库的随机函数进行随机选取的操作。

    \item \textbf{\_\_repr\_\_}
    
    该方法的实现,能把一个对象用字符串的形式表达出来以便辨认。如果没有实现该方法,在控制台打印实例时将出现 <xxx object at 0x......> 的形式。对应的还有一个 \_\_str\_\_ 内置方法,该方法实在 str() 函数被调用时使用,或是使用 print() 打印对象时使用,如果 \_\_str\_\_ 没有被实现,在相关操作中将会调用 \_\_repr\_\_替代。
\end{itemize}

这里只列举了两个例子来说明内置函数的强大之处,更多例子在原书中有解释,但原书也并没有对全部内置函数解释,各个内置函数的具体作用还需要在实践使用中理解。

\lstinputlisting[language=Python]{../../Scripts/Introduction/1-2.py}

\newpage