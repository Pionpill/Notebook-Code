\section{函数装饰器和闭包}

函数装饰器用于在源码中``标记''函数,以某种方式增强函数的行为。这是一项强大的功能,掌握之前必须理解闭包。

\subsection{装饰器基础知识}

装饰器是可调用的对象,其参数是另一个函数(被装饰的函数)。\footnote{也有类装饰器,在后文讲到。}装饰器可能会处理被装饰的函数\footnote{处不处理取决于修饰器函数内部的具体实现},然后把它返回,或者将其转换成另一个函数或可调用对象。

假定有个装饰器 \texttt{decorate}。

\begin{python}
@decorate
def target():
    print('running target')
\end{python}

等效写法:

\begin{python}
def target():
    print('running target')

target = decorate(target)
\end{python}

这两种写法的效果是一致的:被修饰后的函数执行完毕后得到的 \texttt{target} 不一定是原来那个 \texttt{target} 函数,而是 \texttt{decorate(target)} 返回的函数。

下面看一个装饰器的例子:

\lstinputlisting[language=Python]{../../Scripts/Function/7-1.py}

严格来说,装饰器只是语法糖。装饰器可以像常规的可调用对象那样调用,其参数是另一个函数。这样做在元编程中非常方便。

装饰器的一大特性是能把被装饰的函数转换成其他函数。第二个特性是,装饰器在加载模块时立即执行。

\subsection{Python 何时执行装饰器}

装饰器的一个关键特性是,他们在被装饰的函数定义之后立即运行。通常是导入模块时。

\lstinputlisting[language=Python]{../../Scripts/Function/7-2.py}

在上例中可以发现,装饰函数 \texttt{register()} 在主函数运行之前运行了两次。如果是导入该模块,则修饰函数同样会被立即调用两次。

考虑到装饰器在真实代码中的常用方式,示例 7-2 有两个不同寻常的地方。
\begin{itemize}
    \item 装饰器通常在一个单独的模块中定义。
    \item 大多数装饰器会在内部定义一个函数,然后将其返回。
\end{itemize}

\subsection{使用装饰器改进``策略''模式}

在电商示例中,遇到的一个问题是当添加新的折扣时,选择最全局佳折扣需要考虑到新添的折扣。下面例子用装饰器解决了这个问题。

\lstinputlisting[language=Python]{../../Scripts/Function/7-3.py}

与之前的方案相比,使用装饰器有几个优点:
\begin{itemize}
    \item @promotion 装饰器突出了被装饰的函数的作用,还便于临时禁用某个促销策略:只把装饰器注释掉。
    \item 促销折扣策略可以在其他模块中定义,在系统任何地方都行,只需要有装饰器。
\end{itemize}

不过,多数装饰器会使用内部函数修改被装饰的函数。使用内部函数的代码几乎都要靠闭包才能正确运作。为了理解闭包,需要了解 Python 中的变量作用域。

\subsection{变量作用域规则}

下面看一个例子:
\begin{python}
    b = 6
    def f2(a):
    print(a)
    print(b)
    b = 9
    
    f2(3)   # 3 UnboundLocalError...
\end{python}

这里先声明了全局变量 \texttt{b},但是在函数运行时却出错了。原因是:Python 编译函数的定义体时, 它判断 \texttt{b} 是局部变量,因为在函数中给它赋值了。Python 会尝试从本地环境获取 \texttt{b} 。后面调用 \texttt{f2(3)} 时,\texttt{f2} 的定义体会获取并打印局部变量 \texttt{a} 的值。但是尝试获取局部变量 b 的值时,发现 \texttt{b} 没有绑定值。也就是说如果给某个与全局变量同名的局部变量赋值了,那么该局部变量需要在声明后才能正确被调用,而不能反过来使用。

这种情形是设计选择:Python 不要求声明变量,但是假定在函数定义体中赋值的变量是局部变量,无论赋值语句在何处。

如果在函数中赋值时想让 b 作为全局变量,需要使用 global 声明,同时对改变量的修改会影响全局:

\lstinputlisting[language=Python]{../../Scripts/Function/7-5.py}

\subsection{闭包}

人们有时会把闭包和匿名函数弄混,这是因为,在函数内部定义函数不常见,直到开始使用匿名函数才会这样做。而且,只有涉及到嵌套函数时才有闭包问题。

闭包指延伸了作用域的函数,其中包括函数定义体中引用,但是不在定义体中定义的非全局变量。函数是不是匿名没有关系,关键是它能访问定义体之外定义的非全局变量。

假如有个名为 \texttt{avg} 的函数,它的作用是计算不断增加的系列值的均值;例如,整个历史中某个商品的平均收盘价。每天都会增加新价格,因此平均值要考虑至目前为止所有的价格。

初学者可能会这样使用类实现:

\lstinputlisting[language=Python]{../../Scripts/Function/7-8.py}

下面使用高阶函数来实现:

\lstinputlisting[language=Python]{../../Scripts/Function/7-9.py}

调用 \texttt{make\_averager} 时,返回一个 \texttt{averager} 函数对象。每次调用 \texttt{averager} 时,它会把参数添加到系列值中,然后计算当前平均值。

在这两个示例中,调用 \texttt{Average()} 或 \texttt{make\_averager()} 得到一个可调用对象 \texttt{avg},它会随历史更新,然后计算当前均值。

\texttt{Average} 类的实例 \texttt{avg} 在哪里存储历史值很明显: \texttt{self.series} 实例属性。但是第二个实例中呢? \texttt{series} 是 \texttt{make\_averager} 函数的局部变量,因为那个函数的定义体中初始化了 \texttt{series} 为空列表。可是,调用 \texttt{avg(10)} 时,\texttt{make\_averager} 函数已经返回,而它的本地作用域也一去不复返。

在 \texttt{averager} 函数中,\texttt{series} 是自由变量。这是一个技术术语,指未在本地作用域中绑定的变量。

审查返回的 averager 对象,我们发现 Python 在 \texttt{\_\_code\_\_} 属性中保存局部变量和自由变量的名称。

\begin{python}
>>> avg.__code__.co_varnames
('new_value','total')
>>> avg.__code__.co_freevars
('series',)
\end{python}

\texttt{series} 的绑定在返回在 \texttt{avg} 函数的 \texttt{\_\_closure\_\_} 属性中。 \texttt{avg.\_\_closure\_\_} 中的各个元素对应于 \texttt{avg.\_\_code\_\_.co\_freevars} 中的一个名称。这些元素是 \texttt{cell} 对象,有个 \texttt{cell\_contents} 属性,保证着真正的值。

\begin{python}
>>> avg.__code__.co_freevars
('series',)
>>> avg.__colsure__
(<cell at 0x...: list object at 0x...>,)
>>> avg.__colsure__[0].cell_contents
[10,11,12]
\end{python}

综上,闭包是一种函数,它会保留定义函数时存在的自由变量的绑定,这样调用函数时,虽然定义作用域不可用了,但是仍能使用这些绑定。

注意,只有嵌套在其他函数中的函数才可能需要处理不在全局作用域中的外部变量。

\subsection{nonlocal 声明}

在示例 7-9 中, \texttt{mark\_averager} 函数的方法效率并不高。因为我们把所有值存储在历史数列中,然后在每次调用 \texttt{averager} 时使用 \texttt{sum} 求和。更好的方法是,只存储目前的总值和元素个数,然后计算平均值。

\lstinputlisting[language=Python]{../../Scripts/Function/7-13.py}

当我们尝试运行时,会出现如下错误:
\begin{python}
    >>> avg = make_averager()
    >>> avg(10)
    ...
    UnboundLocalError: local variable 'count' referenced before assignment
\end{python}

问题是,\texttt{count} 是数字或任何不可变类型时, \texttt{count += 1} 语句会赋予 \texttt{count} 新值,这会把 \texttt{count} 变成局部变量。 \texttt{total} 变量也受这个影响。

示例 7-9 没有出现这个问题,因为 \texttt{series} 是一个列表,是可变对象。但是对于不可变对象,如果尝试重新绑定,则会隐式地创建局部变量。这样就不再是自由变量了,也不会保存在闭包中。

为了解决这个问题,Python3 引入了 \texttt{nonlocal} 声明,其作用是把变量标记为自由变量,即使在函数中为变量赋予新值了,也会变成自由变量。

\subsection{实现一个简单的装饰器}

下面定义了一个修饰器,在每次调用被修饰的函数时计时。

\lstinputlisting[language=Python]{../../Scripts/Function/7-15.py}

这是修饰器的典型行为:把被装饰的函数替换成新函数,二者接受相同的参数,而且(通常)返回被装饰的函数本该返回的值。同时还会做些格外操作。

例 7-15 中实现的 \texttt{clock} 装饰器有几个缺点:不支持关键字参数,而且遮盖了被装饰函数的 \texttt{\_\_name\_\_} 和 \texttt{\_\_doc\_\_} 属性。下例使用 \texttt{functools.wraps} 装饰器把相关的属性从 \texttt{func} 复制到 \texttt{clocked} 中。此外,这个新版还能正确处理关键字参数。

\lstinputlisting[language=Python]{../../Scripts/Function/7-17.py}

\subsection{标准库中的装饰器}

Python 内置了三个用于装饰方法的函数:\texttt{property},\texttt{classmethod},\texttt{staticmethod}。这将在后面章节做详细说明。另一个常见的装饰器是 \texttt{functools.wraps},它的作用是协助构建行为良好的修饰器。在例 7-17 中已经用过

此外标准库中最值得关注的是 \texttt{lru\_cache} 和 \texttt{singledispath}。这两个装饰器都在 \texttt{functools} 模块中定义。

\subsubsection{使用 \texttt{functools.lru\_cache} 做备忘}

\texttt{functools.lru\_cache} 是非常实用的装饰器,它实现了备忘功能,这是一项技术优化,它会把耗时的函数结果保存起来,避免传入同样的参数时重复计算。LRU 三个字母是 ``Least Recently Used'' 的缩写,表示缓存不会无限增长,一段时间不用的缓存条目会被扔掉。

运行下面例子,获取斐波那契数列:

\lstinputlisting[language=Python]{../../Scripts/Function/7-18.py}

在终端运行的结果如下:
\begin{python}
    [0.00000040s] fibonacci(1) -> 1
    [0.00000080s] fibonacci(0) -> 0
    [0.00050050s] fibonacci(2) -> 1
    [0.00000080s] fibonacci(1) -> 1
    [0.00096630s] fibonacci(3) -> 2
    [0.00000120s] fibonacci(1) -> 1
    [0.00000060s] fibonacci(0) -> 0
    [0.00038900s] fibonacci(2) -> 1
    [0.00166720s] fibonacci(4) -> 3
    [0.00000050s] fibonacci(1) -> 1
    [0.00000070s] fibonacci(0) -> 0
    [0.00029650s] fibonacci(2) -> 1
    [0.00000080s] fibonacci(1) -> 1
    [0.00060670s] fibonacci(3) -> 2
    [0.00255880s] fibonacci(5) -> 5
    [0.00000070s] fibonacci(1) -> 1
    [0.00000070s] fibonacci(0) -> 0
    [0.00034090s] fibonacci(2) -> 1
    [0.00000060s] fibonacci(1) -> 1
    [0.00060460s] fibonacci(3) -> 2
    [0.00000050s] fibonacci(1) -> 1
    [0.00000070s] fibonacci(0) -> 0
    [0.00031580s] fibonacci(2) -> 1
    [0.00113680s] fibonacci(4) -> 3
    [0.00403140s] fibonacci(6) -> 8
    8
\end{python}

浪费时间的地方很明显,\texttt{fibonacci(1)} 调用了 8 次,\texttt{fibonacci(2)} 调用了 5 次......如果我们采用 \texttt{lru\_cache} 则会提高效率。

\lstinputlisting[language=Python]{../../Scripts/Function/7-19.py}

这样一来,执行时间将减半,而且 \texttt{n} 的每个值只调用一次函数:

\begin{python}
[0.00000020s] fibonacci(1) -> 1
[0.00000050s] fibonacci(0) -> 0
[0.00020570s] fibonacci(2) -> 1
[0.00028600s] fibonacci(3) -> 2
[0.00036260s] fibonacci(4) -> 3
[0.00041440s] fibonacci(5) -> 5
[0.00046940s] fibonacci(6) -> 8
8
\end{python}

随着运算量级的增大, \texttt{lru\_cache} 修饰器的作用将更为明显。除此以外,\texttt{lru\_cache} 在从 Web 中获取信息的应用中也发挥着巨大的作用。

特别需要注意的是,\texttt{lru\_cache} 可以使用两个可选的参数来配置,这也是使用他作为修饰时需要加括号的原因:

\texttt{functools.lru\_cache(maxsize = 128,typed = False)}

\texttt{maxsize} 指定存储多少个调用的结果。缓存满了之后,旧的结果会被扔掉。为了保存最佳性能,\texttt{maxsize} 的值应该为 2 的幂。 \texttt{typed} 如果设置为 \texttt{True},则会把不同参数类型得到的结果分开保存。此外, \texttt{lru\_cache} 使用字典存储结果,而且键根据调用时传入的定位参数和关键字参数创建,所有被其修饰的函数,它的所有参数都必须是可散列的。

\subsubsection{单分派泛函数}

因为 Python 不支持重载方法或函数,所以我们不能使用不同的签名定义 函数的变体,也无法使用不同的方法处理不同的数据类型。我们可以使用 \texttt{if/elif/elif} 调用对应的函数,但是这样不便于模块的用户拓展,还显得笨拙,各个函数之间的耦合也会更紧密。

Python 3.4 新增的 \texttt{functools.singledispatch} 装饰器可以把整体方案拆封成多个模块,甚至可以为你无法修改的类提供专门的函数。使用 \texttt{@singledispatch} 装饰的普通函数会变成泛函数(generic function):根据第一个参数的类型,以不同方式执行相同操作的一组函数。 

使用 \texttt{@singledispatch} 装饰一个函数,将定义一个泛型函数,获得分派的依据是第一个参数类型。\footnote{原文为 HTML 相关的示例,不是很直观,这里参考了:\url{https://blog.csdn.net/zjz155/article/details/88593291}}

\begin{python}
from functools import singledispatch

@singledispatch
def fun(arg,verbose=False):
    if verbose:
        print("Let me just say,",end = " ")
    print(arg)
\end{python}

使用泛函数的 \texttt{register()} 属性,重载泛函数的实现。泛函数的 \texttt{register()} 属性是一个装饰器。对于有类型注释的函数,这个装饰器将自动匹配第一个参数为该类型的已注册函数重载泛函数。

\begin{python}
@fun.register
def _(arg:int,verbose=False):
    if verbose:
        print("Strength in numbers, eh?", end=" ")
    print(arg)

@fun.register
def _(arg:list,verbose=False):
    if verbose:
        print("Enumerate this:")
    for i,elem in enumerate(arg):
        print(i,elem)
\end{python}

此时,当我们调用泛函数 \texttt{fun} 时,就能根据第一个参数的类型来分派相应的函数来重载实现。

\begin{python}
>>> fun(42,verbose=True)
Strength in number, eh? 42
>>> fun(['spam','spam','eggs'] , verbose=True)
Enumerate this:
0 spam
1 spam
2 eggs
3 spam
\end{python}

如果函数本身没有类型注释,也可以在 register() 装饰器中声明合适的类型。

\begin{python}
@fun.register(complex)
def _(arg , verbose=False)
    if verbose:
        print("Better than complicated", end=" ")
    print(arg.real, arg.img)
\end{python}

为了能注册之前存在的函数和匿名函数,register()属性可以当作功能函数使用。

\begin{python}
def nothing(arg, verbose=False):
print("Nothing.")
    
fun.register(type(None), nothing)
\end{python}

如果泛函数给出的具体类型,没有对应的注册函数的实现,那么泛函数将去寻找更一般化的实现。用\texttt{@singledispatch}装饰的原函数被注册了基本类型 \texttt{object} 类型,也就是说如果找不到更好的实现,那么将使用\texttt{@singledispatch}装饰的原函数:

只要可能,注册专门函数应该处理抽象基类,不要处理具体实现,这样代码支持的兼容类型更广泛。

\texttt{singledispatch} 机制的一个显著特征是,你可以在系统的任何地方和任何模块中注册专门函数。如果后来在新的模块中定义了新的类型,可以轻松地添加一个新的专门函数来处理那个类型。此外,你还可以为不是自己编写地或者不能修改的类型添加自定义函数。

\subsection{叠放装饰器}

由于修饰器是函数,因此可以组合起来使用。下面两个代码块实现的效果是相同的:
\begin{python}
@d1
@d2
def f():
    print('f')
\end{python}

\begin{python}
def f():
    print('f')
f = d1(d2(f))
\end{python}

\subsection{参数化装饰器}

解析源码中的装饰器时,Python 把被装饰的函数作为第一个参数传给装饰器函数。如果要让装饰器接受其他参数,则需要创建一个装饰器工厂函数,把参数传给它,返回一个装饰器,然后再把它应用到要装饰的函数上。

下面以我们见过的最简单的装饰器为例:

\lstinputlisting[language=Python]{../../Scripts/Function/7-22.py}

\subsubsection{一个参数化的注册装饰器}

为了让 \texttt{register} 同时具备可选的注册和注销功能,需要设置一个参数,将其设为 \texttt{False} 时,不注册被装饰的函数。

\lstinputlisting[language=Python]{../../Scripts/Function/7-23.py}

从概念上看,新的 \texttt{register} 函数不是装饰器而是一个装饰器工厂函数。调用它会返回正在的装饰器。这里的关键是, \texttt{register()} 要返回 \texttt{decorate},然后把它应用到被装饰的函数上。

如果不使用 @ 句法,则要像常规函数那样使用 \texttt{register};若想把 \texttt{f} 添加到 \texttt{registry} 中,则装饰 \texttt{f} 的句法是 \texttt{register()(f)},比如说 \texttt{register(active=False)(f1)}。

参数化装饰器的原理相当复杂。参数化装饰器通常会把被装饰的函数替换掉,而且结构上需要多一层嵌套。接下来讨论这种函数金字塔。

\subsubsection{参数化 \texttt{clock} 装饰器}

本节继续讨论 \texttt{clock} 修饰器,添加一个功能:让用户传入一个格式字符串,控制被装饰函数的输出:

\lstinputlisting[language=Python]{../../Scripts/Function/7-25.py}

上面的代码比较复杂,看懂就可以解决平时大部分问题了。

也有人\footnote{原书的技术审校}认为装饰器最好通过实现 \texttt{\_\_call\_\_} 方法的类实现,不应该像本章示例那样通过函数实现。

\newpage