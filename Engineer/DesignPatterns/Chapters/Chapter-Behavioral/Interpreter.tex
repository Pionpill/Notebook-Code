\subsection{Interpreter (解释器)}

\noindent\textbf{意图}

给定一个语言,定义它的文法的一种表示,并定义一个解释器,这个解释器使用该表示来解释语言中的句子。

\noindent\textbf{动机}

如果一种特定类型的问题发生的频率足够高,那么可能就值得将这个问题的各个实例表述为一个简单语言中的句子。这样就可以构建一个解释器,该解释器通过解释这些句子来解决该问题。例如正则表达式。

\noindent\textbf{适用性}

当有一个语言需要解释执行,并且你可将该语言中的句子表示为一个抽象语法树,可以使用解释器模式。而当存在以下情况时该模式效果最好。

\begin{itemize}
    \item \textbf{文法简单}: 对于复杂的文法,文法的类层次变得庞大而无法管理。
    \item \textbf{效率不是关键问题}: 最高效的解释器通常不是通过直接解释语法分析树实现的,而是将它们转换成另一种形式。例如,正则表达式通常被转换成状态机。
\end{itemize}

\noindent\textbf{结构}

\begin{figure}[H]
    \scriptsize
    \centering
    \begin{tikzpicture}[scale = 1]
        \begin{class}[text width=2cm]{Client}{0,0}
        \end{class}
        \begin{class}[text width=2cm]{Context}{4,1.5}
        \end{class}
        \begin{interface}[text width=3cm]{AbstractExpression}{5,0}
            \operation[0]{Interpret(Context)}
        \end{interface}
        \begin{interface}[text width=3cm]{TerminalExpression}{2.5,-2.5}
            \implement{AbstractExpression}
            \operation{Interpret(Context)}
        \end{interface}
        \begin{interface}[text width=3cm]{NonterminalExpression}{7.5,-2.5}
            \implement{AbstractExpression}
            \operation{Interpret(Context)}
        \end{interface}
        \draw[umlcd style,fill=white,->] (Client) -- (AbstractExpression);
        \draw[umlcd style,fill=white,->] (Client) -- (Context);
        \draw[umlcd style,fill=white,fill opacity =0,{diamond}->] (NonterminalExpression) -- ++(2.2cm,0) |- (AbstractExpression);
    \end{tikzpicture}
\end{figure}

\noindent\textbf{参与者}

\begin{itemize}
    \item \textbf{AbstractExpression}: 声明一个抽象的解释操作,这个接口为抽i想语法树中所有的节点所共享。
    \item \textbf{TerminalExpression}: 实现与文法中的终结符相关联的解释操作。一个句子中的每个终结符需要该类的一个实例。
    \item \textbf{NonterminalExpression}: 对文法中的每一条规则都需要一个对应的类。
    \item \textbf{Context}: 包含解释器之外的一些全局信息
\end{itemize}

解释器模式应用场景比较少,这里不做过多介绍。

\newpage