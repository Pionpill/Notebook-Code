\subsection{Iterator (迭代器)}

\noindent\textbf{意图}

提供一种方法顺序访问一个聚合对象中的各个元素,而又不需要暴露该对象的内部表示。

别名: cursor(游标)

\noindent\textbf{动机}

一个聚合对象,如列表(list),应该提供一种方法来让别人可以访问它的元素,而又不需要暴露该对象的内部表示。

简单来说,我们以某种规则(算法)获取某集合对象中的元素,这种算法就可以被单独写成一个迭代器对象,以获得对应集合中的元素。

\noindent\textbf{适用性}

\begin{itemize}
    \item 访问一个聚合对象的内容而无需暴露它的内部表示。
    \item 支持对聚合对象的多种遍历。
    \item 为遍历不同的聚合结构提供一个统一的接口。
\end{itemize}

\noindent\textbf{结构}

\begin{figure}[H]
    \scriptsize
    \centering
    \begin{tikzpicture}[scale = 1]
        \begin{interface}[text width=2cm]{Aggregate}{0,0}
            \operation[0]{CreateIterator()}
        \end{interface}
        \begin{class}[text width=2cm]{Client}{5,0}
        \end{class}
        \begin{interface}[text width=2cm]{Iterator}{10,0}
            \operation[0]{First()}
            \operation[0]{Next()}
            \operation[0]{IsDone()}
            \operation[0]{CurrentItem()}
        \end{interface}
        \begin{class}[text width=3cm]{ConcreteAggregate}{0,-3.5}
            \implement{Aggregate}
            \operation{CreateIterator()}
        \end{class}
        \begin{class}[text width=3cm]{ConcreteIterator}{10,-3.5}
            \implement{Iterator}
        \end{class}
        \draw[umlcd style,fill=white,->] (Client) -- (Aggregate);
        \draw[umlcd style,fill=white,->] (Client) -- (Iterator);
        \draw[umlcd style,fill=white,->] (ConcreteIterator) -- (ConcreteAggregate);
    \end{tikzpicture}
\end{figure}

\noindent\textbf{参与者}
\begin{itemize}
    \item \textbf{Iterator}: 迭代器定义访问和遍历元素的接口。
    \item \textbf{ConcreteIterator}: 具体迭代器实现迭代器接口。对该聚合遍历时跟踪当前位置。
    \item \textbf{Aggregate}: 聚合定义创建相应迭代器对象的接口。
    \item \textbf{ConcreteAggregate}: 具体聚合实现创建相应迭代器的接口,该操作返回 ConcreteIterator 的一个适当的实例。
\end{itemize}

\noindent\textbf{协作}
\begin{itemize}
    \item ConcreteIterator 跟踪聚合中的当前对象,并能够计算出待遍历的后继对象。
\end{itemize}

\noindent\textbf{优缺点}

\begin{itemize}
    \item \textbf{支持以不同的方式遍历一个聚合}
    \item \textbf{简化了聚合的接口}
    \item \textbf{在同一个聚合上可以有多个遍历}
\end{itemize}

\noindent\textbf{例子}

\begin{itemize}
    \item Java: \url{https://blog.csdn.net/zhengzhb/article/details/7610745}
    \item Video: \url{https://www.bilibili.com/video/BV1wK411V791}
\end{itemize}

\lstinputlisting[language=Python]{../../scripts/behavioral/Iterator.py}

\newpage