\subsection{Proxy (代理)}

\noindent\textbf{意图}

为其他对象提供一种代理以控制对这个对象的访问。

别名: Surrogate

\noindent\textbf{动机}

当一个对象进行访问控制的一个原因是: 只有我们确实需要这个对象时才对它进行创建和初始化。

例如我们需要访问谷歌,但由于不可名状的原因,我们无法通过浏览器直接访问。但是我们可以通过一些代理软件简介访问服务器。

总而言之,仅通过代理类才能访问到背后的对象。

\noindent\textbf{适用性}

\begin{itemize}
    \item \textbf{远程代理}: 为一个对象在不同的地址空间提供局部代表。
    \item \textbf{虚代理}: 根据需要创建开销很大的对象。
    \item \textbf{保护代理}: 控制对原始对象的访问。保护代理用于对象应该有不同的访问权限的时候。
    \item \textbf{智能代理}: 取代了简单的指针,再访问对象时执行一些附加操作。
\end{itemize}

\noindent\textbf{结构}

\begin{figure}[H]
    \scriptsize
    \centering
    \begin{tikzpicture}[scale = 1]
        \begin{class}[text width=2cm]{Client}{0,0}
        \end{class}
        \begin{interface}[text width=2cm]{Subject}{5,0}
            \operation[0]{Request()}
        \end{interface}
        \begin{class}[text width=2cm]{RealSubject}{3,-3}
            \implement{Subject}
            \operation{Request()}
        \end{class}
        \begin{class}[text width=2cm]{Proxy}{7,-3}
            \implement{Subject}
            \operation{Request()}
        \end{class}
        \draw[umlcd style,fill=white,->] (Client) -- (Subject);
        \draw[umlcd style,fill=white,->] (Proxy) -- (RealSubject);
    \end{tikzpicture}
\end{figure}

\noindent\textbf{参与者}

\begin{itemize}
    \item \textbf{Proxy}: 保存一个引用使得代理可以访问实体;控制对实体的存取,并可能负责创建和删除它。
    \item \textbf{Subject}: 定义 RealSubject 和 Proxy 的共用接口,这样就在任何使用 RealSubject 的地方都可以使用 Proxy
    \item \textbf{RealSubject}: 定于 Proxy 所代表的实体。
\end{itemize}

\noindent\textbf{协作}

\begin{itemize}
    \item 代理根据其种类,再适当的时候向 RealSubject 转发请求。
\end{itemize}

\noindent\textbf{优缺点}

\begin{itemize}
    \item Peoxy 模式再访问对象时引入了一定程度的间接性。
\end{itemize}

\noindent\textbf{实例}

\begin{itemize}
    \item Java: \url{https://blog.csdn.net/xiaofeng10330111/article/details/105633821}
    \item Video: \url{https://www.bilibili.com/video/BV15V411z7nD}
\end{itemize}

\lstinputlisting[language=Python]{../../scripts/structural/Proxy.py}

\newpage